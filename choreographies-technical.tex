% % arara: pdflatex
% % arara: biber
% % arara: pdflatex
% % arara: pdflatex
% \documentclass[
%   class=../hdeyoung-proposal,
%   crop=false
% ]{standalone}

% \usepackage{ordered-logic}
% \usepackage{basic-atoms}
% \usepackage{tikz-cd}

% \NewDocumentCommand{\chor}{}{X}
% \NewDocumentCommand{\spec}{}{\Sigma}

% \NewDocumentCommand{\erasemsg}{m}{(#1)^{e}}

% \NewDocumentCommand{\trans}{t* t+ o}{%
%   \longrightarrow
%   \IfBooleanT{#1}{^*}\IfBooleanT{#2}{^+}%
%   \IfValueT{#3}{_{#3}}%
% }

% \begin{document}

\subsection{Choreographies, formally}\label{sec:chor-formal}

Hopefully the preceding examples have given some intuition for what counts as a choreography.
To make the definition precise, we need only formalize the locality and specification-preserving properties.

\subsubsection{Locality}\label{sec:locality}

As discussed in \cref{sec:chor-example-counter}, clauses such as $\bit{1} \fuse \inc[<-] \lrimp \monad{\inc[<-] \fuse \bit{0}}$ satisfy locality because the premise consists of a process that receives a message.
Although it is possible to admit clauses like this one, they come at the expense of complicating the statement of locality slightly.
Instead, it is more convenient to require that clauses be given in curried form, such as $\bit{1} \lrimp (\inc[<-] \rimp \monad{\inc[<-] \fuse \bit{0}})$.
% Because currying is easy enough to do in a preprocessing phase, this requirement is not onerous.

For curried clauses, locality holds if there is exactly one process-like atom, $\patom^+$, in the premises and if any message-like atoms, $\matom[<-]$ and $\matom[->]$, are used only with right- and left-ordered implications, respectively.

\begin{definition}[Locality]
  A clause, which must have form $\patom^+ \lrimp A^-$, satisfies \vocab{locality} if it adheres to the following refined grammar.\footnote{An alternative notation for $\with*[i \in I][\parens]{\matom[<-]^+_i \rimp \monad{A^+_i}}$ could be the Lambek-inspired $\with*[i \in I][\parens]{\monad{A_i} \twoheadleftarrow \matom[<-]^+_i}$, which would better emphasize that left-directed messages, $\matom[<-]^+_i$, must arrive from the right.}
  \begin{alignat*}{3}
    &\text{Negative propositions} &\quad&& A^-,B^- &::= \monad{A^+} \mid \with*[i \in I][\parens]{\matom[<-]^+_i \rimp \monad{A^+_i}} \mid \with*[i \in I][\parens]{\matom[->]^+_i \limp \monad{A^+_i}} \\
    % \matom[<-]^+ \rimp \monad{A^+} \mid \matom[->]^+ \limp \monad{A^+} \\
    &\text{Positive propositions}      &&& A^+,B^+ &::= A^+ \fuse B^+ \mid \one \mid \matom[<-]^+ \fuse A^+ \mid A^+ \fuse \matom[->]^+ \mid \patom^+ \mid A^-
  \end{alignat*}
\end{definition}

% \noindent
Notice that, for technical reasons related to the translation to session-typed processes (\cref{sec:translation}), this definition also requires that message-like atoms $\matom[<-]$ and $\matom[->]$ never appear in clause heads except as part of a conjunction---either $\matom[<-] \fuse A^+$ or $A^+ \fuse \matom[->]$.
Because $\one$ is available, this requirement does not restrict expressiveness however.

As a consequence of this document's focus on \emph{well-typed} processes, also notice the more fundamental omission of propositions like $(\matom[->]^+_1 \limp \monad{A^+_1}) \with (\matom[<-]^+_2 \rimp \monad{A^+_2})$.
Intuitively, such a proposition corresponds to an input-guarded nondeterministic choice---the process chooses either to receive $\matom[->]^+_1$ from the left and continue as process $A^+_1$, or to receive $\matom[<-]^+_2$ from the right and continue as process $A^+_2$.
Nondeterministic choice is not typable in the current \acs{SILL} typing scheme of \autocite{Caires+:MSCS13}.
These propositions are ruled out by requiring all arms of an additive conjunction to be consistently left- or right-implications.

With some simple transformations, the choreographies presented earlier do indeed satisfy locality.
For example, the $\inc[<-]$-choreography can be put into the curried form 
\begin{equation*}
  \!\begin{aligned}
    &\eps \lrimp (\inc[<-] \rimp \monad{\eps \fuse \bit{1}}) \\
    &\bit{0} \lrimp (\inc[<-] \rimp \monad{\bit{1}}) \\
    &\bit{1} \lrimp (\inc[<-] \rimp \monad{\inc[<-] \fuse \bit{0}})
    \text{\,.}
  \end{aligned}
\end{equation*}
Likewise, after some transformations, the $\bit{}[->]$-choreography also satisfies locality:
\begin{equation*}
  \inc \lrimp \parens[auto, align=c@{\,}l]{
                    & (\eps[->] \limp \monad{(\one \fuse \eps[->]) \fuse \bit{1}[->]}) \\[0.5\jot]
              \with & (\bit{0}[->] \limp \monad{\one \fuse \bit{1}[->]}) \\[0.5\jot]
              \with & (\bit{1}[->] \limp \monad{\inc \fuse \bit{0}[->]})}
\end{equation*}

\subsubsection{Specification-preserving}\label{sec:spec-pres}

To judge that a signature is specification-preserving, we rely on a notion of erasure that removes the assigned roles, translating message- and process-like atoms to ordinary atoms.
\begin{definition}[Role erasure]
  For atomic propositions, the \vocab{role erasure} $\erasemsg{-}$ is given by
  \begin{align*}
    \erasemsg{\matom[<-]^+} &= \matom^+ = \erasemsg{\matom[->]^+} \\
    % \erasemsg{\matom[<-]^+ \fuse A^+} &= \matom^+ \fuse \erasemsg{A^+} \\
    % \erasemsg{A^+ \fuse \matom[->]^+} &= \erasemsg{A^+} \fuse \matom^+ \\
    \erasemsg{\patom^+} &= \patom^+
    \,.
  \end{align*}
  Role erasures for propositions and contexts, $\erasemsg{A^+}$, $\erasemsg{A^-}$, and $\erasemsg{\octx}$, are defined compositionally, lifting role erasure for atoms.
\end{definition}
For instance, the role erasure of $\inc[<-]$ is $\erasemsg{\inc[<-]} = \inc$, matching the intuition that the message-like atom $\inc[<-]$ in the $\inc[<-]$-choreography (\cref{sec:chor-example-counter}) serves to implement the specification's $\inc$ atom.

Using role erasure, we can define the specification-preserving property as follows:
\begin{definition}[Specification-preserving]
  A signature $\chor$ is \vocab{specification-preserving} for the specification $\spec$ if:
  $\octx \trans[\chor] \octx'$ under the signature $\chor$ if and only if $\erasemsg{\octx} \trans[\spec] \erasemsg{\octx'}$ under the specification $\spec$.
  In other words, $\chor$ is specification-preserving for $\spec$ if $\erasemsg{-}$ is a bisimulation.
\end{definition}

Thus, to be specification-preserving, a choreography must be lock-step equivalent with its specification.
For example, the $\inc[<-]$-choreography for the binary counter specification is indeed specification-preserving because the two are lock-step equivalent.
\Cref{fig:spec-pres-inc} shows how the steps correspond using bisimulation diagrams.

\begin{figure}[!t]
\begin{gather*}
  \begin{tikzcd}[arrow style=math font, ampersand replacement=\&]
    \omatch[^e]{\eps , \inc} \rar \dar[dash, "\erasemsg{-}"'] \& \omatch[^e]{\eps , \bit{1}} \dar[dash, "\erasemsg{-}"] \\
    \omatch{\eps , \inc[<-]} \rar \& \omatch{\eps , \bit{1}}
  \end{tikzcd}
  \qquad\qquad
  \begin{tikzcd}[arrow style=math font, ampersand replacement=\&]
    \omatch[^e]{\bit{0} , \inc} \rar \dar[dash, "\erasemsg{-}"'] \& \omatch[^e]{\bit{1}} \dar[dash, "\erasemsg{-}"] \\
    \omatch{\bit{0} , \inc[<-]} \rar \& \omatch{\bit{1}}
  \end{tikzcd}
  \\[3\jot]
  \begin{tikzcd}[arrow style=math font, ampersand replacement=\&]
    \omatch[^e]{\bit{1} , \inc} \rar \dar[dash, "\erasemsg{-}"'] \& \omatch[^e]{\inc , \bit{0}} \dar[dash, "\erasemsg{-}"] \\
    \omatch{\bit{1} , \inc[<-]} \rar \& \omatch{\inc[<-] , \bit{0}}
  \end{tikzcd}
\end{gather*}
\caption{The $\inc[<-]$-choreography of \cref{sec:chor-example-counter} is specification-preserving.\label{fig:spec-pres-inc}}
\end{figure}


% \subsection{}

% Hopefully the preceding examples have given some intuition for what counts as a choreography.
% To make the definition precise, we need only formalize the locality and specification-preserving properties.

% We use the forward-chaining ordered logic programming language described in \cref{sec:ordered-lp}, with a few restrictions.
% First, we confine ourselves to the purely propositional fragment of ordered logic programming.
% Second, to simplify the statement of locality, we require that each implication have exactly one premise: each implication has the form of $A^+ \rimp \monad{B^+}$, $A^+ \limp \monad{B^+}$, or $A^+ \lrimp \monad{B^+}$.
% These uncurried implications would not seem to place a large burden on the programmer, for it is easy enough to write $\p[_{\mathrm{2}}]^+ \fuse \p[_{\mathrm{1}}]^+ \fuse \p[_{\mathrm{3}}]^+ \lrimp \monad{B^+}$ in place of $\p[_{\mathrm{1}}]^+ \lrimp \p[_{\mathrm{2}}]^+ \limp \p[_{\mathrm{3}}]^+ \rimp \monad{B^+}$, for example.

% These restrictions can be lifted, but at the expense of complicating the presentation; because the restrictions are orthogonal to the main points of discussion, we prefer the simplifying restrictions for the present.

% \begin{alignat*}{2}
%   &\text{Negative propositions}\quad & A^- &::= A^+ \rimp \monad{B^+} \mid A^+ \limp \monad{B^+} \mid A^- \with B^- \\
%   &\text{Positive propositions}      & A^+ &::= A^+ \fuse B^+ \mid \one \mid \p^+ \mid \p[->]^+ \mid \p[<-]^+ \mid A^-
% \end{alignat*}
 


% \subsubsection{Locality}\label{sec:locality}

% \begin{definition}[Locality]
%   A clause $A^+ \lrimp \monad{B^+}$ is \emph{local} if its premise adheres to the grammar refinement $O^+$:
%   \begin{alignat*}{2}
%     O^+ &::= L^+ \fuse O^+ \mid \p^+ \mid O^+ \fuse R^+ \\
%     L^+ &::= \p[->]^+ \mid L^+_1 \fuse L^+_2 \mid \one \\
%     R^+ &::= \p[<-]^+ \mid R^+_1 \fuse R^+_2 \mid \one
%   \end{alignat*}
%   % Similarly, a right implication, $A^+ \rimp \monad{B^+}$, is local if its premise adheres to the grammar of $R^+$.
%   % Dually, a left implication, $A^+ \limp \monad{B^+}$, is local if its premise adheres to the grammar of $L^+$.
% \end{definition}


% A clause $A^+ \lrimp \monad{B^+}$ is \emph{local} if its premise adheres to the grammar refinement $O^+$:
% \begin{alignat*}{2}
%   O^+ &::= L^+ \fuse O^+ \mid \p^+ \mid O^+ \fuse R^+ \\
%   L^+ &::= \p[->]^+ \mid L^+_1 \fuse L^+_2 \mid \one \\
%   R^+ &::= \p[<-]^+ \mid R^+_1 \fuse R^+_2 \mid \one
% \end{alignat*}
% The grammar may appear to be a bit complicated, but the idea behind it is simple enough and matches the intuition behind locality:
% a local premise $O^+$ contains exactly one process atom $\p^+$ that receives right-directed messages, $\p[->]^+$, from its left and left-directed messages, $\p[<-]^+$, from its right.

% Similarly, a right implication, $A^+ \rimp \monad{B^+}$, is local if its premise adheres to the grammar of $R^+$.
% That is, right implications must receive only left-directed messages that are arriving at the implication's right-hand side.
% No process atom may appear in the premise because, as an ephemeral resource, the implication itself acts as the recipient process.
% Dually, a left implication, $A^+ \limp \monad{B^+}$, is local if its premise adhers to the grammar of $L^+$.


% % At the expense of a more complicated grammar, locality could be extended to implications with multiple premises.
% % , we could allow curried clauses, such as $\p[_1, ->]^+ \lrimp \p[_2, ->]^+ \limp \p^+ \rimp \p[_3, <-]^+ \rimp \monad{A^+}$.
% % Uncurrying clauses would not seem to place a large burden on the programmer, for it is easy enough to write $\q[->]^+ \fuse \p[->]^+ \fuse \rr^+ \fuse \s[<-]^+ \lrimp \monad{A^+}$, and this detail is anyway orthogonal to what follows.

% \subsubsection{Specification-preserving}\label{sec:spec-pres}

% To judge that an ordered logic program is specification-preserving, we rely on a notion of erasure that relates message atoms to process atoms.
% \begin{definition}[Message erasure]
%   For atomic propositions, the \vocab{message erasure} $\erasemsg{-}$ is given by
%   \begin{equation*}
%     \erasemsg{\p[->]^+} = \erasemsg{\p[<-]^+} = \erasemsg{\p^+} = \p^+
%     \,.
%   \end{equation*}
%   Message erasure of propositions and contexts, $\erasemsg{A^+}$, $\erasemsg{A^-}$, and $\erasemsg{\octx}$, is defined compositionally as the lifting of message erasure for atoms.
% \end{definition}
% For instance, the message erasure of $\inc[<-]$ is $\erasemsg{\inc[<-]} = \inc$, matching the intuition that the message $\inc[<-]$ serves to implement the specification's atom $\inc$ in the choreography from \cref{sec:chor-example-counter}.

% Using this definition, we can define the property of specification-preserving as follows:
% \begin{definition}[Specification-preserving]
%   An ordered logic program $\chor$ is \vocab{specification-preserving} for specification $\spec$ if:
%   \begin{enumerate}
%   \item\label{defn:specification-preserving:completeness} for each step $\octx \trans[\spec] \octx'$ in the specification $\spec$, there is a non-empty trace $\octx \trans+[\chor] \octx'$ in the program $\chor$; and
%   \item\label{defn:specification-preserving:soundness} for each step $\octx \trans[\chor] \octx'$ in the program $\chor$, either $\erasemsg{\octx} = \erasemsg{\octx'}$ or there is a step $\erasemsg{\octx} \trans[\spec] \erasemsg{\octx'}$ in the specification $\spec$.
%   \end{enumerate}
% \end{definition}
% Thus, to be specification-preserving for $\spec$, a choreography $\chor$ must be a sound and complete implementation of specification $\spec$:
% every step in $\spec$ must be reproducible by a non-empty trace in $\chor$ (part \labelcref{defn:specification-preserving:completeness}, completeness); and every step in $\chor$ must be either silent or its erasure reproducible by a single step in $\spec$ (part \labelcref{defn:specification-preserving:soundness}, soundness).

% % The first part of this definition requires the choreography $\chor$ to be a complete implementation of the specification $\spec$: every step in $\spec$ must be reproducible by a trace in $\chor$.
% % The second part requires $\chor$ to be a sound implementation of $\spec$: every step in $\chor$ must be either silent or reproducible by a single step in $\spec$.

% \tikzcdset{arrow style=math font}
% \tikzset{subscript/.style={shorten >=0.5em, "\ensuremath{#1}" {inner sep=0pt, sloped, at end, below right}}}

% As an example, the object-oriented choreography for the binary counter, given in \cref{sec:chor-example-counter}, is indeed specification-preserving.
% \begin{equation*}
%   \begin{tikzcd}
%     \omatch{\bit{1}, \inc}
%       \arrow[rr, subscript=\spec]
%       \arrow[dr, start anchor=south east, end anchor=north west, subscript=\chor]
%       &
%       & \ofill{\inc, \bit{0}} \\
%     & \ofill{\bit{1}, \inc[<-]} \arrow[ur, start anchor=north east, end anchor=south west, subscript=\chor] &
%   \end{tikzcd}
% \end{equation*}

% \begin{equation*}
%   \begin{tikzcd}
%     \omatch{\bit{1}, \inc[<-]}  \rar[subscript=\chor]  \dar[dash, "\erasemsg{-}" {below, sloped}]
%       & \ofill{\inc, \bit{0}}   \dar[dash, "\erasemsg{-}"' {above, sloped}] \\
%     \ofill[^e]{\bit{1}, \inc}   \rar[subscript=\spec]
%       & \ofill[^e]{\inc, \bit{0}}
%   \end{tikzcd}
% \end{equation*}
% \begin{gather*}
%   \begin{lgathered}
%     \omatch{\bit{1}, \inc} \trans[\spec] \ofill{\inc, \bit{0}} \\
%     \ofill{\bit{1}, \inc} \trans[\chor] \ofill{\bit{1}, \inc[<-]} \trans[\chor] \ofill{\inc, \bit{0}}
%   \end{lgathered}
%   \\
%   \begin{lgathered}
%     \omatch{\eps, \inc} \trans[\spec] \ofill{\eps, \bit{1}} \\
%     \ofill{\eps, \inc} \trans[\chor] \ofill{\eps, \inc[<-]} \trans[\chor] \ofill{\eps, \bit{1}}
%   \end{lgathered}
%   \\
%   \begin{lgathered}
%     \omatch{\bit{0}, \inc} \trans[\spec] \ofill{\bit{1}} \\
%     \ofill{\bit{0}, \inc} \trans[\chor] \ofill{\bit{0}, \inc[<-]} \trans[\chor] \ofill{\bit{1}}
%   \end{lgathered}
% \end{gather*}

% \begin{gather*}
%   \begin{lgathered}
%     \omatch{\inc} \trans[\chor] \ofill{\inc[<-]} \\
%     \erasemsg{\ofill{\inc}} = \ofill[^e]{\inc} = \erasemsg{\ofill{\inc[<-]}}
%   \end{lgathered}
%   \\
%   \begin{lgathered}
%     \omatch{\bit{1}, \inc[<-]} \trans[\chor] \ofill{\inc, \bit{0}} \\
%     \erasemsg{\ofill{\bit{1}, \inc[<-]}} = \ofill[^e]{\bit{1}, \inc} \trans[\spec] \ofill[^e]{\inc, \bit{0}} = \erasemsg{\ofill{\inc, \bit{0}}}
%   \end{lgathered}
%   \\
%   \begin{lgathered}
%     \omatch{\eps, \inc[<-]} \trans[\chor] \ofill{\eps, \bit{1}} \\
%     \ofill{\eps, \inc} \trans[\spec] \ofill{\eps, \bit{1}}
%   \end{lgathered}
%   \\
%   \begin{lgathered}
%     \omatch{\bit{0}, \inc[<-]} \trans[\chor] \ofill{\bit{1}} \\
%     \ofill{\bit{0}, \inc} \trans[\spec] \ofill{\bit{1}}
%   \end{lgathered}
% \end{gather*}

% \end{document}

%%% Local Variables:
%%% TeX-master: "choreographies"
%%% End:
