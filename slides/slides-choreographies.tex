% arara: pdflatex
\documentclass{beamer}
\usepackage{my-proposal-talk}

\usepackage{pifont}


\renewcommand*{\inc}{\color{structure}\mathtt{s}\color{black}}
\renewcommand*{\eps}{\color{structure}\mathtt{e}\color{black}}
\renewcommand*{\bit}[1]{\color{structure}\mathtt{#1}\color{black}}

\NewDocumentCommand \bits { }{ \mathord{\color{structure} B} }
\NewDocumentCommand \val { m }{ \llcorner #1 \lrcorner }

\newcommand*{\zeroq}{\mathtt{z\mkern-2mu \text{\texttt{?}}}}
\newcommand*{\zerott}{\mathtt{t\mkern-2mu t}}
\newcommand*{\zeroff}{\mathtt{f\mkern-2mu f}}


\RenewDocumentCommand \strcons { }
  { \mskip4mu }


\NewDocumentCommand \showcolumnwidth { }{%
  \tikz { \draw [|-|] (0,0) -- (\linewidth,0); }%
}


\newenvironment{checkenv}{%
  \only{%
    \setbeamertemplate{itemize item}{{\color{example}\ding{51}}}%
  }%
}
{}

\newenvironment{crossenv}{%
  \only{%
    \setbeamertemplate{itemize item}{\alert{\ding{55}}}%
  }%
}
{}


\begin{document}


\colorlet{example}{example text.fg}


\begin{frame}
  \frametitle{Letters as processes}

  \begin{equation*}
    \begin{tikzpicture}[trim left=(e.west), ellipse highlight/.append style={overlay}]
      \graph [
        sill graph/.alt=<-+>{0}{0.5}, nodes={process}, nodes/.only=<-.>{text height=1.5ex},
        nodes/.alt=<+->{}{transparent, text opaque},
        edges/.alt=<+->{}{transparent},
        /tikz/channel/.append style/.alt=<+>{}{transparent},
        /utils/exec/.only=<+>,
        /tikz/execute at end scope/.only=<+>{\node [ellipse highlight={($(i2.east)!1.33!(s right)$) ($(i2.east)!.67!(s right)$)}] {};},
        /tikz/execute at end scope/.only=<+>{\node [ellipse highlight={($(e.west)!0.67!(s left)$) ($(e.west)!1.33!(s left)$)}] {};},
        /utils/exec/.only=<+>,
        /tikz/execute at end scope/.only=<+>{\redex[ultra thick em]{(b) (i1)}},
      ] {
        s right / [coordinate]
         -> ["$c_0$" channel, transparent/.only=<.(2)-.(3)>]
        i2 / "\inc" [transparent/.only=<+(2)-+(3)>]
         -> ["$c_1$" channel]
        i1 / "\temporal<.-.(1)>{\inc}{\bit{0}}{\alt<.(2)-.(3)>{\bit{1}}{\inc}}"
         -> ["$c_2$" channel]
        b / "\alt<.-.(1)>{\inc}{\bit{1}}"
         -> ["$c_3$" channel]
        e / "\eps"
         -> ["$c_4$" channel, transparent/.only=<.(-2)->]
        s left / [coordinate];
      };

      \redex<+>[ultra thick em]{(e) (b)}
      \redex<.>[ultra thick em]{(i1) (i2)}

      \redex<+(2)->[ultra thick em]{(b) (i1)}
      \visioncircle<+(2)>{(e.east)}{(i1.west)}{{white}{50}}
      \visioncircle<+(2)->{(b.east)}{(i2.west)}{{white}{50}}

      \path (s right) +(0,2.3em) +(0,-2.3em);
      % \draw [red] (current bounding box.north west) rectangle (current bounding box.south east);
    \end{tikzpicture}
  \end{equation*}

  What if letters are processes? (cf.\ [Miller '92])
  \begin{itemize}
  \item Strings are process and message chains.
  \item Rewriting rules are chain transformations:
    \begin{itemize}
    \item[] \hspace{-1.5em}
      $\begin{tikzpicture}[baseline=(b.base)]
        \graph [sill graph=0.5, nodes={process}] {
          / [coordinate] -> "\inc" -> b / "\bit{1}" -> / [coordinate];
        };
      \end{tikzpicture}
        \trans
      \begin{tikzpicture}[baseline=(b.base)]
        \graph [sill graph=0.5, nodes={process}] {
          / [coordinate] -> b / "\bit{0}" -> "\inc" -> / [coordinate];
        };
      \end{tikzpicture}$, etc.
    \end{itemize}
  \end{itemize}
  \bigskip\medskip

  \onslide<+(-2)->
  This is not a local implementation!
  \begin{itemize}
  % \item It assumes an omniscient coordinator, or bounded-omniscient processes.
  \item Assumes ``processes'' have a degree of omniscience
  \item Similar to a one-dimensional cellular automaton
  \end{itemize}
\end{frame}


\begin{frame}
  \frametitle{Letters as processes \emph{and} messages}

  \onslide<+->
  \begin{equation*}
    \begin{tikzpicture}[trim left=(e.west)]
      \graph [sill graph=0.5, nodes={process}] {
        / [coordinate]
         ->
        i2 / "\inc" [transparent/.only=<+->]
         ->
        i1 / "\inc" [transparent/.only=<.->]
         ->
        b / "\alt<-.(3)>{\bit{1}}{\bit{0}}"
         ->
        e / "\eps"
         -> [transparent]
        / [coordinate];

        (i2.east) -> ["$\inc$" {message, name=i2 msg}, transparent/.only=<-.(-1)>] (i2.west);
        (i1.east) -> ["$\inc$" {message, name=i1 msg, transparent/.only=<+(2)->}, transparent/.only=<-.(-1)>] (i1.west);
      };

      \redex<+(-1)-+>[ultra thick em]{(b) (i1 msg)}

      \visioncircle<+(-1)-+(1)>{(e.east)}{(i1.west)}{{white}{50}}

      \path<+(-1)> (i1) edge ["$\inc$" {message, name=i1 msg', somewhat near end}] (b);
      \redex<.(-1)>[ultra thick em]{(b) (i1 msg')}

      \path<+(-1)-> (b) edge ["$\inc$" {message, name=i1 msg'', somewhat near start}] (e);

      \redex<+->{(e) (i1 msg'')}
      \redex<.->{(b) (i2 msg)}

      \path (e.east) +(0,2.3em) -- +(0,-2.3em);
      % \draw [red] (current bounding box.north west) rectangle (current bounding box.south east);
    \end{tikzpicture}
  \end{equation*}

  What if some letters are processes and some are messages?
  \begin{itemize}
  \item Strings are process chains
  \item Rewriting rules are \emph{local} process implementations
    \begin{itemize}
    \item[] \hspace{-1.5em}
      $\begin{tikzpicture}[baseline=(b.base)]
        \graph [sill graph=0.5, nodes={process}] {
          / [coordinate] ->["$\inc$" message] b / "\bit{1}" -> / [coordinate];
        };
      \end{tikzpicture}
        \trans
      \begin{tikzpicture}[baseline=(b.base)]
        \graph [sill graph=0.5, nodes={process}] {
          / [coordinate] -> b / "\bit{0}" ->["$\inc$" message] / [coordinate];
        };
      \end{tikzpicture}$, etc.
    \end{itemize}
  \end{itemize}
  \bigskip\medskip

  \onslide<+->
  Assignment of letters to process/message roles is a \structure{\emph{choreography}}
  \begin{itemize}
  \item $\eps$, $\bit{0}$, and $\bit{1}$ are processes in the example choreography, whereas $\inc$, $\zeroq$, $\zerott$, and $\zeroff$ are messages.
  \end{itemize}
\end{frame}


\begin{frame}
  \frametitle{Choreographies must be local}

  \onslide<+->

  {\large Not all role assignments are well-formed choreographies.}
  \begin{itemize}[<+- | cross@1->]
  \item<.- | cross@1-> Two or more processes:\enskip% \hfill
    {\footnotesize
    $\begin{tikzpicture}[baseline=(b.base)]
      \graph [sill graph=0.5, nodes={process}] {
        / [coordinate] -> "{}" -> b / "\phantom{\inc}" -> / [coordinate];
      };
    \end{tikzpicture}
      \trans
    \dotsb$}

  \item Only messages:\enskip% \hfill
    {\small
    $\raisebox{.5ex}[.75em+.5ex]{$\begin{tikzpicture}[baseline=(c)]
      \graph [sill graph=0.5, nodes={process}] {
        c / [coordinate] ->["$\phantom{\inc}$" {message, overlay}] / [coordinate] ->["$\phantom{\inc}$" {message, overlay}] / [coordinate];
      };
    \end{tikzpicture}$}
      \trans
    \dotsb$}

  \item Process and two incoming messages:\enskip% \hfill
    {\footnotesize
    $\begin{tikzpicture}[baseline=(b.base)]
      \graph [sill graph=0.5, nodes={process}] {
        / [coordinate] ->["$\phantom{\inc}$" {message, overlay}] b / "\phantom{\inc}" ->["$\phantom{\inc}$" {message=right, somewhat near end}] / [coordinate];
      };
    \end{tikzpicture}
      \trans
    \dotsb$}

  \item Process and an outgoing message:\enskip% \hfill
    {\footnotesize
    $\begin{tikzpicture}[baseline=(b.base)]
      \graph [sill graph=0.5, nodes={process}] {
        / [coordinate] -> b / "\phantom{\inc}" ->["$\phantom{\inc}$" {message, overlay}] / [coordinate];
      };
    \end{tikzpicture}
      \trans
    \dotsb$}
  \end{itemize}
  \medskip

  \onslide<+->

  {\large Well-formed choreographies have premises:}
  \begin{itemize}[<+- | check@1->]
  \item A process and an incoming message:\\\qquad% \hfill
    {\footnotesize
    $\begin{tikzpicture}[baseline=(b.base)]
      \graph [sill graph=0.5, nodes={process}] {
        / [coordinate] ->["$\phantom{\inc}$" message] b / "\phantom{\inc}" -> / [coordinate];
      };
    \end{tikzpicture}
      \trans
    \dotsb$
    {\normalsize or}\enskip
    $\begin{tikzpicture}[baseline=(b.base)]
      \graph [sill graph=0.5, nodes={process}] {
        / [coordinate] -> b / "\phantom{\inc}" ->["$\phantom{\inc}$" {message=right, somewhat near end}] / [coordinate];
      };
    \end{tikzpicture}
      \trans
    \dotsb$}

  \item A process:\enskip% \hfill
    {\footnotesize
    $\begin{tikzpicture}[baseline=(b.base)]
      \graph [sill graph=0.5, nodes={process}] {
        / [coordinate] -> b / "\phantom{\inc}" -> / [coordinate];
      };
    \end{tikzpicture}
      \trans
    \dotsb$}
  \end{itemize}
  \bigskip

  \onslide<+->
  \structure{\large Definition (Locality).}\enskip A choreography is \emph{local} if each rewriting rule's premise then contains exactly one process, at most one incoming message, and no outgoing messages.
\end{frame}


\end{document}
