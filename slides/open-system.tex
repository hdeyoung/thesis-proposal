% arara: pdflatex
% arara: pdflatex
\documentclass[beamer]{standalone}
\usepackage{my-proposal-talk}

\tikzset{
  visible/.style args={<#1>#2}{
    alt={<#1>{#2}{transparent}}
  }
}

\tikzset{
  hypothetical/.style={dashed, fill=none, nearly transparent},
  graphs/visible/.style args={<#1>}{nodes={visible={<#1>}}, edges={visible={<#1>}}},
  graphs/hypothetical graph/.style={nodes={process, hypothetical}, edges={hypothetical}},
}

\begin{document}
\begin{standaloneframe}
  \frametitle{}
  \onslide<+->

  \begin{equation*}
    \begin{tikzpicture}
      \graph [sill graph=0.5] {
        { [hypothetical graph, /utils/exec/.only=<+>, visible=<.(1)->]
          / [coordinate] -> C -> / [coordinate]
        }
         ->
        P
         ->
        { [hypothetical graph, /utils/exec/.only=<+>, visible=<.(-1)->]
          / [coordinate] -> S -> / [coordinate]
        };
      };
    \end{tikzpicture}
  \end{equation*}

  \begin{itemize}
  \item A process offers a service to its client.
  \item A process may also use an outside service.
  \end{itemize}
  
  \begin{equation*}
    \begin{tikzpicture}
      \graph [sill graph=0.5] {
        / [coordinate]
         -> [hypothetical]
        C [process, hypothetical]
         -> ["$\inc$" message]
        / [coordinate]
         -> ["$\inc$" message]
        "\bit{1}" [process]
         ->
        "\eps" [process, hypothetical];
      };
    \end{tikzpicture}
  \end{equation*}
\end{standaloneframe}
\end{document}
