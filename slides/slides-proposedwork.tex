% arara: pdflatex
\documentclass{beamer}
\usepackage{my-proposal-talk}
% \usepackage{multicols}

\usepackage{zlmtt}
\usepackage{listings}


\renewcommand*{\inc}{\color{example}\mathtt{s}\color{black}}
\renewcommand*{\eps}{\color{structure}\mathtt{e}\color{black}}
\RenewDocumentCommand \bit { m O{} }{ \mathord{\color{structure}\text{\texttt{#1}}#2} }

\NewDocumentCommand \bits { }{ \mathord{\color{structure} B} }
\NewDocumentCommand \bitplus { }{ \mathord{\text{\texttt{\color{structure} +}}} }
\NewDocumentCommand \bitequals { O{} }{ \mathord{\color{structure}\text{\texttt{=#1}}} }
\NewDocumentCommand \dec { }{ \mathord{\mathtt{\color{example} d}} }
\NewDocumentCommand \bitlinc { }{ \mathord{\text{\texttt{\color{example} <}}\mkern-4mu\text{\texttt{\color{example} <}}} }
\NewDocumentCommand \bitrskip { }{ \mathord{\text{\texttt{\color{example} >}}\mkern-4mu\text{\texttt{\color{example} >}}} }

\NewDocumentCommand \blenq { }{ \mathord{\mathtt{len}\text{\texttt{?}}} }
\NewDocumentCommand \blen { m }{ \mathop{\mathtt{len}} #1 }

\NewDocumentCommand \valq { O{0} }{ \mathord{\color{example}\mathtt{v}\text{\smaller[#1]\texttt{?}}} }
\NewDocumentCommand \val { m }{ \mathop{\color{example}\mathtt{v}} #1 }

\NewDocumentCommand \dbl { }{ \mathop{\mathsf{dbl}} }

\newcommand*{\zeroq}{\mathtt{z\mkern-2mu \text{\texttt{?}}}}
\newcommand*{\zerott}{\mathtt{t\mkern-2mu t}}
\newcommand*{\zeroff}{\mathtt{f\mkern-2mu f}}

\RenewDocumentCommand \strcons { }
  { \mskip4mu }
\NewDocumentCommand \msetcons { }{ \mathbin{,} }


\lstset{
  basicstyle=\ttfamily,
  escapeinside={\%*}{*)},
  keepspaces=true,
  keywordstyle=\color{purple},
  morekeywords={proc,recv,send,case,of,stype,and},
  emph={eps,bit0,bit1,bit0'}, emphstyle=\color{structure},
  emph={[2]s,z?,tt,ff,v,v?}, emphstyle={[2]\color{example}},
  emph={[3]l,m,r}, emphstyle={[3]\itshape},
  columns=fixed,
  basewidth=0.5em,
}


\begin{document}

\colorlet{example}{example text.fg}

\begin{frame}[fragile]
  \frametitle{Value represented by a bit string}

  \begin{tabular}{@{}l@{\quad}l@{\qquad\quad}l@{}}
    % &&
    % \begin{lstlisting}[gobble=6]
    %   stype Nat = &{ ..., v%*\texttt{\color{example}?}*): Nat' }
    %     and Nat' = +{ v: int /\ Nat }
    % \end{lstlisting}
    % %
    % \\
    % %
    \rlap{\structure{\large Query:}} \\
    %
    &
    $\bit{0} \strcons \valq  \trans  \valq \strcons \bit{0'}$
    &
    \begin{lstlisting}[gobble=6]
      l -> bit0 -> r =
      { case (recv r) of ...
        | v%*\texttt{\color{example}?}*) => send l v%*\texttt{\color{example}?}*);
                l -> bit0%*\texttt{\color{structure}'}*) -> r }
    \end{lstlisting}
    %
    \\[6ex]
    %
    \rlap{\structure{\large Response:}} \\
    %
    &
    $\!\begin{aligned}
      \MoveEqLeft[2]
      \val{(N)} \strcons \bit{\hphantom{'}0'} \strcons \tikzmark{dblp}\dbl(N,N') \\
        &\trans  \bit{0} \strcons \val{(N')}
    \end{aligned}$
    &
    \begin{lstlisting}[gobble=6]
      l -> bit0%*\texttt{\color{structure}'}*) -> r =
      { case (recv l) of
          v => n <- recv l;
               send r v;
               send r (%*\tikzmark{dblf}*)dbl n);
               l -> bit0 -> r }
    \end{lstlisting}
  \end{tabular}

    \begin{tikzpicture}
      \graph [grow right sep=3em] {
        bc / "backward-chaining\\predicate" [align=center]
         <-> [alert, dashed, "?" alert]
        pf / "pure function";
      };

      \draw [overlay] (bc) edge (dblp);
      \draw [overlay] (pf) edge (dblf);
    \end{tikzpicture}

  % \begin{itemize}
  % \item Well-moded backward-chaining predicates vs.\ pure functions
  % \item Specific challenge: mixture of forward- and backward-chaining
  % \end{itemize}

  % \begin{align*}
  %   \begin{tikzpicture}[baseline=(b.base)]
  %     \graph [sill graph=0.5, nodes={process}] {
  %       / [coordinate]
  %        -> ["{$\valq[1]$}" message]
  %       b / "\bit{0}"
  %        ->
  %       / [coordinate];
  %     };
  %   \end{tikzpicture}
  %     &\trans
  %   \begin{tikzpicture}[baseline=(b.base)]
  %     \graph [sill graph=0.5, nodes={process}] {
  %       / [coordinate]
  %        ->
  %       b / "\bit{0'}"
  %        -> ["{$\valq[1]$}" message]
  %       / [coordinate];
  %     };
  %   \end{tikzpicture}
  %   %
  %   \\
  %   %
  %   \begin{tikzpicture}[baseline=(b.base)]
  %     \graph [sill graph=0.5, nodes={process}] {
  %       / [coordinate]
  %        ->
  %       b / "\bit{0'}"
  %        -> ["$\val{}$" {message=right, somewhat near end}]
  %       / [coordinate]
  %        -> ["\smaller[2]$N$" {message=right, text height=\heightof{$\val{}$}}]
  %       / [coordinate];
  %     };
  %   \end{tikzpicture}
  %     &\trans
  %   \begin{tikzpicture}[baseline=(b.base)]
  %     \graph [sill graph=0.5, nodes={process}] {
  %       / [coordinate]
  %        -> ["$\val{}$" message=right]
  %       / [coordinate]
  %        -> ["\smaller[2]$2N$" {message=right, text height=\heightof{$\val{}$}}]
  %       b / "\bit{0}"
  %        ->
  %       / [coordinate];
  %     };
  %   \end{tikzpicture}
  % \end{align*}
\end{frame}


\begin{frame}
  \frametitle{How to add two binary representations?}

  \begin{equation*}
    \begin{tikzpicture}
      \graph [sill graph=0.5, nodes={process}] {
        / [coordinate] -> "\bitequals" [text height=1.5ex] -> / "\bit{0}" -> / "\bit{1}" -> "\bitplus" [text height=1.5ex] -> / "\bit{1}" -> / "\bit{1}" -> "\eps";
      };
    \end{tikzpicture}
  \end{equation*}
  %
  \structure{String rewriting/process chains:} Addition as repeated increment
  \begin{itemize}
  \item Turing-machine--like: $\Omega(N \log N)$ work to compute $M+N$
  % \begin{align*}
  %   \eps \strcons \bit{1} \strcons \bit{1} \strcons \bitplus \strcons \bit{1} \strcons \bit{0} \strcons \bitequals
  %     \trans*
  %   \eps \strcons \bit{1} \strcons \bit{0} \strcons \bit{0} \strcons \bitplus \strcons \bit{0} \strcons \bit{1} \strcons \bitequals
  %     \trans*
  %   \eps \strcons \bit{1} \strcons \bit{0} \strcons \bit{1} \strcons \bitplus \strcons \bit{0} \strcons \bit{0} \strcons \bitequals
  %     \trans*
  %   \eps \strcons \bit{1} \strcons \bit{0} \strcons \bit{1}
  % \end{align*}
  \begingroup\footnotesize
  \begin{spreadlines}{.25ex}
    \begin{align*}
      \MoveEqLeft[0.5]
      \eps \strcons \bit{1} \strcons \bit{1}
      \strcons \bitplus \strcons
      \bit{1} \strcons \bit{0}
      \strcons \bitequals \\
      % 
      &\trans
      % 
      \eps \strcons \bit{1} \strcons \bit{1}
      \strcons \bitplus \strcons
      \bit{1} \strcons \bit{0}
      \strcons \dec \strcons \bitequals['] \\
      % 
      &\trans
      % 
      \eps \strcons \bit{1} \strcons \bit{1}
      \strcons \bitplus \strcons
      \bit{1} \strcons \dec \strcons \bit{1\!'}
      \strcons \bitequals['] \\
      % 
      &\trans
      % 
      \eps \strcons \bit{1} \strcons \bit{1}
      \strcons \bitplus \strcons \bitlinc \strcons
      \bit{0} \strcons \bitrskip \strcons \bit{1\!'}
      \strcons \bitequals['] \\
      % 
      &\trans
      % 
      \eps \strcons \bit{1} \strcons \bit{1}
      \strcons \inc \strcons \bitplus \strcons
      \bit{0} \strcons \bitrskip \strcons \bit{1\!'}
      \strcons \bitequals['] \\
      % 
      &\trans
      % 
      \eps \strcons \bit{1} \strcons \bit{1}
      \strcons \inc \strcons \bitplus \strcons
      \bit{0} \strcons \bit{1} \strcons \bitrskip
      \strcons \bitequals['] \\
      % 
      &\trans
      % 
      \eps \strcons \bit{1} \strcons \bit{1}
      \strcons \inc \strcons \bitplus \strcons
      \bit{0} \strcons \bit{1}
      \strcons \bitequals \\
      % 
      \shortvdotswithin{\trans}
      % 
      &\trans \eps \strcons \bit{1} \strcons \bit{0} \strcons \bit{1}
    \end{align*}
  \end{spreadlines}
  \endgroup

  \item There must be a more efficient way\textellipsis\@
  \end{itemize}
\end{frame}

% e11+10=
% e11+10d=
% e11+1d1=
% e11+<0>1=
% e11s+0>1=
% e11s+01>=
% e11s+01d=
% e11s+0<0>=
% e11s+<00>=
% e11ss+00>=
% e11ss+00=
% e11ss+00d=
% e11ss+0d1=
% e11ss+d11=
% e11ssk11=
% e11ssk1=
% e11ssk=
% e11ss
% e1s0s
% es00s
% e100s
% e101


\begin{frame}
  \frametitle{From process chains to process trees}

  \begin{equation*}
    \begin{tikzpicture}[baseline=(plus.base)]
      \graph [sill graph=0.5, nodes={process}] {
        / [coordinate]
         -> ["$z_0$" channel]
        plus / "\bitplus" [text height=1.5ex]
         ->
        { / "\bit{1}" [> {"$x_0$" channel}] -> ["$x_1$" channel] / "\bit{1}" -> ["$x_2$" channel] / "\eps" ,
          / "\bit{0}" [> {"$y_0$" channel}] -> ["$y_1$" channel] / "\bit{1}" -> ["$y_2$" channel] / "\eps" };
      };
    \end{tikzpicture}
    \enskip\trans\enskip
    \begin{tikzpicture}[baseline=(plus.base)]
      \graph [sill graph=0.5, nodes={process}] {
        / [coordinate]
         -> ["$z_0$" channel]
        / "\bit{1}"
         -> ["$z_1$" channel]
        plus / "\bitplus" [text height=1.5ex]
         ->
        { / "\bit{1}" [> {"$x_1$" channel}] -> ["$x_2$" channel] / "\eps" ,
          / "\bit{1}" [> {"$y_1$" channel}] -> ["$y_2$" channel] / "\eps" };
      };
    \end{tikzpicture}
  \end{equation*}

  \structure{Process trees:}
  \begin{itemize}
  \item \emph{Explicit} use of channels generalizes from chains to trees (SILL)
  \item Processes may now use services along multiple channels
    \begin{equation*}
      \begin{tikzpicture}
        \graph [sill graph=0.5] {
          / [coordinate]
           -> ["$c$" channel]
          p / "P" [process]
           ->
          { c0 / [coordinate, > {"$c_0$" channel}] ,
            cn / [coordinate] };
          (p) -> ["$c_n$" channel] (cn);

          (c0) <-> [dotted] (cn);
        };
      \end{tikzpicture}
    \end{equation*}
  \item Bit-wise addition: $O(\log M + \log N)$ work to compute $M+N$
  \end{itemize}
\end{frame}


\begin{frame}[fragile]
  \frametitle{To what form of rewriting do process trees correspond?}

  \begin{center}
    \begin{tikzpicture}
      \matrix [
        matrix of nodes, column sep=2.5em,
        column 1/.append style={anchor=base east},
        column 2/.append style={anchor=base west}
      ] {
        |(sr)| string rewriting & |(pc)| process chains \\[1ex]
        |(msr)| \llap{\structure{Claim:}\enskip}\emph{multiset} rewriting & |(pt)| process \emph{trees} \\
        |(dst)| destinations & |(ch)| channels \\
      };

      \graph [edges={structure}] {
        (sr) <-> (pc);
        (msr) <-> (pt);
        (dst) <-> (ch);
      };
    \end{tikzpicture}
  \end{center}

  \structure{Example:} Bit-wise addition
  \begin{align*}
    \begin{tikzpicture}[baseline=(plus.base)]
      \graph [sill graph=0.5, nodes={process}] {
        / [coordinate]
         -> ["$z_i$" channel]
        plus / "\bitplus" [text height=1.5ex]
         ->
        { / "\bit{1}" [> {"$x_i$" channel}] -> ["$x_{i+1}$" channel] / [coordinate] ,
          / "\bit{0}" [> {"$y_i$" channel}] -> ["$y_{i+1}$" channel] / [coordinate] };
      };
    \end{tikzpicture}
    \enskip
    &\trans
    \enskip
    \begin{tikzpicture}[baseline=(plus.base)]
      \graph [sill graph=0.5, nodes={process}] {
        / [coordinate]
         -> ["$z_i$" channel]
        / "\bit{1}"
         -> ["$z_{i+1}$" channel]
        plus / "\bitplus" [text height=1.5ex]
         ->
        { / [coordinate, > {"$x_{i+1}$" channel}] ,
          / [coordinate, > {"$y_{i+1}$" channel}] };
      };
    \end{tikzpicture}
    &&
    \text{\footnotesize($z_{i+1}$ fresh)}
    %
    \\[8\jot]
    %
    \!\begin{lgathered}
      \bit{1}(x_{i+1}, x_i) \msetcons {} \\
      \bit{0}(y_{i+1}, y_i) \msetcons {}
    \end{lgathered}
    \bitplus(x_i, y_i, z_i)
    \enskip
    &\trans\enskip
    \!\begin{lgathered}
      \bitplus(x_{i+1}, y_{i+1}, z_{i+1}) \msetcons {} \\
      \bit{1}(z_{i+1}, z_i)
    \end{lgathered}
    &&
    \text{\footnotesize($z_{i+1}$ fresh)}
  \end{align*}
\end{frame}


\begin{frame}
  \frametitle{Multiset rewriting as process reduction}

  \begin{center}
    Rewriting--chreography--process picture
  \end{center}

  \structure{Specific challenges:}
  \begin{itemize}
  \item Linearity of destinations (unique indexes [Simmons '12])
  \item Processes can pass channels.  Analogue for destinations?
  \item Is there a process analogue to persistent propositions without destinations?
  \end{itemize}
\end{frame}


\begin{frame}
  \frametitle{Plan}

  \begin{itemize}
  \item First-order ordered logic programming
    \begin{itemize}
    \item Well-moded backward chaining vs.\ pure functional computation (X months)
      \begin{itemize}
      \item Rely more on related work here?
      \end{itemize}
    \item Nesting of forward-chaining computations inside backward-chaining computations (X months)
    \end{itemize}
  \item Concurrent linear logic programming (first-order)
    \begin{itemize}
    \item 
    \end{itemize}
  \end{itemize}
\end{frame}
\end{document}



\item \structure{\emph{Conjecture:}} corresponds to a fragment of \emph{multiset} rewriting (linear logic programming)
    \begin{equation*}
      \text{Process}\enskip
      %
      \begin{tikzpicture}[baseline=(plus.base)]
        \graph [sill graph=0.5, nodes={process}] {
          / [coordinate]
           -> ["$z$" channel]
          plus / "\bitplus" [text height=1.5ex]
           ->
          { / [coordinate, > {"$x$" channel}] ,
            / [coordinate, > {"$y$" channel}] };
        };
      \end{tikzpicture}
      %
      \enskip\text{corresponds to element}\enskip
      %
      \bitplus(x,y,z)
    \end{equation*}

  \begin{align*}
    \MoveEqLeft[0.5]
    \begin{aligned}[t]
      &\eps(x_2) \msetcons \bit{1}(x_2,x_1) \msetcons \bit{1}(x_1,x_0) \msetcons {} \\
      &\eps(y_2) \msetcons \bit{1}(y_2,y_1) \msetcons \bit{0}(y_1,y_0) \msetcons \bitplus(x_0,y_0,z_0)
    \end{aligned} \\
    &\trans \\
    \MoveEqLeft[0.5]
     \begin{aligned}[t]
       &\eps(x_2) \msetcons \bit{1}(x_2,x_1) \msetcons {} \\
       &\eps(y_2) \msetcons \bit{1}(y_2,y_1) \msetcons \bitplus(x_1,y_1,z_1) \msetcons \bit{1}(z_1,z_0)
     \end{aligned}
  \end{align*}
  \end{itemize}
\end{frame}


% case x of
%   1 => case y of 
%          0 => z' <- + <- x,y;
%               z <- bit1 <- z'


1 0 1 1 1 0 s s
1 0 1 1 1 1 s
1 1 0 0 0 0

\end{document}