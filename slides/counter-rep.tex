% arara: pdflatex
% arara: pdflatex
% arara: pdflatex
\documentclass{beamer}%{standalone}
\usepackage{beamerext-rev}

\usepackage{expl3}
\usepackage{xparse}

\usepackage{array}

\usepackage{mathtools}
\usepackage{tikz}
\usetikzlibrary{scopes,graphs,cd,shapes.misc,shapes.arrows,shapes.symbols,fit,positioning}
% \usegdlibrary{trees}

\NewDocumentCommand{\tikzmark}{m}{%
  \tikz [remember picture, overlay, baseline]
    { \coordinate (#1); }%
}


% \usepackage{../ch/binary-counter}
\NewDocumentCommand{\eps}{}{\mathtt{e}}
\NewDocumentCommand{\bit}{O{} m}{\mathtt{#2}#1}
\NewDocumentCommand{\inc}{}{\mathtt{s}}
\NewDocumentCommand{\prt}{}{\mathtt{p}}
\NewDocumentCommand{\even}{}{\mathtt{ev}}
\NewDocumentCommand{\odd}{}{\mathtt{o}}

\NewDocumentCommand{\strcons}{}{\mskip\thinmuskip}

\NewDocumentCommand{\cntr}{m}{\ulcorner#1\urcorner}
\NewDocumentCommand{\cval}{m}{\llcorner#1\lrcorner}

\NewDocumentCommand{\trans}{s}{\longrightarrow\IfBooleanT{#1}{^*}}

\NewDocumentCommand{\vocab}{m}{\emph{#1}}

% \ExplSyntaxOn
% \NewDocumentCommand{\olpframe}{o m o}{
  
% }
% \ExplSyntaxOff

\ExplSyntaxOn
\NewDocumentCommand{\mathul}{d<> O{} m}{
  \IfValueTF {#1}
    { \mathul:nnn {<#1>} {#2} {#3} }
    { \mathul:nnn    { } {#2} {#3} }
}

\cs_new:Npn \mathul:nnn #1#2#3 {
  \tikzmark { mathul~south~west }
  #3
  \tikzmark { mathul~south~east }
  \tikz [remember~picture, overlay] {
    \draw #1 [very~thick, color=structure, #2]
      ([yshift=-0.25ex]mathul~south~west) -- ([yshift=-0.25ex]mathul~south~east);
  }
}
\ExplSyntaxOff


\makeatletter
\tikzset{%
  graphs/sill graph/.style = {
    grow left sep=#1 * 2.5em, math nodes,
    edges = {-, shorten <=-0.02em, shorten >=-0.02em},
  },
  graphs/sill graph/.default = 1,
  graphs/sill linear graph/.style = {sill graph},
  %
  process/.style = {
    rounded rectangle, rounded rectangle left arc=none,
    minimum size=1.75em,
    fill=structure!25, draw,
    font=\LARGE, % text height=1.5ex
  },
  %
  channel/.style = {
    below right=0.2em and 0.05em, at end, sloped,
    inner sep=0em,
    overlay,
    font=\small,
%    draw
  },
  %
  message@/.style n args = {3}{
    signal, signal to=#1, signal pointer angle=105,
    above=0.3em, midway, sloped,
    inner xsep=#2, inner ysep=0.2em,
    overlay,
    font=\LARGE\hspace{#3},
    fill=example text.fg!20, draw
  },
  message/.is choice,
  message/left/.style = { message@={left}{0.1em}{-0.05em} },
  message/right/.style = { message@={right}{0.05em}{0.05em} },
  message/east/.style = { message=right },
  message/west/.style = { message=left },
  message/.default = left,
  %
  point/.style = {
    circle, fill=red,
    inner sep=0pt, outer sep=0pt,
    minimum size=2pt
  },
  %
  end point/.style = {
    circle, fill,
    inner sep=0, outer sep=0,
    minimum size=4\pgflinewidth
  }
}

\NewDocumentCommand{\drawredex}{m}{%
  \node [draw=red, dashed, overlay, fit=#1] {};
}
\makeatother

\tikzset{
  every picture/.append style={thick}
}

\NewDocumentCommand{\tikzmarktext}{m m}{%
  \tikz [remember picture, baseline] {
    \node (#1) [inner sep=0em, outer sep=0.3333em, anchor=base]
      {#2};
  }%
}

\makeatletter
\NewDocumentCommand{\cover}{}{%
  \alt{\beamer@makecovered}{\beamer@fakeinvisible}%
}
\makeatother

\setbeamertemplate{navigation symbols}{}
\setbeamercovered{transparent}

\begin{document}

\begin{frame}{Binary representation of natural numbers as strings}
  % A \vocab{value} is a string of $\bit{0}$s and $\bit{1}$s beginning with an $\eps$, with no leading $\bit{0}$s.
  % Represent a number by a string of $\bit{0}$s and $\bit{1}$s beginning with an $\eps$, with no leading $\bit{0}$s.
  \structure{Representation:}
  \begin{itemize}
  \item<+-> String of $\bit{0}$s and $\bit{1}$s beginning with an $\eps$, with \tikzmarktext{leading0s}{\alert<.(3)>{no leading $\bit{0}$s.}}
    % Make rhs visible as <.->
    \begin{align*}
      \eps \strcons \bit{1}
        \uncover<+->{&= \cntr{1} \\}
      \eps \strcons \bit{1} \strcons \bit{0}
        \uncover<.->{&= \cntr{2} \\}
      \mathllap{\eps \strcons \bit{1} \strcons \bit{0} \strcons \bit{1} \strcons \bit{0} \strcons \bit{1} \strcons \bit{0}}
        \uncover<.->{&= \cntr{42}}
    \end{align*}
  % 
  \item<+-> Adequately represent natural number $n$ by bit string $\cntr{n}$:
  % \item<+-> More generally, adequately represent number $n$ by string $\cntr{n}$:
    \begin{align*}
      \hphantom{\eps \strcons \bit{1} \strcons \bit{0}} &\hphantom{= \cntr{42}} \\[-\baselineskip]
      \cntr{0} &= \eps \\
      \cntr{2n} &= \cntr{n} \strcons \bit{0}
          \mathrlap{\text{ , \tikzmarktext{leading0s side}{\alert<.(1)>{if $2n > 0$}}%
            \tikz [remember picture, overlay] {
              \draw <.(1)> [alert]
                (leading0s.south)
                  edge [->, out=-90, in=20, in looseness=2] (leading0s side.east);
          }}} \\
      %
      \mathllap{\cntr{2n+1}} &= \cntr{n} \strcons \bit{1}
    \end{align*}
  \end{itemize}
\end{frame}


\begin{frame}{Incrementing binary numbers as string rewriting}
  \fbox{%
  \begin{columns}[T]
  \column{0.42\linewidth}
    \structure{Successor instructions:}
    \begin{itemize}
    \item<+->
      $\cntr{n} \alert<+>{\strcons \inc}
         \onslide<+->{\Alt<-.>[c]{{} \mathrel{\text{``$=$''}} {}}{{} \alert<.(1)>{\trans*} {}}
       \cntr{n+1}}$
    \end{itemize}
  %
  \column{0.5\linewidth}
    \phantom{Successors:}
    \begin{itemize}
    \item<+-> Grade school arithmetic as string rewriting
      \begin{description}
      \item[Ends in $\bit{1}$?] % Flip the bit to $\bit{0}$ and carry the $\inc$.
        \onslide<+->{$\bit{1} \strcons \inc  \trans  \inc \strcons \bit{0}$}
      \item[Ends in $\bit{0}$?] % Flip the bit to $\bit{1}$.
        \onslide<+->{$\bit{0} \strcons \inc  \trans  \bit{1} \phantom{\strcons \inc}$}
      \item[Is $\eps$?] % Add a leading $\bit{1}$.
        \onslide<+->{$\eps \strcons \inc  \trans  \eps \strcons \bit{1}$}
      \end{description}
    \end{itemize}
  \end{columns}}

  \onslide<+->
  \structure{Example.}
  \begin{equation*}
    \Temporal<+->[r]{\cntr{1}}{\eps \strcons \bit{1}}{} \strcons \inc
      \trans*
    \Temporal<+->{\cntr{2}}{\eps \strcons \bit{1} \strcons \bit{0}}{}
  \end{equation*}
\end{frame}



% \begin{frame}{Incrementing binary numbers by string rewriting}
%   \begin{description}[]%\hspace{0.75em}]
%   \item[Increments:]\mbox{}\\
%     \begin{itemize}
%     \item<+-> Grade school arithmetic as \alert<.(3)>{string rewriting}
%       \begin{align*}
%         \cntr{n} \alert<+>{\strcons \inc}
%           \onslide<+->{&\Alt<-.>[c]{{} \mathrel{\text{``$=$''}} {}}{{} \alert<+>{\trans*} {}}
%         \cntr{n+1} \\}
%         %
%         \onslide<+->{%
%         \eps \strcons \bit{1} \strcons \inc
%           &\trans*
%         \eps \strcons \bit{1} \strcons \bit{0}}
%       \end{align*}
%     \end{itemize}
%   \end{description}

%   \begin{equation*}
%     \eps \strcons \bit{1} \strcons \inc
%       \trans*
%     \eps \strcons \bit{1} \strcons \bit{0}
%   \end{equation*}

%   \begin{center}
%     \begin{minipage}{0.5\linewidth}    
%       \begin{description}
%         % \setlength\labelsep{1cm}
%       \item[\onslide<+,+(1)->{Ends in $\bit{1}$?}] % Flip the bit to $\bit{0}$ and carry the $\inc$.
%         \onslide<.(1)->{$\bit{1} \strcons \inc  \trans  \inc \strcons \bit{0}$}
%       \item[\onslide<+(-1),+(1)->{Ends in $\bit{0}$?}] % Flip the bit to $\bit{1}$.
%         \onslide<.(1)->{$\bit{0} \strcons \inc  \trans  \bit{1} \phantom{\strcons \inc}$}
%       \item[\onslide<+(-2),+(1)>{Is $\eps$?}] % Add a leading $\bit{1}$.
%         \onslide<.(1)->{$\eps \strcons \inc  \trans  \eps \strcons \bit{1}$}
%       \end{description}
%     \end{minipage}
%   \end{center}

%   % \begin{center}
%   %   \begin{minipage}{0.5\linewidth}    
%   %     \begin{description}
%   %       % \setlength\labelsep{1cm}
%   %     \item[Ends in $\bit{1}$?] % Flip the bit to $\bit{0}$ and carry the $\inc$.
%   %       $\bit{1} \strcons \inc  \trans  \inc \strcons \bit{0}$
%   %     \item[Ends in $\bit{0}$?] % Flip the bit to $\bit{1}$.
%   %       $\bit{0} \strcons \inc  \trans  \bit{1} \phantom{\strcons \inc}$
%   %     \item[Is $\eps$?] % Add a leading $\bit{1}$.
%   %       $\eps \strcons \inc  \trans  \eps \strcons \bit{1}$
%   %     \end{description}
%   %   \end{minipage}
%   % \end{center}

%   \onslide<1->
%   \structure{Example}
%   \begin{equation*}
%     \cover<2-3>{\eps \strcons} \bit{1} \strcons \inc
% %    \cover<+(1)-+(3)>{\eps \strcons} \bit{1} \strcons \inc
%       % \visible<+(1)->{
%       \trans
%     % \uncover<.(3)->{\eps \strcons}} \visible<+(1)->{\inc \strcons \bit{0}}
%     \eps \strcons \inc \strcons \bit{0}
%       \trans
%     \eps \strcons \bit{1} \strcons \bit{0}
%   \end{equation*}
% %  \onslide<+->
%  % \onslide<+->

%   % \begin{equation*}
%   %   \uncover<+-+(2)>{\cntr{0} \strcons \inc}
%   %     \visible<.(1)->{\uncover<-.(2)>{=} \eps \strcons \inc}
%   %     \trans
%   %     \visible<.(2)->{\eps \strcons \bit{1} \uncover<-.(2)>{=}}
%   %     \uncover<.-.(2)>{\cntr{1}}
%   % \end{equation*}

%   % \pause

%   % \begin{gather*}
%   %   \cntr{2n} \strcons \inc
%   %     = \cntr{n} \strcons \bit{0} \strcons \inc
%   %   \trans
%   %     \cntr{n} \strcons \bit{1}
%   %     = \cntr{2n+1}
%   %   %
%   %   \\
%   %   %
%   %   \cntr{2n+1} \strcons \inc
%   %     = \cntr{n} \strcons \bit{1} \strcons \inc
%   %   \trans
%   %     \cntr{n} \strcons \inc \strcons \bit{0}
%   %   \trans*
%   %     \cntr{n+1} \strcons \bit{0}
%   %     = \cntr{2n+2}
%   % \end{gather*}
% \end{frame}

% \begin{frame}{Incrementing binary numbers by string rewriting}
%   % Allow increments, $\inc$, that are resolved by grade school addition (string rewriting):
%   Increments, $\inc$, are resolved by grade school addition (string rewriting):
%   \begin{equation*}
%     \cntr{n} \strcons \inc  \trans*  \cntr{n+1}
%   \end{equation*}

%   \begin{center}
%     \setlength\tabcolsep{0.5\tabcolsep}
%     \begin{tabular}{@{}>{\color{structure}}rl>{\color{structure}}l@{}}
%       \llap{The string either }ends in $\bit{1}$:& % Flip the bit to $\bit{0}$ and carry the $\inc$.
%         $\bit{1} \strcons \inc  \trans  \inc \strcons \bit{0}$&; \\
%       ends in $\bit{0}$:& % Flip the bit to $\bit{1}$.
%         $\bit{0} \strcons \inc  \trans  \bit{1} \phantom{\strcons \inc}$&; or \\
%       is $\eps$:& % Add a leading $\bit{1}$.
%         $\eps \strcons \inc  \trans  \eps \strcons \bit{1}$&.
%     \end{tabular}
%   \end{center}

% \onslide<1->
%   \structure{Example}
%   \begin{equation*}
%     \cover<2-3>{\eps \strcons} \bit{1} \strcons \inc
% %    \cover<+(1)-+(3)>{\eps \strcons} \bit{1} \strcons \inc
%       % \visible<+(1)->{
%       \trans
%     % \uncover<.(3)->{\eps \strcons}} \visible<+(1)->{\inc \strcons \bit{0}}
%     \eps \strcons \inc \strcons \bit{0}
%       \trans
%     \eps \strcons \bit{1} \strcons \bit{0}
%   \end{equation*}
% %  \onslide<+->
%  % \onslide<+->

%   % \begin{equation*}
%   %   \uncover<+-+(2)>{\cntr{0} \strcons \inc}
%   %     \visible<.(1)->{\uncover<-.(2)>{=} \eps \strcons \inc}
%   %     \trans
%   %     \visible<.(2)->{\eps \strcons \bit{1} \uncover<-.(2)>{=}}
%   %     \uncover<.-.(2)>{\cntr{1}}
%   % \end{equation*}

%   % \pause

%   % \begin{gather*}
%   %   \cntr{2n} \strcons \inc
%   %     = \cntr{n} \strcons \bit{0} \strcons \inc
%   %   \trans
%   %     \cntr{n} \strcons \bit{1}
%   %     = \cntr{2n+1}
%   %   %
%   %   \\
%   %   %
%   %   \cntr{2n+1} \strcons \inc
%   %     = \cntr{n} \strcons \bit{1} \strcons \inc
%   %   \trans
%   %     \cntr{n} \strcons \inc \strcons \bit{0}
%   %   \trans*
%   %     \cntr{n+1} \strcons \bit{0}
%   %     = \cntr{2n+2}
%   % \end{gather*}
% \end{frame}


\begin{frame}[fragile]{\emph{Concurrent} string rewriting}
  \structure{Concurrency:}
  \begin{itemize}
  \item<+-> Several \alert<.(4)>{morally concurrent} successor instructions at once:
    \begin{equation*}
      \begin{tikzcd}[column sep=scriptsize, every cell/.append style={text height=1ex}, every arrow/.append style={line width=2 * rule_thickness}, /tikz/every node/.append style={}]
        & &
        |[alt=<.(5)->{text opacity=1}{text opacity=0}]|
        \eps \strcons \bit{1} \strcons \mathul[example text.fg]{\bit{0} \strcons \inc}
          \only<.(6)->{\ar[dr, example text.fg]}
        \\
        \eps \strcons \mathul<.(1)->[black]{\bit{1} \strcons \inc} \strcons \inc
          \only<.(2)->{\ar[r, black]}
        &
        |[alt=<.(2)->{text opacity=1}{text opacity=0}]|
        \mathul<.(3)->[alt=<.(4)>{alert}{structure}]{\eps \strcons \inc} \strcons \mathul<.(3)->[alt=<.(4)>{alert}{example text.fg}]{\bit{0} \strcons \inc}
          \only<.(4)>{\ar[rr, dashed, alert]}
          \only<.(5)->{\ar[ur, structure]}
          \only<.(7)->{\ar[dr, example text.fg]}
        & &
        |[alt=<.(4)->{text opacity=1}{text opacity=0}]|
        \eps \strcons \bit{1} \strcons \bit{1}
        \\
        & &
        |[alt=<.(7)->{text opacity=1}{text opacity=0}]|
        \mathul{\eps \strcons \inc} \strcons \bit{1}
          \only<.(8)->{\ar[ur, structure]}
      \end{tikzcd}
    \end{equation*}
    \addtocounter{beamerpauses}{8}
  %
  \item<+-> We require that interleavings of independent rewritings\\are indistinguishable, so that they appear to be concurrent.
    \begin{itemize}
    \item This diagram commutes, e.g.
    \end{itemize}
  \item<+-> Concurrency, but where are the processes and messages?
  \end{itemize}
\end{frame}


\begin{frame}{}
  \begin{tikzpicture}[baseline]
    \graph [sill graph=0.5] {
      / [coordinate]
        -> ["$c_0$" channel]
      "\inc" [process]
        -> ["$c_1$" channel]
      / [coordinate];
    };
  \end{tikzpicture}
  \begin{tikzpicture}[baseline]
    \graph [sill graph=1.6] {
      / [coordinate]
        -> ["$c_1{=}c_0$" channel, "$\inc$" message]
      / [coordinate];
    };
  \end{tikzpicture}

  \begin{tikzpicture}
    \graph [sill graph=0.5] {
      / [coordinate]
        ->
      / [coordinate]
        -> ["$c_0$" channel, "$\inc$" {message, at start}]
      "\bit{1}" [process]
        -> ["$c_1$" channel]
      / [coordinate];
    };
  \end{tikzpicture}
  \begin{tikzpicture}
    \graph [sill graph=0.5] {
      / [coordinate]
        -> ["$c_0$" channel]
      "\bit{0}" [process]
        ->
      / [coordinate]
        -> ["$c_{0'}$" channel]
      "\inc" [process]
        -> ["$c_1$" channel]
      / [coordinate];
    };
  \end{tikzpicture}

  \begin{tikzpicture}
    \graph [sill graph] {
      / [coordinate]
        -> ["$c_0$" channel, "$\inc$" message]
      "\eps" [process];
    };
  \end{tikzpicture}
  \begin{tikzpicture}
    \graph [sill graph=0.5] {
      / [coordinate]
        -> ["$c_0$" channel]
      "\bit{1}" [process]
        ->
      / [coordinate]
        -> ["$c_1$" channel]
      "\eps" [process];
    };
  \end{tikzpicture}

  \begin{tikzpicture}
    \graph [sill graph=0.5] {
      / [coordinate]
        ->
      / [coordinate]
        -> ["$c_0$" channel, "$\inc$" {message, at start}]
      "\bit{0}" [process]
        -> ["$c_1$" channel]
      / [coordinate];
    };
  \end{tikzpicture}
  \begin{tikzpicture}
    \graph [sill graph=0.5] {
      / [coordinate]
        -> ["$c_0$" channel]
      "\bit{1}" [process]
        -> ["$c_1$" channel]
      / [coordinate];
    };
  \end{tikzpicture}
\end{frame}


\begin{frame}{}
  Letters are processes (formulas-as-processes)
  % \begin{equation*}
  %   \begin{tikzpicture}
  %     \graph [sill graph,
  %       tree layout, grow=left, level distance=0, level sep=2.5em] {
  %       / [point]
  %         ->["$c$" channel]
  %       "\inc" [process]
  %         -> {
  %           / [point, > {"$d_0$" channel}] -> / [point] ,
  %           / [point, > {"$d_1$" channel}] -> / [point]
  %         };
  %     };
  %   \end{tikzpicture}
  % \end{equation*}

  \begin{equation*}
    \begin{tikzpicture}[channel/.append style={below, at start}]
      \graph [sill graph, grow left sep=1.25em] {
        / [coordinate]
          ->
        / [point]
          -> ["$c_0$" channel]
        "\inc" [process]
          ->
        / [point]
          -> ["$c_1$" channel]
        "\bit{1}" [process]
          ->
        / [point]
          -> ["$c_2$" channel]
        "\eps" [process];
      };
    \end{tikzpicture}
  \end{equation*}

  \begin{equation*}
    \begin{tikzpicture}
      \graph [sill graph] {
        / [coordinate]
          -> ["$c_0$" {channel, name=c0}, "$\inc$" {message, name=inc}]
        b1/"\bit{1}" [process]
          -> ["$c_2$" channel]
        "\eps" [process];
      };
      \node [overlay, fit=(b1) (c0) (inc), draw=red, dashed] {};
    \end{tikzpicture}
  \end{equation*}

  \begin{equation*}
    \begin{tikzpicture}
      \graph [sill graph] {
        / [coordinate]
          -> ["$c_0$" channel]
        "\bit{0}" [process]
          -> ["$c_{1'}$" {channel, name=c1'}]
        inc/"\inc" [process]
          -> ["$c_2$" channel]
        "\eps" [process];
      };
      \coordinate [left=0.9em of inc] (x);
      \node [overlay, fit=(x) (c1') (inc), draw=red, dashed] {};
    \end{tikzpicture}
  \end{equation*}
  
  \begin{equation*}
    \begin{tikzpicture}
      \graph [sill graph] {
        / [coordinate]
          ->["$c_0$" channel]
        "\bit{0}" [process]
          -> ["$c_2$" channel, "$\inc$" message]
        "\eps" [process];
      };
    \end{tikzpicture}
  \end{equation*}

  \begin{equation*}
    \begin{tikzpicture}
      \graph [sill graph] {
        / [coordinate]
          ->["$c_0$" channel]
        "\bit{0}" [process]
          -> ["$c_2$" channel]
        "\bit{1}" [process]
          -> ["$c_{2'}$" channel]
        "\eps" [process];
      };
    \end{tikzpicture}
  \end{equation*}

  \begin{equation*}
    \begin{tikzpicture}
      \graph [sill graph] {
        / [coordinate]
          -> ["$\inc$" message]
        / [coordinate]
          -> ["$c_0$" channel, "$\inc$" message]
        "\bit{1}" [process]
          -> ["$c_2$" channel]
        "\eps" [process];
      };
    \end{tikzpicture}
  \qquad
    \begin{tikzpicture}
      \graph [sill graph] {
        / [coordinate]
          -> ["$c_0$" channel, "$\inc$" {message, name=i0}]
        b1/"\bit{1}" [process]
          -> ["$c_2$" channel, "$\inc$" {message, name=i1}]
        e/"\eps" [process];
      };

      \drawredex{(b1) (i0)}
      \drawredex{(e) (i1)}
    \end{tikzpicture}
  \end{equation*}
\end{frame}


\begin{frame}{}
  \begin{equation*}
    \begin{tikzpicture}
      \graph [sill graph] {
        / [coordinate]
          -> ["$c_0$" channel]
        "\inc" [process]
          -> ["$\bit{1}$" {message=right, near end}]
        / [coordinate]
          -> ["$c_1$" channel, "$\eps$" message=right]
        / [end point];
      };
    \end{tikzpicture}
  \end{equation*}

  \begin{equation*}
    \begin{tikzpicture}
      \graph [sill graph] {
        / [coordinate]
          -> ["$\bit{0}$" {message=right, near end}]
        / [coordinate]
          -> ["$\bit{1}$" {message=right, near end}]
        / [coordinate]
          -> ["$c_0$" channel, "$\eps$" message=right]
        / [end point];
      };
    \end{tikzpicture}
  \end{equation*}
\end{frame}


\begin{frame}{Picture}
  \begin{tikzpicture}
    \graph [
      grow right sep=5em,
      nodes = {text=structure},
      edges = {thick, bend left}
    ] {
        olp/Specification -> chor/Choreography -> proc/Processes;
        { [edges={bend right}] olp <-[dashed] chor <- proc };
    };
  \end{tikzpicture}
\end{frame}


\begin{frame}{Parity}
  \begin{equation*}
    \begin{lgathered}
      \bit{0} \strcons \prt  \trans  \prt \strcons \bit[']{0} \\
      \bit{1} \strcons \prt  \trans  \prt \strcons \bit[']{1} \\
      \eps \strcons \prt  \trans  \eps \strcons \even \\
      \even \strcons \bit[']{0}  \trans  \bit{0} \strcons \even \\
      \even \strcons \bit[']{1}  \trans  \bit{1} \strcons \odd \\
      \odd \strcons \bit[']{0}  \trans  \bit{0} \strcons \odd \\
      \odd \strcons \bit[']{1}  \trans  \bit{1} \strcons \even
    \end{lgathered}
  \end{equation*}
\end{frame}

\end{document}