% arara: pdflatex
\documentclass{beamer}
\usepackage{my-proposal-talk}

\usetikzlibrary{positioning}


\renewcommand*{\inc}{\color{structure}\mathtt{s}\color{black}}
\renewcommand*{\eps}{\color{structure}\mathtt{e}\color{black}}
\renewcommand*{\bit}[1]{\color{structure}\mathtt{#1}\color{black}}

\NewDocumentCommand \val { m }{ \llcorner #1 \lrcorner }

\NewDocumentCommand \circtask { O{} m }{%
  \tikz { \fill [#1] (#2/2,#2/2) circle [radius={#2/2}]; }%
}
\NewDocumentCommand \sqtask { O{} m }{%
  \tikz { \fill [#1] (0,0) rectangle ++(#2,#2); }%
}

\tikzset{
  derivation length/.initial = 1em,
  %
  derivation/.pic = {
    \fill (0,0)
      coordinate ( root)
      -- ++( 60 : \pgfkeysvalueof{/tikz/derivation length})
      coordinate ( right leaf)
      -- ++(180 : \pgfkeysvalueof{/tikz/derivation length})
      coordinate ( left leaf)
      -- ++(-60 : \pgfkeysvalueof{/tikz/derivation length})
      -- cycle;
  },
  %
  derivation layer width/.initial = 1em,
  derivation layer height/.initial = 1em,
  %
  derivation layer/.pic = {
    \fill (0,0)
      coordinate ( bottom left)
      -- ++(  0 : \pgfkeysvalueof{/tikz/derivation layer width})
      coordinate ( bottom right)
      -- ++( 60 : cosec 60 * \pgfkeysvalueof{/tikz/derivation layer height})
      coordinate ( top right)
      -- ++(180 : 2 * cot 60 * \pgfkeysvalueof{/tikz/derivation layer height}
                   + \pgfkeysvalueof{/tikz/derivation layer width})
      coordinate ( top left)
      -- ++(-60 : cosec 60 * \pgfkeysvalueof{/tikz/derivation layer height})
      -- cycle;
  },
}


\NewDocumentCommand \pfreduce { O{} m m }{%
  \begin{tikzpicture}[#1]
    \fill [gray] (0 : 0) -- ++(60 : 2) -- ++(180 : 2) -- ++(-60 : 2) -- cycle;
    \clip (60 : 2 + \gaphyp)
      -- ++(60 : \layerhyp)
      -- ++(180 : 2 + \gaphyp + 2 * cos 60 * \layerhyp)
      -- ++(-60 : \layerhyp)
      -- ++(0 : 2 + \gaphyp)
      -- cycle;

    \fill [fill={#2}]
      (120 : 0.8 + \gaphyp + \layerhyp)
       -- ++( 60 : 1.2)
       -- ++(180 : 1.2)
       -- ++(-60 : 1.2)
       -- cycle;

    \fill [fill={#3}]
      (60 : 0.8 + \gaphyp + \layerhyp)
       -- ++( 60 : 1.2)
       -- ++(180 : 1.2)
       -- ++(-60 : 1.2)
       -- cycle;
  \end{tikzpicture}%
}


\NewDocumentCommand \pfconstruct { O{} m m }{%
  \begin{tikzpicture}[#1]
    \fill [fill={#2}]
      (0 : 0)
       -- ++( 120 : \layerhyp)
       -- ++(   0 : 1.2)
       -- ++(-120 : \layerhyp)
       -- ++( 180 : 1.2 - 2 * cos 60 * \layerhyp)
       -- cycle;

    \fill [fill={#3}]
      (120 : \layerhyp + \gaphyp)
       -- ++( 120 : \layerhyp)
       -- ++(   0 : 1.2 + 2 * cos 60 * \layerhyp + 2 * cos 60 * \gaphyp)
       -- ++(-120 : \layerhyp)
       -- ++( 180 : 1.2 + 2 * cos 60 * \gaphyp)
       -- cycle;
  \end{tikzpicture}%
}


\NewDocumentCommand \showcolumnwidth { }{%
  \tikz { \draw [|-|] (0,0) -- (\linewidth,0); }%
}


% Presentation metadata
\title{Session-typed concurrent logic programming}
\author[DeYoung]{Henry DeYoung}
\institute[Carnegie Mellon]{%
  Computer Science Department\\
  Carnegie Mellon University%
}
\date{\today}
\titlegraphic{\textcolor{gray}{\rule{12em}{9em}}}


\begin{document}


\begin{frame}[plain]
  \titlepage
\end{frame}


\newcommand*\gaphyp{0.05}
\newcommand*\layerhyp{0.5}


\begin{frame}[fragile]
  \frametitle{Computation and deduction}

  \begin{columns}[c]
  \column{0.5\textwidth}
    \begin{center}
      \begin{tikzcd}[arrow style=tikz, /tikz/>={To[scale=0.8]}, arrows={very thick, start anchor=real east, end anchor=real west}, /tikz/nodes={inner xsep=0}]
        \pfreduce[scale=0.75]{structure}{example text.fg}
          \rar[structure] &
        %
        \pfreduce[scale=0.75]{none}{example text.fg}
      \end{tikzcd}
    \end{center}

  \column{0.55\textwidth}
    \begin{center}
      \begin{tikzcd}[arrow style=tikz, /tikz/>={To[scale=0.8]}, arrows={very thick, start anchor=real east, end anchor=real west}, /tikz/nodes={inner xsep=0}]
        \pfconstruct[scale=0.75]{none}{none}  \rar[structure] &  \pfconstruct[scale=0.75]{example text.fg}{structure}
      \end{tikzcd}
    \end{center}
  \end{columns}

  \vspace*{2ex}

  \begin{columns}[T]
  \column{0.5\textwidth}
    \structure{Computation as proof reduction}
    \begin{itemize}
    \item Functional programming
    \item Types --- preservation and progress properties
    \end{itemize}

  \column{0.55\textwidth}
    \structure{Computation as proof construction}
    \begin{itemize}
    \item Committed-choice bottom-up logic programming
    \item Impoverished types --- can get stuck
    \end{itemize}
  \end{columns}

  \vspace*{1ex}

  \begin{center}
    Both have a seemingly sequential notion of transition,
    \begin{tikzcd}[arrow style=tikz, /tikz/>={To[scale=0.8]}, arrows={very thick}, /tikz/nodes={inner xsep=0}]
      \rar & {}
    \end{tikzcd}.
  \end{center}

  \vspace*{-4ex}
\end{frame}


\begin{frame}[fragile]
  \frametitle{From sequentiality to concurrency}
  % \emph{Concurrent} proof reduction and proof construction}

  \begin{columns}[b,onlytextwidth]
  \column{0.3\textwidth}
    \begin{center}
      \fcolorbox{structure!30}{structure!10}{$
        \begin{gathered}
          \text{Tasks:} \\
          \begin{tikzcd}[ampersand replacement=\&, row sep=1.5ex, column sep=1.8em, arrow style=tikz, /tikz/>={To[scale=0.8]}, arrows={very thick, start anchor=real east, end anchor=real west}, /tikz/nodes={inner ysep=0}]
            \circtask[structure]{1.7ex}
              \rar[structure] \&
            \sqtask[gray]{1.7ex}
            \\
            \circtask[example text.fg]{1.7ex}
              \rar[example text.fg] \&
            \sqtask[gray]{1.7ex}
          \end{tikzcd}
        \end{gathered}
      $}
    \end{center}

  \column{0.6\textwidth}
    \begin{tikzcd}[row sep=2ex, arrow style=tikz, /tikz/>={To[scale=0.8]}, arrows={very thick}, /tikz/nodes={inner ysep=0}]
      &
      \sqtask[gray]{1.7ex} \, \circtask[example text.fg]{1.7ex}
        \drar[example text.fg, start anchor=east, end anchor=north west] &
      \\
      \circtask[structure]{1.7ex} \, \circtask[example text.fg]{1.7ex}
        \urar[structure, start anchor=north east, end anchor=west]
        \drar[example text.fg, start anchor=south east, end anchor={[yshift=-axis_height]north west}]
        \ar[rr, alert, dashed, dash phase=0.5em, start anchor=real east, end anchor=real west] &
      &
      \sqtask[gray]{1.7ex} \, \sqtask[gray]{1.7ex}
      \\
      &
      \circtask[structure]{1.7ex} \, \sqtask[gray]{1.7ex}
        \urar[structure, start anchor={[yshift=-axis_height]north east}, end anchor=south west] &
    \end{tikzcd}
  \end{columns}

  \vspace*{3ex}

  % \setbeamerfont*{itemize/enumerate body}{size=\small}
  \structure{Interleaving concurrent semantics:}
  \begin{itemize}
  \item Require that interleavings of independent tasks are indistinguishable,
  so that they appear to be concurrent.
    \begin{itemize}
    \item This diagram commutes.
    \item CLF's concurrent equality [WCPW '02]
    \end{itemize}
  \end{itemize}

  \structure{True concurrent semantics?}
\end{frame}


\begin{frame}[fragile]
  \frametitle{\emph{Concurrent} computation and deduction}

  \begin{columns}[t]
  \column{0.5\textwidth}
    \begin{center}
      \begin{tikzcd}[row sep=0, every cell/.append style={rectangle, inner sep=0}, arrow style=tikz, /tikz/>={To[scale=0.8]}, arrows={very thick, start anchor=east, end anchor=west}]
        % 
        & \drar[example text.fg, end anchor={[yshift=0.75ex]west}] & \\[-1ex]
        % 
        \pfreduce[scale=0.75]{structure}{example text.fg}
          \urar[structure, start anchor={[yshift=0.75ex]east}]
          \drar[example text.fg, start anchor={[yshift=-0.75ex]east}]
          \ar[rr, alert, dashed] & &
        % 
        \pfreduce[scale=0.75]{none}{none} \\[-1ex]
        % 
        & \urar[structure, end anchor={[yshift=-0.75ex]west}] &
      \end{tikzcd}
    \end{center}

  \column{0.55\textwidth}
    \begin{center}
      \begin{tikzcd}[row sep=0, every cell/.append style={rectangle, inner sep=0}, arrow style=tikz, /tikz/>={To[scale=0.8]}, arrows={very thick, start anchor=east, end anchor=west}]
        & \drar[example text.fg, end anchor={[yshift=0.75ex]west}] & \\
        % 
        \pfconstruct[scale=0.75]{none}{none}
          \urar[structure, start anchor={[yshift=0.75ex]east}]
          \drar[example text.fg, start anchor={[yshift=-0.75ex]east}]
          \ar[rr, alert, dashed]
        &&
        \pfconstruct[scale=0.75]{example text.fg}{structure}
        \equiv
        \pfconstruct[scale=0.75]{structure}{example text.fg}
        \\
        & \urar[structure, end anchor={[yshift=-0.75ex]west}] &
      \end{tikzcd}
    \end{center}
  \end{columns}


  \begin{columns}[c]
  \column{0.5\textwidth}
    \structure{Concurrent computation as concurrent proof reduction}    
    \begin{itemize}
    \item SILL: Proofs as session-typed processes
    \end{itemize}

    \showcolumnwidth

  \column{0.55\textwidth}
    \structure{Concurrent computation as concurrent proof construction}    
    \begin{itemize}
    \item CLF: Proofs as multiset rewriting traces
    \end{itemize}

    \showcolumnwidth
  \end{columns}
\end{frame}

\end{document}


e 1 0 1 1
e o v o o v?

e 1 0 1 1 ?
e 1 0 1 1 f

    e 0 ?
    e ? 0
    e t 0 s
e t 1    e 0 t s
e 1 t

