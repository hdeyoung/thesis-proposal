% arara: pdflatex
\documentclass[tikz]{standalone}

\tikzset{
  trapezoid height/.initial=1em,
  trapezoid width/.initial=1em,
  % sin t = o/h
  trapezoid/.pic={
    \pgfmathparse{\pgfkeysvalueof{/tikz/trapezoid width}}
    \show\pgfmathresult
    \pgfkeys{/tikz/trapezoid width/.expand once=\pgfmathresult}

    \fill (0,0)
      -- ++(\pgfkeysvalueof{/tikz/trapezoid width} , 0)
      -- ++(60 : cosec 60 * \pgfkeysvalueof{/tikz/trapezoid height})
      coordinate ( top right)
      -- ++(-2 * cot 60 * \pgfkeysvalueof{/tikz/trapezoid height} - \pgfkeysvalueof{/tikz/trapezoid width} , 0)
      circle [radius=1pt]
      coordinate ( top left)
      -- (0,0)
      -- cycle;
  }
}

% 
\begin{document}
\begin{tikzpicture}[equilateral triangle/.style={shape=regular polygon, regular polygon sides=3}]
  \draw [help lines] (0,0) grid (3cm,3cm);

  \pic [name=lower, trapezoid width=0.5cm, trapezoid height=0.5cm] {trapezoid};
  \pic [name=upper, at=(lower top left), trapezoid width=1.0555cm, trapezoid height=0.5cm] {trapezoid};

  % \node (tri) [equilateral triangle, fill=gray, minimum size=1cm, rotate=-60] {};
  % \node [trapezium, fill=blue, minimum height=0.2cm, trapezium angle=120,
  %        above=0.025cm of tri.corner 1, anchor=bottom right corner] {};
  % \node (tra) [trapezium, fill=green, minimum height=0.2cm, trapezium angle=120,
  %        above=0.025cm of tri.corner 2, anchor=bottom left corner] {};
  % \node [trapezium, fill=red, minimum width=0.75cm, trapezium angle=120, trapezium stretches body,
  %        above=0.025cm of tra] {};
\end{tikzpicture}
\end{document}
