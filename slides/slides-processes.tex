% arara: pdflatex
\documentclass{beamer}
\usepackage{my-proposal-talk}

% \usepackage[varl]{zi4}
\usepackage{zlmtt}
\usepackage{listings}


\renewcommand*{\inc}{\color{example}\mathtt{s}\color{black}}
\renewcommand*{\eps}{\color{structure}\mathtt{e}\color{black}}
\renewcommand*{\bit}[1]{\color{structure}\mathtt{#1}\color{black}}

\NewDocumentCommand \bits { }{ \mathord{\color{structure} B} }
\NewDocumentCommand \val { m }{ \llcorner #1 \lrcorner }

\newcommand*{\zeroq}{\mathtt{z\mkern-2mu \text{\texttt{?}}}}
\newcommand*{\zerott}{\mathtt{t\mkern-2mu t}}
\newcommand*{\zeroff}{\mathtt{f\mkern-2mu f}}


\RenewDocumentCommand \strcons { }
  { \mskip4mu }


\NewDocumentCommand \showcolumnwidth { }{%
  \tikz { \draw [|-|] (0,0) -- (\linewidth,0); }%
}



\lstset{
  basicstyle=\ttfamily,
  escapeinside={\%*}{*)},
  keepspaces=true,
  keywordstyle=\color{purple},
  morekeywords={proc,recv,send,case,of},
  emph={eps,bit0,bit1,bit0'}, emphstyle=\color{structure},
  emph={[2]s,z?,tt,ff}, emphstyle={[2]\color{example}},
  emph={[3]l,m,r}, emphstyle={[3]\itshape},
  columns=fixed,
  basewidth=0.5em,
}


\NewDocumentCommand{\proc}{m}{%
  \text{\ttfamily\color{structure}#1}%
}
\NewDocumentCommand{\chan}{m}{%
  \text{\ttfamily\itshape#1}%
}
\NewDocumentCommand{\spawn}{m m m}{%
  #1 \mathbin{\text{\ttfamily->}} #2 \mathbin{\text{\ttfamily->}} #3%
}
\NewDocumentCommand{\exec}{}{\mathop{\mathsf{exec}}}
\NewDocumentCommand{\msg}{}{
  \mathop{\mathsf{msg}}
}
\NewDocumentCommand{\send}{m m m}{
  \text{\ttfamily\color{purple} send}
  \mskip\thinmuskip
  #1
  \mkern\medmuskip
  #2
  \text{\ttfamily ;}
  #3
}

\begin{document}


\colorlet{example}{example text.fg}


\begin{frame}[fragile]
  \frametitle{From choreography to process definitions}

  \begin{columns}[onlytextwidth]
  \column{0.5\textwidth}
    \structure{\large Choreography}
  \column{0.5\textwidth}
    \structure{\large Process definition}
  \end{columns}
  \medskip

  \onslide<+->
  \begin{columns}[T,onlytextwidth]
  \column{0.5\textwidth}
    \begin{flalign*}
      \quad
      \begin{tikzpicture}[baseline=(b.base)]
        \graph [sill graph=0.5, nodes={process}] {
          / [coordinate]
           -> ["$\inc$" message, "$\text{\ttfamily\itshape r\vphantom{l}}$" channel]
          b/"\bit{1}"
           -> ["$\text{\ttfamily\itshape l}$" channel]
          / [coordinate];
        };
        %
        \fill<.(3)> [white, fill opacity=.7]
          (current bounding box.south west) rectangle (current bounding box.north east);
      \end{tikzpicture}
        &\trans
      \begin{tikzpicture}[baseline=(b.base)]
        \graph [sill graph=0.5, nodes={process}] {
          / [coordinate]
           -> ["$\text{\ttfamily\itshape r\vphantom{l}}$" channel]
          b/"\bit{0}"
           ->["$\inc$" message, "$\text{\ttfamily\itshape l}$" channel]
          / [coordinate];
        };
        %
        \fill<.(2)> [white, fill opacity=.7]
          (current bounding box.south west) rectangle (current bounding box.north east);
      \end{tikzpicture}&
      % \\[4\jot]
      % \begin{tikzpicture}[baseline=(b.base)]
      %   \graph [sill graph=0.5, nodes={process}] {
      %     / [coordinate]
      %      -> ["$\zeroq$" message, "$\text{\ttfamily\itshape r\vphantom{l}}$" channel]
      %     b/"\bit{1}"
      %      -> ["$\text{\ttfamily\itshape l}$" channel]
      %     / [coordinate];
      %   };
      % \end{tikzpicture}
      %   &\trans
      % \begin{tikzpicture}[baseline=(b.base)]
      %   \graph [sill graph=0.5, nodes={process}] {
      %     / [coordinate]
      %      -> ["$\zeroff$" {message=right, overlay}, "$\text{\ttfamily\itshape r\vphantom{l}}$" channel]
      %     b/"\bit{0}"
      %      ->["$\text{\ttfamily\itshape l}$" channel]
      %     / [coordinate];
      %   };
      % \end{tikzpicture}&
    \end{flalign*}
    % \showcolumnwidth

  \onslide<+->
  \column{0.5\textwidth}
    \begin{tabular}{@{\quad}c@{}}\begin{lstlisting}[gobble=6]
      %*\tikzmark{proc}*)l -> bit1 -> r =%*\tikzmark{eq}*)
      { %*\tikzmark{case}*)case (recv r) of%*\tikzmark{of}*)
          s => %*\tikzmark{send}*)send l s;
               l -> bit0 -> r%*\tikzmark{bit0r}*) }
    \end{lstlisting}\end{tabular}
    % \showcolumnwidth
  \tikz [remember picture, overlay] {
    \fill<+> [white, fill opacity=.7] ([yshift=1.5ex]send) rectangle ([xshift=.2em]bit0r);
    \fill<+> [white, fill opacity=.7] ([yshift=1.75ex]proc) |- ([yshift=-.75ex]proc -| case) |- ([yshift=-.25ex]send) |- ([yshift=-.25ex]of) |- (eq) |- ([yshift=1.75ex]proc) -- cycle; }
  \end{columns}
  \bigskip

  \onslide<+->
  \begin{columns}[T,onlytextwidth]
  \column{0.5\textwidth}
    \begin{flalign*}
      \quad
      \begin{tikzpicture}[baseline=(e.base)]
        \graph [sill graph=0.5, nodes={process}] {
          / [coordinate]
           -> ["$\inc$" message, "$\text{\ttfamily\itshape r}$" channel]
          e/"\eps";
        };
        %
        \fill<.(2)> [white, fill opacity=.7]
          (current bounding box.south west) rectangle (current bounding box.north east);
      \end{tikzpicture}
        &\trans
      \begin{tikzpicture}[baseline=(e.base)]
        \graph [sill graph=0.5, nodes={process}] {
          / [coordinate]
           -> ["$\text{\ttfamily\itshape r}$" channel]
          b/"\bit{1}"
           -> ["$\text{\ttfamily\itshape m}$" channel]
          e/"\eps";
        };
        %
        \fill<.(1)> [white, fill opacity=.7]
          (current bounding box.south west) rectangle (current bounding box.north east);
      \end{tikzpicture}&
    \end{flalign*}
    % \showcolumnwidth

  \column{0.5\textwidth}
    \begin{tabular}{@{\quad}c@{}}\begin{lstlisting}[gobble=6]
      %*\tikzmark{proc}*)eps -> r =%*\tikzmark{eq}*)
      { %*\tikzmark{case}*)case (recv r) of%*\tikzmark{of}*)
          s => %*\tikzmark{eps}*)eps -> m;
               m -> bit1 -> r%*\tikzmark{bit1r}*) }
    \end{lstlisting}\end{tabular}
    \tikz [remember picture, overlay] {
      \fill<+> [white, fill opacity=.7] ([yshift=1.5ex]eps) rectangle ([xshift=.2em]bit1r);
      \fill<+> [white, fill opacity=.7] ([yshift=1.75ex]proc) |- ([yshift=-.75ex]proc -| case) |- (eps) |- (of) |- (eq) |- ([yshift=1.75ex]proc) -- cycle; }
    % \showcolumnwidth
  \end{columns}
  \bigskip

  \onslide<+->
  \begin{columns}[T,onlytextwidth]
  \column{0.5\textwidth}
    \begin{flalign*}
      \quad
      \begin{tikzpicture}[baseline=(b.base)]
        \graph [sill graph=0.5, nodes={process}] {
          / [coordinate]
           -> ["$\inc$" message, "$\text{\ttfamily\itshape r\vphantom{l}}$" channel]
          b/"\bit{0}"
           -> ["$\text{\ttfamily\itshape l}$" channel]
          / [coordinate];
        };
      \end{tikzpicture}
        &\trans
      \begin{tikzpicture}[baseline=(b.base)]
        \graph [sill graph=0.5, nodes={process}] {
          / [coordinate]
           -> ["$\text{\ttfamily\itshape r\vphantom{l}}$" channel]
          b/"\bit{1}"
           -> ["$\text{\ttfamily\itshape l}$" channel]
          / [coordinate];
        };
      \end{tikzpicture}&
    \end{flalign*}
    % \showcolumnwidth

  \column{0.5\textwidth}
    \begin{tabular}{@{\quad}c@{}}\begin{lstlisting}[gobble=6]
      l -> bit0 -> r =
      { case (recv r) of
          s => l -> bit1 -> r }
    \end{lstlisting}\end{tabular}
    % \showcolumnwidth
  \end{columns}
\end{frame}


\begin{frame}[fragile]
  \frametitle{Equivalence of a choreography and its process definition}

  \onslide<+->
  \begin{columns}[onlytextwidth]
  \column{0.5\textwidth}
    \structure{\large Choreography}
  \column{0.5\textwidth}
    \structure{\large Process definition and \rlap{semantics}}
  \end{columns}
  \medskip

  \begin{columns}[T,onlytextwidth]
  \column{0.5\textwidth}
    \begin{flalign*}
      \quad
      \begin{tikzpicture}[baseline=(b.base)]
        \graph [sill graph=0.5, nodes={process}] {
          / [coordinate]
           -> ["$\inc$" message, "$\text{\ttfamily\itshape r\vphantom{l}}$" channel]
          b/"\bit{1}"
           -> ["$\text{\ttfamily\itshape l}$" channel]
          / [coordinate];
        };
      \end{tikzpicture}
        &\trans
      \begin{tikzpicture}[baseline=(b.base)]
        \graph [sill graph=0.5, nodes={process}] {
          / [coordinate]
           -> ["$\text{\ttfamily\itshape r\vphantom{l}}$" channel]
          b/"\bit{0}"
           ->["$\inc$" message, "$\text{\ttfamily\itshape l}$" channel]
          / [coordinate];
        };
      \end{tikzpicture}&
    \end{flalign*}
    % \showcolumnwidth

  \column{0.5\textwidth}
    \begin{tabular}{@{\quad}c@{}}\begin{lstlisting}[gobble=6]
      l -> bit1 -> r =
      { case (recv r) of
          s => send l s;
               l -> bit0 -> r }
    \end{lstlisting}\end{tabular}

  % If executions are fair, then the second step is invertible.
    % \showcolumnwidth
  \end{columns}

  \bigskip

  \onslide<+->
  % \hspace{\fill}\structure{\large Equivalence}\hspace{\fill}
  \begin{equation*}
    \qquad
    \begin{tikzpicture}
      \matrix [column sep=5em, column 2/.append style={inner sep=0, anchor=west}] {
        \graph [sill graph=.5, nodes={process}] {
          c1 / [coordinate]
           -> ["$\inc$" message, "$\text{\ttfamily\itshape r\vphantom{l}}$" channel]
          b1 / "\bit{1}"
           -> ["$\text{\ttfamily\itshape l}$" channel]
          / [coordinate];
        };
        &
        \node (b1 s)
        {$\exec(\spawn{\chan{l}}{\proc{bit1}}{\chan{r}})
          \strcons
          \msg(\inc)$};
        % 
        \\[5ex]
        % 
        &
        \node (snd)
        {$\mathmakebox[\widthof{$\exec(\spawn{\chan{l}}{\proc{bit1}}{\chan{r}}) \strcons \msg(\inc)$}][l]{
            \exec(\send{\chan{l}}{\inc}{\spawn{\chan{l}}{\proc{bit0}}{\chan{r}}})
          }$
        };
        % 
        \\[5ex]
        % 
        \graph [sill graph=.5, nodes={process}] {
          c0 / [coordinate]
           -> ["$\text{\ttfamily\itshape r\vphantom{l}}$" channel]
          b0 / "\bit{0}"
           -> ["$\inc$" {message, overlay}, "$\text{\ttfamily\itshape l}$" channel]
          / [coordinate];
        };
        &
        \node (s b0)
        {$\msg(\inc)
          \strcons
          \exec(\spawn{\chan{l}}{\proc{bit0}}{\chan{r}})$};
        \\
      };
      %
      \draw (b1) edge [->, shorten <=1ex, shorten >=1ex, line width=0.5pt] (b0);
      %
      \draw (b1 s) edge [->, shorten <=1ex, shorten >=1ex, line width=0.5pt] (snd);
      \draw (snd) edge [->, shorten <=1ex, shorten >=1ex, line width=0.5pt] (s b0);
      %
      \draw (c1) edge [shorten <=1em, shorten >=1em, line width=0.5pt, double equal sign distance, double] (b1 s);
      \draw (c0) edge [shorten <=1em, shorten >=1em, line width=0.5pt, double equal sign distance, double] (s b0);
    \end{tikzpicture}
  \end{equation*}
\end{frame}


\end{document}
