% arara: pdflatex
% arara: biber
% arara: pdflatex
% arara: pdflatex
\documentclass[
  class=../hdeyoung-proposal,
  crop=false
]{standalone}

\usepackage{ordered-logic}
\usepackage{basic-atoms}
\usepackage{ordered-lp-terms}
\usepackage{proof}
\usepackage{mathpartir}

\NewDocumentEnvironment{infers}{o O{}}%
  {%
    \noindent\IfValueT{#1}{\fbox{#1}#2}%
    \begin{mathpar}\ignorespaces
  }%
  {\end{mathpar}\ignorespacesafterend}

\NewDocumentCommand{\irlabel}{m m o}%
  {{#2}\text{\textsc{#1}}\IfValueT{#3}{_{#3}}}
\NewDocumentCommand{\rlab}{m o}%
  {\IfValueTF{#2}{\irlabel{r}{#1}[#2]}{\irlabel{r}{#1}}}
\NewDocumentCommand{\llab}{m o}%
  {\IfValueTF{#2}{\irlabel{l}{#1}[#2]}{\irlabel{l}{#1}}}

\NewDocumentCommand{\tctx}{}{\Psi}
\NewDocumentCommand{\tctxe}{}{\cdot}

% \NewDocumentCommand{\trans}{t* t+ o}{%
%   \longrightarrow
%   \IfBooleanT{#1}{^*}\IfBooleanT{#2}{^+}%
%   \IfValueT{#3}{_{#3}}%
% }
% \NewDocumentCommand{\ntrans}{}{
%   \longarrownot\trans
% }

\begin{document}

\subsection{Technical details}\label{sec:ordered-lp:technical}

The previous \lcnamecrefs{sec:olp-intuition:binary-counter} have hopefully served to provide intuition for ordered logical specifications.
In this \lcnamecref{sec:ordered-lp:technical}, we review the technical details, generally following the lead of \citeauthor{Simmons:CMU12}'s SLS framework~\autocite*{Simmons:CMU12}, but confining ourselves to a propositional fragment and using a weakly focused proof-construction strategy~\autocite{Simmons+Pfenning:CMU11}.
The reader should feel free to skip this \lcnamecref{sec:ordered-lp:technical}, since the technical details are not critical to an understanding of the rest of this proposal.

\subsubsection{Propositions, terms, and traces}\label{sec:props-terms-traces}

\paragraph{Propositions.}\label{sec:propositions}

Propositions are polarized into \vocab{positive} and \vocab{negative} classes:
\begin{alignat*}{2}
  &\text{Positive propositions}\quad & A^+ &::= \p^+ \mid A^- \mid A^+ \fuse B^+ \mid \one \\
  &\text{Negative propositions}      & A^- &::= A^+ \rimp B^- \mid A^+ \limp B^- \mid A^- \with B^- \mid \monad{A^+}
\end{alignat*}
The positive propositions, $A^+$, are those whose left rules are invertible, whereas the negative propositions, $A^-$, are those whose right rules are invertible.
% As an example, the ordered implications $A^+ \rimp B^-$ and $A^+ \limp B^-$ are classified as negative because the $\rlab{\rimp}$ and $\rlab{\limp}$ rules are invertible.

Positive atomic propositions, $\p^+$, stand in for arbitrary positive propositions.
Negative propositions, $A^-$, are included in positive ones.
The lax modality, or monad, $\monad{A^+}$, will be responsible for typing traces; it gives logical force to the separation of the two classes of propositions, yet still allows positive propositions to be included in negative ones.
% Second, negative propositions can similiarly be treated as positive ones via the implicit inclusion $A^-$.

\paragraph{Contexts.}\label{sec:contexts}
The set of all ordered contexts, $\octx$, forms the free monoid over the alphabet of hypotheses $x{:}A^+$ and $x{:}A^-$: concatenation is written as $\octx_1, \octx_2$ and its unit is the empty context, $\octxe$.
\begin{alignat*}{2}
  &\text{Ordered contexts}\quad & \octx &::= \octxe \mid x{:}A^+ \mid x{:}A^- \mid \octx_1, \octx_2
\end{alignat*}
Because the proof-construction strategy is weakly focused, non-atomic positive propositions, such as $A^+ \fuse B^+$, are indeed permissible hypotheses.

In addition to ordered contexts, we include \vocab{frames}.
Frames, $\ofrm$, can be thought of as ordered contexts with one hole; i.e., $\ofrm$ is morally the same as $\octx_L, \Box, \octx_R$, for some $\octx_L$ and $\octx_R$.
The hole may be filled with an ordered context, so that $\ofill{\octx}$ is the ordered context $\octx_L, \octx, \octx_R$.
To ease the notational clutter in typing rules, we also allow contexts to be matched against filled frames.

\paragraph{Terms.}\label{sec:terms}

As previously mentioned, a focused proof-construction strategy~\autocite{Andreoli:JLC92} forms the basis of ordered logical specifications.
Specifically, we choose a weakly focused strategy~\autocite{Simmons+Pfenning:CMU11}.
Each form of sequent in the weakly focused calculus corresponds to a syntactic class of terms.

We begin in \cref{fig:traces} with traces, $T$, which are typed with a judgment $\tof{T :: \octx \trans* \octx'}$ that describes the change of state that the trace effects.
\begin{figure}
  \begin{infers}[$\tof{T :: \octx \trans* \octx'}$]
    \begin{alignedat}{3}
      &\text{Traces} &\quad&& T &::= \tnil \mid \tseq{T_1; T_2} \mid S
    \end{alignedat}
    \\
    \infer{\tof{\tnil :: \octx \trans* \octx}}{
      }
    \and
    \infer{\tof{\tseq{T_1; T_2} :: \octx \trans* \octx''}}{
      \tof{T_1 :: \octx \trans* \octx'} &
      \tof{T_2 :: \octx' \trans* \octx''}}
    \and
    \infer{\tof{S :: \octx \trans* \octx'}}{
      \tof{S :: \octx \trans \octx'}}
  \end{infers}
  \caption{Traces\label{fig:traces}}
\end{figure}
Traces are either empty, $\tnil$, or the (nominally) sequential composition of two traces, $\tseq{T_1; T_2}$, or a single step, $S$.
The empty trace effects no change to the state;
the sequential composition $\tseq{T_1; T_2}$ first carries out trace $T_1$, and then continues with the remainder of the trace, $T_2$.
A single step is also typed by the change of state that it effects: $\tof{S :: \octx \trans \octx'}$ (\cref{fig:steps}).
\begin{figure}
  \begin{infers}[$\tof{S :: \octx \trans \octx'}$]
    \begin{alignedat}{3}
      &\text{Steps} &\quad&& S &::= \tstep{x <- R} \mid \tinv{\pfuse{y_1}{y_2} <- x} \mid \tinv{\pone <- x}
    \end{alignedat}
    \\
    \infer{\tof{\tstep{x <- R} :: \omatch{\octx} \trans \ofill{x{:}A^+}}}{
      \octx \seq \aof{R : \susp-{\monad{A^+}}}}
    \\
    \infer{\tof{\tinv{\pfuse{y_1}{y_2} <- x} :: \omatch{x{:}A^+_1 \fuse A^+_2} \trans \ofill{y_1{:}A^+_1 , y_2{:}A^+_2}}}{
      }
    \and
    \infer{\tof{\tinv{\pone <- x} :: \omatch{x{:}\one} \trans \ofill{\octxe}}}{
      }
    \and
  \end{infers}
  \caption{Steps\label{fig:steps}}
\end{figure}
Each step is either a left-focusing phase, with term $\tstep{x <- R}$, or one of two inversion steps, with terms $\tinv{\pfuse{y_1}{y_2} <- x}$ and $\tinv{\pone <- x}$.
In a left-focusing phase, if some piece of state $\octx$ within $\omatch{\octx}$ satisfies the requirements imposed by atomic term $R$ of type $\monad{A^+}$, then $\octx$ is replaced with $x{:}A^+$---no inversion is performed because the proof-construction stategy is weakly focused.
The inversion occurs as discrete steps that decompose non-atomic positive propositions.

% \Cref{fig:patterns} gives the typing rules for patterns $p$.
% \begin{figure}
%   \begin{infers}[Patterns: $\octx \pseq \pof{p : A^+}$]
%     \infer{x{:}\susp+{\p^+} \pseq \pof{x : \p^+}}{
%       }
%     \and
%     \infer{\octx_1, \octx_2 \pseq \pof{\pfuse{p_1}{p_2} : A^+_1 \fuse A^+_2}}{
%       \octx_1 \pseq \pof{p_1 : A^+_1} &
%       \octx_2 \pseq \pof{p_2 : A^+_2}}
%     \and
%     \infer{\octxe \pseq \pof{\pone : \one}}{
%       }
%   \end{infers}
%   \caption{Patterns\label{fig:patterns}}
% \end{figure}
% These rules correspond to the left inversion pattern rules of higher-order focusing~\autocite{Zeilberger:POPL08}.
% Patterns are either variables, $x$, 

The typing rules for atomic terms, $R$, according to the judgement $\octx \seq \aof{R : \susp-{C^-}}$ are shown in \cref{fig:atomic-terms}.
\begin{figure}
  \begin{infers}[$\octx \seq \aof{R : \susp-{C^-}}$]
    \begin{alignedat}{3}
      &\text{Atomic terms} &\quad&& R &::= \atm{c . \spine} \mid \atm{x . \spine}
    \end{alignedat}
    \\
    \infer{\omatch{\octxe} \seq \aof{\atm{c . \spine} : \susp-{C^-}}}{
      c{:}A^- \in \sig &
      \ofill{\lfoc{A^-}} \seq \sof{\spine : \susp-{C^-}}}    
    \and
    \infer{\omatch{x{:}A^-} \seq \aof{\atm{x . \spine} : \susp-{C^-}}}{
      \ofill{\lfoc{A^-}} \seq \sof{\spine : \susp-{C^-}}}
  \end{infers}
  \caption{Atomic terms\label{fig:atomic-terms}}
\end{figure}
These rules correspond to those for beginning a left-focusing phase from a stable sequent.
Atomic terms consist of a head---either a constant $c$ or variable $x$---followed by a spine, $\spine$, that completes the focusing phase.

Spines, as shown in \cref{fig:spines}, are either empty ($\snil$), a value application followed by a spine ($\app{V ; \spine}$), or a projection followed by a spine ($\fst{\spine}$ or $\snd{\spine}$).
\begin{figure}[!t]
  \begin{infers}[$\omatch{\lfoc{A^-}} \seq \sof{\spine : \susp-{C^-}}$]
    \begin{alignedat}{3}
      &\text{Spines} &\quad&& \spine &::= \snil \mid \app{V ; \spine} \mid \fst{\spine} \mid \snd{\spine}
    \end{alignedat}
    \\
    \infer{\lfoc{A^-} \seq \sof{\snil : \susp-{A^-}}}{
      }
    \\
    \infer{\omatch{\lfoc{A^+ \rimp B^-}, \octx} \seq \sof{\app{V ; \spine} : \susp-{C^-}}}{
      \octx \seq \vof{V : \rfoc{A^+}} &
      \ofill{\lfoc{B^-}} \seq \sof{\spine : \susp-{C^-}}}
    \and
    \infer{\omatch{\octx, \lfoc{A^+ \limp B^-}} \seq \sof{\app{V ; \spine} : \susp-{C^-}}}{
      \octx \seq \vof{V : \rfoc{A^+}} &
      \ofill{\lfoc{B^-}} \seq \sof{\spine : \susp-{C^-}}}
    \and
    \infer{\omatch{\lfoc{A^-_1 \with A^-_2}} \seq \sof{\fst{\spine} : \susp-{C^-}}}{
      \ofill{\lfoc{A^-_1}} \seq \sof{\spine : \susp-{C^-}}}
    \and
    \infer{\omatch{\lfoc{A^-_1 \with A^-_2}} \seq \sof{\snd{\spine} : \susp-{C^-}}}{
      \ofill{\lfoc{A^-_2}} \seq \sof{\spine : \susp-{C^-}}}    
  \end{infers}
  \caption{Spines\label{fig:spines}}
\end{figure}
The spine typing judgement $\omatch{\lfoc{A^-}} \seq \sof{\spine : \susp-{C^-}}$ corresponds to the left-focused sequent form.
Notice that spine typing can succeed even at non-atomic propositions, provided that the proposition under focus matches the consequent.
Also, value applications require that values, $V$, be typed by positive propositions under focus.

The value typing judgment, $\octx \seq \vof{V : \rfoc{A^+}}$, is shown in \cref{fig:values}.
\begin{figure}
  \begin{infers}[$\octx \seq \vof{V : \rfoc{A^+}}$]
    \begin{alignedat}{3}
      &\text{Values} &\quad&& V &::= x \mid N \mid \vfuse{V_1}{V_2} \mid \vone
    \end{alignedat}
    \\
    \infer{x{:}\p^+ \seq \vof{x : \rfoc{\p^+}}}{
      }
    \and
    \infer{\octx \seq \vof{N : \rfoc{A^-}}}{
      \octx \seq \nof{N : A^-}}
    \and
    \infer{\octx_1, \octx_2 \seq \vof{\vfuse{V_1}{V_2} : \rfoc{A^+_1 \fuse A^+_2}}}{
      \octx_1 \seq \vof{V_1 : \rfoc{A^+_1}} &
      \octx_2 \seq \vof{V_2 : \rfoc{A^+_2}}}
    \and
    \infer{\octxe \seq \vof{\vone : \rfoc{\one}}}{
      }
  \end{infers}
  \caption{Values\label{fig:values}}
\end{figure}
Of particular note is that normal terms, $N$, are included as values that type negative propositions that are included as positive ones.

Normal terms, $N$, are typed by negative propositions, $A^-$, under the judgment $\octx \seq \nof{N : A^-}$ shown in \cref{fig:normal-terms}.
\begin{figure}
  \begin{infers}[$\octx \seq \nof{N : A^-}$]
    \begin{alignedat}{3}
      &\text{Normal terms} &\quad&& N &::= \lam{x.N} \mid \pair{N_1, N_2} \mid \lett{T in V}
    \end{alignedat}
    \\
    \infer{\octx \seq \nof{\lam{x.N} : A^+ \rimp B^-}}{
      \octx, x{:}A^+ \seq \nof{N : B^-}}
    \and
    \infer{\octx \seq \nof{\lam{x.N} : A^+ \limp B^-}}{
      x{:}A^+, \octx \seq \nof{N : B^-}}
    \and
    \infer{\octx \seq \nof{\pair{N_1, N_2} : A^-_1 \with A^-_2}}{
      \octx \seq \nof{N_1 : A^-_1} &
      \octx \seq \nof{N_2 : A^-_2}}
    \and
    \infer{\octx \seq \nof{\lett{T in V} : \monad{A^+}}}{
      \tof{T :: \octx \trans* \octx'} &
      \octx' \seq \vof{V : \rfoc{A^+}}}
  \end{infers}
  \caption{Normal terms\label{fig:normal-terms}}
\end{figure}
Each normal term is either an abstraction ($\lam{x.N}$), a pair ($\pair{N_1, N_2}$), or a trace capped by a value ($\lett{T in V}$).
The typing judgment for normal terms corresponds to eager inversion on the right.

% \begin{alignat*}{3}
%   &\text{Traces} &\quad&& T &::= \tnil \mid \tstep{p <- R; T} \\
%   &\text{Atomic terms} &&& R &::= \atm{c . S} \mid \atm{x . S} \\
%   &\text{Spines} &&& S &::= \rapp{V ; S} \mid \lapp{V ; S} \mid \fst{S} \mid \snd{S} \mid \snil \\
%   &\text{Values} &&& V\! &::= x \mid \vfuse{V_1}{V_2} \mid \vone \\
%   &\text{Patterns} &&& p &::= x \mid \pfuse{p_1}{p_2} \mid \pone
% \end{alignat*}



\subsubsection{Concurrent equality}\label{sec:concurrent-equality}

\NewDocumentCommand{\inputs}{m}{\prescript{\mathord{\bullet}}{}{#1}}
\NewDocumentCommand{\outputs}{m}{#1^{\mathord{\bullet}}}
\NewDocumentCommand{\set}{m}{\{#1\}}
\NewDocumentCommand{\FV}{m}{\mathop{\mathrm{FV}}#1}

As described in \cref{sec:olp-intuition:concurrency}, \vocab{concurrent equality} formalizes the idea that different interleavings of independent rewritings should be indistinguishable.

As described in \cref{sec:olp-intuition:concurrency}, the different interleavings of independent rewritings should be indistinguishable.
Traces $T$ thus form a trace monoid in which independent rewriting steps commute.
Following \textcite{Cervesato+:LFMTP12}, this independence relation is defined on the sets of input and output variables, $\inputs{S}$ and $\outputs{S}$, of a step $S$.

Specifically, $\inputs{S}$ and $\outputs{S}$ are given by: 
\begin{alignat*}{3}
  \inputs{(\tstep{x <- R})} &= \FV{(R)} &\qquad&& \outputs{(\tstep{x <- R})} &= \set{x} \\
  \inputs{(\tinv{\pfuse{y_1}{y_2} <- x})} &= \set{x} &&& \outputs{(\tinv{\pfuse{y_1}{y_2} <- x})} &= \set{y_1, y_2} \\
  \inputs{(\tinv{\pone <- x})} &= \set{x} &&& \outputs{(\tinv{\pone <- x})} &= \emptyset
\end{alignat*}

Consider the trace $\tseq{S_1; S_2}$.
These two neighboring steps are independent and commute if $\outputs{S_1{}} \cap \inputs{S_2} = \emptyset$.
In other words, if $\outputs{S_1{}} \cap \inputs{S_2} = \emptyset$, then $\tof{\tseq{S_1; S_2} :: \octx \trans* \octx''}$ if and only if $\tof{\tseq{S_2; S_1} :: \octx \trans* \octx''}$.
Concurrent equality is the congruence relation that is obtained by associativity and unit axioms and this partial commutativity.

% Step $S_1$ must precede step $S_2$ if $S_1$ binds as outputs any variables that are used by $S_2$, i.e., if $\outputs{S_1{}} \cap \inputs{S_2} \neq \emptyset$.
% Conversely, steps $S_1$ and $S_2$ may be permuted if $\outputs{S_1{}} \cap \inputs{S_2} = \emptyset$.
% So, the traces $\tcons{S_1; \tcons{S_2; T}}$ and $\tcons{S_2; \tcons{S_1; T}}$ are equivalent whenever $\outputs{S_1{}} \cap \inputs{S_2} = \emptyset$.




% Following \textcite{Cervesato+:LFMTP12}, concurrent equality is defined by characterizing the steps in a trace that may be permuted.



% Traces can be adapted to a trace monoid with the independence relation defined on the sets of input and output variables, $\inputs{S}$ and $\outputs{S}$, of a step $S$.



% Consider the trace $\tcons{S_1; \tcons{S_2; T}}$.
% Step $S_1$ must precede step $S_2$ if $S_1$ binds as outputs any variables that are used by $S_2$, i.e., if $\outputs{S_1{}} \cap \inputs{S_2} \neq \emptyset$.
% Conversely, steps $S_1$ and $S_2$ may be permuted if $\outputs{S_1{}} \cap \inputs{S_2} = \emptyset$.
% So, the traces $\tcons{S_1; \tcons{S_2; T}}$ and $\tcons{S_2; \tcons{S_1; T}}$ are equivalent whenever $\outputs{S_1{}} \cap \inputs{S_2} = \emptyset$.

% This

% Step $S_2$ depends on following step $S_1$ if $S_2$ uses variables that are bound as outputs of $S_1$, i.e., if $\outputs{S_1} \cap \inputs{S_2} \neq \emptyset$.

Bound variables can always be renamed to be distinct from the input and output variables of previous steps in the trace.






% \subsubsection{Fairness}\label{sec:fairness}

% \NewDocumentCommand{\head}{m}{#1^\flat}
% \begin{definition}[Weak Transition-Fairness]
%   A trace $\tof{T :: \octx_0 \trans[\omega][]}$ is \vocab{weakly transition-unfair} if there exists an $i$ such that:
%   \begin{itemize}
%   \item for every $j \geq i$, there is a step $\tof{S'_j :: \octx_j \trans \octx'_j}$ for which $\head{(S'_j)} = \head{(S'_i)} \neq \head{(S_j)}$.
%   \end{itemize}
% \end{definition}
% In other words, trace $T$ is weakly transition-unfair if there is some transition $\head{(S'_i)}$ that is eventually (viz.\ from $i$ onward) persistently enabled but never taken by trace $T$.

% Transitions are defined from steps $S$ using the $\head{(-)}$ function:
% \begin{equation*}
%   \!\begin{aligned}
%     \head{(\tstep{p <- R})} &= \head{R} \\
%     %
%       \head{(\rapp{V ; S})} &= \rapp{\square ; \head{S}} \\
%       \head{(\lapp{V ; S})} &= \lapp{\square ; \head{S}} \\
%       \head{(\fst{S})} &= \fst{\head{S}} \\
%       \head{(\snd{S})} &= \snd{\head{S}} \\
%       \head{(\snil)} &= \snil
%     \end{aligned}
% \end{equation*}

\end{document}

%%% Local Variables:
%%% TeX-master: "ordered-lp"
%%% End:
