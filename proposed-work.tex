\section{Proposed work}\label{sec:extensions}

Should be easy:
\begin{itemize}
\item First-order quantifiers in the ordered logic setting
\item Adapt notion of choreography and compiler to linear logic
\end{itemize}
Somewhat harder:
\begin{itemize}
\item Modalities for sharing ($\bang$ and $\gnab$)
\item Infer choreographies and their types
\end{itemize}

This proposed work seems more like engineering.
In general, I am not sure what big picture understanding the compilation gives; it feels like all that we've done is to obfuscate the SSOS.

Maybe an understanding of what functional programs are possible for a given logic program would be a better goal?
This goal, however, doesn't need to rely on the compilation/functional aspect very much;
we could equally well ask what styles of choreography are possible for a given logic program.


\subsection{Schedule}\label{sec:schedule}

%%% Local Variables:
%%% TeX-master: "proposal"
%%% End:
