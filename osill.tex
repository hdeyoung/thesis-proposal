% arara: lualatex
\documentclass{article}

% \usepackage[T1]{fontenc}
% \usepackage[no-math]{fontspec}
% \usepackage[lining,semibold]{libertine}
% \usepackage[scaled=0.9,varqu,varl]{zi4}
% \usepackage{biolinum}
% \usepackage[scaled=0.9,varqu,varl]{zi4}


% \usepackage[lining,semibold]{libertine}
% 



% \usepackage[type1]{biolinum}
\usepackage[type1,lining,semibold,scale=0.95,tt=false]{libertine}
\usepackage[varqu,varl]{zi4}
\usepackage{amsmath}
\usepackage{amssymb}
\usepackage{mathtools}
\usepackage[libertine,slantedGreek]{newtxmath}
% \usepackage[no-math]{fontspec}
% \usepackage[osf,semibold,type1]{libertineRoman}
\usepackage[heavycircles]{stmaryrd}
\useosf

% \newcommand*\binampersand{\&}
% \newcommand*\shortleftarrow{\leftarrow}


\usepackage{expl3}
\usepackage{xparse}

\usepackage{mathpartir}
\usepackage{proof}

\usepackage{tikz}
\usetikzlibrary{graphs,graphdrawing}
\usegdlibrary{trees}

\usepackage{acro}
\DeclareAcronym{SILL}{
  short = \MakeLowercase{SILL},
  long = session-typed intuitionistic linear logic,
  short-format = \scshape
}

\ExplSyntaxOn

\DeclarePairedDelimiter \parens { \lparen } { \rparen }
\DeclarePairedDelimiter \bagged { \lbag } { \rbag }

\NewDocumentCommand \tseq { >{ \SplitArgument{1}{|-} } m }
  { \tseq:nn #1 }
\cs_new:Npn \tseq:nn #1#2 { \tseq_ctx:n {#1} \vdash #2 }
\cs_new:Npn \tseq_ctx:n #1 { #1 }

\NewDocumentCommand \oseq { >{ \SplitArgument{1}{|-} } m }
  { \oseq:nn #1 }
\cs_new:Npn \oseq:nn #1#2 { \oseq_ctxs:n {#1} \vdash #2 }
\cs_new:Npn \oseq_ctxs:n #1 {
  \oseq_ctxs_aux:w #1 ; \q_no_value ; \q_stop
}
\cs_new:Npn \oseq_ctxs_aux:w #1 ; #2 ; #3 \q_stop {
  #1
  \quark_if_no_value:nF {#2} { \mathrel{;} #2 }
}

\NewDocumentCommand \tctx {} { \Psi }
\NewDocumentCommand \octx {} { \lctx }
\NewDocumentCommand \lctx {} { \Delta }
\NewDocumentCommand \lctxe {} { \cdot }
\NewDocumentCommand \uctx {} { \Gamma }

\NewDocumentCommand \sub { >{ \SplitArgument{1}{/} } m m }
  { \sub:nnn #1 {#2} }
\cs_new:Npn \sub:nnn #1#2#3 { [#1/#2]#3 }

\NewDocumentCommand \fwd {} { \mathord{\leftrightarrow} }
\NewDocumentCommand \id { m } { \text{\textsc{\MakeLowercase{Id}}}\sb{#1} }
\NewDocumentCommand \cut { m } { \text{\textsc{\MakeLowercase{Cut}}}\sb{#1} }
\NewDocumentCommand \subst { m } { \text{\textsc{\MakeLowercase{Subst}}}\sb{#1} }

\NewDocumentCommand \comp { >{ \SplitArgument{1}{|} } m }
  { \comp:nn #1 }
\cs_new:Npn \comp:nn #1#2 { #1 \parallel #2 }

\NewDocumentCommand \with { O{} g }
  { \IfValueTF {#2} { \with:nn {#1} {#2} } \with: }
\cs_new:Npn \with:nn #1#2 {
  \mathord{\binampersand}
  \bagged {
    \seq_set_split:Nnn \l_tmpa_seq {,} {#2}
    \seq_use:Nn \l_tmpa_seq {,}
  } \sb{#1}
}
\cs_new:Npn \with: { \mathbin{\binampersand} }

\NewDocumentCommand \ssor { O{} g }
  { \IfValueTF {#2} { \ssor:nn {#1} {#2} } \ssor: }
\cs_new:Npn \ssor:nn #1#2 {
  \mathord{\ssor:}
  \bagged {
    \seq_set_split:Nnn \l_tmpa_seq {,} {#2}
    \seq_use:Nn \l_tmpa_seq {,}
  } \sb{#1}
}
\cs_new:Npn \ssor: { \oplus }

\NewDocumentCommand \caseR { s O{} m }
  {
    \IfBooleanTF {#1}
      { \case:nnnn { \mskip\thinmuskip \bagged } { \mathsf{caseR} } {#2} {#3} }
      { \case:nnnn \parens { \mathsf{caseR} } {#2} {#3} }
  }
\NewDocumentCommand \caseL { s O{} m }
  {
    \IfBooleanTF {#1}
      { \case:nnnn { \mskip\thinmuskip \bagged } { \mathsf{caseL} } {#2} {#3} }
      { \case:nnnn \parens { \mathsf{caseL} } {#2} {#3} }
  }
\cs_new:Npn \case:nnnn #1#2#3#4 {
  #2
  #1 {
    \seq_set_split:Nnn \l_tmpa_seq {|} {#4}
    \seq_clear:N \l_tmpb_seq
    \seq_map_inline:Nn \l_tmpa_seq
      { \seq_put_right:Nn \l_tmpb_seq { \case_branch:n {##1} } }
    \seq_use:Nn \l_tmpb_seq { \mid }
  } \sb{#3}
}
\cs_new:Npn \case_branch:n #1 { \case_branch_aux:w #1 \q_stop }
\cs_new:Npn \case_branch_aux:w #1 => #2 \q_stop {
  #1 \Rightarrow #2
}

\NewDocumentCommand \selectL { >{ \SplitArgument{1}{;} } m }
  { \select:nnn { \mathsf{selectL} } #1 }
\NewDocumentCommand \selectR { >{ \SplitArgument{1}{;} } m }
  { \select:nnn { \mathsf{selectR} } #1 }
\cs_new:Npn \select:nnn #1#2#3 {
  #1 \mskip4mu #2 ; #3
}

\NewDocumentCommand \inl {} { \inj{ \mathsf{1} } }
\NewDocumentCommand \inr {} { \inj{ \mathsf{2} } }

\NewDocumentCommand \inj { m } { \mathsf{in}\sb{#1} }


\NewDocumentCommand \one {} { \mathord { \mathbf{1} } }

\NewDocumentCommand \quitR {} { \mathsf{quitR} }
\NewDocumentCommand \waitL { m } { \mathsf{waitL} ; #1 }


\NewDocumentCommand \imp {} { \mathbin{\supset} }

\NewDocumentCommand \sendR { m m }
  { \send:nnn { \mathsf{sendR} } {#1} {#2} }
\NewDocumentCommand \sendL { m m }
  { \send:nnn { \mathsf{sendL} } {#1} {#2} }
\cs_new:Npn \send:nnn #1#2#3 {
  #1 \mskip4mu #2 ; #3
}

\NewDocumentCommand \recvL { m m }
  { \recv:nnn { \mathsf{recvL} } {#1} {#2} }
\NewDocumentCommand \recvR { m m }
  { \recv:nnn { \mathsf{recvR} } {#1} {#2} }
\cs_new:Npn \recv:nnn #1#2#3 {
  #2 \mskip4mu \mathord{\shortleftarrow} \mskip4mu #1 ; #3
}

\NewDocumentCommand \vcut { m } { \text{\textsc{\MakeLowercase{Vcut}}}\sb{#1} }
\NewDocumentCommand \cpy { } { \text{\textsc{\MakeLowercase{Copy}}} }

\NewDocumentCommand \cp { m m } { \mathsf{copy} \mskip\thinmuskip #1 ; #2 }
\NewDocumentCommand \vcomp { m >{ \SplitArgument{1}{|} } m }
  { \vcomp:nnn {#1} #2 }
\cs_new:Npn \vcomp:nnn #1#2#3 { #2 \parallel\sb{#1} #3 }
\NewDocumentCommand \vrecvR { m m }
  { \recv:nnn { \mathsf{vrecvR} } {#1} {#2} }
\NewDocumentCommand \vsendL { m m }
  { \send:nnn { \mathsf{vsendL} } {#1} {#2} }


\NewDocumentCommand \rrule { o m } {
  \IfValueTF {#1}
    { \rrule:nn {#2} {#1} }
    { \rrule:n {#2} }
}
\cs_new:Npn \rrule:nn #1#2 { {#1}\text{\textsc{\MakeLowercase{R}}}\sb{#2} }
\cs_new:Npn \rrule:n #1 { {#1}\text{\textsc{\MakeLowercase{R}}} }

\NewDocumentCommand \lrule { o m } {
  \IfValueTF {#1}
    { \lrule:nn {#2} {#1} }
    { \lrule:n {#2} }
}
\cs_new:Npn \lrule:nn #1#2 { {#1}\text{\textsc{\MakeLowercase{L}}}\sb{#2} }
\cs_new:Npn \lrule:n #1 { {#1}\text{\textsc{\MakeLowercase{L}}} }

\NewDocumentCommand \reduces {} { \longrightarrow }

\ExplSyntaxOff

\title{Session-typing for linearly ordered process networks}
\author{Henry DeYoung and Frank Pfenning}
\date{\today} 

\begin{document}

\maketitle

\section{Introduction}\label{sec:introduction}

In \ac{SILL}, session-typed sequents have the form
\begin{equation*}
  \underbrace{
    x_0{:}A_0 , x_1{:}A_1 , \dots , x_n{:}A_n
  }_{\textstyle \lctx}
  \vdash
  P :: x_{n+1}{:}A_{n+1}
\end{equation*}
where the channel names, $x_i$, are needed to unambiguously refer to hypotheses and the consequent.
The cut rule of \ac{SILL} types a parallel composition of processes,
\begin{equation*}
  \infer[\cut{A}]{\lctx , \lctx' \vdash (\nu x)(P \mid Q) :: z{:}C}{
    \lctx \vdash P :: x{:}A &
    \lctx' , x{:}A \vdash Q :: z{:}C}
\end{equation*}
Top-level $\cut{}$s thus arrange processes in a tree-shaped network, such as the following, in which some processes are clients of more than one process (i.e., some nodes have more than one child).
\begin{center}
  \begin{tikzpicture}
    \graph [tree layout, grow=left, empty nodes, nodes={draw, circle}] {
      / [draw=none] <-
      x <- { y0 <- z0 <- { w0 , w1 } ,
             y1 ,
             y2 <- { z1 , z2 <- w2 } };
    };
  \end{tikzpicture}
\end{center}
In this note, we are interested in developing a restriction of \ac{SILL} in which the process networks are linearly ordered:
\begin{center}
  \begin{tikzpicture}
    \graph [tree layout, grow=left, empty nodes, nodes={draw, circle}] {
      / [draw=none] <-
      x <- y <- z <- w;
    };
  \end{tikzpicture}
\end{center}

Each edge in the above \ac{SILL} tree represents a top-level $\cut{}$, and each top-level $\cut{}$ corresponds to a hypothesis that serves as its principal formula.
% Each hypothesis gives rise to a top-level $\cut{}$, represented as an edge in the above graph. 
Therefore, for the process network to be linearly ordered, contexts $\lctx$ must be either singletons or empty.
In this special case, sequents have one of the simpler forms
\begin{equation*}
  x_0{:}A_0 \vdash P :: x_1{:}A_1
  \quad\text{or}\quad
  \lctxe \vdash P :: x_1{:}A_1
  \,.
\end{equation*}
Additionally, now that there is at most one hypothesis, we no longer need channel names.
We can instead refer to the channels generically as the left channel (hypothesis) and the right channel (consequent), so that the forms of sequent are more simply
\begin{equation*}
  A_0 \vdash P :: A_1
  \quad\text{and}\quad
  \lctxe \vdash P :: A_1
  \,.
\end{equation*}



% If we restrict contexts $\lctx$ to be either empty or singletons, then sequents have one of the simpler forms
% \begin{equation*}
%   \lctxe \vdash P :: x_1{:}A_1
%   \quad\text{or}\quad
%   x_0{:}A_0 \vdash P :: x_1{:}A_1
%   \,.
% \end{equation*}
% Now that there is at most one hypothesis, we no longer need channel names --- we can instead refer to the channels generically as the left channel (hypothesis) and the right channel (consequent).
% \begin{equation*}
%   \lctxe \vdash P :: A_1
%   \quad\text{or}\quad
%   A_0 \vdash P :: A_1
%   \,.
% \end{equation*}

\section{Judgmental principles}\label{sec:judeg-princ}

When restricting sequents to have at most one hypothesis, the identity and cut principles become
\begin{mathpar}
  \infer[\id{A}]{\oseq{A |- A}}{
    }
  \and
  \infer[\cut{A}]{\oseq{\octx |- C}}{
    \oseq{\octx |- A} &
    \oseq{A |- C}}
\end{mathpar}

As in \ac{SILL}, the identity rule types a forwarding process, which we write here as $\fwd$.
The cut rule continues to type a process composition, but, without channel names, process composition is no longer commutative.
To reinforce this difference, we use a different notation, $\comp{P | Q}$, for this form of composition.
\begin{mathpar}
  \infer[\id{A}]{\oseq{A |- \fwd :: A}}{
    }
  \and
  \infer[\cut{A}]{\oseq{\octx |- \comp{P | Q} :: C}}{
    \oseq{\octx |- P :: A} &
    \oseq{A |- Q :: C}}
\end{mathpar}

\subsection{Cut reduction as computation}\label{sec:cut-reduction-as-1}

The cut reductions involving $\id{A}$ are
\begin{gather*}
  \infer[\cut{A}]{\oseq{A |- \comp{\fwd | Q} :: C}}{
    \infer[\id{A}]{\oseq{A |- \fwd :: A}}{
      } &
    \oseq{A |- Q :: C}}
  %
  \reduces
  %
  \oseq{A |- Q :: C}
  %
  \\
  \shortintertext{and}
  %
  \infer[\cut{A}]{\oseq{\octx |- \comp{P | \fwd} :: A}}{
    \oseq{\octx |- P :: A} &
    \infer[\id{A}]{\oseq{A |- \fwd :: A}}{
      }}
  %
  \reduces
  %
  \oseq{\octx |- P :: A}
\end{gather*}
and correspond to the process reductions
\begin{equation*}
  (\comp{\fwd | Q}) \reduces Q
  \quad\text{and}\quad
  (\comp{P | \fwd}) \reduces P
  \,.
\end{equation*}


\section{Additives}\label{sec:additives}

\subsection{Additive conjunction as branching}\label{sec:addit-conj-as-branch}

When restricted to singleton or empty contexts, the sequent calculus rules for the additive conjunction $A_1 \with A_2$ are as follows.
Note that in this case the left rules' contexts are forced to be the singleton $A_1 \with A_2$.
\begin{mathpar}
  \infer[\rrule{\with}]{\oseq{\octx |- A_1 \with A_2}}{
    \oseq{\octx |- A_1} &
    \oseq{\octx |- A_2}}
  \and
  \infer[{\lrule[1]{\with}}]{\oseq{A_1 \with A_2 |- C}}{
    \oseq{A_1 |- C}}
  \and
  \infer[{\lrule[2]{\with}}]{\oseq{A_1 \with A_2 |- C}}{
    \oseq{A_2 |- C}}
\end{mathpar}

Just as a proof of $A_1 \with A_2$ may be used as a proof of either $A_1$ or $A_2$, a process of type $A_1 \with A_2$ offers its client the choice of either service $A_1$ or service $A_2$.
Thus, the session-typing rules for $A_1 \with A_2$ are
\begin{mathpar}
  \infer[\rrule{\with}]{\oseq{\octx |- \caseR{\inl => P_1 | \inr => P_2} :: A_1 \with A_2}}{
    \oseq{\octx |- P_1 :: A_1} &
    \oseq{\octx |- P_2 :: A_2}}
  \\
  \infer[{\lrule[1]{\with}}]{\oseq{A_1 \with A_2 |- \selectL{\inl ; Q} :: C}}{
    \oseq{A_1 |- Q :: C}}
  \and
  \infer[{\lrule[2]{\with}}]{\oseq{A_1 \with A_2 |- \selectL{\inr ; Q} :: C}}{
    \oseq{A_2 |- Q :: C}}
\end{mathpar}

\subsubsection{Cut reduction as computation}\label{sec:cut-reduction-as}

There are two principal cut reductions for type $A_1 \with A_2$.
The first of these involves the $\rrule{\with}$ and $\lrule[1]{\with}$ rules:
\begin{gather*}
  \infer[\cut{A_1 \with A_2}]{\oseq{\octx |- \comp{\caseR{\inl => P_1 | \inr => P_2} | (\selectL{\inl ; Q})} :: C}}{
    \infer[\rrule{\with}]{\oseq{\octx |- \caseR{\inl => P_1 | \inr => P_2} :: A_1 \with A_2}}{
      \oseq{\octx |- P_1 :: A_1} &
      \oseq{\octx |- P_2 :: A_2}} &
    \infer[{\lrule[1]{\with}}]{\oseq{A_1 \with A_2 |- \selectL{\inl ; Q} :: C}}{
      \oseq{A_1 |- Q :: C}}}
  \\
  \reduces
  \\
  \infer[\mathrlap{\cut{A_1}}]{\oseq{\octx |- \comp{P_1 | Q} :: C}}{
    \oseq{\octx |- P_1 :: A_1} &
    \oseq{A_1 |- Q :: C}}
\end{gather*}
It corresponds to the process reduction in which the client selects the first branch,
\begin{equation*}
  \comp{\caseR{\inl => P_1 | \inr => P_2} | (\selectL{\inl; Q})}
    \reduces \comp{P_1 | Q}
  \,.
\end{equation*}
The second principal cut reduction for type $A_1 \with A_2$, involving the $\rrule{\with}$ and $\lrule[1]{\with}$ rules, is analogous and corresponds to the process reduction in which the client selects the second branch:
\begin{equation*}
  \comp{\caseR{\inl => P_1 | \inr => P_2} | (\selectL{\inr; Q})}
    \reduces \comp{P_2 | Q}
  \,.
\end{equation*}

\subsubsection{Identity expansion as forwarding}\label{sec:ident-expans-as}

The identity expansion for $A_1 \with A_2$ is  
\begin{gather*}
  \infer[\mathrlap{\id{A_1 \with A_2}}]{\oseq{A_1 \with A_2 |- \fwd :: A_1 \with A_2}}{
    }
  \\
  \equiv
  \\
  \infer[\rrule{\with}]{\oseq{A_1 \with A_2 |- \caseR{\inl => \selectL{\inl ; \fwd} | \inr => \selectL{\inr ; \fwd}} :: A_1 \with A_2}}{
    \infer[{\lrule[1]{\with}}]{\oseq{A_1 \with A_2 |- \selectL{\inl ; \fwd} :: A_1}}{
      \infer[\id{A_1}]{\oseq{A_1 |- \fwd :: A_1}}{
        }} &
    \infer[{\lrule[2]{\with}}]{\oseq{A_1 \with A_2 |- \selectL{\inr ; \fwd} :: A_2}}{
      \infer[\id{A_2}]{\oseq{A_1 |- \fwd :: A_2}}{
        }}}
\end{gather*}
It corresponds to an observational equivalence on session-typed processes:
\begin{equation*}
  (\fwd :: A_1 \with A_2) \equiv \caseR{\inl => \selectL{\inl ; \fwd} | \inr => \selectL{\inr ; \fwd}}
  \,.
\end{equation*}
  

\subsubsection{$n$-ary generalization}\label{sec:n-ary-generalization}

For the programmer's convenience, we generalize the binary additive conjunction to a labeled $n$-ary additive conjunction, $\with{\ell_i : A_i}$.
The session-typing rules are thus
\begin{mathpar}
  \infer[\rrule{\with}]{\oseq{\octx |- \caseR*{\ell_i => P_i} :: \with{\ell_i : A_i}}}{
    \bagged{\oseq{\octx |- P_i :: A_i}}}
  \and
  \infer[\lrule{\with}]{\oseq{\with{\ell_i : A_i} |- \selectL{\ell_k ; Q} :: C}}{
    \oseq{A_k |- Q :: C}}
\end{mathpar}
with a single principal cut reduction that corresponds to the process reduction
\begin{equation*}
  \comp{(\caseR*{\ell_i => P_i}) | (\selectL{\ell_k ; Q})} \reduces \comp{P_k | Q}
\end{equation*}
and an identity expansion that corresponds to the equivalence   
\begin{equation*}
  (\fwd :: \with{\ell_i: A_i}) \equiv \caseR*{\ell_i => \selectL{\ell_i ; \fwd}}
  \,.
\end{equation*}

\subsection{Additive disjunction as choice}\label{sec:addit-disj-as}

The additive disjunction $A_1 \ssor A_2$ is dual to the additive conjunction $A_1 \with A_2$.
% \begin{mathpar}
%   \infer{\oseq{\octx |- A_1 \ssor A_2}}{
%     \oseq{\octx |- A_1}}
%   \and
%   \infer{\oseq{\octx |- A_1 \ssor A_2}}{
%     \oseq{\octx |- A_2}}
%   \and
%   \infer{\oseq{A_1 \ssor A_2 |- C}}{
%     \oseq{A_1 |- C} &
%     \oseq{A_2 |- C}}
% \end{mathpar}
A process that offers service $A_1 \ssor A_2$ is one that offers its own choice, not the client's choice, of services $A_1$ or $A_2$.
When annotated with process terms, the right and left sequent rules for $\ssor$ are therefore: 
\begin{mathpar}
  \infer[{\rrule[1]{\ssor}}]{\oseq{\octx |- \selectR{\inl ; P} :: A_1 \ssor A_2}}{
    \oseq{\octx |- P :: A_1}}
  \and
  \infer[{\rrule[2]{\ssor}}]{\oseq{\octx |- \selectR{\inr ; P} :: A_1 \ssor A_2}}{
    \oseq{\octx |- P :: A_2}}
  \and
  \infer[\lrule{\ssor}]{\oseq{A_1 \ssor A_2 |- \caseL{\inl => Q_1 | \inr => Q_2} :: C}}{
    \oseq{A_1 |- Q_1 :: C} &
    \oseq{A_2 |- Q_2 :: C}}
\end{mathpar}
Again, notice that the context restriction forces the left rule's context to be only the principal formula, $A_1 \ssor A_2$.

The two principal cut reductions are dual to those for $A_1 \with A_2$; they correspond to the process reductions
\begin{align*}
  \comp{(\selectR{\inl; P}) | \caseL{\inl => Q_1 | \inr => Q_2}}
    &\reduces \comp{P | Q_1} \\
  %
  \shortintertext{and}
  %
  \comp{(\selectR{\inr; P}) | \caseL{\inl => Q_1 | \inr => Q_2}}
    &\reduces \comp{P | Q_2}
  \,.
\end{align*} 

The identity expansion is also dual to that of $A_1 \with A_2$; it corresponds to the observational equivalence
\begin{equation*}
  (\fwd :: A_1 \ssor A_2) \equiv \caseL{\inl => \selectR{\inl ; \fwd} | \inr => \selectR{\inr ; \fwd}}
  \,.
\end{equation*}

Finally, for the programmer's convenience, we again generalize the binary additive disjunction to labeled, $n$-ary additive disjunction.
The session-typing rules are
\begin{mathpar}
  \infer[\rrule{\ssor}]{\oseq{\octx |- \selectR{\ell_k ; P} :: \ssor{\ell_i : A_i}}}{
    \oseq{\octx |- P :: A_k}}
  \and
  \infer[\lrule{\ssor}]{\oseq{\ssor{\ell_i : A_i} |- \caseL*{\ell_i => Q_i} :: C}}{
    \bagged{\oseq{A_i |- Q_i :: C}}}
\end{mathpar}
with the principal cut reduction that corresponds to the process reduction
\begin{equation*}
  \comp{(\selectR{\ell_k ; P}) | (\caseL*{\ell_i => Q_i})}
    \reduces \comp{P | Q_k}
\end{equation*}
and the identity expansion that corresponds to the observational equivalence
\begin{equation*}
  (\fwd :: \ssor{\ell_i: A_i}) \equiv \caseL*{\ell_i => \selectR{\ell_i ; \fwd}}
  \,.
\end{equation*}


\section{Multiplicative unit as termination}\label{sec:mult-unit-as}

Once contexts are restricted at most one hypothesis, the right and left sequent rules for $\one$ are:
\begin{mathpar}
  \infer[\rrule{\one}]{\oseq{\lctxe |- \one}}{
    }
  \and
  \infer[\lrule{\one}]{\oseq{\one |- C}}{
    \oseq{\lctxe |- C}}
\end{mathpar}
As session-typing rules, these rules type termination processes:
\begin{mathpar}
  \infer[\rrule{\one}]{\oseq{\lctxe |- \quitR :: \one}}{
    }
  \and
  \infer[\lrule{\one}]{\oseq{\one |- \waitL{Q} :: C}}{
    \oseq{\lctxe |- Q :: C}}
\end{mathpar}

The principal cut reduction for $\one$ is
\begin{equation*}
  \infer[\cut{\one}]{\oseq{\lctxe |- \comp{\quitR | (\waitL{Q})} :: C}}{
    \infer[\rrule{\one}]{\oseq{\lctxe |- \quitR :: \one}}{
      } &
    \infer[\lrule{\one}]{\oseq{\one |- \waitL{Q} :: C}}{
      \oseq{\lctxe |- Q :: C}}}
  \reduces
  \oseq{\lctxe |- Q :: C}
\end{equation*}
It corresponds to the process reduction
\begin{equation*}
  \comp{\quitR | (\waitL{Q})}  \reduces  Q
  \,.
\end{equation*}

The identity expansion for $\one$ is
\begin{equation*}
  \infer[\id{\one}]{\oseq{\one |- \fwd :: \one}}{
    }
  \equiv
  \infer[\lrule{\one}]{\oseq{\one |- \waitL{\quitR} :: \one}}{
    \infer[\rrule{\one}]{\oseq{\lctxe |- \quitR :: \one}}{
      }}
\end{equation*}
and corresponds to the observational equivalance
\begin{equation*}
  (\fwd :: \one) \equiv \waitL{\quitR}
  \,.
\end{equation*}


\section{First-order connectives}\label{sec:data}

\subsection{First-order conjunction as data output}\label{sec:first-order-conj}

The conjunction of a proposition and a first-order type is used to describe the behavior of a process that offers the output of a first-order term of type $\tau$.
The right and left session-typing rules are 
% \begin{mathpar}
%   \infer[\rrule{\land}]{\oseq{\tctx ; \octx |- A \land \tau}}{
%     \tseq{\tctx |- M : \tau} &
%     \oseq{\tctx ; \octx |- A}}
%   \and
%   \infer[\lrule{\land}]{\oseq{\tctx ; A \land \tau |- C}}{
%     \oseq{\tctx, x:\tau ; A |- C}}
% \end{mathpar}
\begin{mathpar}
  \infer[\rrule{\land}]{\oseq{\tctx ; \octx |- \sendR{M}{P} :: A \land \tau}}{
    \tseq{\tctx |- M : \tau} &
    \oseq{\tctx ; \octx |- P :: A}}
  \and
  \infer[\lrule{\land}]{\oseq{\tctx ; A \land \tau |- \recvL{x}{Q} :: C}}{
    \oseq{\tctx, x:\tau ; A |- Q :: C}}
\end{mathpar}

The principal cut reduction for $A \land \tau$ is
\begin{gather*}
  \infer[\cut{A \land \tau}]{\oseq{\tctx ; \octx |- \comp{(\sendR{M}{P}) | (\recvL{x}{Q})} :: C}}{
    \infer[\rrule{\land}]{\oseq{\tctx ; \octx |- \sendR{M}{P} :: A \land \tau}}{
      \tseq{\tctx |- M : \tau} &
      \oseq{\tctx ; \octx |- P :: A}} &
    \infer[\lrule{\land}]{\oseq{\tctx ; A \land \tau |- \recvL{x}{Q} :: C}}{
      \oseq{\tctx, x:\tau ; A |- Q :: C}}}
  \\
  \reduces
  \\
  \infer[\cut{A}]{\oseq{\tctx ; \octx |- \comp{P | (\sub{M/x}{Q})} :: C}}{
    \oseq{\tctx ; \octx |- P :: A} &
    \infer[\subst{\tau}]{\oseq{\tctx ; A |- \sub{M/x}{Q} :: C}}{
      \tseq{\tctx |- M : \tau} &
      \oseq{\tctx, x:\tau ; A |- Q :: C}}}
\end{gather*}
which corresponds to the process reduction
\begin{equation*}
  \comp{(\sendR{M}{P}) | (\recvL{x}{Q})} \reduces \comp{P | (\sub{M/x}{Q})}
  \,.
\end{equation*}
The identity expansion at type $A \land \tau$ is
\begin{gather*}
  \infer[\id{A \land \tau}]{\oseq{\tctx ; A \land \tau |- \fwd :: A \land \tau}}{
    }
  \\
  \equiv
  \\
  \infer[\lrule{\land}]{\oseq{\tctx ; A \land \tau |- \recvL{x}{\sendR{x}{\fwd}} :: A \land \tau}}{
    \infer[\rrule{\land}]{\oseq{\tctx, x:\tau ; A |- \sendR{x}{\fwd} :: A \land \tau}}{
      \tseq{\tctx, x:\tau |- x : \tau} &
      \infer[\id{A}]{\oseq{\tctx, x:\tau ; A |- \fwd :: A}}{
        }}}
\end{gather*}
which corresponds to an observational equivalence on typed processes:
\begin{equation*}
  (\fwd :: A \land \tau) \equiv \recvL{x}{\sendR{x}{\fwd}}
  \,.
\end{equation*}

\subsection{First-order implication as data input}\label{sec:first-order-impl}

The connective $\imp$ is dual to $\land$, having right and left session-typing rules
% \begin{mathpar}
%   \infer[\rrule{\imp}]{\oseq{\tctx ; \octx |- \tau \imp A}}{
%     \oseq{\tctx, x:\tau ; \octx |- A}}
%   \and
%   \infer[\lrule{\imp}]{\oseq{\tctx ; \tau \imp A |- C}}{
%     \tseq{\tctx |- M : \tau} &
%     \oseq{\tctx ; A |- C}}
% \end{mathpar}
\begin{mathpar}
  \infer[\rrule{\imp}]{\oseq{\tctx ; \octx |- \recvR{x}{P} :: \tau \imp A}}{
    \oseq{\tctx, x:\tau ; \octx |- P :: A}}
  \and
  \infer[\lrule{\imp}]{\oseq{\tctx ; \tau \imp A |- \sendL{M}{Q} :: C}}{
    \tseq{\tctx |- M : \tau} &
    \oseq{\tctx ; A |- Q :: C}}
\end{mathpar}
The principal cut reduction for $\tau \imp A$ is also analogous to that of $A \land \tau$, and corresponds to the process reduction
\begin{equation*}
  \comp{(\recvR{x}{P}) | (\sendL{M}{Q})} \reduces \comp{(\sub{M/x}{P}) | Q}
  \,.
\end{equation*}
Likewise, the identity expansion for $\tau \imp A$ gives rise to the observational equivalence
\begin{equation*}
  (\fwd :: \tau \imp A) \equiv \recvR{x}{\sendL{x}{\fwd}}
\end{equation*}
that is analogous to the one for $A \land \tau$.


\section{Example: Binary natural numbers}\label{sec:exampl-binary-natur}

\begin{verbatim}
datatype bin = 0 | 1

stype Bit = &{ s: Bit ,
               z?: bool /\ Bit ,
               p: +{ fail: Bit , succ: Bit } ,
               c: Bit' }
 and Bit' = +{ a: bin /\ Bit' , q: Bit }

eps :: |- Bit =
{ caseR of 
    s => eps || bit1
  | z? => sendR true; eps
  | p => selectR fail; eps
  | c => selectR q; eps }

bit0 :: Bit |- Bit =                       bit1 :: Bit |- Bit =       
{ caseR of                                 { caseR of                 
    s => bit1                                  s => selectL s; bit0
  | z? => selectL z?; b <- recvL;            | z? => sendR false; bit1
          sendR b; bit0                      | p => selectR succ; bit0
  | p => selectL p;                         | c => selectR a1;
         caseL of                                   selectL c; bit1' }
           fail => selectR fail; bit0
         | succ => selectR succ; bit1
  | c => selectR a0;
         selectL c; bit0' }

bit0' :: Bit' |- Bit' =            bit1' :: Bit' |- Bit' =    
{ caseL of                         { caseL of                 
    a => b <- recvL;                   a => b <- recvL;
         selectR a; sendR b;                selectR a; sendR b;
         bit0'                              bit1'
  | q => selectR q; bit0 }           | q => selectR q; bit1 }

copy :: Bit' |- Bit =
{ caseL of 
    a => b <- recvL;
         case b of
           0 => copy || bit0
         | 1 => copy || bit1
  | q => <-> }
\end{verbatim}


\section{}

\begin{mathpar}
  \infer[\cpy]{\oseq{\uctx , u:A ; \lctxe |- \cp{u}{Q} :: C}}{
    \oseq{\uctx , u:A ; A |- Q :: C}}
  \and
  \infer[\vcut{A}]{\oseq{\uctx ; \lctx' |- \vcomp{u}{P | Q} :: C}}{
    \oseq{\uctx ; \lctxe |- P :: A} &
    \oseq{\uctx , u:A ; \lctx' |- Q :: C}}
  \and
  \infer[\rrule{\imp}]{\oseq{\uctx ; \lctx |- \vrecvR{u}{P} :: A \imp B}}{
    \oseq{\uctx , u:A ; \lctx |- P :: B}}
  \and
  \infer[\lrule{\imp}]{\oseq{\uctx ; A \imp B |- \vsendL{Q_1}{Q_2} :: C}}{
    \oseq{\uctx ; \lctxe |- Q_1 :: A} &
    \oseq{\uctx ; B |- Q_2 :: C}}
\end{mathpar}

\begin{gather*}
  \infer[\vcut{A}]{\oseq{\uctx ; \lctx' |- \vcomp{u}{P | (\cp{u}{Q})} :: C}}{
    \oseq{\uctx ; \lctxe |- P :: A} &
    \infer[\cpy]{\oseq{\uctx , u:A ; \lctxe |- \cp{u}{Q} :: C}}{
      \oseq{\uctx , u:A ; A |- Q :: C}}}
  \\
  \reduces
  \\
  \infer[\vcut{A}]{\oseq{\uctx ; \lctxe |- \vcomp{u}{P | (\comp{P | Q})} :: C}}{
    \oseq{\uctx ; \lctxe |- P :: A} &
    \infer[\cut{A}]{\oseq{\uctx , u:A ; \lctxe |- \comp{P | Q} :: C}}{
      \oseq{\uctx , u:A ; \lctxe |- P :: A} &
      \oseq{\uctx , u:A ; A |- Q :: C}}}
\end{gather*}


\begin{gather*}
  \infer[\cut{A \imp B}]{\oseq{\uctx ; \lctx |- \comp{(\vrecvR{u}{P}) | (\vsendL{Q_1}{Q_2})} :: C}}{
    \infer[\rrule{\imp}]{\oseq{\uctx ; \lctx |- \vrecvR{u}{P} :: A \imp B}}{
      \oseq{\uctx , u:A ; \lctx |- P :: B}} &
    \infer[\lrule{\imp}]{\oseq{\uctx ; A \imp B |- \vsendL{Q_1}{Q_2} :: C}}{
      \oseq{\uctx ; \lctxe |- Q_1 :: A} &
      \oseq{\uctx ; B |- Q_2 :: C}}}
  \\
  \reduces
  \\
  \infer[\vcut{A}]{\oseq{\uctx ; \lctx |- \vcomp{u}{Q_1 | (\comp{P | Q_2})} :: C}}{
    \oseq{\uctx ; \lctxe |- Q_1 :: A} &
    \infer[\cut{B}]{\oseq{\uctx , u:A ; \lctx |- \comp{P | Q_2} :: C}}{
      \oseq{\uctx , u:A ; \lctx |- P :: B} &
      \oseq{\uctx , u:A ; B |- Q_2 :: C}}}
\end{gather*}

\begin{gather*}
  \infer[\mathrlap{\id{A \imp B}}]{\oseq{\uctx ; A \imp B |- \fwd :: A \imp B}}{
    }
  \\
  \equiv
  \\
  \infer[\rrule{\imp}]{\oseq{\uctx ; A \imp B |- \vrecvR{u}{\vsendL{(\cp{u}{\fwd})}{\fwd}} :: A \imp B}}{
    \infer[\lrule{\imp}]{\oseq{\uctx , u:A ; A \imp B |- \vsendL{(\cp{u}{\fwd})}{\fwd} :: B}}{
      \infer[\cpy]{\oseq{\uctx , u:A ; \lctxe |- \cp{u}{\fwd} :: A}}{
        \infer[\id{A}]{\oseq{\uctx , u:A ; A |- \fwd :: A}}{
          }} &
      \infer[\id{B}]{\oseq{\uctx , u:A ; B |- \fwd :: B}}{
        }}}
\end{gather*}



\begin{center}
  \begin{tikzpicture}[valid/.style={fill=gray!50}]
    \graph [tree layout, grow=left, empty nodes, nodes={circle,draw}] {
      // [tree layout] {
        x0 [draw=none] <- x1 <- x2 <- x3;
      };

      // [tree layout, sibling distance=0.5em] {
        x2 [grow=120] <- { [nodes={valid}] u0 <- { u1 , u2 , u3 } };
      };

      // [tree layout, sibling distance=0.5em] {
        x3 [grow=-120] <- { [nodes={valid}] u4 <- { u5 , u6 } };
      };
    };
  \end{tikzpicture}
\qquad
  \begin{tikzpicture}[valid/.style={fill=gray!50}]
    \graph [tree layout, grow=left, sibling distance=0.5em, empty nodes, nodes={circle,draw}] {
      // [tree layout] {
        x0 [draw=none] <- x1 <- x2 <- x3 <- x5 <- { [nodes={valid}] u5' , u6' };
      };

      // [tree layout] {
        x2 [grow=120] <- { [nodes={valid}] u0 <- { u1 , u2 , u3 } };
      };

      // [tree layout] {
        x3 [grow=-120] <- { [nodes={valid}] u4 <- { u5 , u6 } };
      };
    };
  \end{tikzpicture}
\end{center}

\end{document}

LM =
  selectR copy;
  <->

bit0' =
  caseL of
    copy => selectR app0;
            bit0 | LM
  | app0 => selectR app0;
            bit0'
  | app1 => ...

Rm 
  caseL of 
    app0 => RM | bit0



e LM -> e F
0 LM -> LM 0' A0
A0 0' -> 0' A0
A0 Rm -> Rm 0
F 0' -> 0 F
F Rm -> .




e S L T R T -->* e 

e S 0 L T R T

Bit = &{ cp: Bit' }
Bit' = +{ app0: Bit' , app1: Bit' , quit: Bit }

eps : |- Bit
eps =
{ caseR of
    cp => selectR quit;
          eps }

bit0 : Bit |- Bit
bit0 =
{ caseR of
    cp => selectR app0;
          selectL cp;
          bit0' }

bit0' =
{ caseL of
    app0 => selectR app0;
            bit0'
  | app1 => ...
  | quit => selectR quit;
            bit0 }

copy : Bit' |- Bit
copy =
{ caseL of
    app0 => copy | bit0
  | app1 => ...
  | quit => <-> }


stype Bit = &{ s: Bit ,
               z?: bool /\ Bit ,
               p: +{ fail: Bit , succ: Bit } ,
               c: Bit' }
 and Bit' = +{ a0: Bit' , a1: Bit' , q: Bit }

eps : |- Bit
eps =
{ caseR of 
    s => eps | bit1
  | z? => sendR true; eps
  | p => selectR fail; eps
  | c => selectR q; eps }

bit0 : Bit |- Bit
bit0 =
{ caseR of
    s => bit1
  | z? => selectL z?; b <- recvL;
          sendR b; bit0
  | p => selectL p;
         caseL of
           fail => selectR fail; bit0
         | succ => selectR succ; bit1
  | c => selectR a0;
         selectL c; bit0' }

bit0' : Bit' |- Bit'
bit0' =
{ caseL of
    a0 => selectR a0; bit0'
  | a1 => selectR a1; bit0'
  | q => selectR q; bit0 }

bit1 : Bit |- Bit
bit1 =
{ caseR of
    s => selectL s; bit0
  | z? => sendR false; bit1
  | p => selectR succ; bit0
  | c => selectR a1;
         selectL c; bit1' }

bit1' : Bit' |- Bit'
bit1' =
{ caseL of
    a0 => selectR a0; bit1'
  | a1 => selectR a1; bit1'
  | q => selectR q; bit1 }

copy : Bit' |- Bit
copy =
{ caseL of 
    a0 => copy | bit0
  | a1 => copy | bit1
  | q => <-> }


e 1 0 c cp
e 1 c 0' a0 cp
e c 1' a1 0' cp 0
e q 1' 0' a1 cp 0
e 1 q 0' cp 1 0
e 1 0 q cp 1 0
e 1 0 1 0

e 1 0 c cp c cp
e 1 c 0' a0 cp c cp
e c 1' a1 0' cp 0 c cp
e q 1' 0' a1 cp c 0' a0 cp
e 1 q 0' cp 1 c 0' cp 0
e 1 0 q cp c 1' a1 0' cp 0
e 1 0 c 1' 0' a1 cp 0
e 1 c 0' a0 1' 0' cp 1 0
e c 1' a1 0' 1' a0 0' cp 1 0
e q 1' 0' a1 1' 0' a0 cp 1 0
e 1 q 0' 1' a1 0' cp 0 1 0
e 1 0 q 1' 0' a1 cp 0 1 0
e 1 0 1 q 0' cp 1 0 1 0
e 1 0 1 0 q cp 1 0 1 0
e 1 0 1 0 1 0 1 0

e 1 0 c cp s c cp
e 1 c 0' a0 cp s c cp
e c 1' a1 0' cp 0 s c cp
e q 1' 0' a1 cp 1 c cp
e 1 q 0' cp 1 c 1' a1 cp
e 1 0 q cp c 1' a1 1' cp 1
e 1 0 c 1' 1' a1 cp 1
e 1 c 0' a0 1' 1' cp 1 1
e c 1' a1 0' 1' a0 1' cp 1 1
e q 1' 0' a1 1' 1' a0 cp 1 1
e 1 q 0' 1' a1 1' cp 0 1 1
e 1 0 q 1' 1' a1 cp 0 1 1
e 1 0 1 q 1' cp 1 0 1 1
e 1 0 1 1 q cp 1 0 1 1
e 1 0 1 1 1 0 1 1