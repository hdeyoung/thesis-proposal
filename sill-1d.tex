% % arara: lualatex
% % arara: lualatex
% % arara: biber
% % arara: lualatex
% % arara: lualatex
% % \documentclass{../hdeyoung-proposal}
% \documentclass[
%   class=../hdeyoung-proposal,
%   crop=false
% ]{standalone}


% \usepackage{linear-logic}
% \usepackage{ordered-logic}
% \usepackage{proof}
% \usepackage{mathpartir}

% \usepackage{tikz}
% \usetikzlibrary{shapes.misc,graphs,quotes,graphdrawing}
% \usegdlibrary{trees}

% \usepackage{scalerel}

% \ExplSyntaxOn

% % \DeclarePairedDelimiter \parens { \lparen } { \rparen }
% \DeclarePairedDelimiter \braced:wn { \lbrace } { \rbrace }
% \NewDocumentCommand{ \braced }{ s o m o }
%   {
%     \IfBooleanTF {#1}
%       { \braced:wn* {#3} }
%       {
%         \IfValueTF {#2}
%           { \braced:wn[#2] {#3} }
%           { \braced:wn {#3} }
%       }
%     \IfValueT {#4} { \sb{#4} }
%   }

% \DeclarePairedDelimiter \bagged:wn { \lbag } { \rbag }
% \NewDocumentCommand{ \bagged }{ s o m o }
%   {
%     \IfBooleanTF {#1}
%       { \bagged:wn* {#3} }
%       {
%         \IfValueTF {#2}
%           { \bagged:wn[#2] {#3} }
%           { \bagged:wn {#3} }
%       }
%     \IfValueT {#4} { \sb{#4} }
%   }


% \NewDocumentCommand \oseq { >{ \SplitArgument{1}{|-} } m }
%   { \oseq:nn #1 }
% \cs_new:Npn \oseq:nn #1#2 { \oseq_ctxs:n {#1} \vdash #2 }
% \cs_new:Npn \oseq_ctxs:n #1 {
%   \seq_set_split:Nnn \l_tmpa_seq {;} {#1}
%   \seq_use:Nn \l_tmpa_seq { \mathrel{;} }
% }

% \NewDocumentCommand \procof { m m } { #1 \dblcolon #2 }
% \NewDocumentCommand \hypof { m } { #1 }


% \NewDocumentCommand \cut { m } { \text{\textsc{\MakeLowercase{Cut}}}\sb{#1} }
% \NewDocumentCommand \id { m } { \text{\textsc{\MakeLowercase{Id}}}\sb{#1} }

% \NewDocumentCommand \comp { >{ \SplitArgument{1}{|} } m }
%   { \comp:nn #1 }
% \cs_new:Npn \comp:nn #1#2 { #1 \parallel #2 }

% \NewDocumentCommand \fwd {} { \mathord{\leftrightarrow} }


% \RenewDocumentCommand \with { s }
%   { \IfBooleanTF {#1} \with:n \with: }
% \cs_new:Npn \with:n #1 {
%   \mathord{\binampersand}
%   \braced {
%     \seq_set_split:Nnn \l_tmpa_seq {,} {#1}
%     \seq_use:Nn \l_tmpa_seq {,}
%   }
% }
% \cs_new:Npn \with: { \mathbin{\binampersand} }

% \NewDocumentCommand \ssor { s }
%   { \IfBooleanTF {#1} \ssor:n \ssor: }
% \cs_new:Npn \ssor:n #1 {
%   \mathord{\ssor:}
%   \braced {
%     \seq_set_split:Nnn \l_tmpa_seq {,} {#1}
%     \seq_use:Nn \l_tmpa_seq {,}
%   }
% }
% \cs_new:Npn \ssor: { \oplus }

% \NewDocumentCommand \caseR { s m o }
%   {
%     \IfBooleanTF {#1}
%       {
%         \IfValueTF {#3}
%           { \case:nNnn { \mathsf{caseR} } \parens {#2} { \sb{#3} } }
%           { \case:nNnn { \mathsf{caseR} } \parens {#2} {} }
%       }
%       { \case:nNnn { \mathsf{caseR} } \parens {#2} {} }
%   }
% \NewDocumentCommand \caseL { s m o }
%   {
%     \IfBooleanTF {#1}
%       {
%         \IfValueTF {#3}
%           { \case:nNnn { \mathsf{caseL} } \parens {#2} { \sb{#3} } }
%           { \case:nNnn { \mathsf{caseL} } \parens {#2} {} }
%       }
%       { \case:nNnn { \mathsf{caseL} } \parens {#2} {} }
%   }
% \cs_new:Npn \case:nNnn #1#2#3#4 {
%   #1#4 \mskip\thinmuskip
%   #2 {
%     \seq_set_split:Nnn \l_tmpa_seq {|} {#3}
%     \seq_clear:N \l_tmpb_seq
%     \seq_map_inline:Nn \l_tmpa_seq
%       { \seq_put_right:Nn \l_tmpb_seq { \case_branch:n {##1} } }
%     \seq_use:Nn \l_tmpb_seq { \talloblong }
%   }
% }
% \cs_new:Npn \case_branch:n #1 { \case_branch_aux:w #1 \q_stop }
% \cs_new:Npn \case_branch_aux:w #1 => #2 \q_stop {
%   #1 \Rightarrow #2
% }

% \NewDocumentCommand \selectL { >{ \SplitArgument{1}{;} } m }
%   { \select:nnn { \mathsf{selectL} } #1 }
% \NewDocumentCommand \selectR { >{ \SplitArgument{1}{;} } m }
%   { \select:nnn { \mathsf{selectR} } #1 }
% \cs_new:Npn \select:nnn #1#2#3 {
%   \!\mathord{}\mathop{#1} #2 ; #3
% }


% \NewDocumentCommand \inj { m } { \mathsf{in}\sb{#1} }

% \NewDocumentCommand \inl {} { \inj{ \mathsf{1} } }
% \NewDocumentCommand \inr {} { \inj{ \mathsf{2} } }


% \RenewDocumentCommand \one {} { \mathord { \mathbf{1} } }

% \NewDocumentCommand \closeR {} { \mathsf{closeR} }
% \NewDocumentCommand \waitL { m } { \mathsf{waitL} ; #1 }


% \NewDocumentCommand \rrule { o m } {
%   \IfValueTF {#1}
%     { \rrule:nn {#2} {#1} }
%     { \rrule:n {#2} }
% }
% \cs_new:Npn \rrule:nn #1#2 { {#1}\text{\textsc{\MakeLowercase{R}}}\sb{#2} }
% \cs_new:Npn \rrule:n #1 { {#1}\text{\textsc{\MakeLowercase{R}}} }

% \NewDocumentCommand \lrule { o m } {
%   \IfValueTF {#1}
%     { \lrule:nn {#2} {#1} }
%     { \lrule:n {#2} }
% }
% \cs_new:Npn \lrule:nn #1#2 { {#1}\text{\textsc{\MakeLowercase{L}}}\sb{#2} }
% \cs_new:Npn \lrule:n #1 { {#1}\text{\textsc{\MakeLowercase{L}}} }


% \NewDocumentCommand \bnd { >{\SplitArgument{1}{=}}m } { \bnd:nn #1 }
% \cs_new:Npn \bnd:nn #1#2 { \mathsf{bnd} \mskip\thinmuskip #1 \mskip\thinmuskip #2 }
% \NewDocumentCommand \exec { } { \mathsf{exec} \mskip\thinmuskip }
% \NewDocumentCommand \msgL { } { \mathsf{msgL} \mskip\thinmuskip }
% \NewDocumentCommand \msgR { } { \mathsf{msgR} \mskip\thinmuskip }
% \NewDocumentCommand \msgQ { } { \mathsf{msgQ} }

% \ExplSyntaxOff




% \addbibresource{../proposal.bib}

% \NewDocumentCommand{\ie}{}{i.e.}


% \usepackage{listings}
% \crefname{listing}{listing}{listings}
% \Crefname{listing}{Listing}{Listings}

% \newlength{\mywidth}
% \settowidth{\mywidth}{\ttfamily A}
% \lstset{basicstyle=\ttfamily, basewidth=\mywidth}

% \captionsetup[lstlisting]{%
%   box=colorbox, boxcolor=gray,
%   font={normalfont, sf, color=white},
%   labelfont=bf,
%   justification=justified, singlelinecheck=false
% }

% \lstnewenvironment{sillcode}[1][]
%   {\lstset{language={},frame=bottomline,framerule=0.8ex,rulecolor=\color{gray},float,#1}}%
%   {}

% \lstnewenvironment{sillcode*}[1][]
%   {\lstset{language={},#1}}%
%   {}

% \NewDocumentCommand{\sillinline}{o}{%
%   \IfValueTF{#1}{\lstinline[#1]}{\lstinline}%
% }


% \NewDocumentCommand{\pctx}{}{\Psi}
% \ExplSyntaxOn
% \NewDocumentCommand{\ctxmonad}{>{\SplitArgument{1}{<-}}m}{
%   \{\use_ii:nn #1 \vdash \use_i:nn #1\}
% }
% \NewDocumentCommand \spawn { >{ \SplitArgument{1}{;} } m } { \spawn:nn #1 }
% \cs_new:Npn \spawn:nn #1#2 {
%   \mathsf{spawn}
%   \tl_if_empty:nF {#1} {
%     \mskip\thinmuskip #1 ; #2
%   }
% }
% \NewDocumentCommand{\mbind}{>{\SplitArgument{1}{;}}m}{
%   \use_i:nn#1 ; \use_ii:nn#1
% }
% \NewDocumentCommand{\mletrec}{o m m}{
%   \mathsf{letrec}\IfValueT{#1}{\sb{#1}}
%   \mskip\thinmuskip
%   #2
%   \mskip\thinmuskip
%   \mathsf{in} \mskip\thinmuskip #3
% }
% \NewDocumentCommand{\mprocdef}{m}{
%   #1
% }
% \ExplSyntaxOff


% \newcommand*\kay{k}



% \DeclareAcronym{BHK}{
  short = BHK,
  long  = Brouwer-Heyting-Kolmogorov
}

\DeclareAcronym{JILL}{
  short = JILL,
  long  = judgmental intuitionistic linear logic
}

\DeclareAcronym{ILL}{
  short = ILL,
  long  = intuitionistic linear logic
}

\DeclareAcronym{SML}{
  short = SML,
  long  = Standard ML
}

\DeclareAcronym{SOS}{
  short = SOS,
  short-indefinite = an,
  long = structural operational semantics
}

\DeclareAcronym{SSOS}{
  short = SSOS,
  short-indefinite = an,
  long = substructural operational semantics
}

\DeclareAcronym{SILL}{
  short = SILL,
  long = session-typed intuitionistic linear logic
}

\DeclareAcronym{CLF}{
  short = CLF,
  long = the concurrent logical framework
}



% \begin{document}

\section{Session-typed processes from singleton linear logic}\label{sec:sill}

Thus far, we have followed a proof-construction approach to computation, having in \cref{sec:ordered-lp} reviewed an interpretation of ordered logical specifications as concurrent string rewriting and in \cref{sec:choreographies} identified a fragment in which those specifications have a message-passing character.
In this \lcnamecref{sec:sill}, we turn to a proof-reduction view of computation.

Recently, \textcite{Caires+Pfenning:CONCUR10} with Toninho~\autocite*{Caires+:MSCS13} have established a Curry--Howard isomorphism, dubbed \acs{SILL}, between the sequent calculus for intuitionistic linear logic and a session-typed $\pi$-calculus, in which propositions are session types, proofs are session-typed processes, and cut reductions are process reductions.\footnote{\Textcite{Wadler:JFP14} later developed a correspondence between classical linear logic and a session-typed $\pi$-calculus, but \citeauthor{Caires+:MSCS13}'s intuitionistic correspondence turns out to be better suited to our goals here, for reasons we will explain shortly.}
This gives a proof-reduction view of concurrency that differs, apparently substantially, from the proof-construction perspective.
But, by the end of this proposal document, we will have shown that the differences are not as substantial as they first appear.

In this section, we present a reformulation of \citeauthor{Caires+:MSCS13}'s \ac{SILL} for a restriction of intuitionistic linear logic, which we call singleton linear logic, that is a better fit for comparisons with ordered logical specifications.
% As hinted below and further justified in \cref{?}, \ac{SISLL} is a better fit for comparisons with ordered logical specifications.

% [[
% We begin our presentation of \ac{SISLL} with a review of the session-typing judgements of \citeauthor{Caires+:MSCS13}'s \ac{SILL}.
% ]]

\subsection{Toward singleton linear logic}

In a session-based model of concurrency, pairs of processes interact in well-defined sessions, with one process offering a service that its session partner uses.
Session types, pioneered by \textcite{Honda:CONCUR93}, describe the interaction protocol to which a process adheres when offering its service.
% the processes in that session must adhere.
When processes interact, the session type changes: one process now offers, and the other uses, the continuation of the initial service.
As shown by \textcite{Caires+:MSCS13}, the logical reading of session-based concurrency is linear logic exactly because it can express this change of state.

Because a process offers its service along a distinguished channel, the basic session-typing judgment of \citeauthor{Caires+:MSCS13}'s \ac{SILL} is $P :: x{:}A$, meaning \enquote{process $P$ offers a service of session type $A$ along channel $x$}.
However, $P$ itself may rely on services offered by yet other processes, and so, more generally, the \ac{SILL} session-typing judgment is a linear sequent annotated as
\begin{equation*}
  \underbrace{
    x_1{:}A_1 , x_2{:}A_2 , \dots , x_n{:}A_n
  }_{\textstyle \lctx}
  \vdash
  P :: x{:}A
  \quad
  \text{($n \geq 0$)}
  \,,
\end{equation*}
% where the channel names, $x_i$, are needed to unambiguously refer to hypotheses and the consequent.
meaning \enquote{Using services $A_i$ offered along channels $x_i$, the process $P$ offers service $A$ along channel $x$.}
(The channels $x_i$ and $x$ must all be distinct and are binding occurrences with scope over the process $P$.)

In \ac{SILL}, the linear sequent calculus's inference rules thus become session-typing rules for processes.
Just as the inference rules arrange sequents into a proof tree, so do the \ac{SILL} session-typing rules arrange processes into a tree-shaped network in which some processes are clients of more than one process (i.e., some nodes have more than one child).
The following is one such example.
% The way in which \ac{SILL} processes use and offer services along distinct channels thus arranges processes in a tree-shaped network, such as the following, in which some processes are clients of more than one process (i.e., some nodes have more than one child).
\begin{equation*}
  \begin{tikzpicture}[channel/.style = {text depth=0, midway, sloped, above}]
    \graph [
      tree layout, grow=left, math nodes, % empty nodes,
      nodes={
        rounded rectangle, rounded rectangle left arc=none,
        draw, minimum size=3ex,
      },
      edges={-},
    ] {
      / [draw=none] <-["$\scriptstyle x$"' channel]
      P <- { / [> {"$\scriptstyle x_1$"' channel}] <- / <- { / , / } ,
             / [> {"$\scriptstyle x_2$"' channel}] ,
             / [> {"$\scriptstyle x_3$"' channel}] <- { / , / <- / } };
    };
  \end{tikzpicture}
\end{equation*}
% Each edge in the above \ac{SILL} tree represents a top-level $\cut{}$, and each top-level $\cut{}$ corresponds to a hypothesis that serves as its principal formula.

% With this session-typing judgment, the inference rules of the linear sequent calculus become \ac{SILL} session-typing rules for concurrent processes.
% For instance, the cut rule of \ac{SILL} types a parallel composition of processes (shown in $\pi$-calculus syntax here):
% \begin{equation*}
%   \infer[\cut{A}]{\lctx , \lctx' \vdash (\nu x)(P \mid Q) :: z{:}C}{
%     \lctx \vdash P :: x{:}A &
%     \lctx' , x{:}A \vdash Q :: z{:}C}
% \end{equation*}
% Top-level $\cut{}$s thus arrange processes in a tree-shaped network, such as the following, in which some processes are clients of more than one process (i.e., some nodes have more than one child).
% \begin{equation*}
%   % \begin{tikzpicture}
%   %   \graph [
%   %     tree layout, grow=left, math nodes, % empty nodes,
%   %     nodes={
%   %       rounded rectangle, rounded rectangle left arc=none,
%   %       draw, minimum size=2ex,
%   %     }
%   %   ] {
%   %     / [draw=none] <-
%   %     Q <- { "P_1" <- / <- { / , / } ,
%   %            / ,
%   %            / <- { / , / <- / } };
%   %   };
%   % \end{tikzpicture}
% \end{equation*}
% Each edge in the above \ac{SILL} tree represents a top-level $\cut{}$, and each top-level $\cut{}$ corresponds to a hypothesis that serves as its principal formula.
% % Each hypothesis gives rise to a top-level $\cut{}$, represented as an edge in the above graph. 

% Recall that the goal of this thesis proposal is to establish a connection between ordered logical specifications and 

In this thesis proposal, we are interested in a restriction of \ac{SILL} that will match concurrent ordered logical specifications.
To match the ordered context of these specifications, we cannot directly use \ac{SILL}---at least not in its full generality.
Instead, we need a restriction of \ac{SILL} in which process networks are chains, not arbitrary trees.


One might expect that process chains should arise from \emph{ordered} logic.
But, taking into account the way that the structure of the \ac{SILL} session-typing judgment induces a tree-shape for process networks, it becomes apparent that process chains, in fact, arise when processes are restricted to use at most one service each---that is, when contexts $\lctx$ are restricted to be either empty or singletons.
The session-typing judgment becomes
\begin{equation*}
  \lctxe \vdash P :: x{:}A
  \quad\text{or}\quad
  x_1{:}A_1 \vdash P :: x{:}A
\end{equation*}
and the process networks are necessarily chains:
\begin{equation*}
  \begin{tikzpicture}[channel/.style = {text depth=0, midway, sloped, above}]
    \graph [
      tree layout, grow=left, math nodes, % empty nodes,
      nodes={
        rounded rectangle, rounded rectangle left arc=none,
        draw, minimum size=3ex,
      },
      edges={-},
    ] {
      / [draw=none] <-["$\scriptstyle x$"' channel]
      P <-
      / [> {"$\scriptstyle x_1$"' channel}] <-
      / <-
      / ;
    };
  \end{tikzpicture}
\end{equation*}

Having made this restriction, we can simplify the judgments: if there are at most two channels, they can always be unambiguously named \enquote{left} and \enquote{right}, rather than bothering with fresh names like $x_1$ and $x$.
Moreover, since the channel names are now fixed by position (rather like de~Bruijn indices), we may as well omit them altogether from the session-typing judgments:
\begin{equation*}
  \lctxe \vdash P :: A
  \quad\text{or}\quad
  A_1 \vdash P :: A
  \,.
\end{equation*}

The following \lcnamecrefs{sec:cut-as-composition} describe this restriction of \ac{SILL}.
% It's worth emphasizing that, even in the presence of the restriction to singleton antecedents, we still have a Curry--Howard isomorphism: although we choose to present the logical rules and process assignment simultaneously, singleton linear logic is indeed a well-defined logic in its own right.
It's worth emphasizing that although we choose to present the logical rules and process assignment simultaneously, singleton linear logic is indeed a well-defined logic in its own right: even in the presence of the restriction to singleton antecedents, we still have a Curry--Howard isomorphism.

% Finally, it's worth emphasizing that we still have a Curry--Howard isomorphism in the presence of the restriction to singleton antecedents: although we choose to present the logical rules and process assignment simultaneously, singleton linear logic is indeed a well-defined logic in its own right.


% This is a Curry--Howard isomophism in that the same restriction---contexts must be empty or singletons---still yields a well-defined logic when applied to linear logic.

% In this note, we are interested in developing a restriction of \ac{SILL} in which the process networks are linearly ordered:
% \begin{center}
%   % \begin{tikzpicture}
%   %   \graph [tree layout, grow=left, empty nodes, nodes={draw, circle}] {
%   %     / [draw=none] <-
%   %     x <- y <- z <- w;
%   %   };
%   % \end{tikzpicture}
% \end{center}

% Therefore, for the process network to be linearly ordered, contexts $\lctx$ must be either singletons or empty.
% In this special case, sequents have one of the simpler forms
% \begin{equation*}
%   x_0{:}A_0 \vdash P :: x_1{:}A_1
%   \quad\text{or}\quad
%   \lctxe \vdash P :: x_1{:}A_1
%   \,.
% \end{equation*}


% % exec(spawn P; Q) -> {exec P * exec Q}

\subsection{Cut as composition}\label{sec:cut-as-composition}

% Recall the $\cut{}$ rule for intuitionistic linear logic:
% \begin{equation*}
%   \infer[{\cut{A}}]{\oseq{\lctx, \lctx' |- \comp{x_0}{P_0 | P} :: C}}{
%     \oseq{\lctx |- \procof{P_0}{x_0{:}A}} &
%     \oseq{\lctx', x_0{:}A |- \procof{P}{x{:}C}}}
% \end{equation*}
% When restricted to empty or singleton antecedents, the $\cut{}$ rule becomes
% \begin{equation*}
%   \infer[\cut{A}]{\oseq{\lctx |- \comp{x_0}{P_0 | P} :: C}}{
%     \oseq{\lctx |- \procof{P_0}{x_0{:}A}} &
%     \oseq{x_0{:}A |- \procof{P}{x{:}C}}}
% \end{equation*}

The cut rule of intuitionistic linear logic composes a plan for obtaining resource $A$ with another plan that uses resource $A$:
\begin{equation*}
  \infer{\oseq{\lctx, \lctx' |- C}}{
    \oseq{\lctx |- A} &
    \oseq{\lctx', A |- C}}
  \,.
\end{equation*}

When antecedents are restricted to be either empty or singletons, that rule becomes the cut rule for singleton linear logic;
it retains the character of a composition:
% The cut rule of (intuitionistic) singleton linear logic restricts the antecedent of each sequent to be either empty or a singleton, but still retains the character of a composition:
\begin{equation*}
  \infer[\cut{A}]{\lseq{\lctx |- C}}{
    \lseq{\lctx |- A} &
    \lseq{A |- C}}
  \,.
\end{equation*}
Because proofs are to be processes, this suggests that the process interpretation of the cut rule should compose a process that offers service $A$ with another process that uses service $A$.
The $\cut{A}$ rule thus becomes a typing rule for process composition:
\begin{equation*}
  \infer[\cut{A}]{\lseq{\lctx |- \procof{\spawn{P; Q}}{C}}}{
    \lseq{\lctx |- \procof{P}{A}} &
    \lseq{A |- \procof{Q}{C}}}
  \,,
\end{equation*}
where $\spawn{P; Q}$ means \enquote{Spawn process $P$ to the left, and then continue as process $Q$.}
% The syntax is reminiscent of do notation for monadic computations, with the channel c being bound within c <- spawnP <- ;Qc.

To complete the description of $\spawn{}$, we must make its operational semantics precise.
Rather than using \iacl*{SOS}, it is convenient to describe the semantics as an ordered logical specification~\autocites{Pfenning:APLAS04}{Pfenning+Simmons:LICS09}, in the style known as \iacf{SSOS}.
In our \ac{SSOS}, we will use the atomic proposition $\exec{P}$ to represent an executing process $P$.
The rule for executing a $\spawn{}$ is
\begin{equation*}
  \exec{(\spawn{P; Q})} \lrimp \monad{\exec{P} \fuse \exec{Q}}
  \,.
\end{equation*}
Thus, to execute the $\spawn{}$, we execute the processes $P$ and $Q$ side-by-side, with process $P$ offering a service that $Q$ uses.


\subsection{Additive conjunction as branching}\label{sec:addit-conj-as-branch}

So far we have discussed only cut, a judgmental principle that applies to all services\footnote{The other judgmental  principle, identity, is postponed to \cref{sec:ident-as-forw} in the interest of presenting only what is necessary for a first, simple example.}; specific services are defined by the right and left rules of the logical connectives.

The singleton linear sequent calculus rules for the additive conjunction $A_1 \with A_2$ are as follows.
Other than the restriction to singleton or empty antecedents, these rules are the same as those from linear logic.
% Note that in this case the left rules' contexts are forced to be the singleton $A_1 \with A_2$.
\begin{mathpar}
  \infer[\rrule{\with}]{\lseq{\lctx |- A_1 \with A_2}}{
    \lseq{\lctx |- A_1} &
    \lseq{\lctx |- A_2}}
  \and
  \infer[{\lrule[1]{\with}}]{\lseq{A_1 \with A_2 |- C}}{
    \lseq{A_1 |- C}}
  \and
  \infer[{\lrule[2]{\with}}]{\lseq{A_1 \with A_2 |- C}}{
    \lseq{A_2 |- C}}
\end{mathpar}
The right rule, $\rrule{\with}$, says that to prove $A_1 \with A_2$ we must prove both $A_1$ and $A_2$ (using the same resource $\lctx$) so that we are prepared for whichever of the two resources, $A_1$ or $A_2$, is eventually chosen by a $\lrule[1]{\with}$ or $\lrule[2]{\with}$ left rule.

Correspondingly, a process that offers service $A_1 \with A_2$ gives its client a choice of services $A_1$ and $A_2$; the process must be prepared to offer whichever service the client chooses.
Based on this intuition, we interpret the $\rrule{\with}$ rule as typing a binary guarded choice:
\begin{equation*}
  \infer[\rrule{\with}]{\lseq{\lctx |- \procof{\caseR{\inl => P_1 | \inr => P_2}}{A_1 \with A_2}}}{
    \lseq{\lctx |- \procof{P_1}{A_1}} &
    \lseq{\lctx |- \procof{P_2}{A_2}}}
\end{equation*}
where $\caseR{\inl => P_1 | \inr => P_2}$ means \enquote{Input either $\inl$ or $\inr$ along the right-hand channel, and then continue as process $P_1$ or $P_2$, respectively.}

Conversely, the client sitting to the right that uses service $A_1 \with A_2$ must behave in a complementary way: the client should select either service $A_1$ or service $A_2$ and then, having notified the offering process of its choice (as $\inl$ or $\inr$), continue the session by using that service.
The left rules for type $A_1 \with A_2$ are thus:
\begin{mathpar}
  \infer[{\lrule[1]{\with}}]{\lseq{\hypof{A_1 \with A_2} |- \procof{\selectL{\inl; Q}}{C}}}{
    \lseq{\hypof{A_1} |- \procof{Q}{C}}}
  \and
  \infer[{\lrule[2]{\with}}]{\lseq{\hypof{A_1 \with A_2} |- \procof{\selectL{\inr; Q}}{C}}}{
    \lseq{\hypof{A_2} |- \procof{Q}{C}}}
\end{mathpar}
where $\selectL{\inj{\mathsf{1/2}} ; Q}$ means \enquote{Send label $\inj{\mathsf{1/2}}$ along the left-hand channel and then continue as process $Q$.}

Our intuition about the behavior of the guarded choice processes is made precise by their operational semantics.
First, the $\inl$ branch:
\begin{equation*}
  \begin{lgathered}
    \exec(\selectL{\inl ; Q}) \lrimp \monad{\msgL{\inl} \fuse \exec{Q}} \\
    \exec(\caseR{\inl => P_1 | \inr => P_2}) \fuse \msgL{\inl} \lrimp \monad{\exec{P_1}}
      \,.
  \end{lgathered}
\end{equation*}
To execute the selection process $\selectL{\inl ; Q}$, we asynchronously send to the left a message containing the label $\inl$, which is represented in the \ac{SSOS} as the proposition $\msgL{\inl}$, and then immediately continue the session by executing process $Q$.
When this message arrives, the destination process $\caseR{\inl => P_1 | \inr => P_2}$ resumes execution as $P_1$.

The operational semantics of the $\inr$ branch is symmetric to that of the $\inl$ branch:
\begin{equation*}
  \begin{lgathered}
    \exec(\selectL{\inr ; Q}) \lrimp \monad{\msgL{\inr} \fuse \exec{Q}} \\
    \exec(\caseR{\inl => P_1 | \inr => P_2}) \fuse \msgL{\inr} \lrimp \monad{\exec{P_2}}
      \,.
  \end{lgathered}
\end{equation*}

\paragraph{Practical considerations.}

% To make the language more palatable for the programmer, we diverge slightly from a pure propositions-as-types interpretation of one-dimensional linear logic by including $n$-ary labeled additive conjunctions $\with*{\ell_i: A_i}[i \in I]$ as a primitive.
% Formaly, the conjunction is over a mutiset of label-type pairs,
% % expressed as a parametric comprehension
% indexed by set $I$
% \autocite{Cervesato+Sans:FI14}

To make the language more palatable for the programmer, we diverge slightly from a pure propositions-as-types interpretation of singleton linear logic by including $n$-ary labeled additive conjunctions, $\with*{\ell: A_\ell}[\ell \in L]$, as a primitive.
% Formally, these types $\with*{\ell_i: A_i}[i \in I]$ are conjunctions over multiset comprehensions~\autocite{Cervesato+Sans:FI14} of label--type pairs.

By analogy with the binary conjunction, the typing rules and operational semantics for $\with*{\ell: A_\ell}[\ell \in L]$ are as follows.
Notice that the $\rrule{\with}$ has a set of premises, one for each label $\ell \in L$.
\begin{gather*}
  \infer[\rrule{\with}]{\lseq{\lctx |- \procof{\caseR[\ell \in L]{\ell => P_\ell}}{\with*{\ell: A_\ell}[\ell \in L]}}}{
    \forall \ell \in L \mathpunct{:}\enskip \lseq{\lctx |- \procof{P_\ell}{A_\ell}}}
  \qquad
  % \raisebox{0.5\baselineskip}{$\bagged{\,\raisebox{-0.5\baselineskip}{
  % \scaleleftright[0.4em]{\lbag}{
    \infer[{\lrule{\with}}]{\lseq{\hypof{\with*{\ell: A_\ell}[\ell \in L]} |- \procof{\selectL{\kay; Q}}{C}}}{
      \lseq{\hypof{A_{\kay}} |- \procof{Q}{C}} &
      \text{($\kay \in L$)}}
  % }{\rbag}_{k \in I}
  % }}[k \in I]$}
  \\[2\jot]
  % \bagged{\,
  % \scaleleftright[0.4em]{\lbag}{
    \begin{lgathered}
      \exec{(\selectL{\kay ; Q})} \lrimp \monad{\msgL{\kay} \fuse \exec{Q}} \\
      \exec{(\caseR[\ell \in L]{\ell => P_\ell})} \fuse \msgL{\kay} \lrimp \monad{\exec{P_{\kay}}} \mathrlap{\qquad\text{($\kay \in L$)}}
    \end{lgathered}
  % }{\rbag}_{k \in I}
  % \,}[k \in I]
\end{gather*}
% According to the multiset nature of the type, our presentation uses the multiset-oriented inference rules of \textcite{Cervesato+Sans:FI14}: the $\rrule{\with}$ right rule has a multiset of premises---one for each index $i \in I$---and there are multisets of $\lrule[k]{\with}$ left rules and \ac{SSOS} rules.

Another possibility would be to simply treat $n$-ary labeled conjunctions as syntactic sugar for nested binary conjunctions, but this would
% turn out to
introduce a communication overhead because we would be sending multiple $\inj{\mathsf{1/2}}$s separately rather than a single label $\kay$.


\subsection{Recursive session types and process definitions}\label{sec:recurs-sess-types}

Concurrent processes frequently exhibit unbounded or infinite, yet well-defined, behavior;
for instance, we may wish to have a counter that offers an increment service indefinitely.
\Textcite{Toninho+:TGC14} have proposed an extension of their \ac{SILL} type theory that incorporates inductive and coinductive session types.
However, to keep matters simpler, we instead rely on general recursion here and choose to be content with the departure from a pure Curry--Howard isomorphism.

Session types thus include general recursive types, $\mu t.A$, and type variables, $t$.
The type $\mu t.A$ is interpreted equi-recursively, being identified with the unfolding $\subst{(\mu t.A)/t}{A}$.
Processes correspondingly include mutually recursive process definitions, via \sillinline`letrec`, and process variables, $X$.

We extend the session-typing judgment with a context, $\pctx$, of process-variable typings.
% Because a process is typed according to the services that it uses and offers, process variables are typed as $X : \ctxmonad{A <- \vec{B}}$, meaning that process $X$ can offer service $A$ if provided with channels along which services $\vec{B}$ are offered.
Because a process is typed according with a sequent of services that it uses and offers, process variables are typed as ${X : \ctxmonad{A <- \lctx}}$ if process $X$ can offer service $A$ 
% when provided with channels along which services $\vec{B}$ are offered.
by using services $\lctx$.
When channels of appropriate types are available, the process $X$ can be called:
\begin{equation*}
  \infer[\text{\scshape call}]{\lseq{\pctx, X{:}\ctxmonad{A <- \lctx} ; \lctx |- \procof{X}{A}}}{
    }
  \:.
\end{equation*}
A common idiom is $\spawn{X ; Q}$, which spawns a call to $X$ that is run in parallel with some process $Q$.
We will frequently abbreviate this with the syntactic sugar $\mbind{X ; Q}$.

% Mutually recursive process definitions add process variables to the context.
% Process variables are added to the context to allow for mutual recursion.
In mutually recursive process definitions,
% , the programmer declares each process with a type that must be checked.
the process bodies may refer to any of the mutually recursive processes via process variables.
The typing rule is
\begin{equation*}
  \infer[\text{\scshape letrec}]
  {\lseq{\pctx ; \lctx |- \procof{\mletrec[X \in \mathcal{X}]{(\mprocdef{X{:}\ctxmonad{A_X <- \lctx_X} = P_X})}{Q}}{C}}}{
    \text{($\pctx' = \braced{X{:}\ctxmonad{A_X <- \lctx_X}}[X \in \mathcal{X}]$)}
    &
    % \bagged{
    \forall X \in \mathcal{X}\mathpunct{:}\enskip
      \lseq{\pctx, \pctx' ; \lctx_X |- \procof{P_X}{A_X}}
    % }[j \in I]
    &
    \lseq{\pctx, \pctx' ; \lctx |- \procof{Q}{C}}}
  \,.
\end{equation*}

The operational semantics of these constructs are as follows:
\begin{equation*}
  \begin{lgathered}
    \exec{(\mletrec[X \in \mathcal{X}]{(\mprocdef{X = P_X})}{Q})} \lrimp \monad{\braced{\bang \bnd{X = P_X}}[X \in \mathcal{X}] \fuse \exec{Q}}
    \\
    \bang \bnd{X = P} \fuse \exec{X} \lrimp \monad{\exec{P}}
  \end{lgathered}
\end{equation*}
The environment of bindings of process variables to process expressions is represented in the \ac{SSOS} as a collection of $\bnd{X = P}$ hypotheses.
To execute a group of mutually recursive process definitions, $\mletrec[X \in \mathcal{X}]{(\mprocdef{X = P_X})}{Q}$, bindings are introduced for each of the process variables and then the body $Q$ is executed.
To execute a free process variable $X$, instead execute the process expression to which $X$ is bound.


%
Now, having presented recursion, we can finally give a simple example program.


\subsection{Example: Binary counter}\label{sec:exampl-binary-count}

We can implement a simple session-typed counter on natural numbers as shown in \cref{lst:counter-inc}.%
\footnote{This example is adapted from one by \textcite{Toninho+:ESOP13}.}
%
\begin{sillcode}[
  caption={A simple binary counter supporting an increment operation},
  label={lst:counter-inc},
  floatplacement=tb,
  gobble=2
]
  stype Cntr = &{ inc: Cntr }
  
  eps : { |- Cntr } =
  { caseR of
      inc => eps; bit1 }
  
  bit0 : { Cntr |- Cntr } =
  { caseR of
      inc => bit1 }
  
  bit1 : { Cntr |- Cntr } =
  { caseR of
      inc => selectL inc; bit0 }
\end{sillcode}
%
The counter is a chain of \sillinline`bit0` and \sillinline`bit1` processes, one for each bit in the binary representation of the counter's value, and is terminated at the most significant end with an \sillinline`eps` process.
For instance, the process chain \sillinline`eps; bit1; bit0` represents a counter with value $2$.

The counter offers a very simple service: the client may only choose to increment the counter, with the same service being offered recursively after the increment.
This service, \sillinline`Cntr`, is therefore a recursive (unary) additive conjunction, declared in the concrete syntax as \sillinline`stype Cntr = &{ inc: Cntr }`.
The \sillinline`eps` process offers this service outright, and thus has type \sillinline`{ |- Cntr }`.
The \sillinline`bit0` and \sillinline`bit1` processes, on the other hand, use the service offered by their more significant neighbors, and thus have type \sillinline`{ Cntr |- Cntr }`.

The process definitions of \sillinline`eps`, \sillinline`bit0`, and \sillinline`bit1` are mutually recursive.
When an \sillinline`eps` process receives an \sillinline`inc` message, it creates a new most significant bit by spawning a new \sillinline`eps` process and then making a recursive call to a \sillinline`bit1` process.
% that uses the service offered by the new \sillinline`eps`.
When a \sillinline`bit0` process receives an \sillinline`inc`, the bit is flipped by virtue of a recursive call to a \sillinline`bit1` process.
Lastly, when a \sillinline`bit1` process receives an \sillinline`inc`, the bit is flipped and a carry is propagated; this is accomplished by first sending \sillinline`inc` along the left to the \sillinline`Cntr` offered by the next more significant bit and then making a recursive call to a \sillinline`bit0` process.

Informally, we can see that, as implemented, the \sillinline`inc` operation respects a counter's denotation: whenever a counter representing natural number $N$ is incremented, the resulting counter represents $N+1$.
Note, however, that this adequacy property is not enforced by the type \sillinline`Cntr`.
An appropriate dependent session type could enforce increment adequacy, but, for simplicity of exposition, we prefer the simple type here.

\subsection{Additive disjunction as choice}\label{sec:addit-disj-as}

In the singleton linear sequent calculus, additive disjunction, $A \ssor B$, is dual to additive conjunction, $A \with B$.
\begin{mathpar}
  \infer[{\rrule[1]{\ssor}}]{\lseq{\lctx |- A_1 \ssor A_2}}{
    \lseq{\lctx |- A_1}}
  \and
  \infer[{\rrule[2]{\ssor}}]{\lseq{\lctx |- A_1 \ssor A_2}}{
    \lseq{\lctx |- A_2}}
  \and
  \infer[\lrule{\ssor}]{\lseq{A_1 \ssor A_2 |- C}}{
    \lseq{A_1 |- C} &
    \lseq{A_2 |- C}}
\end{mathpar}

We should expect this duality to also appear in the process assignment.
Whereas a process of type $A \with B$ offers its client at the right a choice of services $A$ and $B$, a process of type $A \ssor B$ chooses between offering service $A$ or service $B$ to its client at the right.
The client waits to be notified of the offering process's choice and then uses that service.
\begin{mathpar}
  \infer[{\rrule[1]{\ssor}}]{\lseq{\lctx |- \procof{\selectR{\inl; P}}{A_1 \ssor A_2}}}{
    \lseq{\lctx |- \procof{P}{A_1}}}
  \and
  \infer[{\rrule[2]{\ssor}}]{\lseq{\lctx |- \procof{\selectR{\inr; P}}{A_1 \ssor A_2}}}{
    \lseq{\lctx |- \procof{P}{A_2}}}
  \and
  \infer[\lrule{\ssor}]{\lseq{A_1 \ssor A_2 |- \procof{\caseL{\inl => Q_1 | \inr => Q_2}}{C}}}{
    \lseq{A_1 |- \procof{Q_1}{C}} &
    \lseq{A_2 |- \procof{Q_2}{C}}}
\end{mathpar}
Confirming the intuition that $\selectR{\inj{\mathsf{1/2}}; P}$ sends along the right-hand channel and $\caseL{\inl => Q_1 | \inr => Q_2}$ receives along the left-hand channel are the \ac{SSOS} rules:
\begin{equation*}
  \begin{lgathered}
    \exec{(\selectR{\inl ; P})} \lrimp \monad{\exec{P} \fuse \msgR{\inl}} \\
    \msgR{\inl} \fuse \exec{(\caseL{\inl => Q_1 | \inr => Q_2})} \lrimp \monad{\exec{Q_1}}
    %
    \\[\jot]
    %
    \exec{(\selectR{\inr ; P})} \lrimp \monad{\exec{P} \fuse \msgR{\inr}} \\
    \msgR{\inr} \fuse \exec{(\caseL{\inl => Q_1 | \inr => Q_2})} \lrimp \monad{\exec{Q_2}}
      \,.
  \end{lgathered}
\end{equation*}
Because the operational semantics is asynchronous, it's important to distinguish the $\msgR{}$ predicate, which represents messages that flow to the right, from the $\msgL{}$ predicate, which represents messages that flow to the left.
Otherwise, a selection process's continuation could mistakenly capture the message that was just sent, as might happen in executing $\selectR{\inl ; \caseR{\inl => P_1 | \inr => P_2}}$, for example.

Once again, to make the language more convenient for the programmer, we include $n$-ary labeled additive disjunctions $\ssor*{\ell: A_\ell}[\ell \in L]$.
The typing rules and operational semantics are thus more generally
\begin{gather*}
  % \bagged{
    \infer[{\rrule{\ssor}}]{\lseq{\lctx |- \procof{\selectR{\kay; P}}{\ssor*{\ell: A_\ell}[\ell \in L]}}}{
      \lseq{\lctx |- \procof{P}{A_{\kay}}} &
      \text{($\kay \in L$)}}
%   }[k \in I]
  \qquad
  \infer[\lrule{\ssor}]{\lseq{\ssor*{\ell: A_\ell}[\ell \in L] |- \procof{\caseL[\ell \in L]{\ell => Q_\ell}}{C}}}{
    \forall \ell \in L\mathpunct{:}\enskip \lseq{A_\ell |- \procof{Q_\ell}{C}}}
  \\[2\jot]
  % \bagged{
    \begin{lgathered}
      \exec{(\selectR{\kay; P})} \lrimp \monad{\exec{P} \fuse \msgR{\kay}} \\
      \msgR{\kay} \fuse \exec{(\caseL[\ell \in L]{\ell => Q_\ell})} \lrimp \monad{\exec{Q_{\kay}}} \mathrlap{\qquad\text{($\kay \in L$)}}
    \end{lgathered}
  % }[k \in I]
\end{gather*}


\subsection{Example: Binary counter with decrements}\label{sec:exampl-binary-count-1}

\begin{sillcode}[
  caption={A binary counter supporting increments and decrements},
  label={lst:counter-dec},
  floatplacement=tb,
  gobble=2
]
  stype Cntr = &{ inc: Cntr , dec: Cntr' }
    and Cntr' = +{ ok: Cntr , fail: Cntr }

  eps : { |- Cntr } =
  { caseR of
      inc => eps; bit1
    | dec => selectR fail; eps }
  
  bit0 : { Cntr |- Cntr } =
  { caseR of
      inc => bit1
    | dec => selectL dec; bit0' }
  
  bit0' : { Cntr' |- Cntr' } =
  { caseL of
      ok => selectR ok; bit1
    | fail => selectR fail; bit0 }

  bit1 : { Cntr |- Cntr } =
  { caseR of
      inc => selectL inc; bit0
    | dec => selectR ok; bit1 }
\end{sillcode}

\Cref{lst:counter-dec} shows a counter that takes advantage of additive disjunction to support a truncated decrement operation. 
According to the type declaration, a process offering the \sillinline`Cntr` service gives its client a choice of increment or decrement services.
If the client chooses to decrement, the offering process will choose to reply with either \sillinline`fail` or \sillinline`ok` and then recursively offer the \sillinline`Cntr` service.

As implemented, decrementing the counter gives \sillinline`fail` and leaves the process network unchanged if the counter represents $0$; if it represents some $N > 0$, then decrementing the counter gives \sillinline`ok` after decrementing to $N - 1$.
Once again, these adequacy properties are not enforced by the type \sillinline`Cntr`, although they could be with an appropriate dependent session type.


\subsection{Identity as forwarding}\label{sec:ident-as-forw}

In singleton linear logic, in addition to the cut principle, there is an identity principle that states that one way to obtain a resource is to directly use an existing resource:
\begin{equation*}
  \infer[\id{A}]{\lseq{A |- A}}{
    }
  \,.
\end{equation*}
Under the process interpretation, a process can offer service $A$ by acting as a forwarding intermediary between its clients and another process that offers service $A$.
The $\id{A}$ rule thus types a forwarding process between two channels:
\begin{equation*}
  \infer[\id{A}]{\lseq{A |- \procof{\fwd}{A}}}{
    }
  \,.
\end{equation*}
Rather than making the forwarding explicit in the operational semantics, we can simply eliminate the middleman, adjoining the neighboring processes:
\begin{equation*}
  \exec{(\fwd)} \lrimp \monad{\one}
  \,.
\end{equation*}

% \begin{sillcode}[
%   caption={A binary counter as a stream transformer},
%   label={lst:counter-bit},
%   floatplacement=tb,
%   gobble=2
% ]
%   stype Bin = +{ eps: 1 , bit0: Bin , bit1: Bin }
  
%   inc : {Bin |- Bin} =
%   { caseL of
%       eps => selectR bit1; selectR eps; <->
%     | bit0 => selectR bit1; <->
%     | bit1 => selectR bit0; inc }
% \end{sillcode}



\subsection{Other session types}\label{sec:other-session-types}

In addition to the those already mentioned, the other connectives of singleton linear logic correspond to session types.

\paragraph{Multiplicative unit.}
Like its \ac{SILL} cousin, the multiplicative unit $\one$ is the service that terminates without any interaction.
Its right rule, $\rrule{\one}$, types a process that immediately terminates; its left rule, $\lrule{\one}$, types a process that waits for the left-hand side to terminate:
\begin{mathpar}
  \infer[\rrule{\one}]{\lseq{\lctxe |- \procof{\closeR}{\one}}}{
    }
  \and
  \infer[\lrule{\one}]{\lseq{\hypof{\one} |- \procof{\waitL{Q}}{C}}}{
    \lseq{\lctxe |- \procof{Q}{C}}}
\end{mathpar}
The operational semantics is asynchronous, with the $\closeR$ process sending a quit message, $\msgQ$:
\begin{equation*}
  \begin{lgathered}
    \exec{\closeR} \lrimp \monad{\msgQ} \\
    \msgQ \fuse \exec{(\waitL{Q})} \lrimp \monad{\exec{Q}}
  \end{lgathered}
\end{equation*}


\paragraph{First-order universal and existential quantification.}
The first-order quantifiers type processes that exchange functional values.
A process offering service $\forall x{:}\tau. A_x$ (or using service $\exists x{:}\tau. A_x$) first inputs a value $x$ of functional type $\tau$ and then offers (resp., uses) service $A_x$.
Dually, a process offering service $\exists x{:}\tau. A_x$ (or using service $\forall x{:}\tau. A_x$) asynchronously outputs the value of some functional term $M$ of type $\tau$ and then offers (resp., uses) service $\subst{M/x}{A_x}$. 
The first-order quantifiers are thus dependent session types; in the non-dependent case, we write the types as $\tau \vimp A$ and $A \vand \tau$.
% The syntax for value inputs and outputs is $\recv x <- inputc;Pxand outputc M;Q.

Since value inputs and outputs are not critical to the remainder of this proposal, the reader who is interested in further details of their static and dynamic semantics in \ac{SILL} should refer to the papers by \textcites{Toninho+:ESOP13}{Toninho+:PPDP11}; we leave the extrapolation to singleton linear logic as an exercise for the reader.

\paragraph{Multiplicative conjunction and linear implication.}
The restriction to sequents with singleton antecedents proves fatal to attempts to include multiplicative conjunction ($A \tensor B$) and linear implication ($A \lolli B$) as connectives in singleton linear logic.
For multiplicative conjunction, the left rule is problematic because it breaks down one hypothesis into two; for linear implication, the right rule is problematic because it introduces a new hypothesis even if one is already there.
Fortunately, for the examples in which we are interested, the absence of $\tensor$ and $\lolli$ is not an issue.

% \end{document}
