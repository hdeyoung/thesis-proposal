\subsection{What counts as a choreography?}\label{sec:what-counts-choreo}

\begin{align*}
  &\inc \lrimp \inc[<-] \\
  &\eps \lrimp (\inc[<-] \rimp \eps \fuse \bit{1}) \\
  &\bit{0} \lrimp (\inc[<-] \rimp \bit{1}) \\
  &\bit{1} \lrimp (\inc[<-] \rimp \inc \fuse \bit{0})
\end{align*}

\begin{align*}
  &\eps \lrimp \eps[->] \\
  &\bit{0} \lrimp \bit{0}[->] \\
  &\bit{1} \lrimp \bit{1}[->] \\
  &\inc \lrimp \parens[auto, align=c@{\,}l]{
                     & (\eps[->] \limp \eps \fuse \bit{1}) \\[2pt]
               \with & (\bit{0}[->] \limp \bit{1}) \\[2pt]
               \with & (\bit{1}[->] \limp \inc \fuse \bit{0})}
\end{align*}

\NewDocumentCommand{\fch}{o m}{\IfValueTF{#1}{\monad[#1]}{\monad}{#2}}

\begin{align*}
  &\elem{M} \lrimp \elem[->]{M} \\
  &\elem{N} \lrimp (\elem[->]{M} \limp ((M > N) \uimp \fch{\elem{N} \fuse \elem{M}}))
\end{align*}

\begin{itemize}
\item Grammar of choreographies.
\item Each choreography rule must be able to fire independently of the other processes (although it may depend on messages).
\end{itemize}

%%% Local Variables:
%%% TeX-master: "proposal"
%%% End:
