\subsection{Linear logic}\label{sec:linear-logic}

\begin{figure}
  \begin{mathpar}
    \infer[\lab{id}]{\uctx ; P \seq P}{
      }
    \and
    \infer[\lab{copy}]{\uctx, A ; \lctx' \seq J}{
      \uctx, A ; \lctx', A \seq J}
    \\
    \infer[\rlab{{\lolli}}]{\uctx ; \lctx \seq A \lolli B}{
      \uctx ; \lctx, A \seq B}
    \and
    \infer[\llab{{\lolli}}]{\uctx ; \lctx'_1, \lctx'_2, A \lolli B \seq J}{
      \uctx ; \lctx'_1 \seq A &
      \uctx ; \lctx'_2, B \seq J}
    \\
    \infer[\rlab{{\with}}]{\uctx ; \lctx \seq A \with B}{
      \uctx ; \lctx \seq A &
      \uctx ; \lctx \seq B}
    \and
    \infer[{\llab{{\with}}[1]}]{\uctx ; \lctx', A \with B \seq J}{
      \uctx ; \lctx', A \seq J}
    \and
    \infer[{\llab{{\with}}[2]}]{\uctx ; \lctx', A \with B \seq J}{
      \uctx ; \lctx', B \seq J}
    \\
    \infer[\rlab{{\tensor}}]{\uctx ; \lctx_1, \lctx_2 \seq A \tensor B}{
      \uctx ; \lctx_1 \seq A &
      \uctx ; \lctx_2 \seq B}
    \and
    \infer[\llab{{\tensor}}]{\uctx ; \lctx', A \tensor B \seq J}{
      \uctx ; \lctx', A, B \seq J}
    \\
    \infer[\rlab{\one}]{\uctx ; \lctxe \seq \one}{
      }
    \and
    \infer[\llab{\one}]{\uctx ; \lctx', \one \seq J}{
      \uctx ; \lctx' \seq J}
    \\
    \infer[\rlab{\bang}]{\uctx ; \lctxe \seq \bang A}{
      \uctx ; \lctxe \seq A}
    \and
    \infer[\llab{\bang}]{\uctx ; \lctx', \bang A \seq J}{
      \uctx, A ; \lctx' \seq J}
  \end{mathpar}
  \begin{mathpar}
    \infer-[\lab{id$_A$}]{\uctx ; A \seq A}{
      }
    \and
    \infer-[\lab{cut$_A$}]{\uctx ; \lctx, \lctx' \seq J}{
      \uctx ; \lctx \seq A &
      \uctx ; \lctx', A \seq J}
    \and
    \infer-[\lab{cut$^{\bang}_A$}]{\uctx ; \lctx' \seq J}{
      \uctx ; \lctxe \seq A &
      \uctx, A ; \lctx' \seq J}
  \end{mathpar}
  \caption{Sequent calculus rules for \acf{JILL}~\autocite{Chang+:CMU03}.}
\end{figure}

\begin{itemize}
\item Should I give a brief introduction to linear logic here?  Or, is this overkill for committee members?
\item Present ordered logic in its own right here, or present ordered logic only in the context of bottom-up logic programming?
\end{itemize}

\subsection{Proof search as computation: Bottom-up linear logic programming}\label{sec:linear-lp}

\begin{itemize}
\item Notion of transition obtained by reading sequent calculus left rules  (bipoles) bottom-up.
\end{itemize}

\subsubsection{Bottom-up ordered logic programming}\label{sec:ordered-lp}

\begin{itemize}
\item Example: binary counter with increment.
\end{itemize}

\subsection{Proof reduction as computation: Session-typed linear logic}\label{sec:async-sill}

\begin{itemize}
\item Present monadic language only?  Disadvantage is that proof reduction is not quite as clear or simple to see since it involves run-time typing.
\end{itemize}

%%% Local Variables:
%%% TeX-master: "proposal"
%%% End:
