% arara: pdflatex: { shell: yes }
% arara: bibtex
% arara: pdflatex: { shell: yes }
% arara: pdflatex: { shell: yes }
\documentclass{article}

\usepackage{fixltx2e}

\usepackage[T1]{fontenc}
\usepackage[sc,osf]{mathpazo}
\linespread{1.05}
\usepackage{microtype}

\usepackage{cleveref}

\usepackage[
  backend=bibtex,
  style=authoryear-comp,maxcitenames=2%
]{biblatex}
\addbibresource{proposal.bib}

\usepackage[single]{acro}
\DeclareAcronym{JILL}{short = JILL, long = judgmental intuitionistic linear logic}
\DeclareAcronym{SML}{short = SML, long = Standard ML}

\usepackage{logic}
\usepackage{predicates}

\usepackage{verbments}
% \NewDocumentEnvironment{smlcode}{}{\begin{pyglist}[language=sml,gobble=2]}{\end{pyglist}}
\usepackage{listings}
\newlength{\mywidth}
\settowidth{\mywidth}{\ttfamily A}
\lstset{basicstyle=\ttfamily, basewidth=\mywidth}
\NewDocumentCommand{\sml}{o}{%
  \IfValueTF{#1}{\lstinline[#1]}{\lstinline}%
}

\NewDocumentCommand{\vocab}{m}{\emph{#1}}
\NewDocumentCommand{\wc}{m o}{%
  #1\IfValueT{#2}{ [#2]}\textsuperscript{?}%
}

\NewDocumentCommand{\rulename}{m}{\text{\texttt{\textup{#1}}}}

\begin{document}

\title{Untitled Thesis Proposal}
\author{Henry DeYoung\\\texttt{hdeyoung@cs.cmu.edu}}
\date{\today}
\maketitle

\begin{abstract}
  Despite both being rooted in proof theory, logic programming and functional programming each have distinct advantages and disadvantages.
  Logic programs are remarkably concise and expressive, whereas well-typed functional programs provide guarantees about their behavior.
  This thesis proposes to identify a class of (bottom-up) logic programs that do, in fact, enjoy the same kinds of behavioral guarantees as functional programs, thereby affording programmers the best of both worlds.
\end{abstract}

\section{Introduction}\label{sec:introduction}

% arara: pdflatex
% arara: pdflatex
% arara: biber
% arara: pdflatex
% arara: pdflatex
\documentclass[
  class=../hdeyoung-proposal,
  crop=false
]{standalone}

\usepackage[subpreambles]{standalone}

\addbibresource{../proposal.bib}

\usepackage{pifont}
\usepackage{tikz}
\usetikzlibrary{matrix,quotes,graphs}
\usepackage{ordered-logic}
\usepackage{binary-counter}

\DeclareAcronym{CLF}{
  short = CLF,
  long = Concurrent Logical Framework
}

\begin{document}

\section{Introduction}\label{sec:introduction}

With the increasingly complex, distributed nature of today's software systems, concurrency is ubiquitous.
% % With the ever-increasing complexity of today's software systems, concurrency is ubiquitous.
% Concurrency structures systems as nondeterministic compositions of simpler subsystems [components].
Concurrency facilitates distributed computation by structuring systems as nondeterministic compositions of simpler subsystems [components].
% Concurrency attenuates [helps to manage] complexity by structuring systems as nondeterministic compositions of simpler subsystems [components].
% Concurrent systems are structured as nondeterministic compositions of simpler subsystems [components].
But, concomitant with nondeterminism, concurrent systems are notoriously tricky to get right:
subtle races and deadlocks can occur
% even in systems subjected to the most rigorous testing.
even in the most rigorously tested of systems.

% With the ever-increasing complexity and distribution of software systems,
% concurrency has become a pervasive method for structuring computations.
% But, like mutable state, concurrency is notoriously tricky to get right.
% % apply correctly.
% ...

At the same time,
% Decades of research into mathematical logics [proof theory] and programming languages have firmly established the power of deductive computation to ensure programs' clarity and correctness.
decades of research into connections between proof theory and programming languages have firmly established the 
% computation-as-deduction framework as the gold standard for improving programs' clarity and expressiveness and ensuring their correctness.
principle of \vocab{computation as deduction} as the gold standard
% for improving programs' clarity and expressiveness and ensuring their correctness.
framework for clear, expressive, and provably correct programs.
Examples abound: lax logic for monadic computation~\autocite{Benton:JFP98};
% S5 modal logic for distributed computation~\autocites{Murphy:CMU08}{Jia+Walker:ESOP04};
temporal logic for functional reactive programming~\autocite{Jeffrey:PLPV12}; linear logic for graph-based algorithms~\autocite{Cruz+:ICLP14}; etc.

% Can a computation-as-deduction approach make it similarly easier to write clear, expressive, provably correct concurrent programs?
Can a computation-as-deduction approach make it similarly easier to clearly and concisely specify, as well as correctly implement, concurrent programs?
% improve the clarity and expressiveness and ensure the correctness of concurrent computations?

\vspace{0.25\baselineskip}
\noindent \hspace*{\fill}\scalebox{0.75}{\color{black!50}\ding{70}}\hspace*{\fill}
\vspace{0.25\baselineskip}

\noindent
Computation-as-deduction
% principle
comes in two flavors: \vocab{proof-construction-as-computation} and \vocab{proof-reduction-as-computation}.
% , each of which has been (separately) applied to the problem of clearly specifying and correctly simulating or implementing concurrent systems.
Proof-construction-as-computation views the search for a proof, according to a fixed strategy, as the basis of computation; it is the foundation for logic programming~\autocites{Miller+:PAL91}{Andreoli:JLC92}.
% \relax[, such as the Prolog and Datalog languages];
Proof-reduction-as-computation, on the other hand, revolves around a correspondence, known as the Curry--Howard Isomorphism~\autocite{Howard:Curry80}, between propositions and types, proofs and programs, and proof simplification, or reduction, and program evaluation;
it is the foundation for typed functional programming~\autocite{Martin-Lof:LMPS80}.
% [, such as the ML and Haskell languages].
% originates from the {BHK} interpretation of intuitionistic logic.

Both the proof-construction and proof-reduction approaches have been applied to concurrent programming, stemming from \citeauthor{Girard:TCS87}'s~\autocite*{Girard:TCS87} suggestion of connections between linear logic and concurrency.
In the proof-construction vein, \acifused{CLF}{\ac{CLF}~\autocite{Watkins+:CMU02}}{the \acl{CLF}~\autocite[\acs{CLF};][]{Watkins+:CMU02}} has been used to specify a variety of concurrent systems, ranging from the $\pi$-calculus to security protocols, such as Needham--Schroeder, and even emergent story narratives~\autocites{Cervesato+:CMU02}{Martens+:LPNMR13}.
% And, using the Lollimon~\autocite{Lopez+:PPDP05} and Celf~\autocite{Schack-Nielsen:ITU11} logic programming engines that derive from \ac{CLF}, these same concurrent systems can be simulated according to their \ac{CLF} specifications.
Although these same concurrent systems can be simulated according to their \ac{CLF} specifications by the Lollimon~\autocite{Lopez+:PPDP05} and Celf~\autocite{Schack-Nielsen:ITU11} logic programming engines, the programs ultimately remain specifications, not actual distributed implementations.

Taking the proof-reduction tack,
% [perspective],
\textcite{Abramsky:TCS93}, \textcite{Bellin+Scott:TCS94}, and later \textcite{Caires+Pfenning:CONCUR10} with Toninho~\autocites*{Caires+:TLDI12}{Caires+:MSCS13} have given correspondences between sequent calculus proofs or proof nets in linear logic and
% [concurrent] 
processes, between cut elimination and concurrent process execution.
Moreover, in \citeauthor{Caires+:MSCS13}'s work, the correspondence is a full Curry--Howard isomorphism in that intuitionistic linear propositions are also types---%
\vocab{session types}~\autocite{Honda:CONCUR93} that describe the protocol to which a process adheres.
% % session types that describe a process's behavior throughout a protocol 
% ---and that it yields actual distributed implementations~\autocite{Toninho+:ESOP13}.
Unlike proof construction, the proof-reduction approach yields actual distributed implementations~\autocite{Toninho+:ESOP13}.

In spite of their common [logical] basis in linear logic, the proof-construction and proof-reduction approaches to concurrent computation appear at first glance to be strikingly disparate.
They have different dynamics (?); they offer different guarantees (progress for proof reduction); and, perhaps most importantly, they serve very different roles in programming [practice].
Although these same concurrent systems can be simulated according to their \ac{CLF} specifications by the Lollimon~\autocite{Lopez+:PPDP05} and Celf~\autocite{Schack-Nielsen:ITU11} logic programming engines, the programs ultimately remain specifications, not actual distributed implementations.
%
Proof construction is better suited to specification, whereas proof reduction is better suited to implementation.

To ..., we'd like to minimize the gap between specification and implementation.
Despite the apparent disparity between proof construction and proof reduction, is there perhaps some fragment of linear logic in which the two coincide?
Stated differently, is there a class of concurrent specifications from which distributed concurrent implementations can be automatically extracted?

Despite their apparent disparity, is there perhaps some \emph{fragment} of linear logic in which [the dynamics of] proof construction and proof reduction coincide?
Identifying such a fragment would provide a better understanding of the relationship between proof construction and proof reduction
Such a fragment would be of philosophical [conceptual] interest, but would also provide practical benefit

% Nevertheless, is there some \emph{fragment} of linear logic in which proof reduction and proof construction coincide---in which [well-typed] implementations can be mechanically extracted from specifications?

% In other words, proof construction is more suited to specification, whereas proof reduction is more suited to implementation.





% The proof-constuction and proof-reduction approaches each have their own advantages and disadvantages.
% Proof construction leads to concurrent programs that are more declarative [clear] but may get stuck in an ill-defined state.
% Proof reduction, on the other hand, provides progress and preservation properties that ensure well-typed concurrent processes never get stuck, but process definitions are less declarative.
% % The proof-construction approach is more declarative [clear] than the proof-reduction approach, but proof reduction comes with its own advantages.
% % Whereas proof construction may get stuck in an ill-defined state, proof reduction can always progress.

% % As always greedy researchers, 
% But is there some \emph{fragment} of linear logic in which proof construction and proof reduction coincide---in which the advantages of \emph{both} approaches, proof-construction and proof-reduction, are retained?
% %---can we have our cake and eat it too?

\vspace{0.25\baselineskip}
\noindent \hspace*{\fill}\scalebox{0.75}{\color{black!50}\ding{70}}\hspace*{\fill}
\vspace{0.25\baselineskip}

\noindent
The thesis is that, yes, we can indeed have our cake and eat it too:
\begin{quotation}
\noindent
Thesis statement.
\itshape Session types form a bridge between distinct notions of concurrency in computational interpretations of intuitionistic linear logic based on proof construction, on one hand, and proof reduction, on the other hand.
% Session types form a bridge between different notions of concurrency that arise in computational interpretations of linear logic: computation-as-proof-search, on one hand, and computation-as-proof-reduction, on the other hand.
\end{quotation}

The contributions of this thesis can be viewed from several perspectives.
\begin{itemize}
\item This work can be seen as a proof-theoretic [logical] reconstruction of multiparty session types~\autocite{Honda+:POPL08}.
In multiparty session types, binary session types are generalized to conversations among several parties.
Conversations in their entirety are specified using global session types.
Global types can be projected to binary session types for each pair of participants, which very nearly are implementations.
\item This work can be seen to further understanding of proof construction and proof reduction.
\item Gives types to logic programs.
Guarantees deadlock-freedom.
\end{itemize}
In addition to the practical benefit of 


The remainder of this document aims to establish this thesis as a plausible one.
% To do so, we turn our attention from linear logic to (non-modal) intuitionistic ordered logic~\autocites{Lambek:AMM58}{Polakow+Pfenning:MFPS99}, a restriction of linear logic in which [the context of] hypotheses are totally ordered, forming a list rather than a multiset.
To do so, we turn our attention from linear logic to (non-modal) intuitionistic ordered logic~\autocites{Lambek:AMM58}{Polakow+Pfenning:MFPS99}---a restriction of linear logic in which the context of hypotheses forms a list rather than a multiset or bag---and defend the thesis in this restricted setting.
% The proposed research is to extend the argument to linear logic.
The proposed thesis research is to relax the restrictions and expand the ideas in this document to intuitionistic linear logic.

Specifically, this document describes ... as depicted in \cref{fig:outline}.
First, \cref{?} reviews a string rewriting interpretation of proof construction in a [non-modal] fragment of intuitionistic ordered logic~\autocite{Simmons:CMU12}.
String rewriting specifications in this fragment are equipped with a natural notion of concurrency based on treating  as equivalent the different interleavings of independent rewriting steps.
[equivalence classes of proofs.]

Despite being concurrent, these string rewriting specifications lack an immediate notion of \emph{process} or \emph{process identity}.
Toward this end, \cref{?} introduces \vocab{choreographies}, a further restriction of string rewriting specifications obtained when [in which] atomic propositions are assigned roles as either process-like atoms or message-like atoms.
(Message-like atoms, such as $\inc[<-]$ in \cref{fig:outline}, are indicated with an arrow decoration.)
A specification may admit several choreographies, but, as described in \cref{?}, a well-formed choreography must be (lock-step) equivalent with the specification.
% \Cref{?} also describes a lock-step equivalence that must hold between a specification and its well-formed choreography.

Even with process-like atoms, choreographies remain string rewriting specifications, not actual distributed implementations of processes.
Nevertheless, choreographies are a stepping-stone to process implementations. 
In \cref{?}, we develop a session-typed process calculus from a Curry--Howard interpretation of a fragment of linear logic; 
\Cref{?} shows how choreographies can be compiled to 



Choreographies serve as a stepping stone to full-fledged process definitions.



\begin{figure}[!t]
  \centering
  \begin{tikzpicture}[node font=\small]
    \matrix [column sep=6em] {
      \node (srs) [align=center] {%
        $\bit{1} \fuse \inc \lrimp \monad{\inc \fuse \bit{0}}$\\[0.5ex]
        \textbf{String rewriting specifications (\cref{?})}\\[0.5ex]
        \textit{Proof construction in a fragment of}\\\textit{propositional ordered logic}%
      };
      &
      \node (ch) [align=center] {%
        $\bit{1} \fuse \inc[<-] \lrimp \monad{\inc[<-] \fuse \bit{0}}$\\[0.5ex]
        \textbf{Choreographies (\cref{?})}\\[0.5ex]
        \textit{Proof construction in a fragment of}\\\textit{propositional ordered logic}%
      };
      \\[8ex]
      \node (ssos) [align=center] {%
        % $\bit{1} = \caseR{\inc[<-] <= \selectL{\inc[<-]; \bit{0}}}$\\[0.5ex]
        \textbf{String rewriting specifications (\cref{?})}\\[0.5ex]
        \textit{Proof construction in a fragment of}\\\textit{first-order ordered logic}%
      };
      &
      \node (proc) [align=center] {%
        % $\exec{\bit{1}} \fuse \msg{\inc[<-]} \lrimp \monad{\selectL{\inc[<-]; \bit{0}}}$\\
        % $\exec{(\selectL{\inc[<-]; \bit{0}})} \lrimp \monad{\msg{\inc[<-]} \fuse \exec{\bit{0}}}$\\[0.5ex]
        \textbf{Session-typed processes (\cref{?})}\\[0.5ex]
        \textit{Proof reduction in}\\\textit{singleton linear logic}%
      };
      \\
    };

    \graph [edges={node font=\footnotesize}] {
      (srs) ->["role assignment"]
      (ch) ->["session types"]
      (proc) ->["SSOS"' align=center]
      (ssos) ->["specialization"]
      (srs);
    };
  \end{tikzpicture}
  \caption{Proof construction to proof reduction---there and back again\label{fig:outline}}
\end{figure}


% Then, by showing that session types bridge the notions of concurrency that arise in a proof-construction-as-computation interpretation of ordered logic and a proof-reduction-as-computation of a further restricted logic.

% a special case of the thesis is defended
% Then, by showing that session types bridge the notions of concurrency that arise in a proof-construction-as-computation interpretation of ordered logic and a proof-reduction-as-computation of a further restricted logic.

% Specifically, we show how session types bridge the notions of concurrency that arise in a proof-construction-as-computation interpretation of ordered logic and a proof-reduction-as-computation of a further restricted logic.

% Rather than considering linear logic from the outset, we restrict the logic to ordered logic
% In it, we show how session types bridge the notions of concurrency that arise in a proof-construction-as-computation interpretation of ordered logic and a proof-reduction-as-computation of a further restricted logic.

% The proposed research thesis is supported



% \section{}


% Similarly, from a proof-reduction perspective, 
% \textcite{Abramsky:TCS93} and later \textcite{Bellin+Scott:TCS94} gave correspondences between proofs in classical linear logic and concurrent processes, with proof reduction corresponding to concurrent process execution.
% These works fall short of a full Curry--Howard isomorphism in that propositions do not clearly correspond to a notion of type.
% More recently, \textcite{Caires+Pfenning:CONCUR10} with Toninho~\autocites*{Caires+:TLDI12}{Caires+:MSCS13} have developed a proof-as-processes corresponsdence for intuitionistic linear logic that indeed treats propositions as types---session types that describe a process's behavior.





% \vspace{\baselineskip}






% Under a proof-construction-as-computation viewpoint, computation arises from the act of searching, according to a fixed strategy, for a proof;
% it is the foundation of logic programming~\autocites{Miller+:PAL91}{Andreoli:JLC92}.
% Proof construction naturally lends itself to the specification and simulation of concurrent systems because
% % proof construction itself can proceed concurrently 
% independent subproofs can occur in any order, just as independant program steps can execute in any order.
% % be constructed concurrently.
% % Because independent parts of a proofcan be constructed concurrently, the proof-construction-as-computation naturally lends itself to specification and simulation of concurrent systems.
% As examples, \acifused{CLF}{}{the }\ac{CLF}~\autocite{Watkins+:CMU02} has been used to specify a variety of concurrent systems, ranging from the $\pi$-calculus to security protocols such as Needham--Schroeder~\autocite{Cervesato+:CMU02}.




% The computation-as-deduction principle comes in two flavors, \vocab{proof-reduction-as-computation} and \vocab{proof-construction-as-computation}, each of which has been (separately) applied to the problem of clearly specifying and correctly simulating or implementing concurrent systems.
% Under a proof-construction-as-computation viewpoint, computation arises from the act of searching, according to a fixed strategy, for a proof;
% it is the foundation of logic programming~\autocites{Miller+:PAL91}{Andreoli:JLC92}.
% Proof construction naturally lends itself to the specification and simulation of concurrent systems because
% % proof construction itself can proceed concurrently 
% independent subproofs can occur in any order, just as independant program steps can execute in any order.
% % be constructed concurrently.
% % Because independent parts of a proofcan be constructed concurrently, the proof-construction-as-computation naturally lends itself to specification and simulation of concurrent systems.
% As examples, \acifused{CLF}{}{the }\ac{CLF}~\autocite{Watkins+:CMU02} has been used to specify a variety of concurrent systems, ranging from the $\pi$-calculus to security protocols such as Needham--Schroeder~\autocite{Cervesato+:CMU02}.




% Computation-as-deduction comes in two flavors, \vocab{proof-reduction-as-computation} and \vocab{proof-construction-as-computation}, each of which has been (separately) applied to the problem of clearly specifying and correctly simulating or implementing concurrent systems.
% Proof-reduction-as-computation is the foundation for typed functional programming~\autocite{Martin-Lof:LMPS80}, such as the ML and Haskell languages, and revolves around a correspondence, known as the Curry--Howard Isomorphism~\autocite{Howard:Curry80}, between propositions and types, proofs and programs, and proof reduction, or simplification, and program evaluation.
% % originates from the {BHK} interpretation of intuitionistic logic.
% Proof-construction-as-computation is the foundation for logic programming~\autocites{Miller+:PAL91}{Andreoli:JLC92}, such as the Prolog and Datalog languages, and 



% From the proof-construction-as-computation viewpoint, computation arises from the act of searching, according to a fixed strategy, for a proof;
% it is the foundation of logic programming~\autocites{Miller+:PAL91}{Andreoli:JLC92}.
% Because independent parts of a proofcan be constructed concurrently, the proof-construction-as-computation naturally lends itself to specification and simulation of concurrent systems.
% For example, \acifused{CLF}{}{the }\ac{CLF}~\autocite{Watkins+:CMU02} has been used to specify a variety of concurrent systems, ranging from the $\pi$-calculus to security protocols such as Needham--Schroeder~\autocite{Cervesato+:CMU02}.





% Both the proof-reduction and proof-construction techniques have been separately applied to the problem of clearly specifying and correctly similating or implementing concurrent systems.
% Using proof-construction, for example, the {CLF}~\autocite{Watkins+:CMU02} has been used to specify a variety of concurrent systems, ranging from the $\pi$-calculus to the Needham--Schroder public key protocol~\autocite{Cervesato+:CMU02}.
% Those< same concurrent systems can be simulated using the Lollimon~\autocite{Lopez+:PPDP05} and Celf~\autocite{Schack-Nielsen:ITU11} logic programming interpreters.

% Using proof-reduction, SILL has been used to implement 

\end{document}


\section{Background}\label{sec:background}

\section{Background}\label{sec:background}

\subsection{Linear logic}\label{sec:linear-logic}

Traditional intuitionistic logic is concerned with the \emph{truth} of propositions.
% , such as \enquote{$n + 0 = n$ for every natural number $n$}.
% truth are concerned with 
% the object of study is truth.
Truths are forever;
having proved a proposition true, there is no \wc{harm}[cost] in reusing or even forgetting the proof.
Traditional intuitionistic logic therefore validates several structural principles: \vocab{contraction}, that hypotheses may be duplicated; \vocab{weakening}, that irrelevant hypotheses may be ignored; and \vocab{exchange}, that the order of hypotheses is unimportant.

Linear logic~\autocite{Girard:TCS87}, on the other hand, treats proofs as precious resources.
Accordingly, linear logic cannot validate the contraction and weakening principles: resources should neither be freely duplicated (as\fxnote{\ \st{they would be}} with contraction), nor be freely discarded (as\fxnote{\ \st{they would be}} with weakening).
The order of resources remains unimportant, however, so linear logic does validate the exchange principle.

We will use a sequent calculus presentation of intuitionistic linear logic based on the judgmental reconstruction by~\textcite{Chang+:CMU03}.
The main judgment is a \wc{sequent}[hypothetical judgment] which states that by using resources~$A_1, \dotsc, A_n$ we can obtain resource~$A$:%
\fxwarning{\ Resource judgment labels, \eg\ $\mathsf{use}$ and $\mathsf{obt}$?}
\begin{equation*}
  \underbrace{A_1, \dotsc, A_n}_{\textstyle \lctx} \seq A \:.
\end{equation*}
Because resources may neither be duplicated nor discarded, the context of resources~$A_1, \dotsc, A_n$ is treated as a multiset.
\fxnote{\st{To formally distinguish the resources~$A_i$, they are labeled with unique variables~$x_i$; we often omit the variables, however, to reduce notational clutter.}\ }%
Following tradition, we use the metavariable~$\lctx$ to stand for an arbitrary context of resources.

\NewDocumentCommand{\lmatch}{O{} m}{\lctx#1, #2}%
\ExplSyntaxOn
\NewDocumentCommand{\lfill}{O{} m}{
  \lctx#1\tl_if_empty:nF{#2}{, #2}
}
\ExplSyntaxOff
%
What are the rules for obtaining and using resources?
One way to obtain a resource $A$ is to use it directly.
Dually, if we can use some of our resources to obtain an $A$, then we are justified in later using that $A$ (along with the remaining resources).
These judgmental principles are expressed by the identity and cut rules, respectively:
\begin{mathpar}
  \infer-[\lab{id$_A$}]{A \seq A}{
    }
  \and
  \infer-[\lab{cut$_A$}]{\lmatch[']{\lctx} \seq C}{
    \lctx \seq A &
    \lfill[']{A} \seq C} \:.
\end{mathpar}
Notice that the identity rule uses only an $A$; allowing any other resources on the left would cause them to be incorrectly disposed here.
The identity and cut rules can be shown by simple inductions to be admissible~\autocite{Chang+:CMU03}, provided that identity for atomic propositions is a rule (hence the dotted inference rule notation).

In addition to the judgmental rules of identity and cut, the sequent calculus contains right and left rules that give meaning to each propositional connective.
% A proof of $A \lolli B$ is a plan for using resource $A$ to achieve goal $B$.
For instance, the \wc{linear implication} $A \lolli B$ (pronounced \enquote{$A$ linearly implies $B$} or \enquote{$A$ lolli $B$}) internalizes the linear hypothetical judgment as a proposition:\fxnote{\ the proposition/a proof of} $A \lolli B$ is a plan for using resource $A$ to obtain resource $B$.
% To make our plan, we presume to have resource $A$ and show how to acheive $B$.
To obtain such a plan, we can presume to have resource $A$\fxnote{\ available for use} and show how to obtain $B$:
\begin{equation*}
  \infer[\rlab{{\lolli}}]{\lctx \seq A \lolli B}{
    \lctx, A \seq B}
  \:.
\end{equation*}
% Conversely, we can use our plan to produce $B$ if we can obtain an $A$.
Conversely, if we can obtain an $A$, then we can carry out, \ie\ use, our plan to make a $B$ available for use:
% To use our plan, we must first obtain $A$
\begin{equation*}
  \infer[\llab{{\lolli}}]{\lmatch[']{\lctx'_A, A \lolli B} \seq C}{
    \lctx'_A \seq A &
    \lfill[']{B} \seq C}
  \:.
\end{equation*}
Thus, the right and left rules explain how to obtain and use, respectively, the resource $A \lolli B$.

The resource $A \tensor B$ (pronounced \enquote{$A$ tensor $B$}) is a pair of the resources $A$ and $B$.
Once again, the right and left rules give meaning to $A \tensor B$:
\begin{mathpar}
  \infer[\rlab{{\tensor}}]{\lctx_1, \lctx_2 \seq A \tensor B}{
    \lctx_1 \seq A &
    \lctx_2 \seq B}
  \and
  \infer[\llab{{\tensor}}]{\lmatch[']{A \tensor B} \seq C}{
    \lfill[']{A, B} \seq C}
\end{mathpar}
To obtain an $A \tensor B$, we must use some of our resources to obtain an $A$ and the rest to obtain a $B$.
To use an $A \tensor B$, we must use \emph{both} an $A$ and a $B$: using only one would cause the resources that went into obtaining the other to be incorrectly disposed.
%These \wc{properties} are expressed in the right and left rules, respectively:
Thus, $A \tensor B$ is a conjunction in which both $A$ and $B$ must be used.

The resource $A \with B$ (pronounced \enquote{$A$ with $B$}) is a different form of conjunction:
\begin{mathpar}
  \infer[\rlab{{\with}}]{\lctx \seq A \with B}{
    \lctx \seq A &
    \lctx \seq B}
  \and
  \infer[{\llab{{\with}}[1]}]{\lmatch[']{A \with B} \seq C}{
    \lfill[']{A} \seq C}
  \and
  \infer[{\llab{{\with}}[2]}]{\lmatch[']{A \with B} \seq C}{
    \lfill[']{B} \seq C}
\end{mathpar}
To obtain an $A \with B$, we must obtain both $A$ and $B$ using \emph{all} of our resources, $\lctx$, in each case.
To use an $A \with B$, we must then choose exactly one of $A$ and $B$ to use; using both would be an underhanded duplication of the resources $\lctx$.

The rules for the disjunction $A \llor B$ are dual to the rules for $A \with B$:
\begin{mathpar}
  \infer[{\rlab{{\llor}}[1]}]{\lctx \seq A \llor B}{
    \lctx \seq A}
  \and
  \infer[{\rlab{{\llor}}[2]}]{\lctx \seq A \llor B}{
    \lctx \seq B}
  \and
  \infer[\llab{{\llor}}]{\lmatch[']{A \llor B} \seq C}{
    \lfill[']{A} \seq C &
    \lfill[']{B} \seq C}
\end{mathpar}
A resource $A \llor B$ is thus either an $A$ or a $B$.
When using a $A \llor B$, we must therefore be prepared to use what we get in either case.
% This time, instead of choosing one of $A$ and $B$ when using $A \llor B$, the environment has made that choice for us; we must be prepared for either case.

\NewDocumentCommand{\univq}{u: u.}{\forall #1{:}#2.\,}
\NewDocumentCommand{\existq}{u: u.}{\exists #1{:}#2.\,}
The universal and existential quantifiers, $\univq x:\tau.A$ and $\existq x:\tau.A$, 

For convenience, these rules, along with rules for the first-order universal and existential quantifiers, are summarized in \cref{fig:seq-jill}.%
\fxerror{\ [Decide whether to present $\bang A$ and $\monad{A}$ here.]}

% In this section, we briefly review a judgmental intuitionistic formulation of \citeauthor{Girard:TCS87}'s linear logic~\autocite*{Girard:TCS87}, a substructural logic of resources that is especially suited to modeling stateful systems.


\NewDocumentCommand{\tctx}{}{\Psi}
\NewDocumentCommand{\subst}{m m}{[#1]#2}

\begin{figure}
  \begin{mathpar}
    \infer[\lab{id}]{\uctx ; P \seq P}{
      }
    \and
    \infer[\lab{copy}]{\uctx, A ; \lctx' \seq J}{
      \uctx, A ; \lctx', A \seq J}
    \\
    \infer[\rlab{{\lolli}}]{\uctx ; \lctx \seq A \lolli B}{
      \uctx ; \lctx, A \seq B}
    \and
    \infer[\llab{{\lolli}}]{\uctx ; \lmatch[']{\lctx'_A, A \lolli B} \seq J}{
      \uctx ; \lctx'_A \seq A &
      \uctx ; \lfill[']{B} \seq J}
    \\
    \infer[\rlab{{\with}}]{\uctx ; \lctx \seq A \with B}{
      \uctx ; \lctx \seq A &
      \uctx ; \lctx \seq B}
    \and
    \infer[{\llab{{\with}}[1]}]{\uctx ; \lmatch[']{A \with B} \seq J}{
      \uctx ; \lfill[']{A} \seq J}
    \and
    \infer[{\llab{{\with}}[2]}]{\uctx ; \lmatch[']{A \with B} \seq J}{
      \uctx ; \lfill[']{B} \seq J}
    \\
    \infer[\rlab{{\tensor}}]{\uctx ; \lctx_1, \lctx_2 \seq A \tensor B}{
      \uctx ; \lctx_1 \seq A &
      \uctx ; \lctx_2 \seq B}
    \and
    \infer[\llab{{\tensor}}]{\uctx ; \lmatch[']{A \tensor B} \seq J}{
      \uctx ; \lfill[']{A, B} \seq J}
    \\
    \infer[\rlab{\one}]{\uctx ; \lctxe \seq \one}{
      }
    \and
    \infer[\llab{\one}]{\uctx ; \lmatch[']{\one} \seq J}{
      \uctx ; \lfill[']{} \seq J}
    \\
    \infer[\rlab{\forall}]{\tctx ; \uctx ; \lctx \seq \univq x:\tau. A}{
      \tctx, x{:}\tau ; \uctx ; \lctx \seq A}
    \and
    \infer[\llab{\forall}]{\tctx ; \uctx ; \lmatch[']{\univq x:\tau. A} \seq J}{
      \tctx \seq t : A &
      \tctx ; \uctx ; \lfill[']{\subst{t/x}{A}} \seq J}
    \\
    \infer[\rlab{\exists}]{\tctx ; \uctx ; \lctx \seq \existq x:\tau. A}{
      \tctx \seq t : A &
      \tctx ; \uctx ; \lctx \seq \subst{t/x}{A}}
    \and
    \infer[\llab{\exists}]{\tctx ; \uctx ; \lmatch[']{\existq x:\tau. A} \seq J}{
      \tctx, x{:}\tau ; \uctx ; \lfill[']{A} \seq J}
    \\
    \infer[\rlab{\bang}]{\uctx ; \lctxe \seq \bang A}{
      \uctx ; \lctxe \seq A}
    \and
    \infer[\llab{\bang}]{\uctx ; \lmatch[']{\bang A} \seq J}{
      \uctx, A ; \lfill[']{} \seq J}
  \end{mathpar}
  \begin{mathpar}
    \infer-[\lab{id$_A$}]{\uctx ; A \seq A}{
      }
    \and
    \infer-[\lab{cut$_A$}]{\uctx ; \lmatch[']{\lctx} \seq J}{
      \uctx ; \lctx \seq A &
      \uctx ; \lfill[']{A} \seq J}
    \and
    \infer-[\lab{cut$^{\bang}_A$}]{\uctx ; \lctx' \seq J}{
      \uctx ; \lctxe \seq A &
      \uctx, A ; \lctx' \seq J}
  \end{mathpar}
  \caption{Sequent calculus rules for a propositional fragment of \acf{JILL}~\autocite{Chang+:CMU03}.\label{fig:seq-jill}}
\end{figure}

\subsection{Proof search as computation: Bottom-up linear logic programming}\label{sec:linear-lp}

\begin{itemize}
\item Notion of transition obtained by reading sequent calculus left rules  (bipoles) bottom-up.
\end{itemize}

\subsection{Proof reduction as computation: Session-typed linear logic}\label{sec:async-sill}

Having seen a proof-reduction-as-computation correspondence between intuitionistic logic and functional computation, it's natural to ask if there is a computational interpretation of the intuitionistic linear sequent calculus.


% as session-type discipline for concurrent processes:
Giving session-typed concurrency a logical footing, \textcite{Caires+Pfenning:CONCUR10} along with Toninho~\autocite*{Caires+:TLDI12} have developed a Curry-Howard interpretation of the intuitionistic linear sequent calculus in which: proofs are processes, propositions are session types, and proof reduction is interprocess communication.

\ExplSyntaxOn
\NewDocumentCommand{\minput}{>{\SplitArgument{1}{;}}m}{
  \minput_:nn #1
}
\NewDocumentCommand{\minput_:nn}{>{\SplitArgument{1}{<-}}m m}{
  \use_i:nn #1 \leftarrow \mathtt{input}\: \use_ii:nn #1
  \mathtt{;}\: #2
}
\NewDocumentCommand{\moutput}{m >{\SplitArgument{1}{<-}}m m}{
  \mathtt{output}\:#1\:(\use_i:nn #2 \leftarrow \use_ii:nn #2)\mathtt{;}\:#3
}
\ExplSyntaxOff
Recall the sequent calculus right rule for linear implication:
\begin{equation*}
  \infer[\rlab{{\lolli}}]{\lctx \seq A \lolli B}{
    \lctx, A \seq B} \:.
\end{equation*}
It says that to obtain $A \lolli B$ we can obtain $B$ while presuming to have $A$.
Correspondingly, a process that offers service $A \lolli B$ should first input a channel offering service $A$ and then continue the session by using that service to offer service $B$.
Based on this intuition, we assign an input process to the $\rlab{{\lolli}}$:
\begin{equation*}
  \infer[\rlab{{\lolli}}]{\lctx \seq \minput{y <- x; P_y} :: x{:}A \lolli B}{
    \lctx, y{:}A \seq P_y :: x{:}B} \:.
\end{equation*}
The syntax $\minput{y <- x; P_y}$ means \enquote{Input channel $y$ along channel $x$ and then continue as process $P_y$}.

The left rule for linear implication uses the service offerred by the right rule, so it should be a matching output:
\begin{equation*}
  \infer[\llab{{\lolli}}]{\lmatch[']{\lctx'_A, x{:}A \lolli B} \seq \moutput{x}{y <- Q}{R} :: z{:}C}{
    \lctx'_A \seq Q :: y{:}A &
    \lfill[']{x{:}B} \seq R :: z{:}C}
\end{equation*}



\begin{itemize}
\item Present monadic language only?  Disadvantage is that proof reduction is not quite as clear or simple to see since it involves run-time typing.
\end{itemize}





\subsection{Ordered logic}\label{sec:ordered-logic}

\NewDocumentCommand{\ofrm}{m}{\Theta\{#1\}}
\begin{figure}
  \begin{mathpar}
    \infer[\lab{id}]{P \seq P}{
      }
    \\
    \infer[\rlab{{\rimp}}]{\octx \seq A \rimp B}{
      \octx, A \seq B}
    \and
    \infer[\llab{{\rimp}}]{\ofrm{A \rimp B, \octx'} \seq J}{
      \octx' \seq A &
      \ofrm{B} \seq J}
    \\
    \infer[\rlab{{\limp}}]{\octx \seq A \limp B}{
      A, \octx \seq B}
    \and
    \infer[\llab{{\limp}}]{\ofrm{\octx', A \limp B} \seq J}{
      \octx' \seq A &
      \ofrm{B} \seq J}
    \\
    \infer[\rlab{{\with}}]{\octx \seq A \with B}{
      \octx \seq A &
      \octx \seq B}
    \and
    \infer[{\llab{{\with}}[1]}]{\ofrm{A \with B} \seq J}{
      \ofrm{A} \seq J}
    \and
    \infer[{\llab{{\with}}[2]}]{\ofrm{A \with B} \seq J}{
      \ofrm{B} \seq J}
    \\
    \infer[\rlab{{\fuse}}]{\octx_1, \octx_2 \seq A \fuse B}{
      \octx_1 \seq A &
      \octx_2 \seq B}
    \and
    \infer[\llab{{\fuse}}]{\ofrm{A \fuse B} \seq J}{
      \ofrm{A, B} \seq J}
    \\
    \infer[\rlab{\one}]{\octxe \seq \one}{
      }
    \and
    \infer[\llab{\one}]{\ofrm{\one} \seq J}{
      \ofrm{\octxe} \seq J}
  \end{mathpar}
  \begin{mathpar}
    \infer-[\lab{id$_A$}]{A \seq A}{
      }
    \and
    \infer-[\lab{cut$_A$}]{\ofrm{\octx} \seq J}{
      \octx \seq A &
      \ofrm{A} \seq J}
  \end{mathpar}
  \caption{Sequent calculus rules for ordered logic~\autocite{Polakow+Pfenning:MFPS99,Simmons:CMU12}.}
\end{figure}




\begin{itemize}
\item Should I give a brief introduction to linear logic here?  Or, is this overkill for committee members?
\item Present ordered logic in its own right here, or present ordered logic only in the context of bottom-up logic programming?
\end{itemize}

\subsection{Bottom-up ordered logic programming}\label{sec:ordered-lp}

\NewPredicate{\eps}{0}
\NewPredicate{\bitz}[bit_0]{0}
\NewPredicate{\bito}[bit_1]{0}
\ExplSyntaxOn
\NewDocumentCommand{\bit}{m}{
  \int_case:nnF {#1}
    {
      {0} {\bitz}
      {1} {\bito}
    }
    { Error! }
}
\ExplSyntaxOff
\NewPredicate{\inc}{0}

\begin{align*}
  &\eps \fuse \inc \lrimp \eps \fuse \bit{1} \\
  &\bit{0} \fuse \inc \lrimp \bit{1} \\
  &\bit{1} \fuse \inc \lrimp \inc \fuse \bit{0}
\end{align*}

\begin{itemize}
\item Example: binary counter with increment.
\end{itemize}


%%% Local Variables:
%%% TeX-master: "proposal"
%%% End:


\section{Choreographies}\label{sec:choreographies}

% arara: pdflatex
% arara: biber
% arara: pdflatex
% arara: pdflatex
\documentclass[
  class=../hdeyoung-proposal,
  crop=false
]{standalone}

\usepackage[subpreambles]{standalone}

\usepackage{ordered-logic}
\usepackage{binary-counter}

\addbibresource{../proposal.bib}

\begin{document}

\section{Choreographies}\label{sec:choreographies}

Traditionally, concurrency is phrased as the composition of interacting, \wc{locally executing}[\st{distributed}] processes.
As the binary counter example from \cref{sec:exampl-binary-count-2,sec:exampl-binary-count-3} demonstrates, a notion of concurrency based on indistinguishable interleavings of independent rewrites arises naturally in ordered logic programming.
% a notion of concurrency based on indestinguishable interleavings arises naturally in ordered logic programming.
% However, it is not as clear how to identify a notion of process
But where are the \wc{locally executing}[\st{distributed}] processes?

Taking a formula-as-process view \autocites{Miller:ELP92}{Cervesato+Scedrov:IC09}, the processes are the ordered logic program's atomic propositions.
% This thesis proposes that the atomic propositions in an ordered logic program are the processes.
% \fxnote{[How much of this is already implied by a formula-as-process interpretation?]}%
The program's clauses, accordingly, serve to specify the valid interactions among processes. 
In the binary counter,
% \wc{specification}[\st{program}], 
for example, $\eps$, $\bit{0}$, $\bit{1}$, and $\inc$, among others,
% $\dec$, $\bit[']{0}$, $\zero$, and $\succ$ 
are
% all 
atoms-as-processes,
% whose interactions are governed by the program's rules.
% In particular, the rule
and the clause
\begin{equation*}
  \bit{1} \fuse \inc \lrimp \monad{\inc \fuse \bit{0}}
\end{equation*}
% says that neighboring $\bit{1}$ and $\inc$ processes \fxnote{should be able to} interact to form neighboring $\inc$ and $\bit{0}$ processes.
says that one valid interaction is for neighboring $\bit{1}$ and $\inc$ processes to \wc{coordinate}[\st{cooperate}] to become neighboring $\inc$ and $\bit{0}$ processes.
% (with similar readings for the other clauses).

The program's clauses don't tell the full story, however:
the clauses specify\fxnote{\ globally} \emph{what} are valid interactions but not \emph{how} to realize \wc{them}[\st{those interactions}]\fxnote{\ \st{locally}}.
% The program is thus only a \vocab{specification}, with the how instead being supplied by the logic programming language's operational semantics.
In ordered logic programming, the how is\fxnote{\ \st{instead}} traditionally supplied by an operational semantics in which a\fxnote{\st{n omniscient}} central conductor, having the benefit of a global view of all atoms, directs the atoms' interactions according to the program's clauses.
% The how is instead supplied by the language's operational semantics.
% % The  language's operational semantics instead supplies the how.
% % The program is thus only a \vocab{specification}, with
% In the usual operational semantics for ordered logic programming, there is a central \enquote{conductor} who, having the benefit of a global view of all atoms, directs the atoms' interactions according to the program's clauses.
But because they rely so heavily on the central conductor, processes using this semantics are no more than superficially \wc{local}[\st{distributed}].
% Under this semantics, however, processes are only nominally distributed because they rely so heavily on the central conductor.


To be truly \wc{local}[\st{distributed}], the processes should instead communicate directly with their neighbors to identify which, if any, of the valid interactions are possible for them at that moment.
% For instance, by communicating directly with its left-hand neighbor, an $\inc$ process might learn that that neighbor is a $\bit{1}$ process and that the above clause therefore applies; further direct communication between the two processes would effect
For instance, by communicating directly with its left-hand neighbor, an $\inc$ process might learn that that neighbor is a $\bit{1}$ process; with further direct communication, the $\bit{1}$ and $\inc$ processes could coordinate to effect the above\fxnote{\ globally specified} interaction.
% How does $\bit{1}$, for example, learn that its right-hand neighbor is $\inc$ and that the above clause therefore applies?


So, the distinction being drawn here is one between a \vocab{specification} and its \vocab{choreography}---the what and the how.
% \fxnote{\st{The original program is only a specification of the valid process interactions, whereas a choreography is a pattern of communication that implements that specification.}}
A specification is the original program, which serves as a \emph{global} description of the valid process interactions; a choreography is a \emph{local}\fxnote{\ message-passing} implementation of that specification.%
% A specification is the original program used to describe the valid process interactions, whereas a choreography is a pattern of communication that implements that specification%
% \footnote{Notice that we always speak of a choreography relative to a specification, just as an implementation is always relative to an abstraction.}%
% , and which must be given by the programmer, at least implicitly.%
\footnote{We borrow the term \enquote*{choreography} from the literature on session-based concurrency.
The analogy is intended only as a loose one, however, and should not be taken to imply a precise, technical correspondence.}%
%, which must also be given by the programmer, at least implicitly.%
\footnote{Notice that a choreography is always relative to a given specification.}

% Designing a one-size-fits-all distributed operational semantics appears to be difficult, however.
% We could try to design a one-size-fits-all local operational semantics, but this appears to be difficult.
Ideally, an operational semantics would automatically generate a choreography from the specification supplied by the programmer, but designing such a semantics\fxnote{\ \st{unfortunately}} appears to be difficult.
% % Designing a distributed operational semantics that is uniformly suitable appears to be difficult, however.
% % Designing a uniformly suitable distributed operational semantics appears to be difficult, however.
Different specifications will often require different patterns of interprocess communication;
sometimes a specification will even admit several choreographies, and, more often than not, the programmer will want to exercise control in those cases.
% % Therefore, rather than relying on a one-size-fits-all operational semantics, the programmer must indicate the intended pattern of communication for each program.
% Not having a one-size-fits-all local operational semantics, the programmer himself must indicate the intended pattern of communication for each program.
Therefore, not having a one-size-fits-all local operational semantics, the programmer himself must supply the choregraphy, at least implicitly.

In the previous \lcnamecref{sec:??}, we saw several examples of specifications (characterized as forward-chaining ordered logic programs), including the aforementioned binary counter supporting increment and decrement operations.
We'll now describe what counts as a choreography, first with informal examples and then with formal definitions.



% % Designing a one-size-fits-all distributed operational semantics appears to be difficult, however.
% % We could try to design a one-size-fits-all local operational semantics, but this appears to be difficult.
% Ideally, the operational semantics would localize programs in this way, but, unfortunately, designing such a semantics\fxnote{\ \st{that is also uniformly suitable}} appears to be difficult.
% % % Designing a distributed operational semantics that is uniformly suitable appears to be difficult, however.
% % % Designing a uniformly suitable distributed operational semantics appears to be difficult, however.
% Different programs will often require different patterns of interprocess communication;
% sometimes a program will even admit several communication patterns, and, more often than not, the programmer will want to exercise control in those cases.
% % Therefore, rather than relying on a one-size-fits-all operational semantics, the programmer must indicate the intended pattern of communication for each program.
% Not having a one-size-fits-all local operational semantics, the programmer himself must indicate the intended pattern of communication for each program.

% % The distinction being drawn here is one between a \vocab{specification} and its \vocab{choreography}---the what and the how.
% % The original program serves only as a specification of what are valid process interactions, whereas the choregraphy is the programmer's intended 

% So, the distinction being drawn here is one between a \vocab{specification} and its \vocab{choreography}---the what and the how.
% % \fxnote{\st{The original program is only a specification of the valid process interactions, whereas a choreography is a pattern of communication that implements that specification.}}
% A specification is the original program, which serves as a global description of the valid process interactions; a choreography is a local\fxnote{, message-passing} implementation of that specification.%
% % A specification is the original program used to describe the valid process interactions, whereas a choreography is a pattern of communication that implements that specification%
% % \footnote{Notice that we always speak of a choreography relative to a specification, just as an implementation is always relative to an abstraction.}%
% % , and which must be given by the programmer, at least implicitly.%
% \footnote{We borrow the term \enquote*{choreography} from the literature on session-based concurrency.
% The analogy is intended only as a loose one, however, and should not be taken to imply a precise, technical correspondence.}%
% %, which must also be given by the programmer, at least implicitly.%
% \footnote{Notice that a choreography is always relative to a given specification.}

% In the previous \lcnamecref{sec:??}, we saw several examples of specifications (characterized as forward-chaining ordered logic programs), including a binary counter supporting increment and decrement operations.
% We'll now describe what counts as a choreography, first with informal examples and then with formal definitions.

% To build intuition, we'll now describe choregraphies by example


% , adapting terminology from the concurrency literature,


% So, unfortunately, 


% Unfortunately, because different programs will require different patterns of communication among processes, we won't be able to leave the how up to the operational semantics.
% The programmer will want control over the communication patterns.



% Borrowing terminology from the literature on sessions

% In session terminology, the logic program with a centralized operational semantics is known as an orchestration of processes, whereas the desired distributed semantics is known as a choreography.




% \mbox{}\\

% The program's clauses don't tell the full story, however:
% The clauses specify what are valid interactions but not \emph{how} to realize those interactions; the \enquote*{how} is instead supplied by the logic programming language's operational semantics.
% In the usual operational semantics, there is a central \enquote{conductor} who, having the benefit of a global view of all atoms, directs the atoms' interactions according to the program's clauses.

% However, because they rely so heavily on the central conductor, processes using this semantics are no more than superficially distributed.
% % Under this semantics, however, processes are only nominally distributed because they rely so heavily on the central conductor.
% To be truly distributed, the processes should instead communicate directly with their neighbors to identify which, if any, of the valid interactions are possible for them at that moment.

% It's difficult to argue that this centralized \enquote{how} is suitable for \emph{distributed} processes, however.
% The distributed processes should instead communicate directly with their neighbors to identify which, if any, of the valid interactions are possible for them at that moment.
% How does $\bit{1}$, for example, learn that its right-hand neighbor is $\inc$ and that the above clause therefore applies?

% In session terminology, the logic program with a centralized operational semantics is known as an orchestration of processes, whereas the desired distributed semantics is known as a choreography.




% The program's clauses do not tell the full story, however: the clauses specify what are valid interactions but not \emph{how} to realize those interactions.
% In the usual operational semantics, the \enquote{how} is supplied by providing a central \enquote{conductor} that, having the benefit of a global view\fxnote{\ \st{of all atoms}}, manipulates the atoms according to the program's clauses.
% But it's difficult to argue that this centralized \enquote{how} is suitable for \emph{distributed} processes.
% Distributed processes should instead communicate directly with their neighbors to identify which, if any, of the valid interactions are possible for them at that moment.
% How does $\bit{1}$, for example, learn that its right-hand neighbor is $\inc$ and that the above clause therefore applies?






% This isn't the full story, however:
% % The program's clauses specify \emph{what} are valid interactions but not \emph{how} to achieve those interactions.
% the program's clauses specify what are valid interactions but not \emph{how} to achieve them.
% The \enquote{how} is provided by the logic programming language's operational semantics.
% The usual operational semantics


% How do $\bit{1}$ and $\inc$, for example, learn that they are neighbors and that the above clause therefore applies?


% the \enquote{how}it is provided by the logic programming language's operational semantics.
% The usual operational semantics for logic programming 

% This isn't the full story, however.
% The usual operational semantics for ordered logic programming assumes a central \enquote{puppeteer} that has a global view of all atoms and manipulates them according to the program's clauses.
% It's difficult to argue that this centralization is appropriate for distributed processes, however.
% Instead, the processes should communicate directly to identify their neighbors and thereby deduce which, if any, of the valid interactions are possible for them at that moment.
% But this communication is left unspecified in the original logic program.
% How does $\bit{1}$, for example, learn that its right-hand neighbor is $\inc$ and that the above clause therefore applies?

% % What's left unspecified in the ordered logic program is how the distributed processes communicate to identify their neighbors and thereby deduce which, if any, of the valid interactions are possible for them at that moment.
% % % Using a communication protocol that is left unspecified in the program, the atoms deduce 
% % How does $\bit{1}$, for example, learn that its right-hand neighbor is $\inc$ and that the above clause therefore applies?

% Orchestration vs. choreography


% % arara: pdflatex
% % arara: pdflatex
% % arara: biber
% % arara: pdflatex
% % arara: pdflatex
% % \documentclass{../hdeyoung-proposal}
% \documentclass[
%   class=../hdeyoung-proposal,
%   crop=false
% ]{standalone}

% \usepackage{ordered-logic}
% \usepackage{basic-atoms}
% \usepackage{binary-counter}

% \crefname{choreography}{chor.}{chors.}
% \Crefname{choreography}{Chor.}{Chors.}

% \DeclareAcronym{BHK}{
  short = BHK,
  long  = Brouwer-Heyting-Kolmogorov
}

\DeclareAcronym{JILL}{
  short = JILL,
  long  = judgmental intuitionistic linear logic
}

\DeclareAcronym{ILL}{
  short = ILL,
  long  = intuitionistic linear logic
}

\DeclareAcronym{SML}{
  short = SML,
  long  = Standard ML
}

\DeclareAcronym{SOS}{
  short = SOS,
  short-indefinite = an,
  long = structural operational semantics
}

\DeclareAcronym{SSOS}{
  short = SSOS,
  short-indefinite = an,
  long = substructural operational semantics
}

\DeclareAcronym{SILL}{
  short = SILL,
  long = session-typed intuitionistic linear logic
}

\DeclareAcronym{SISLL}{
  short = singleton \acs{SILL},
  long = session-typed intuitionistic singleton linear logic
}

\DeclareAcronym{CLF}{
  short = CLF,
  long = the Concurrent Logical Framework
}


% \begin{document}

\subsection{Choreographies by example}\label{sec:chor-by-example}

\subsubsection{The binary counter}\label{sec:chor-example-counter}

In giving the intuition behind the binary counter specification (\cref{sec:olp-intuition:binary-counter}), we described the $\inc$ atoms % as moving --- moving past any $\bit{1}$s and eventually stopping at the $\eps$ or right-most $\bit{0}$.
as moving up the counter.
% % , a subliminal hint that $\inc$s are like messages.
% % This suggests a choreography in which $\inc$ processes take the active lead:
% This hints that $\inc$s are a bit like messages, and suggests a choreography in which $\inc$ processes initiate the interaction:
% First, each $\inc$ process sends a message, $\inc[<-]$, to its left-hand neighbor, thereby notifying that neighbor of its existence, and then the $\inc$ process terminates.
% If the neighbor is $\eps$, $\bit{0}$, or $\bit{1}$, then, upon receiving the $\inc$'s message, that neighbor takes full responsibility for completing the corresponding interaction.
This hints at a choreography in which $\inc$ atoms act as messages that trigger the increment action:
Whenever an $\inc$ message arrives at an $\eps$, $\bit{0}$, or $\bit{1}$ process, that process takes responsiblity for completing the increment action.
% First, each $\inc$ process sends a message, $\inc[<-]$, to its left-hand neighbor, thereby notifying that neighbor of its existence, and then the $\inc$ process terminates.
% If the neighbor is $\eps$, $\bit{0}$, or $\bit{1}$, then, upon receiving the $\inc$'s message, that neighbor takes full responsibility for completing the corresponding interaction.


% In giving the intuition behind the binary counter specification (\cref{sec:olp-intuition:binary-counter}), we described the $\inc$ atoms as moving up the counter.
% % This hints that $\inc$s are a bit like messages, and suggests a choreography in which $\inc$ processes initiate the interaction:
% % First, each $\inc$ process sends a message, $\inc[<-]$, to its left-hand neighbor, thereby notifying that neighbor of its existence, and then the $\inc$ process terminates.
% % If the neighbor is $\eps$, $\bit{0}$, or $\bit{1}$, then, upon receiving the $\inc$'s message, that neighbor takes full responsibility for completing the corresponding interaction.
% This hints at a choreography in which $\inc$ atoms are messages that trigger the increment action by $\eps$, $\bit{0}$, and $\bit{1}$ atoms that act as processes.
% When the $\eps$, $\bit{0}$, or $\bit{1}$, processes receive the $\inc[<-]$ message,  then, upon receiving the $\inc$'s message, that neighbor takes full responsibility for completing the corresponding interaction.

Expressed as an annotation of the original ordered logical specification, this choreography is:
\begin{equation}\label[choreography]{chor:oop-counter}
  \!\begin{aligned}
    &\eps \fuse \inc[<-] \lrimp \monad{\eps \fuse \bit{1}} \\
    &\bit{0} \fuse \inc[<-] \lrimp \monad{\bit{1}} \\
    &\bit{1} \fuse \inc[<-] \lrimp \monad{\inc[<-] \fuse \bit{0}}
    \text{\,,}
  \end{aligned}
\end{equation}
where the $\eps$, $\bit{0}$, and $\bit{1}$ atoms are viewed as processes, but the $\inc[<-]$ atoms are viewed as messages.
%
Two properties are crucial:
\begin{description}[font=\normalfont\itshape, leftmargin=\parindent, labelindent=\leftmargin, listparindent=\parindent, parsep=0pt]
\item[Locality.]
  Each clause's premise depends on exactly one process-like atom and (at most) one message-like atom.
  Consequently, each process's decisions are entirely local: the $\eps$, $\bit{0}$, and $\bit{1}$ processes act (independently) only after receiving an $\inc[<-]$ message.%
  \footnote{In {SSOS} terminology, processes that wait to receive a message, like $\eps$, $\bit{0}$, and $\bit{1}$ here, would be termed \vocab{latent} propositions; and messages, like $\inc[<-]$ here, would be termed \vocab{passive} propositions.}

  Locality serves to ensure that the choreography describes sensible message-passing behaviors.
  A clause such as $\inc[<-] \lrimp \monad{{\dots}}$, whose premise does not contain a process-like atom, is not message-passing because no process receives the $\inc[<-]$ message.
%
\item[Specification-preserving.]
% % % Second, notice that
% % The choreography exposes the same $\eps$, $\bit{}$, and $\inc$ processes as the original binary counter specification; the last three clauses of the choreography differ from the specification's clauses only in the substitution of $\inc[<-]$ for $\inc$ in their premises.
% The choreography exposes the same $\eps$, $\bit{}$, and $\inc$ processes as the original binary counter specification.
% Its clauses differ from those of the specification only in the substitution of $\inc[<-]$ for $\inc$ in their premises.
% (The choreography also includes an $\inc \lrimp \monad{\inc[<-]}$ clause to justify that substitution.)
% In this sense, there is a very strong equivalence between the two programs.
% The choreography does not fundamentally alter the specification---it only refines that specification by making the communication patterns explicit.
%
The choreography exposes the same behaviors for $\eps$, $\bit{}$, and $\inc$ as in the original specification.
Its clauses are exactly those of the specification, except that each $\inc$ atom in the specification has been annotated as an $\inc[<-]$ message-like atom in the choreography.

In this sense, there is a very strong, lock-step equivalence between the choreography and its specification.
The choreography does not fundamentally alter the specification---it only refines that specification by making the communication patterns explicit.
\end{description}
%
% In this sense, there is a strong equivalence between the 
% The choreography does not fundamentally alter the implementation given in the original program---it only refines that implementation by making the communication patterns explicit.
% In this sense, there is a strong equivalence between, which will be made precise in \cref{??}
%
% Notice that this choreography \wc{refactors} the original program so that each new clause depends on exactly one process atom and at most one message atom.
% In this way, each process's decisions are completely local: the $\inc$ process always sends $\inc[<-]$ regardless of its neighbors, and the $\eps$ and $\bit{}$ processes act only after receiving an $\inc[<-]$ message.%
% \footnote{In \ac{SSOS} terminology, processes that act regardless of their neighbors, like $\inc$, would be termed \vocab{active} propositions; processes that wait to receive a message, like $\eps$, $\bit{0}$, and $\bit{1}$, would be termed \vocab{latent} propositions; and messages, like $\inc[<-]$, would be termed \vocab{passive} propositions.}
%
It's convenient to think of the programmer as supplying this choreography in full, but in practice the programmer might only give the assignment of roles to atoms, \eg\ $\inc[<-]$ for $\inc$.

\subsubsection{Messages can flow in both directions}\label{sec:chor-binary-count}

In our binary counter specification with decrements (\cref{sec:olp-intuition:decrements}), $\dec$ atoms propagate up the counter similarly to $\inc$s, with the difference that each $\dec$ atom eventually gives rise to either a $\fail$ or $\suc$ atom that travels back down the counter.
Once again, this hints at a choreography in which $\dec$, $\fail$, and $\suc$ atoms are message-like:
\begin{itemize}
\item Whenever a $\dec[<-]$ message arrives at an $\eps$, $\bit{0}$, or $\bit{1}$ process's right-hand side, that process completes the local decrement action:
      the $\eps$ and $\bit{1}$ processes send a $\fail[->]$ or $\suc[->]$ message, respectively, to their right;
      the $\bit{0}$ process forwards the $\dec[<-]$ message to its left and continues as a $\bit[']{0}$ process.
\item Whenever a $\fail[->]$ or $\suc[->]$ message arrives at a $\bit[']{0}$ process's left-hand side, that process forwards the message to its right-hand neighbor and continues as a $\bit{0}$ or $\bit{1}$ process, respectively.
% \item Each $\dec$ process sends a message, $\dec[<-]$, to its left-hand neighbor and terminates.
%       If the neighbor is $\eps$, $\bit{0}$, or $\bit{1}$, then, upon receiving the message, that neighbor completes the corresponding interaction given in the specification.
% \item Each $\fail$ or $\suc$ process sends a message, $\fail[->]$ or $\suc[->]$, respectively, to its \emph{right-hand} neighbor and terminates.
%       If the neighbor is $\bit[']{0}$, then, upon receiving the message from $\fail$ or $\suc$, that neighbor completes the corresponding interaction.
\end{itemize}
To account for decrements, the binary counter's choreography is therefore extended with the following clauses:
\begin{equation}
  \!\begin{aligned}
    &\eps \fuse \dec[<-] \lrimp \monad{\eps \fuse \fail[->]} \\
    &\bit{0} \fuse \dec[<-] \lrimp \monad{\dec[<-] \fuse \bit[']{0}} \\
    &\bit{1} \fuse \dec[<-] \lrimp \monad{\bit{0} \fuse \suc[->]} \\[1.5\jot]
    % 
    &\fail[->] \fuse \bit[']{0} \lrimp \monad{\bit{0} \fuse \fail[->]} \\
    &\suc[->] \fuse \bit[']{0} \lrimp \monad{\bit{1} \fuse \suc[->]}
    \,.
  \end{aligned}
\end{equation}
Once again, these clauses are just an annotation of the original specification's clauses, with $\dec$, $\suc$, and $\fail$ annotated as $\dec[<-]$, $\suc[->]$, and $\fail[->]$.
% Once again, the atoms that are decorated with arrows are formally distinct from their undecorated counterparts.
% (As before, the atoms that are decorated with arrows are formally distinct from their undecorated counterparts.)
The extended choreography thus continues to be specification-preserving.

This extended choreography illustrates that message atoms may be either left-directed, like $\inc[<-]$ and $\dec[<-]$, or right-directed, like $\fail[->]$ and $\suc[->]$.
% Moreover, a message's direction determines the structure of premises in which it is received:
% a left-directed (right-directed) message must arrive at the receiving process's right (resp., left) side, otherwise the message would not be traveling from left to right (resp., right to left).
% 
% Because it is traveling left-to-right, a left-directed message must always arrive at the right-hand side of its recipient; dually, a right-directed message must always arrive at the left-hand side of its recipient.
Because a left-directed message travels from right to left, it must always arrive at the right-hand side of its recipient; dually, a right-directed message must always arrive at the left-hand side of its recipient.
This directionality is another aspect of locality, and it further constrains the structure of a choreography's premises.
For example, this choreography's premises are well-formed because each message flows toward its recipient, whereas premises of the forms $\matom[<-] \fuse \patom$ or $\patom \fuse \matom[->]$ are not well-formed because process $\patom$ doesn't receive the $\matom[<-]$ or $\matom[->]$ message.
% For instance, $\bit{1} \fuse \inc[<-]$ and $\fail[->] \fuse \bit[']{0}$ are well-formed premises because each message flows toward its recipient, whereas premises of the forms $\matom[<-] \fuse \p$ or $\p \fuse \matom[->]$ are not well-formed because process $\p$ doesn't receive the $\matom[<-]$ or $\matom[->]$ message.

% As well as retaining locality, notice that this extended choreography continues to be specification-preserving:
% the choreography's clauses differ from those of the specification only in using the decorated forms
% % the substitution of
% $\dec[<-]$, $\fail[->]$, and $\suc[->]$
% in place of
% % for
% their undecorated counterparts.


\subsubsection{Choreographies are not always unique}\label{sec:mult-chor-are}

As alluded to previously, multiple choreographies are possible for some specifications.

This is true of our binary counter specification, for instance.
(To simplify the example, we'll ignore decrements for now.)
In the $\inc[<-]$-choreography (\cref{sec:chor-example-counter}),
% the $\inc$ processes initiate the interaction but leave all remaining work to the $\eps$, $\bit{0}$, and $\bit{1}$ processes alone.
the counter's value is represented by a chain of $\eps$, $\bit{0}$, and $\bit{1}$ processes that are acted upon by $\inc[<-]$ messages.
%
% Alternatively, the $\inc$ processes could wait for $\eps$, $\bit{0}$, or $\bit{1}$ to initiate the interaction, but thereafter take full responsibility for its completion.
Alternatively, the counter's value could be represented by a sequence of $\eps[->]$, $\bit{0}[->]$, and $\bit{1}[->]$ messages; when fed such a message sequence, an $\inc$ process would emit another sequence that represents the result:
%
% Specifically, each $\eps$, $\bit{0}$, and $\bit{1}$ process sends an identifying message, $\eps[->]$, $\bit{0}[->]$, or $\bit{1}[->]$, to its right-hand neighbor and then terminates.
% If the neighbor is $\inc$, then, upon receiving the message, that $\inc$ completes the corresponding interaction.
% % takes responsibility for carrying out the corresponding clause of the specification.
\begin{equation}
  \!\begin{aligned}
    &\eps[->] \fuse \inc \lrimp \monad{\eps[->] \fuse \bit{1}[->]} \\
    &\bit{0}[->] \fuse \inc \lrimp \monad{\bit{1}[->]} \\
    &\bit{1}[->] \fuse \inc \lrimp \monad{\inc \fuse \bit{0}[->]}
      \,.
  \end{aligned}
\end{equation}
Once again, this choreography possesses the locality and specification-preserving properties.

% Owing to the difference in roles held by, these two choreographies have distinct flavors.
% These two choreographies
These two choreographies
% presented thus far
have distinct flavors, owing to the different process and message roles that they assign to the $\inc$ and $\eps$, $\bit{0}$, and $\bit{1}$ atoms.
The $\inc[<-]$-choreography has an object-oriented character: by sending an $\inc[<-]$ message, the increment method dispatches on the receiving object's class---either $\eps$, $\bit{0}$, or $\bit{1}$.
In contrast, this new $\bit{}[->]$-choreography has a functional character: $\inc$ is a function that receives its argument as a sequence of messages---either $\eps[->]$, $\bit{0}[->]$, or $\bit{1}[->]$.

% The increment method dispatches on 
% $\inc$ invokes the increment method on the neighboring object by sending an $\inc[<-]$ message
% There, the $\inc$ method sends an $\inc[<-]$ message like a method that dispatches on the class of the recipient object---either $\eps$, $\bit{0}$, or $\bit{1}$.
% Our first choreography has an object-oriented flavor, with $\inc$ like a method that dispatches on the class of the recipient object---either $\eps$, $\bit{0}$, or $\bit{1}$.
% In contrast, this second choreography has a more functional flavor, with 

% This alternate choreography has a funcitonal flavor: $\inc$ can be viewed as a function on the $\eps$-and-$\bit{}$ representation of data.
% In contrast, the previous choreography has a more object-oriented flavor

% The difference in sender and recipient between this alternate choreography and the previous one gives the two choreographies different flavors.
% In this alternate choreography
% In constrast, the previous choreography has a more object-oriented flavor, with $\inc$ being a method that dispatches on the class of the recipient---either $\eps$, $\bit{0}$, or $\bit{1}$.

% If we view the $\eps$ and $\bit{}$ processes as data, then this alternate choreography has a functional flavor.
% In contrast, the previous choreography has an object-oriented flavor, with a dynamic dispatch of $\inc[<-]$ on the recipient.

\subsubsection{Two non-choreographies}\label{sec:non-choreographies}

Another, slightly more complex reformulation of the binary counter specification chooses to treat the $\inc$ atom as a simple process, not a message:
\begin{equation}
  \!\begin{aligned}
    &\inc \lrimp \monad{\incm[<-]} \\[1.5\jot]
    % 
    &\eps \fuse \incm[<-] \lrimp \monad{\eps \fuse \bit{1}} \\
    &\bit{0} \fuse \incm[<-] \lrimp \monad{\bit{1}} \\
    &\bit{1} \fuse \incm[<-] \lrimp \monad{\inc \fuse \bit{0}}
      \,.
  \end{aligned}
\end{equation}
In fact, the $\inc$ process does nothing but send an $\incm[<-]$ message.

This signature is equivalent to the binary counter specification in that it ultimately exposes the same $\eps$, $\bit{}$, and $\inc$ behaviors.
However, it is \emph{not} specification-preserving under the informal definition that we have used thus far.
% In contrast with the choreographies, this implementation's main clauses are not a simple decoration of the specification's clauses: each of the specification's clauses is spread across several clauses here.
In contrast with the choreographies, this formulation does more than simply refine the specification by making the communication explicit: it introduces a new message-like atom, $\incm[<-]$, a new clause, $\inc \lrimp \monad{\incm[<-]}$, and modifies the existing clauses.






% Another, more complex implementation of the binary counter specification divides the work of completing the interaction among the $\eps$ and $\bit{}$ and (the continuation of) the $\inc$ processes:
% \begin{equation}
%   \!\begin{aligned}
%     &\inc \lrimp \monad{\inc[^l, <-] \fuse \inc[^r]} \\[1.5\jot]
%     % 
%     &\eps \fuse \inc[^l, <-] \lrimp \monad{\eps \fuse \eps[^r, ->]} \\
%     &\bit{0} \fuse \inc[^l, <-] \lrimp \monad{\bit{0}[^r, ->]} \\
%     &\bit{1} \fuse \inc[^l, <-] \lrimp \monad{\inc \fuse \bit{1}[^r, ->]} \\[1.5\jot]
%     % 
%     &\eps[^r, ->] \fuse \inc[^r] \lrimp \monad{\bit{1}} \\
%     &\bit{0}[^r, ->] \fuse \inc[^r] \lrimp \monad{\bit{1}} \\
%     &\bit{1}[^r, ->] \fuse \inc[^r] \lrimp \monad{\bit{0}} \,.
%   \end{aligned}
% \end{equation}
% In this implementation, $\inc$ first sends an $\inc[^l, <-]$ message to its left-hand neighbor and then waits for a response as process $\inc[^r]$.
% Upon receiving an $\inc[^l, <-]$ message, the recipient process, either $\eps$, $\bit{0}$, or $\bit{1}$, partially completes the interaction and sends an identifying message to its right-hand neighbor, which is necessarily an $\inc[^r]$ process.
% The $\inc[^r]$ process finishes the interaction once it receives the identifying message.

% This implementation is equivalent to the binary counter specification in that it ultimately exposes the same $\eps$, $\bit{}$, and $\inc$ processes.
% However, it is not specification-preserving under the informal definition that we have used thus far.
% % In contrast with the choreographies, this implementation's main clauses are not in one-to-one correspondence with those of the specification.
% % In contrast with the choreographies, this implementation's main clauses are not a simple decoration of the specification's clauses: each of the specification's clauses is spread across several clauses here.
% In contrast with the choreographies, this implementation does more than simply refine the specification by making the communication explicit: using a temporary $\inc[^r]$ process, it spreads each of the specification's clauses across several clauses.

So, this signature is not specification-preserving, and therefore not a choreography, for the binary counter specification.
But that doesn't mean that the programmer cannot achieve the same behavior anyway: the programmer is free to rewrite the \emph{specification} to incorporate the behavior at the specification level.
% Although this implementation is not specification-preserving for the binary counter specification, and therefore not a choreography, the programmer can nevertheless achieve the same behavior by changing the \emph{specification}.
If the specification is changed to be
\begin{equation}
  \!\begin{aligned}
    &\inc \lrimp \monad{\incm} \\[1.5\jot]
    % 
    &\eps \fuse \incm \lrimp \monad{\eps \fuse \bit{1}} \\
    &\bit{0} \fuse \incm \lrimp \monad{\bit{1}} \\
    &\bit{1} \fuse \incm \lrimp \monad{\inc \fuse \bit{0}}
      \,,
  \end{aligned}
\end{equation}
then the above signature is indeed a choreography for \emph{this} specification.

% Although this implimentation is not specification-preserving, and therefore not a choreography, for the binary counter specification (\cref{??}), the programmer can nevertheless acheive the same behavior by changing the \emph{specification}.
% Instead of using the binary counter specification from \cref{??}, the following specification could be used because one of its choreographies, which uses $\inc[^l, <-]$, $\eps[^r, ->]$, $\bit{0}[^r, ->]$, and $\bit{1}[^r, ->]$ messages, is nearly identical the disallowed implimentation.
% \begin{equation*}
%   % \!\begin{aligned}[t]
%   %   &\inc[^l] \lrimp \monad{\inc[^l, <-]} \\
%   %   &\eps[^r] \lrimp \monad{\eps[^r, ->]} \\
%   %   &\bit{0}[^r] \lrimp \monad{\bit{0}[^r, ->]} \\
%   %   &\bit{1}[^r] \lrimp \monad{\bit{1}[^r, ->]}
%   % \end{aligned}
%   % \qquad
%   \!\begin{aligned}[t]
%     &\inc \lrimp \monad{\inc[^l] \fuse \inc[^r]} \\
%     &\eps \fuse \inc[^l] \lrimp \monad{\eps \fuse \eps[^r]} \\
%     &\bit{0} \fuse \inc[^l] \lrimp \monad{\bit{0}[^r]} \\
%     &\bit{1} \fuse \inc[^l] \lrimp \monad{\inc \fuse \bit{1}[^r]} \\
%     &\eps[^r] \fuse \inc[^r] \lrimp \monad{\bit{1}} \\
%     &\bit{0}[^r] \fuse \inc[^r] \lrimp \monad{\bit{1}} \\
%     &\bit{1}[^r] \fuse \inc[^r] \lrimp \monad{\bit{0}} \,.
%   \end{aligned}
% \end{equation*}


% Although this implementation ultimately exposes the same $\eps$, $\bit{}$, and $\inc$ processes

% Even so\fxnote{\ \st{Even though this choreography introduces and auxiliary process atom}}, we still consider it to be a valid choreography: $\inc[']$ is only temporary, leaving the underlying specification fundamentally unchanged.

Another signature that is equivalent to the binary counter specification, in the sense that the two track the same value, is
\begin{equation}
  \!\begin{aligned}
    &\num{N} \fuse \inc[<-] \lrimp \monad{\num{(N{+}1)}} \,.
  \end{aligned}
\end{equation}
Nevertheless, we wouldn't consider this to be a choreography of the binary counter specification because, by using a single number held by $\num{}$ instead of a string of $\bit{}$s, it fundamentally alters the specification.
We would, however, consider this signature to be a choreography of a different, simple counter specification, namely $\num{N} \fuse \inc \lrimp \monad{\num{(N{+}1)}}$.

% % Although it is equivalent to the binary counter in the sense that it tracks the same value, we wouldn't consider the following program to be a choreography of the binary counter specification because it fundamentally alters the implementation by using a single $\num{}$ instead of a string of $\bit{}$s.
% The following program is also equivalent to the binary counter specification, in the sense that the two track the same value.
% Nevertheless, we wouldn't consider it to be a choreography of the binary counter because it fundamentally alters the specification by using a single number held by $\num{}$ instead of a string of $\bit{}$s.
% \begin{align*}
%   &\inc \lrimp \monad{\inc[<-]} \\
%   &\num{N} \fuse \inc[<-] \lrimp \monad{\num{(N{+}1)}}
% \end{align*}
% We would, however, consider it to be a choreography of a different, simple counter specification: $\num{N} \fuse \inc \lrimp \monad{\num{(N{+}1)}}$.



% \end{document}

%%% Local Variables:
%%% TeX-master: "choreographies"
%%% End:


% arara: pdflatex
% arara: biber
% arara: pdflatex
% arara: pdflatex
\documentclass[
  class=../hdeyoung-proposal,
  crop=false
]{standalone}

\usepackage{ordered-logic}
\usepackage{basic-atoms}

\NewDocumentCommand{\chor}{}{X}
\NewDocumentCommand{\spec}{}{\Sigma}

\NewDocumentCommand{\erasemsg}{m}{#1^{e}}

\NewDocumentCommand{\trans}{t* t+ o}{%
  \longrightarrow
  \IfBooleanT{#1}{^*}\IfBooleanT{#2}{^+}%
  \IfValueT{#3}{_{#3}}%
}

% \cs_new_protected:Nn \trans: {
%   \peek_meaning:nTF {*}
%     { \@@_trans_star: }
%     { \peek_meaning:nTF {+}
%         { \@@_trans_plus: }
%    \longrightarrow
%   \IfBooleanT{#1}{^*}\IfBooleanT{#2}{^+}%
%   \IfValueT{#3}{_{#3}}%
% }

\begin{document}

\subsection{Choreographies, formally}\label{sec:chor-formal}

Hopefully the preceding examples have given some intuition for what counts as a choreography.
To make the definition precise, we need only formalize the locality and specification-preserving properties.

We use the forward-chaining ordered logic programming language described in \cref{??}, with a few restrictions.


\subsubsection{Locality}\label{sec:locality}

Each clause must have the form $U^+ \lrimp \monad{A^+}$, where $U^+$ is an \vocab{uncurried local premise} that adheres to the following grammar:
\begin{alignat*}{2}
  A^- &::= L^+ \limp \monad{A^+} \mid R^+ \rimp \monad{A^+} \mid A^-_1 \with A^-_2 \\
  A^+ &::= \p^+ \mid \p[->]^+ \mid \p[<-]^+ \mid A^+_1 \fuse A^+_2 \mid \one \mid A^- \\
  U^+ &::= L^+ \fuse U^+ \mid \p^+ \mid U^+ \fuse R^+ \\
  L^+ &::= \p[->]^+ \mid L^+_1 \fuse L^+_2 \mid \one \\
  R^+ &::= \p[<-]^+ \mid R^+_1 \fuse R^+_2 \mid \one
\end{alignat*}
The grammar is a bit complicated, but the idea behind it is simple and matches the intuition behind locality:
an uncurried local premise $U^+$ contains exactly one process atom $\p^+$ that receives right-directed messages, $\p[->]^+$, from its left and left-directed messages, $\p[<-]^+$, from its right.

At the expense of a more complicated grammar, we could allow curried clauses, such as $\p[_1, ->]^+ \lrimp \p[_2, ->]^+ \limp \p^+ \rimp \p[_3, <-]^+ \rimp \monad{A^+}$.
Uncurrying clauses would not seem to place a large burden on the programmer, for it is easy enough to write $\q[->]^+ \fuse \p[->]^+ \fuse \rr^+ \fuse \s[<-]^+ \lrimp \monad{A^+}$, and this detail is anyway orthogonal to what follows.

\subsubsection{Specification-preserving}\label{sec:spec-pres}

\begin{definition}[Specification-preserving]
  An ordered logic program $\chor$ is \vocab{specification-preserving} for specification $\spec$ if:
  \begin{enumerate}
  \item for each step $\octx \trans[\spec] \octx'$ in the specification $\spec$, there is a non-empty trace $\octx \trans+[\chor] \octx'$ in the program $\chor$; and
  \item for each step $\octx \trans[\chor] \octx'$ in the program $\chor$, either $\erasemsg{(\octx)} = \erasemsg{(\octx')}$ or there is a step $\erasemsg{(\octx)} \trans[\spec] \erasemsg{(\octx')}$ in the specification $\spec$.
  \end{enumerate}
\end{definition}

\end{document}

%%% Local Variables:
%%% TeX-master: "choreographies"
%%% End:


\end{document}

\section{Compiling choreographies}\label{sec:compile-choreo}

\section{Compiling choreographies}\label{sec:compile-choreo}

\NewDocumentCommand{\compile}{m m m m}{%
  \llbracket #2\rrbracket^{#1}_{#3} = #4%
}

\begin{mathpar}
  \infer{\compile{d}{H_1 \fuse H_2}{c}{\mbind{\bv{c'} <- \mspawn{P_1} <- d; P_2}}}{
    \compile{d}{H_1}{c'}{P_1} &
    \compile{c'}{H_2}{c}{P_2}}
  \and
  \infer{\compile{d}{\one}{c}{\mfwd{c <- d}}}{
    }
  \and
  \infer{\compile{d}{p}{c}{\mbind{c <- p <- d}}}{
    }
\end{mathpar}

\NewDocumentCommand{\compilectx}{m m m m}{%
  \compile{#1}{#2}{#3}{#4}%
}

\begin{mathpar}
  \infer{\compilectx{d}{\octx_1, \octx_2}{c}{\existq c'. A^+_1 \tensor A^+_2}}{
    \compilectx{d}{\octx_1}{c'}{A^+_1} &
    \compilectx{c'}{\octx_2}{c}{A^+_2}}
  \and
  \infer{\compilectx{d}{\octxe}{c}{c \eq d}}{
    }
  \\
  \infer{\compilectx{d}{H}{c}{\exec{P}}}{
    \compile{d}{H}{c}{P}}
\end{mathpar}


\subsection{Correctness}\label{sec:correctness}

\begin{itemize}
\item Well-typed choreographies compile to well-typed processes.
\item Executions of a choreography and its compiled form are bisimilar.
\end{itemize}

%%% Local Variables:
%%% TeX-master: "proposal"
%%% End:


\section{Proposed work}\label{sec:extensions}

% % arara: lualatex
% % arara: lualatex
% % arara: biber
% % arara: lualatex
% % arara: lualatex
% % \documentclass{../hdeyoung-proposal}
% \documentclass[
%   class=../hdeyoung-proposal,
%   crop=false
% ]{standalone}


% \usepackage{linear-logic}
% \usepackage{ordered-logic}
% \usepackage{proof}
% \usepackage{mathpartir}

% \usepackage{tikz}
% \usetikzlibrary{shapes.misc,graphs,quotes,graphdrawing}
% \usegdlibrary{trees}

% \usepackage{scalerel}

% \ExplSyntaxOn

% % \DeclarePairedDelimiter \parens { \lparen } { \rparen }
% \DeclarePairedDelimiter \bagged:wn { \lbag } { \rbag }
% \NewDocumentCommand{ \bagged }{ s o m o }
%   {
%     \IfBooleanTF {#1}
%       { \bagged:wn* {#3} }
%       {
%         \IfValueTF {#2}
%           { \bagged:wn[#2] {#3} }
%           { \bagged:wn {#3} }
%       }
%     \IfValueT {#4} { \sb{#4} }
%   }


% \NewDocumentCommand \oseq { >{ \SplitArgument{1}{|-} } m }
%   { \oseq:nn #1 }
% \cs_new:Npn \oseq:nn #1#2 { \oseq_ctxs:n {#1} \vdash #2 }
% \cs_new:Npn \oseq_ctxs:n #1 {
%   \seq_set_split:Nnn \l_tmpa_seq {;} {#1}
%   \seq_use:Nn \l_tmpa_seq { \mathrel{;} }
% }

% \NewDocumentCommand \procof { m m } { #1 \dblcolon #2 }
% \NewDocumentCommand \hypof { m } { #1 }


% \NewDocumentCommand \cut { m } { \text{\textsc{\MakeLowercase{Cut}}}\sb{#1} }
% \NewDocumentCommand \id { m } { \text{\textsc{\MakeLowercase{Id}}}\sb{#1} }

% \NewDocumentCommand \comp { >{ \SplitArgument{1}{|} } m }
%   { \comp:nn #1 }
% \cs_new:Npn \comp:nn #1#2 { #1 \parallel #2 }

% \NewDocumentCommand \fwd {} { \mathord{\leftrightarrow} }


% \RenewDocumentCommand \with { s }
%   { \IfBooleanTF {#1} \with:n \with: }
% \cs_new:Npn \with:n #1 {
%   \mathord{\binampersand}
%   \bagged {
%     \seq_set_split:Nnn \l_tmpa_seq {,} {#1}
%     \seq_use:Nn \l_tmpa_seq {,}
%   }
% }
% \cs_new:Npn \with: { \mathbin{\binampersand} }

% \NewDocumentCommand \ssor { s }
%   { \IfBooleanTF {#1} \ssor:n \ssor: }
% \cs_new:Npn \ssor:n #1 {
%   \mathord{\ssor:}
%   \bagged {
%     \seq_set_split:Nnn \l_tmpa_seq {,} {#1}
%     \seq_use:Nn \l_tmpa_seq {,}
%   }
% }
% \cs_new:Npn \ssor: { \oplus }

% \NewDocumentCommand \caseR { s m }
%   {
%     \IfBooleanTF {#1}
%       { \case:nNn { \mathsf{caseR} } \bagged {#2} }
%       { \case:nNn { \mathsf{caseR} } \parens {#2} }
%   }
% \NewDocumentCommand \caseL { s m }
%   {
%     \IfBooleanTF {#1}
%       { \case:nNn { \mathsf{caseL} } \bagged {#2} }
%       { \case:nNn { \mathsf{caseL} } \parens {#2} }
%   }
% \cs_new:Npn \case:nNn #1#2#3 {
%   #1 \mskip\thinmuskip
%   #2 {
%     \seq_set_split:Nnn \l_tmpa_seq {|} {#3}
%     \seq_clear:N \l_tmpb_seq
%     \seq_map_inline:Nn \l_tmpa_seq
%       { \seq_put_right:Nn \l_tmpb_seq { \case_branch:n {##1} } }
%     \seq_use:Nn \l_tmpb_seq { \talloblong }
%   }
% }
% \cs_new:Npn \case_branch:n #1 { \case_branch_aux:w #1 \q_stop }
% \cs_new:Npn \case_branch_aux:w #1 => #2 \q_stop {
%   #1 \Rightarrow #2
% }

% \NewDocumentCommand \selectL { >{ \SplitArgument{1}{;} } m }
%   { \select:nnn { \mathsf{selectL} } #1 }
% \NewDocumentCommand \selectR { >{ \SplitArgument{1}{;} } m }
%   { \select:nnn { \mathsf{selectR} } #1 }
% \cs_new:Npn \select:nnn #1#2#3 {
%   \!\mathord{}\mathop{#1} #2 ; #3
% }


% \NewDocumentCommand \inj { m } { \mathsf{in}\sb{#1} }

% \NewDocumentCommand \inl {} { \inj{ \mathsf{1} } }
% \NewDocumentCommand \inr {} { \inj{ \mathsf{2} } }


% \RenewDocumentCommand \one {} { \mathord { \mathbf{1} } }

% \NewDocumentCommand \quitR {} { \mathsf{quitR} }
% \NewDocumentCommand \waitL { m } { \mathsf{waitL} ; #1 }


% \NewDocumentCommand \rrule { o m } {
%   \IfValueTF {#1}
%     { \rrule:nn {#2} {#1} }
%     { \rrule:n {#2} }
% }
% \cs_new:Npn \rrule:nn #1#2 { {#1}\text{\textsc{\MakeLowercase{R}}}\sb{#2} }
% \cs_new:Npn \rrule:n #1 { {#1}\text{\textsc{\MakeLowercase{R}}} }

% \NewDocumentCommand \lrule { o m } {
%   \IfValueTF {#1}
%     { \lrule:nn {#2} {#1} }
%     { \lrule:n {#2} }
% }
% \cs_new:Npn \lrule:nn #1#2 { {#1}\text{\textsc{\MakeLowercase{L}}}\sb{#2} }
% \cs_new:Npn \lrule:n #1 { {#1}\text{\textsc{\MakeLowercase{L}}} }


% \NewDocumentCommand \exec { } { \mathsf{exec} \mskip\thinmuskip }
% \NewDocumentCommand \msg { } { \mathsf{msg} \mskip\thinmuskip }

% \ExplSyntaxOff




% \addbibresource{../proposal.bib}

% \NewDocumentCommand{\ie}{}{i.e.}


% \usepackage{listings}
% \crefname{listing}{listing}{listings}
% \Crefname{listing}{Listing}{Listings}

% \newlength{\mywidth}
% \settowidth{\mywidth}{\ttfamily A}
% \lstset{basicstyle=\ttfamily, basewidth=\mywidth}

% \captionsetup[lstlisting]{%
%   box=colorbox, boxcolor=gray,
%   font={normalfont, sf, color=white},
%   labelfont=bf,
%   justification=justified, singlelinecheck=false
% }

% \lstnewenvironment{sillcode}[1][]
%   {\lstset{language={},frame=bottomline,framerule=0.8ex,rulecolor=\color{gray},float,#1}}%
%   {}

% \lstnewenvironment{sillcode*}[1][]
%   {\lstset{language={},#1}}%
%   {}

% \NewDocumentCommand{\sillinline}{o}{%
%   \IfValueTF{#1}{\lstinline[#1]}{\lstinline}%
% }


% \NewDocumentCommand{\pctx}{}{\Psi}
% \ExplSyntaxOn
% \NewDocumentCommand{\ctxmonad}{>{\SplitArgument{1}{<-}}m}{
%   \{\use_ii:nn #1 \vdash \use_i:nn #1\}
% }
% \NewDocumentCommand \spawn { >{ \SplitArgument{1}{;} } m } { \spawn:nn #1 }
% \cs_new:Npn \spawn:nn #1#2 {
%   \mathsf{spawn}
%   \tl_if_empty:nF {#1} {
%     \mskip\thinmuskip #1 ; #2
%   }
% }
% \NewDocumentCommand{\mbind}{>{\SplitArgument{1}{;}}m}{
%   \use_i:nn#1 ; \use_ii:nn#1
% }
% \NewDocumentCommand{\mletrec}{m m}{
%   \mathsf{letrec} \mskip\thinmuskip #1 \mskip\thinmuskip \mathsf{in} \mskip\thinmuskip #2
% }
% \NewDocumentCommand{\mprocdef}{m}{
%   #1
% }
% \ExplSyntaxOff




% \DeclareAcronym{BHK}{
  short = BHK,
  long  = Brouwer-Heyting-Kolmogorov
}

\DeclareAcronym{JILL}{
  short = JILL,
  long  = judgmental intuitionistic linear logic
}

\DeclareAcronym{ILL}{
  short = ILL,
  long  = intuitionistic linear logic
}

\DeclareAcronym{SML}{
  short = SML,
  long  = Standard ML
}

\DeclareAcronym{SOS}{
  short = SOS,
  short-indefinite = an,
  long = structural operational semantics
}

\DeclareAcronym{SSOS}{
  short = SSOS,
  short-indefinite = an,
  long = substructural operational semantics
}

\DeclareAcronym{SILL}{
  short = SILL,
  long = session-typed intuitionistic linear logic
}

\DeclareAcronym{SISLL}{
  short = singleton \acs{SILL},
  long = session-typed intuitionistic singleton linear logic
}

\DeclareAcronym{CLF}{
  short = CLF,
  long = the Concurrent Logical Framework
}



% \begin{document}

\section{Proposed work}\label{sec:proposed-work}

In this document, we have shown how the session types that arise from singleton linear logic form a bridge between a class of ordered logical specifications and well-typed processes---between proof-construction-as-computation and proof-reduction-as-computation.
% There are three areas of proposed work.
Most of the proposed work involves generalizing this connection along several dimensions: 
\begin{enumerate*}[label=\emph{\roman*}), itemjoin={{; }}, itemjoin*={{; and }}]
\item a more expressive logic for specifications
% ---relating \emph{linear logical}, not just ordered logical, specifications to well-typed processes;
\item a more expansive translation that covers generative invariants
\item a more permissive session-type system
% the strength of the session-type system---relating a larger class of logical specifications to untyped or more weakly typed processes.
% \item a coarser equivalence for the specification-preserving property
\end{enumerate*}.
We now outline that proposed work.
% In this \lcnamecref{sec:proposed-work},


% The primary area of proposed work is to generalize this connection:
% instead of showing only how certain ordered logical specifications can translate to process chains that are typed by singleton linear logic, I propose to show how certain \emph{linear logical} specifications can translate to process trees that are typed by 

%  to (a class of) linear logical specifications and \ac{SILL} process definitions typed in linear logic.
% % To defend the proposed thesis, this connection must be extended to (a class of) linear logical specifications and \ac{SILL} process definitions typed in linear logic.

\subsection{From ordered logical to linear logical specifications}\label{sec:from-ordered-logical}

The primary area of proposed work is to generalize the logic used for specifications from ordered logic to the more expressive linear logic.
The process chains used in this proposal will be correspondingly generalized to \citeauthor{Caires+:MSCS13}'s~\autocite*{Caires+:MSCS13} \ac{SILL} process trees.
We'll motivate this generalization with an example: addition of binary representations.

\paragraph{Logical specification.}
By adapting ideas from Turing machines, it is possible---though undoubtedly awkward---to give an ordered logical specification for adding two binary numbers.
First, the numbers are arranged end-to-end, separated by a $\plus$ atom and terminated by an $\equals$ atom.
For instance, the string
\begin{equation*}
  \eps \fuse \bit{1} \fuse \bit{0} \fuse \plus \fuse \bit{1} \fuse \bit{0} \fuse \equals
\end{equation*}
represents a request to evaluate $2+2$.
Next, repeatedly decrement the second number and increment the first number.
When the second number reaches $0$, the first number holds the desired sum.
%
\begin{figure}
  \begin{equation*}
    \begin{alignedat}{2}
      &\equals \lrimp \monad{\dec \fuse \equals[']} \\[1.5\jot]
      % 
      &\bit{0} \fuse \dec \lrimp \monad{\dec \fuse \bit[']{0}} &\quad\enskip& \bit{0} \fuse \skp \lrimp \monad{\skp \fuse \bit{0}} \\
      &\bit{1} \fuse \dec \lrimp \monad{\skp \fuse \bit{0} \fuse \ok} && \bit{1} \fuse \skp \lrimp \monad{\skp \fuse \bit{1}} \\
      &\plus \fuse \dec \lrimp \monad{\fail} && \plus \fuse \skp \lrimp \monad{\inc \fuse \plus} \\[1.5\jot]
      % 
      &\ok \fuse \smash{\bit[']{0}} \lrimp \monad{\bit{1} \fuse \ok} && \fail \fuse \smash{\bit[']{0}} \lrimp \monad{\fail} \\
      &\ok \fuse \smash{\equals[']} \lrimp \monad{\equals} && \fail \fuse \smash{\equals[']} \lrimp \monad{\one}
    \end{alignedat}
  \end{equation*}
  \caption{An ordered logical specification of Turing-machine--like binary addition\label{fig:turing-binary-add}}
\end{figure}
%
The ordered logical specification
% and corresponding well-typed process definitions
of this addition algorithm is shown in \cref{fig:turing-binary-add}.


% \begin{sillcode}
%   plus =
%   { caseR of
%       dec => selectR fail; <->
%     | skip_inc => selectL inc; plus }

%   bit0 =
%   { caseR of
%       dec => selectL dec; bit0'
%     | skip_inc => selectL skip_inc; bit0 }

%   bit0' =
%   { caseL of
%       ok => selectR ok; bit1
%     | fail => selectR fail; <-> }

%   bit1 =
%   { caseR of
%       dec => selectR ok; selectL skip_inc; bit0
%     | skip_inc => selectL skip_inc; bit1 }

%   equals =
%   { selectL dec; equals' }

%   equals' =
%   { caseL of
%       ok => equals
%     | fail => <-> }
% \end{sillcode}


Unfortunately, this algorithm is not especially efficient: it takes $\Omega(N\log N)$ work to compute $M+N$.
It would be better to add the two binary representations bit-by-bit using the usual grade-school algorithm.
However, bit-by-bit addition demands that we can \emph{locally} access the least significant bit of each number and, separately, produce output bits---which is not possible in an ordered logical specification.
% A better algorithm would add the two numbers bit-by-bit.

It is possible, however, in a \emph{destination-passing} linear logical specification~\autocite{Cervesato+:CMU02}.
Even without the ordering constraint, a tree structure can be recovered via destinations that thread the $\bit{}$ atoms together with a $\plus$ parent atom.
% into linked-list--like structures that are joined at a $\plus$ parent.
Pictorially, the request to compute $2+2$ would be expressed as the state
\begin{equation*}
  \!\begin{aligned}[c]
    \eps(c_2) \tensor \bit{1}(c_2 , c_1) \tensor \bit{0}(c_1 , c_0) & \\
    \eps(d_2) \tensor \bit{1}(d_2 , d_1) \tensor \bit{0}(d_1 , d_0) &
  \end{aligned}
  \tensor \plus(c_0 , d_0 , c)
\end{equation*}
where $c$ and the $c_i$s and $d_j$s are all destinations and where the sum will be output at destination $c$.
Thus, the destination-passing rule for adding two numbers that both end in $\bit{0}$ is 
\begin{equation*}
  \bit{0}(C_1 , C_0) \tensor \bit{0}(D_1 , D_0) \tensor \plus(C_0 , D_0 , C)
    \lolli \monad{\exists c'_0.\, \plus(C_1 , D_1 , c'_0) \tensor \bit{0}(c'_0 , C)}
  \,.
\end{equation*}
It says that if both inputs end in $\bit{0}$, then their sum also ends in $\bit{0}$, with the more significant bits obtained by inductively adding the more significant bits of the two inputs.
When this rule is applied to the above state, the state changes and the first bit of output is produced:
\begin{equation*}
  \!\begin{aligned}[c]
    \eps(c_2) \tensor \bit{1}(c_2 , c_1) & \\
    \eps(d_2) \tensor \bit{1}(d_2 , d_1) &
  \end{aligned}
  \tensor \plus(c_1 , d_1 , c_0') \tensor \bit{0}(c_0' , c)
  \,.
\end{equation*}


% \begin{figure}
%   \begin{equation*}
%     \begin{lgathered}
%       \eps(C_0) \tensor \inc(C_0 , C) \lolli \monad{\eps(C)} \\
%       \bit{0}(C_1 , C_0) \tensor \inc(C_0 , C) \lolli \monad{\bit{1}(C_1 , C)} \\
%       \bit{1}(C_1 , C_0) \tensor \inc(C_0 , C) \lolli \monad{\exists c'_0.\, \inc(C_1 , c'_0) \tensor \bit{0}(c'_0 , C)}
%     \end{lgathered}
%   \end{equation*}

%   \begin{equation*}
%     \begin{lgathered}
%       \eps(C_0) \tensor \eps(D_0) \tensor \plus(C_0 , D_0 , C) \lolli \monad{\eps(C)} \\
%       \eps(C_0) \tensor \bit{0}(D_1 , D_0) \tensor \plus(C_0 , D_0 , C) \lolli \monad{\bit{0}(D_1 , C)} \\
%       \eps(C_0) \tensor \bit{1}(D_1 , D_0) \tensor \plus(C_0 , D_0 , C) \lolli \monad{\bit{1}(D_1 , C)} \\
%       %
%       \bit{0}(C_1 , C_0) \tensor \bit{0}(D_1 , D_0) \tensor \plus(C_0 , D_0 , C) \lolli \monad{\exists c'_0.\, \plus(C_1 , D_1 , c'_0) \tensor \bit{0}(c'_0 , C)} \\
%       \bit{0}(C_1 , C_0) \tensor \bit{1}(D_1 , D_0) \tensor \plus(C_0 , D_0 , C) \lolli \monad{\exists c'_0.\, \plus(C_1 , D_1 , c'_0) \tensor \bit{1}(c'_0 , C)} \\
%       \bit{1}(C_1 , C_0) \tensor \bit{0}(D_1 , D_0) \tensor \plus(C_0 , D_0 , C) \lolli \monad{\exists c'_0.\, \plus(C_1 , D_1 , c'_0) \tensor \bit{1}(c'_0 , C)} \\
%       %
%       \bit{1}(C_1 , C_0) \tensor \bit{1}(D_1 , D_0) \tensor \plus(C_0 , D_0 , C) \lolli \monad{\exists c'_0, c''_0.\, \plus(C_1 , D_1 , c'_0) \tensor \inc(c'_0 , c''_0) \tensor \bit{0}(c''_0 , C)}
%     \end{lgathered}
%   \end{equation*}
% \end{figure}


\paragraph{Concurrent processes.}
Now, let's consider how we might add two binary numbers using \ac{SILL} process trees.
Suppose that we represent each number as before---each number is a chain of $\bit{}$ processes (a degenerate process subtree if you will)---and that we include a $\plus$ parent process that uses the two numbers to offer the sum.
For example, the following process network represents a request to compute $2+2$.%
\footnote{The process names have been abbreviated to $\mathsf{e}$, $\mathsf{0}$, $\mathsf{1}$, and $\mathsf{+}$ for this picture.}
\begin{equation*}
  \begin{tikzpicture}[channel/.style = {text depth=0, midway, sloped, above}]
    \graph [
      tree layout, grow=left, typeset=$\mathsf{\tikzgraphnodetext}$, % empty nodes,
      nodes={
        rounded rectangle, rounded rectangle left arc=none,
        draw, minimum size=3ex,
      },
      edges={-},
      simple
    ] {
      / [draw=none] <-["$\scriptstyle c$"' channel]
      p / "\mathord{\mathclap{+}}" <- { [name separator=]
        { [name=c] 0 <-["$\scriptstyle c_1$"' channel] 1 <-["$\scriptstyle c_2$"' channel] e } ,
        { [name=d] 0 <-["$\scriptstyle d_1$"' channel] 1 <-["$\scriptstyle d_2$"' channel] e }
      };

      c0 ->["$\scriptstyle c_0$" channel] p;
      d0 ->["$\scriptstyle d_0$" channel] p;
    };
  \end{tikzpicture}
\end{equation*}
where the $c_i$s and $d_j$s are channels.
Notice the remarkable similarity of this network with the initial linear logical state shown above: destinations become channels and atoms become processes.

We would expect the $\plus$ process to be implemented in such a way that the above process network eventually transforms to the following network.
\begin{equation*}
  \begin{tikzpicture}[channel/.style = {text depth=0, midway, sloped, above}]
    \graph [
      tree layout, grow=left, typeset=$\mathsf{\tikzgraphnodetext}$, % empty nodes,
      nodes={
        rounded rectangle, rounded rectangle left arc=none,
        draw, minimum size=3ex,
      },
      edges={-},
      simple
    ] {
      / [draw=none] <-["$\scriptstyle c$"' channel]
      0 <-["$\scriptstyle c_0'$"' channel]
      p / "\mathord{\mathclap{+}}" <- { [name separator=]
        { [name=c] 1 <-["$\scriptstyle c_2$"' channel] e } ,
        { [name=d] 1 <-["$\scriptstyle d_2$"' channel] e }
      };

      c1 ->["$\scriptstyle c_1$" channel] p;
      d1 ->["$\scriptstyle d_1$" channel] p;
    };
  \end{tikzpicture}
\end{equation*}
Once again, there is a remarkable similarity between this network and the linear logical state after producing one output bit.

The proposed work is to make these similarities precise.
Just as in this document, the overall goal will be to identify a class of linear logical specifications that can be translated to \ac{SILL} process trees.
This item of proposed work is of primary importance.

\paragraph{Specific goals.}
This proposed work involves several components.
\begin{itemize}
\item \emph{Identify the class of linear logical specifications that act as choreographies.}
  Not all linear logical specifications will describe process-like behaviors.
  As in the ordered case, choreographies will need to be both local and specification-preserving.
  Now, however, locality depends not on adjacency in the ordered context, but on sharing a destination.

  The key challenge here, therefore, will be to ensure that destinations are used in a channel-like way within the choreography.
  Each atom should \enquote{offer} along one destination and \enquote{use} possibly several distinct destinations, and each destination should have one occurrence as an \enquote{offer} and one occurrence as a \enquote{use}.
  The machinery of destination uniqueness and index sets~\autocite{Simmons:CMU12} will likely be useful here.

\item \emph{Develop a translation of choreographies to \ac{SILL} processes.}
  In addition to translating destinations to channels, the main challenges here will be expanding the class of choreographies to allow translation to processes of $\tensor$, $\lolli$, and $\bang$ type.
  The $\tensor$ and $\lolli$ types are not possible in singleton linear logic (as mentioned in \cref{sec:other-session-types}) and so nothing similar was considered in the translation of ordered choreographies to process chains.  
  The $\bang$ type was, by choice, not considered in the ordered translation to keep the initial development simple.  
  
\item \emph{Give a type system for choreographies.}
  I expect to follow the same pattern as in the ordered case: derive the choreography typing rules from the process typing rules by looking at the process to which a choreography translates.
  While everything will be notationally more complex, I do not expect many surprises here.

\item \emph{Relate the results for ordered logic to those for linear logic.}
  \Textcite{Simmons+Pfenning:HOSC11} show how to encode ordered logical specifications in linear logic using destinations.
  For instance, under their destination-adding translation, the clause for incrementing $\bit{1}$ from our $\inc[<-]$-choreography becomes the following linear logical clause.
  \begin{equation*}
    \bit{1} \fuse \inc[<-] \lrimp \monad{\inc[<-] \fuse \bit{0}}
    \qquad\mathord{\leftrightsquigarrow}\qquad
    \!\begin{aligned}[t]
      \MoveEqLeft[1]
      \bit{1}(C_1, C_0) \tensor \inc[<-](C_0, C) \\[-0.5\jot]
        &\lolli \monad{\exists c'_0.\, \inc[<-](C_1, c'_0) \tensor \bit{0}(c'_0, C)}
    \end{aligned}
  \end{equation*}
  There should be a similar \enquote{channel-adding} translation from \ac{SISLL} process chains to \ac{SILL} processes.
  \begin{equation*}
    \bit{1} = \caseR{\inc[<-] => \selectL{\inc[<-]; \bit{0}}}
    \qquad\mathord{\leftrightsquigarrow}\qquad
 %   \!\begin{aligned}[t]
%      \MoveEqLeft[1]
      \begin{array}[t]{@{}l@{}}
      c \shortleftarrow \bit{1} \shortleftarrow d = {}\\
        \mathsf{case}\,c\,\mathsf{of}\\
        \quad\inc[<-] \Rightarrow \begin{array}[t]{@{}l@{}}
        \mathsf{select}\,d\,\inc[<-];\\
        c \shortleftarrow \bit{0} \shortleftarrow d
      \end{array}
      \end{array}
  %  \end{aligned}
  \end{equation*}
  Moreover, the translation from choreography to process should respect the destination-adding translation: adding destinations to an ordered choreography and then translating it to a process should give the same result as first translating the choreography to a process chain and then adding channels.
  This will serve as a sanity check on our design of the translation from linear choreographies to \ac{SILL} processes.
\end{itemize}


\subsection{Generative invariants as session types}

In this proposal, we have used the non-modal fragment of ordered logic to specify concurrent systems, whereas that fragment of ordered logic was originally developed by \textcite{Lambek:AMM58} to describe sentence structure.
However, these two modes of use of ordered logic are not as different as they might first appear.

Recall that, in our running example of an incrementable binary counter, the counter is represented as a string of $\bit{0}$, $\bit{1}$, and $\inc$ atoms terminated at the most significant end by an $\eps$.
More precisely, a string is a well-formed binary counter if it can be generated from the $\Cntr$ nonterminal by the context-free grammar
\begin{equation*}
  \Cntr ::= \eps \mid \Cntr \fuse \bit{0} \mid \Cntr \fuse \bit{1} \mid \Cntr \fuse \inc
  \,,
\end{equation*}
which is notation for four distinct productions.

Building on \citeauthor{Lambek:AMM58}'s work, the same context-free grammar can be described in ordered logic using \vocab{generative invariants}~\autocite{Simmons:CMU12}.
Each production in the grammar (below, left) becomes a clause (below, right), with the atomic proposition $\cntr$ acting as the nonterminal:
\begin{equation*}
  \left.
  \!\begin{aligned}
    &\Cntr \to \eps \\
    &\Cntr \to \Cntr \fuse \bit{0} \\
    &\Cntr \to \Cntr \fuse \bit{1} \\
    &\Cntr \to \Cntr \fuse \inc
  \end{aligned}
  \qquad\middle\vert\qquad
  \!\begin{aligned}
    &\cntr \lrimp \monad{\eps} \\
    &\cntr \lrimp \monad{\cntr \fuse \bit{0}} \\
    &\cntr \lrimp \monad{\cntr \fuse \bit{1}} \\
    &\cntr \lrimp \monad{\cntr \fuse \inc}
    \,.
  \end{aligned}
  \right.
\end{equation*}
Just as all binary counters are generated from the $\Cntr$ nonterminal according to the above productions, so too are all binary counters generated as maximal rewritings of the $\cntr$ atom according to these clauses:
%
\begin{definition}[Counter well-formedness]\label{def:counter-wf}
  String $S$ is a well-formed counter if $S$ is a maximal rewriting of $\cntr$ under the signature $\sig_{\cntr}$, that is, if $\cntr \trans+[\sig_{\cntr}] S \ntrans[\sig_{\cntr}]$.
\end{definition}
%
\noindent
For example, the maximal trace
\begin{equation*}
  \mathul{\cntr}
    \trans[\sig_{\cntr}] \mathul{\cntr} \fuse \inc
    \trans[\sig_{\cntr}] \mathul{\cntr} \fuse \bit{1} \fuse \inc
    \trans[\sig_{\cntr}] \eps \fuse \bit{1} \fuse \inc
    \ntrans[\sig_{\cntr}]
\end{equation*}
witnesses that $\eps \fuse \bit{1} \fuse \inc$ is a well-formed binary counter.

As observed by~\textcite{Simmons:CMU12}, generative invariants like $\sig_{\cntr}$ serve a similar purpose for ordered logical specifications as types do for functional programs: both describe the valid states and enable preservation and progress properties for their respective notions of computation.
%
% \begin{theorem*}[Safety of $\inc$]\leavevmode
%   \begin{itemize}[nosep]
%   \item If $S$ is a well-formed counter and $S \trans[\sig_{\inc}] S'$, then $S'$ is a well-formed counter.
%   \item If $S$ is a well-formed counter, then either $S$ is $\inc$-free or $S \trans[\sig_{\inc}] S'$.
%   \end{itemize}
% \end{theorem*}
%

% \noindent
Given the translation from choreographies (i.e., ordered logical specifications) to well-typed processes that was presented in \cref{sec:translation}, it's thus natural to ask how that translation interacts with a generative invariant.
Being the choreography's \enquote{type}, does the generative invariant become the process's session type?

It appears that the answer is likely yes.  Compare, for example, the generative invariant for the $\inc[<-]$-choreography with the types of the \sillinline`eps`, \sillinline`bit0`, and \sillinline`bit1` processes and the recursive type \sillinline`Cntr` from \cref{sec:exampl-binary-count}:
% It appears that the generative invariant for a choreography is closely connected to the session type of the process that results from translating the choreography.
% Compare the generative invariant for the $\inc[<-]$-choreography with the types of the \sillinline`eps`, \sillinline`bit0`, and \sillinline`bit1` processes and the recursive type \sillinline`Cntr` from \cref{?}:
\begin{equation*}
  \left.
  \!\begin{aligned}
    &\cntr \lrimp \monad{\eps} \\
    &\cntr \lrimp \monad{\cntr \fuse \bit{0}} \\
    &\cntr \lrimp \monad{\cntr \fuse \bit{1}} \\
    &\cntr \lrimp \monad{\cntr \fuse \inc[<-]}
  \end{aligned}
  \qquad\middle\vert\qquad
  \!\begin{aligned}
    &\text{\sillinline`eps : \{ |- Cntr \}`} \\
    &\text{\sillinline`bit0 : \{ Cntr |- Cntr \}`} \\
    &\text{\sillinline`bit1 : \{ Cntr |- Cntr \}`} \\
    &\text{\sillinline`stype Cntr = &\{ inc: Cntr \}`}
    \,.
  \end{aligned}
  \right.
\end{equation*}
Similar correspondences between generative invariants and session types exist for the $\dec[<-]$-choreography, the $\bit{}[->]$-choreography, and all other examples that we've considered.
It seems much too tantalizing to be pure coincidence.

Therefore, if all else goes smoothly, I propose to develop a translation of generative invariants to session types and prove that it is respected by the translation from choreographies to processes.
I plan to follow the pattern of this thesis proposal, first developing the translation for the special case of ordered generative invariants before extending the results to linear generative invariants.
%
This item of proposed work is of somewhat lesser importance than the generalization from ordered logic to linear logic for specifications, but has appeal in giving a more compelling explanation of the choreography types presented in \cref{sec:chor-types}.

% C(c) -> ?d. C(d) * 0(d,c)
% C(c) -> ?d0,d1. C(d0) * C(d1) * a(d0,d1,c)

% a : {c:C -| d0:C , d1:C}


% C -> e
% C -> C * 0
% C -> C * 1

% C = +{e: 1, 0: C, 1: C}
% i : C |- C


% The generative invariant describes the valid choreography states; the session type describes the valid process states.

% C -> {e}
% C -> {C * 0}
% C -> {C * 1}
% C -> {C * i}
% C' -> {C * d}
% C' -> {C * s}
% C' -> {C * z}
% C' -> {C' * 0'}
% C' -> {C' * 1'}

% C = &{i: C, d: C'}
% C' = +{s: C, z: C}






% % For example, the maximal trace
% % \begin{equation*}
% %   \mathul{\cntr}
% %     \trans[\sig_{\cntr}] \mathul{\cntr} \fuse \inc
% %     \trans[\sig_{\cntr}] \mathul{\cntr} \fuse \bit{1} \fuse \inc
% %     \trans[\sig_{\cntr}] \eps \fuse \bit{1} \fuse \inc
% %     \ntrans[\sig_{\cntr}]
% %   \,.
% % \end{equation*}
% % witnesses that $\eps \fuse \bit{1} \fuse \inc$ is a well-formed binary counter.
% % Note how important the choice of signature is: if we also allowed increment clauses from $\sig_{\inc}$ here, this trace would no longer be maximal.

% \begingroup
%   \RenewPredicate{\cntr}[Cntr]{1}%
% In fact, generative signatures generalize context-free grammars.
% For example, here $\cntr{}$ could be predicated on a natural number, effectively generalizing the context-free grammar to have a countably infinite family of nonterminals:
% % One slight generalization would be to predicate $\cntr{}$ on a natural number, effectively giving a countably infinite family of nonterminals:
% \begin{equation*}
%   \sig_{\cntr{}} =
%   \!\begin{aligned}[t]
%     &\cntr{0} \lrimp \monad{\eps} \,, \\
%     &\cntr{(2N)} \lrimp \monad{\cntr{N} \fuse \bit{0}} \,, \\
%     &\cntr{(2N{+}1)} \lrimp \monad{\cntr{N} \fuse \bit{1}} \,, \\
%     &\cntr{(N{+}1)} \lrimp \monad{\cntr{N} \fuse \inc}
%     \,.
%   \end{aligned}
% \end{equation*}

% This generative signature allows us to formally state and prove adequacy of the incrementable binary counter specification, $\sig_{\inc}$.
% %
% \begin{definition}[Counter well-formedness]\label{def:counter-wf}
%   String $S$ is a well-formed counter that represents natural number $N$ (or, more simply, $S$ represents $N$) if $S$ is a maximal rewriting of $\cntr{N}$ under the signature $\sig_{\cntr{}}$, that is, if $\cntr{N} \trans+[\sig_{\cntr{}}] S \ntrans[\sig_{\cntr{}}]$.
% \end{definition}
% %
% For example, the maximal trace
% \begin{equation*}
%   \mathul{\cntr{2}}
%     \trans[\sig_{\cntr{}}] \mathul{\cntr{1}} \fuse \inc
%     \trans[\sig_{\cntr{}}] \mathul{\cntr{0}} \fuse \bit{1} \fuse \inc
%     \trans[\sig_{\cntr{}}] \eps \fuse \bit{1} \fuse \inc
%     \ntrans[\sig_{\cntr{}}]
% \end{equation*}
% witnesses that $\eps \fuse \bit{1} \fuse \inc$ is a well-formed binary counter that represents $2$.
% Adequacy of $\inc$ can be stated as follows.
% (We elide the proof---it is straightforward but requires definitions and lemmas that would only obscure our main point.)
% %
% \begin{theorem}[Adequacy of $\inc$]
%   For all natural numbers $N$ and $N'$ and every well-formed counter $S$ that represents $N$, the equality $N + 1 = N'$ holds if and only if $S \fuse \inc \trans+[\sig_{\inc}] S' \ntrans[\sig_{\inc}]$ for some $S'$ that represents $N'$.
% \end{theorem}
% %
% As first observed by~\textcite{Simmons:CMU12}, generative signatures like $\sig_{\cntr{}}$ serve a similar purpose for ordered logical specifications as types do for functional programs: both enable preservation and progress properties for their respective notions of transition.

% Given the translation from ordered logical specifications (i.e., choreographies) to well-typed processes that was presented in \cref{?}, it's natural to ask how that translation interacts with a generative signature that acts as a specification's \enquote{type}.



\subsection{Translating untyped choreographies to untyped processes}

In this proposal, we have been concerned only with \emph{well-typed} processes and a corresponding class of well-typed choreographies.
The logically grounded session-type discipline ensures that well-typed processes (and, consequently, well-typed choreographies) enjoy communication safety, session fidelity, and deadlock freedom (i.e., global progress).
However, by demanding such a strong form of progress, the current session-type discipline forbids \emph{all} racy processes, even if the races are benign or non-critical.

% For example, consider the $\pi$-calculus process $x(y).z(w).P + z(w).x(y).P$, which waits to receive---in either order---both $y$ along channel $x$ and $w$ along channel $z$.
% This process is certainly racy because it's impossible, in general, to predict the order in which $y$ and $w$ will arrive.
% But, just as certainly, this race is benign because \emph{both} $y$ and $w$ must arrive before continuing with process $P$.
% This and other benign races should be permitted by the session-type discipline.

\NewPredicate{\okL}{0}
\NewPredicate{\okR}{0}

For example, consider the following process:
\begin{equation*}
  \caseL{\okL => \caseR{\okR => P}} + \caseR{\okR => \caseL{\okL => P}} \,,
\end{equation*}
which waits to receive---in either order---$\okL$ and $\okR$ labels from both its left- and right-hand neighbors, respectively.
(The process constructor $+$ denotes nondeterministic choice.)
This process is certainly racy because it's impossible, in general, to predict the order in which the $\okL$ and $\okR$ labels will arrive.
But, even so, this race is benign: execution continues with the process $P$ once, and only once, both labels arrive in either order.
% This and other benign races should be permitted by the session-type discipline.

Choreographies may serve as a stepping-stone toward a more permissive, yet still logically grounded, session-type discipline that allows this and other benign races.
The above example can be cast as the choreography
\begin{equation*}
  (\okL[->] \limp \monad{\okR[<-] \rimp \monad{A^+}}) \with (\okR[<-] \rimp \monad{\okL[->] \limp \monad{A^+}}) \,.
\end{equation*}
By considering how the proposed translation from generative invariants to session types might apply to a generative invariant for this choreography, we may gain insight into a session-type discipline that allows benign races.
I also propose to develop a translation of a broader class of choreographies to untyped processes, which may provide different insight than just looking at the existing session-type discipline.

% In this proposal, we have been concerned with translating only \emph{well-typed} choreographies to \emph{well-typed} processes.
% By concentrating on well-typed processes, we ensure that well-typed choreographies enjoy the same, strong progress and preservation properties as their process counterparts.
% At the same time, however, by demanding such a strong form of progress, we forbid all racy processes and choreographies, even if those races are benign or non-critical.



\subsection{Session-typed Turing machines}

Finally, as the example in \cref{sec:from-ordered-logical} shows, some Turing machines can be session-typed: by translating the ordered logical specification from \cref{fig:turing-binary-add}, we get a well-typed, Turing-machine--like process for adding two binary representations.
In particular, the chain structure of singleton linear logic suggests a fit with the one-way infinite tapes of Turing machines.

Although not directly related to my proposed thesis statement, if time permits, I would like to explore further the possible connections between singleton linear logic and Turing machines.
This is the most open-ended item of proposed work and the least related to my proposed thesis statement, but, if successful, may have some interest to researchers outside the programming languages community, e.g., those working in the theory of computation.


% \subsubsection{Earlier draft}

% First, we need a few \lcnamecrefs{def:counter-wf}.
% %
% \begin{definition}[Counter well-formedness and other properties]\label{def:counter-wf}
%   \mbox{}
%   \begin{itemize}[nosep]
%   \item String $S$ is a well-formed counter that represents natural number $N$ (or, more simply, $S$ represents $N$) if $S$ is a maximal rewriting of $\cntr{N}$ under the signature $\sig_{\cntr{}}$, that is, if $\cntr{N} \trans+[\sig_{\cntr{}}] S \ntrans[\sig_{\cntr{}}]$.
%   \item String $S$ is a well-formed counter if there is some $N$ for which $S$ represents $N$.
%   \item Well-formed counter $S$ is $\inc$-free if the well-formedness trace does not use the $\inc$ clause from the $\sig_{\cntr{}}$ signature.
%   \item Likewise, well-formed counter $S$ has no leading $\bit{0}$s if the well-formedness trace uses the $\bit{0}$ clause only when $N > 0$.
%   \end{itemize}
% \end{definition}
% %
% For example, the maximal trace
% \begin{equation*}
%   \mathul{\cntr{2}}
%     \trans[\sig_{\cntr{}}] \mathul{\cntr{1}} \fuse \inc
%     \trans[\sig_{\cntr{}}] \mathul{\cntr{0}} \fuse \bit{1} \fuse \inc
%     \trans[\sig_{\cntr{}}] \eps \fuse \bit{1} \fuse \inc
%     \ntrans[\sig_{\cntr{}}]
% \end{equation*}
% witnesses that $\eps \fuse \bit{1} \fuse \inc$ is a well-formed binary counter that represents $2$; it is not $\inc$-free, but it does have no leading $\bit{0}$s.
% Note how important the choice of signature is: if we also allowed increment clauses from $\sig_{\inc}$ here, this trace would no longer be maximal.

% With these \lcnamecrefs{def:counter-wf} in hand, we can establish a bijection between the natural numbers and equivalence classes of well-formed counters.
% \begin{theorem}[Adequacy of counters]\label{thm:counter-adequacy}
%   \mbox{}
%   \begin{subtheorems}{theorem}[nosep]
%   \item\label{thm:counter-adequacy:value}
%     For each natural number $N$, there is a unique well-formed counter $S$ that represents $N$, is $\inc$-free, and has no leading $\bit{0}$s.
%   \item\label{thm:counter-adequacy:counter}
%     For each well-formed counter $S$, there is a unique natural number $N$ such that $S$ represents $N$.
%   \end{subtheorems}
% \end{theorem}
% \begin{proof}
%   \Cref{thm:counter-adequacy:value} is by induction on $N$, and \cref{thm:counter-adequacy:counter} is by induction on the structure of the maximal trace that witnesses the well-formedness of $S$.
% \end{proof}
% %
% As the following \lcnamecrefs{thm:counter-adequacy} show, the represented value is invariant under $\sig_{\inc}$-rewriting and $\sig_{\inc}$-rewriting always terminates in an $\inc$-free counter.
% It follows that $\inc$s adequately specify increments.
% %
% % \begin{lemma}
% %   \mbox{}
% %   \begin{itemize}[nosep]
% %   \item If $S_0 \fuse \bit{0} \trans[\sig_{\inc}] S'$, then $S' = S'_0 \fuse \bit{0}$ and $S_0 \trans[\sig_{\inc}] S'_0$.
% %   \item If $S_0 \fuse \bit{1} \trans[\sig_{\inc}] S'$, then $S' = S'_0 \fuse \bit{1}$ and $S_0 \trans[\sig_{\inc}] S'_0$.
% %   \item If $S_0 \fuse \inc \trans[\sig_{\inc}] S'$, then either:
% %     \begin{itemize}[nosep]
% %     \item $S' = S'_0 \fuse \inc$ and $S_0 \trans[\sig_{\inc}] S'_0$;
% %     \item $S_0 = \eps$ and $S' = \eps \fuse \bit{1}$;
% %     \item $S_0 = S_{00} \fuse \bit{0}$ and $S' = S_{00} \fuse \bit{1}$; or
% %     \item $S_0 = S_{00} \fuse \bit{1}$ and $S' = S_{00} \fuse \inc \fuse \bit{0}$.
% %     \end{itemize}
% %   \end{itemize}
% % \end{lemma}
% % 
% \begin{theorem}[Preservation]\label{thm:counter-preservation}
%   For every well-formed counter $S$ that represents $N$, if $S \trans[\sig_{\inc}] S'$, then $S'$ is also a well-formed counter that represents $N$.
%   % \begin{enumerate}[nosep]
%   % \item For every well-formed counter $S$, if $S \trans[\sig_{\inc}] S'$, then $S'$ is a well-formed counter.
%   % \item For all well-formed counters $S$ and $S'$, if $S$ represents $N$ and $S \trans[\sig_{\inc}] S'$, then $S'$ also represents $N$.
%   % \end{enumerate}
% \end{theorem}
% \begin{proof}
%   By induction on the structure of the maximal trace that witnesses the well-formedness of $S$, relying on an inversion lemma for $\sig_{\inc}$-steps:
%   \begin{itemize}[nosep]
%   \item If $S_0 \fuse \bit[_{\mathit{b}}]{} \trans[\sig_{\inc}] S'$, then $S_0 \trans[\sig_{\inc}] S'_0$ and $S' = S'_0 \fuse \bit[_{\mathit{b}}]{}$.
%   \item If $S_0 \fuse \inc \trans[\sig_{\inc}] S'$, then either:
%     \begin{itemize}[nosep]
%     \item $S_0 \trans[\sig_{\inc}] S'_0$ and $S' = S'_0 \fuse \inc$;
%     \item $S_0 = \eps$ and $S' = \eps \fuse \bit{1}$;
%     \item $S_0 = S'_0 \fuse \bit{0}$ and $S' = S'_0 \fuse \bit{1}$; or
%     \item $S_0 = S'_0 \fuse \bit{1}$ and $S' = S'_0 \fuse \inc \fuse \bit{0}$.
%     \end{itemize}
%   \end{itemize}
% \end{proof}

% % \begin{theorem}[Progress]
% %   \mbox{}
% %   \begin{enumerate}[nosep]
% %   \item For every well-formed counter $S$, if $S$ is $\inc$-free, then $S \ntrans[\sig_{\inc}]$.
% %   \item For every well-formed counter $S$, either $S$ is $\inc$-free or $S \trans[\sig_{\inc}] S'$ for some $S'$.
% %   \end{enumerate}
% % \end{theorem}
% % \begin{proof}
% %   By induction on the structure of the maximal trace that generates $S$.
% % \end{proof}

% \begin{theorem}[Termination]\label{thm:inc-termination}
%   \mbox{}
%   \begin{subtheorems}{theorem}[nosep]
%   \item\label{thm:inc-termination:inc-free}
%     For every well-formed counter $S$, string $S$ is $\inc$-free if and only if $S \ntrans[\sig_{\inc}]$.
%   \item\label{thm:inc-termination:finite}
%     For every well-formed counter $S$, there is no infinite $\sig_{\inc}$-rewriting of $S$.
%   \end{subtheorems}
% \end{theorem}
% \begin{proof}
%   \Cref{thm:inc-termination:inc-free} is by induction on the structure of the maximal trace that witnesses the well-formedness of $S$.
%   %
%   \DeclarePairedDelimiter{\meas}{\lVert}{\rVert}%
%   \DeclarePairedDelimiter{\size}{\lvert}{\rvert}%
%   To prove \cref{thm:inc-termination:finite}, define $\meas{S}$ to be a measure in which each $\inc$ in $S$ contributes an amount equal to the length of its higher-order substring:%
%   \footnote{If desired, this measure can also be defined using a generative signature.}
%   \begin{equation*}
%     \!\begin{aligned}[t]
%       \meas{\eps} &= 0 \\
%       \meas{S \fuse \bit[_{\mathit{b}}]{}} &= %\meas{S} \\
%       % \meas{S \fuse \bit{1}} = 
%       \meas{S} \\
%       \meas{S \fuse \inc} &= \meas{S} + \size{S}
%     \end{aligned}
%     \qquad
%     \!\begin{aligned}[t]
%       \size{\eps} &= 1 \\
%       \size{S \fuse \bit[_{\mathit{b}}]{}} &=
%       % \size{S \fuse \bit{1}} =
%       \size{S \fuse \inc} = \size{S} + 1
%     \end{aligned}
%   \end{equation*}
%   % Define
%   % \begin{align*}
%   %   &\meas{0,0} \lrimp \monad{\eps} \\
%   %   &\meas{M,(L{+}1)} \lrimp \monad{\meas{M,L} \fuse \bit{0}} \\
%   %   &\meas{M,(L{+}1)} \lrimp \monad{\meas{M,L} \fuse \bit{1}} \\
%   %   &\meas{(M{+}L),(L{+}1)} \lrimp \monad{\meas{M,L} \fuse \inc}
%   % \end{align*}
%   One can show that $\meas{\mathord{-}}$ is strictly decreasing for each step $S \trans[\sig_{\inc}] S'$, from which \cref{thm:inc-termination:finite} follows.
% \end{proof}

% % Cntr 0 <- eps
% % Cntr N+1 <- Cntr N * bit0
% % Cntr 

% \begin{corollary}[Adequacy of $\inc$]
%   For all natural numbers $N$ and $N'$ and every well-formed counter $S$ that represents $N$, the equality $N + 1 = N'$ holds if and only if $S \fuse \inc \trans+[\sig_{\inc}] S' \ntrans[\sig_{\inc}]$ for some $S'$ that represents $N'$.
%   % For all natural numbers $N$ and $N'$ and every well-formed counter $S$, if $S$ represents $N$, then $N + 1 = N'$ if and only if $S \fuse \inc \trans+[\sig_{\inc}] S' \ntrans[\sig_{\inc}]$ for some $S'$ that represents $N'$.
%   % % \begin{enumerate}
%   % % \item For all natural numbers $N$ and $N'$, if $S$ represents $N$, $S'$ represents $N'$, and $N + 1 = N'$, then $S \fuse \inc \trans+ S' \ntrans$.
%   % % %
%   % % \item For all well-formed counters $S$ and $S'$, if $S$ represents $N$, $S'$ represents $N'$, and $S \fuse \inc \trans+ S' \ntrans$, then $N + 1 = N'$.
%   % % \end{enumerate}
% \end{corollary}

% % \begin{falseclaim}
% %   For all natural numbers $N$ and $N'$ and well-formed counters $S$ and $S'$, if $S$ represents $N$ and $S'$ represents $N'$ and is $\inc$-free, then $N + 1 = N'$ if and only if $S \fuse \inc \trans+[\sig_{\inc}] S' \ntrans[\sig_{\inc}]$.
% %   % \begin{enumerate}
% %   % \item For all natural numbers $N$ and $N'$, if $S$ represents $N$, $S'$ represents $N'$, and $N + 1 = N'$, then $S \fuse \inc \trans+ S' \ntrans$.
% %   % %
% %   % \item For all well-formed counters $S$ and $S'$, if $S$ represents $N$, $S'$ represents $N'$, and $S \fuse \inc \trans+ S' \ntrans$, then $N + 1 = N'$.
% %   % \end{enumerate}
% % \end{falseclaim}

% In this way, the generative signature $\sig_{\cntr{}}$ serves a similar purpose for logic programming as types do for functional programming~\autocite{Simmons:CMU12}: both enable preservation and progress properties for their respective notions of transition.
% We will return to this point in \cref{sec:proposed-work}.

% \endgroup


% \end{document}

\section{Related work}\label{sec:related-work}

Functional logic languages, such as Curry\cite{Hanus:Ganzinger13}, view functional programs as logic programs.
This proposal takes the opposite approach, compiling a class of logic programs to functional programs.

\citeauthor{Simmons-Zerny:LICS13}'s correspondence between natural semantics (functional program) and abstract machines (logic program)\autocite{Simmons-Zerny:LICS13}.

%%% Local Variables:
%%% TeX-master: "proposal"
%%% End:


\printbibliography

\end{document}
