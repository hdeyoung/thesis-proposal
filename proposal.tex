% arara: lualatex
% arara: biber
% arara: lualatex
% arara: lualatex
% arara: lualatex
\documentclass{hdeyoung-proposal}

\usepackage{proposal-macros}
\usetikzlibrary{positioning,shapes.misc,graphs,quotes,graphdrawing}
\usegdlibrary{trees}

\DeclareAcronym{BHK}{
  short = BHK,
  long  = Brouwer-Heyting-Kolmogorov
}

\DeclareAcronym{JILL}{
  short = JILL,
  long  = judgmental intuitionistic linear logic
}

\DeclareAcronym{ILL}{
  short = ILL,
  long  = intuitionistic linear logic
}

\DeclareAcronym{SML}{
  short = SML,
  long  = Standard ML
}

\DeclareAcronym{SOS}{
  short = SOS,
  short-indefinite = an,
  long = structural operational semantics
}

\DeclareAcronym{SSOS}{
  short = SSOS,
  short-indefinite = an,
  long = substructural operational semantics
}

\DeclareAcronym{SILL}{
  short = SILL,
  long = session-typed intuitionistic linear logic
}

\DeclareAcronym{SISLL}{
  short = singleton \acs{SILL},
  long = session-typed intuitionistic singleton linear logic
}

\DeclareAcronym{CLF}{
  short = CLF,
  long = the Concurrent Logical Framework
}

\addbibresource{proposal.bib}

\begin{document}

\title{Session-Typed Concurrent\\Logical Specifications}
\author{Henry DeYoung\\\texttt{hdeyoung@cs.cmu.edu}}
\date{February 5, 2015}
\maketitle

\begin{abstract}
% Concurrency arises naturally in a proof-construction-as-computation setting: if all interleavings of independent [proof] steps in a logical specification are treated as indistinguishable, then those steps appear to be concurrent. 
% Concurrency also arises naturally in a proof-reduction-as-computation setting: there is a Curry--Howard isomorphism between sequent proofs in intuitionistic linear logic and session-typed processes in the $\pi$-calculus.
% 
% Using session types as a bridge, we propose to identify a fragment of intuitionistic linear logic in which these two notions of concurrency coincide.
% Specifically, we propose to show that, given an assignment of message and process roles to atomic propositions, a class of concurrent linear logical specifications can be translated to session-typed processes.
% In addition to the practical benefits of generating well-typed implementations from logical specifications, the proposed work can be seen as giving a proof-theoretic reconstruction of work on global session types for multiparty communication, and furthering an understanding of connections between proof construction and proof reduction, despite their apparent disparity.
% 
% This document aims to establish the plausibility of justifies the proposed work's feasibility by carrying out the program in the simpler case of concurrent ordered logical specifications.
% 
% 
% 
\noindent
% In computational interpretations of intuitionistic linear logic, two notions of concurrency arise.
Concurrency arises naturally in a proof-construction-as-computation interpretation of intuitionistic linear logic: if all interleavings of independent proof steps in a logical specification are treated as indistinguishable, then those steps appear to be concurrent. 
Concurrency also arises naturally in proof-reduction-as-computation: there is a Curry--Howard isomorphism, due to \citeauthor{Caires+Pfenning:CONCUR10}, between sequent proofs in intuitionistic linear logic and session-typed processes in the $\pi$-calculus.

In this proposal, we put forward the thesis that session types form a bridge between these two apparently disparate notions of concurrency.
Specifically, we propose to show that, given an assignment of process and message roles to atomic propositions, a class of concurrent linear logical specifications can be translated to session-typed processes.
In addition to the practical benefits of generating well-typed implementations from logical specifications, the proposed work can be seen as giving a proof-theoretic reconstruction of work on multiparty session types; as assigning behavioral types to a class of logical specifications, thereby ensuring deadlock freedom for those specifications; and as furthering an understanding of the relationship between proof construction and proof reduction.

This document aims to establish the thesis's plausibility by defending it in the restricted setting of intuitionistic ordered logic.
% We show that a class of concurrent ordered logical specifications can be translated to session-typed processes.
The primary area of proposed research will then be to relax that restriction, extending the ideas in this document to linear logic.


% Concurrency arises naturally in a proof-construction-as-computation interpretation of intuitionistic linear logic: if all interleavings of independent proof steps in a logical specification are treated as indistinguishable, then those steps appear to be concurrent. 
% Concurrency also arises naturally in proof-reduction-as-computation: there is a Curry--Howard isomorphism, due to \citeauthor{Caires+Pfenning:CONCUR10}, between sequent proofs in intuitionistic linear logic and session-typed processes in the $\pi$-calculus.

% In this proposal, we put forward the thesis that session types form a bridge between these two apparently disparate notions of concurrency in computational interpretations of intuitionistic linear logic.
% Specifically, we propose to show that, given an assignment of process and message roles to atomic propositions, a class of concurrent linear logical specifications can be translated to session-typed processes.
% In addition to the practical benefits of generating well-typed implementations from logical specifications, the proposed work can be seen as giving a proof-theoretic reconstruction of work on multiparty session types; assigning behavioral types to a class of logical specifications, thereby ensuring deadlock freedom for those specifications; and furthering an understanding of the relationship between proof construction and proof reduction.

% This document aims to establish the thesis's plausibility by defending it in the restricted setting of intuitionistic ordered logic.
% % We show that a class of concurrent ordered logical specifications can be translated to session-typed processes.
% The primary area of proposed research will then be to relax that restriction, extending the ideas in this document to linear logic.

\vspace{\baselineskip}
\noindent\textbf{Keywords:} substructural logics, proof reduction, proof construction, concurrency, session types
\end{abstract}

\tableofcontents
\clearpage

% arara: pdflatex
% arara: pdflatex
% arara: biber
% arara: pdflatex
% arara: pdflatex
\documentclass[
  class=../hdeyoung-proposal,
  crop=false
]{standalone}

\usepackage[subpreambles]{standalone}

\addbibresource{../proposal.bib}

\usepackage{pifont}
\usepackage{tikz}
\usetikzlibrary{matrix,quotes,graphs}
\usepackage{ordered-logic}
\usepackage{binary-counter}

\DeclareAcronym{CLF}{
  short = CLF,
  long = Concurrent Logical Framework
}

\begin{document}

\section{Introduction}\label{sec:introduction}

With the increasingly complex, distributed nature of today's software systems, concurrency is ubiquitous.
% % With the ever-increasing complexity of today's software systems, concurrency is ubiquitous.
% Concurrency structures systems as nondeterministic compositions of simpler subsystems [components].
Concurrency facilitates distributed computation by structuring systems as nondeterministic compositions of simpler subsystems [components].
% Concurrency attenuates [helps to manage] complexity by structuring systems as nondeterministic compositions of simpler subsystems [components].
% Concurrent systems are structured as nondeterministic compositions of simpler subsystems [components].
But, concomitant with nondeterminism, concurrent systems are notoriously tricky to get right:
subtle races and deadlocks can occur
% even in systems subjected to the most rigorous testing.
even in the most rigorously tested of systems.

% With the ever-increasing complexity and distribution of software systems,
% concurrency has become a pervasive method for structuring computations.
% But, like mutable state, concurrency is notoriously tricky to get right.
% % apply correctly.
% ...

At the same time,
% Decades of research into mathematical logics [proof theory] and programming languages have firmly established the power of deductive computation to ensure programs' clarity and correctness.
decades of research into connections between proof theory and programming languages have firmly established the 
% computation-as-deduction framework as the gold standard for improving programs' clarity and expressiveness and ensuring their correctness.
principle of \vocab{computation as deduction} as the gold standard
% for improving programs' clarity and expressiveness and ensuring their correctness.
framework for clear, expressive, and provably correct programs.
Examples abound: lax logic for monadic computation~\autocite{Benton:JFP98};
% S5 modal logic for distributed computation~\autocites{Murphy:CMU08}{Jia+Walker:ESOP04};
temporal logic for functional reactive programming~\autocite{Jeffrey:PLPV12}; linear logic for graph-based algorithms~\autocite{Cruz+:ICLP14}; etc.

% Can a computation-as-deduction approach make it similarly easier to write clear, expressive, provably correct concurrent programs?
Can a computation-as-deduction approach make it similarly easier to clearly and concisely specify, as well as correctly implement, concurrent programs?
% improve the clarity and expressiveness and ensure the correctness of concurrent computations?

\vspace{0.25\baselineskip}
\noindent \hspace*{\fill}\scalebox{0.75}{\color{black!50}\ding{70}}\hspace*{\fill}
\vspace{0.25\baselineskip}

\noindent
Computation-as-deduction
% principle
comes in two flavors: \vocab{proof-construction-as-computation} and \vocab{proof-reduction-as-computation}.
% , each of which has been (separately) applied to the problem of clearly specifying and correctly simulating or implementing concurrent systems.
Proof-construction-as-computation views the search for a proof, according to a fixed strategy, as the basis of computation; it is the foundation for logic programming~\autocites{Miller+:PAL91}{Andreoli:JLC92}.
% \relax[, such as the Prolog and Datalog languages];
Proof-reduction-as-computation, on the other hand, revolves around a correspondence, known as the Curry--Howard Isomorphism~\autocite{Howard:Curry80}, between propositions and types, proofs and programs, and proof simplification, or reduction, and program evaluation;
it is the foundation for typed functional programming~\autocite{Martin-Lof:LMPS80}.
% [, such as the ML and Haskell languages].
% originates from the {BHK} interpretation of intuitionistic logic.

Both the proof-construction and proof-reduction approaches have been applied to concurrent programming, stemming from \citeauthor{Girard:TCS87}'s~\autocite*{Girard:TCS87} suggestion of connections between linear logic and concurrency.
In the proof-construction vein, \acifused{CLF}{\ac{CLF}~\autocite{Watkins+:CMU02}}{the \acl{CLF}~\autocite[\acs{CLF};][]{Watkins+:CMU02}} has been used to specify a variety of concurrent systems, ranging from the $\pi$-calculus to security protocols, such as Needham--Schroeder, and even emergent story narratives~\autocites{Cervesato+:CMU02}{Martens+:LPNMR13}.
% And, using the Lollimon~\autocite{Lopez+:PPDP05} and Celf~\autocite{Schack-Nielsen:ITU11} logic programming engines that derive from \ac{CLF}, these same concurrent systems can be simulated according to their \ac{CLF} specifications.
Although these same concurrent systems can be simulated according to their \ac{CLF} specifications by the Lollimon~\autocite{Lopez+:PPDP05} and Celf~\autocite{Schack-Nielsen:ITU11} logic programming engines, the programs ultimately remain specifications, not actual distributed implementations.

Taking the proof-reduction tack,
% [perspective],
\textcite{Abramsky:TCS93}, \textcite{Bellin+Scott:TCS94}, and later \textcite{Caires+Pfenning:CONCUR10} with Toninho~\autocites*{Caires+:TLDI12}{Caires+:MSCS13} have given correspondences between sequent calculus proofs or proof nets in linear logic and
% [concurrent] 
processes, between cut elimination and concurrent process execution.
Moreover, in \citeauthor{Caires+:MSCS13}'s work, the correspondence is a full Curry--Howard isomorphism in that intuitionistic linear propositions are also types---%
\vocab{session types}~\autocite{Honda:CONCUR93} that describe the protocol to which a process adheres.
% % session types that describe a process's behavior throughout a protocol 
% ---and that it yields actual distributed implementations~\autocite{Toninho+:ESOP13}.
Unlike proof construction, the proof-reduction approach yields actual distributed implementations~\autocite{Toninho+:ESOP13}.

In spite of their common [logical] basis in linear logic, the proof-construction and proof-reduction approaches to concurrent computation appear at first glance to be strikingly disparate.
They have different dynamics (?); they offer different guarantees (progress for proof reduction); and, perhaps most importantly, they serve very different roles in programming [practice].
Although these same concurrent systems can be simulated according to their \ac{CLF} specifications by the Lollimon~\autocite{Lopez+:PPDP05} and Celf~\autocite{Schack-Nielsen:ITU11} logic programming engines, the programs ultimately remain specifications, not actual distributed implementations.
%
Proof construction is better suited to specification, whereas proof reduction is better suited to implementation.

To ..., we'd like to minimize the gap between specification and implementation.
Despite the apparent disparity between proof construction and proof reduction, is there perhaps some fragment of linear logic in which the two coincide?
Stated differently, is there a class of concurrent specifications from which distributed concurrent implementations can be automatically extracted?

Despite their apparent disparity, is there perhaps some \emph{fragment} of linear logic in which [the dynamics of] proof construction and proof reduction coincide?
Identifying such a fragment would provide a better understanding of the relationship between proof construction and proof reduction
Such a fragment would be of philosophical [conceptual] interest, but would also provide practical benefit

% Nevertheless, is there some \emph{fragment} of linear logic in which proof reduction and proof construction coincide---in which [well-typed] implementations can be mechanically extracted from specifications?

% In other words, proof construction is more suited to specification, whereas proof reduction is more suited to implementation.





% The proof-constuction and proof-reduction approaches each have their own advantages and disadvantages.
% Proof construction leads to concurrent programs that are more declarative [clear] but may get stuck in an ill-defined state.
% Proof reduction, on the other hand, provides progress and preservation properties that ensure well-typed concurrent processes never get stuck, but process definitions are less declarative.
% % The proof-construction approach is more declarative [clear] than the proof-reduction approach, but proof reduction comes with its own advantages.
% % Whereas proof construction may get stuck in an ill-defined state, proof reduction can always progress.

% % As always greedy researchers, 
% But is there some \emph{fragment} of linear logic in which proof construction and proof reduction coincide---in which the advantages of \emph{both} approaches, proof-construction and proof-reduction, are retained?
% %---can we have our cake and eat it too?

\vspace{0.25\baselineskip}
\noindent \hspace*{\fill}\scalebox{0.75}{\color{black!50}\ding{70}}\hspace*{\fill}
\vspace{0.25\baselineskip}

\noindent
The thesis is that, yes, we can indeed have our cake and eat it too:
\begin{quotation}
\noindent
Thesis statement.
\itshape Session types form a bridge between distinct notions of concurrency in computational interpretations of intuitionistic linear logic based on proof construction, on one hand, and proof reduction, on the other hand.
% Session types form a bridge between different notions of concurrency that arise in computational interpretations of linear logic: computation-as-proof-search, on one hand, and computation-as-proof-reduction, on the other hand.
\end{quotation}

The contributions of this thesis can be viewed from several perspectives.
\begin{itemize}
\item This work can be seen as a proof-theoretic [logical] reconstruction of multiparty session types~\autocite{Honda+:POPL08}.
In multiparty session types, binary session types are generalized to conversations among several parties.
Conversations in their entirety are specified using global session types.
Global types can be projected to binary session types for each pair of participants, which very nearly are implementations.
\item This work can be seen to further understanding of proof construction and proof reduction.
\item Gives types to logic programs.
Guarantees deadlock-freedom.
\end{itemize}
In addition to the practical benefit of 


The remainder of this document aims to establish this thesis as a plausible one.
% To do so, we turn our attention from linear logic to (non-modal) intuitionistic ordered logic~\autocites{Lambek:AMM58}{Polakow+Pfenning:MFPS99}, a restriction of linear logic in which [the context of] hypotheses are totally ordered, forming a list rather than a multiset.
To do so, we turn our attention from linear logic to (non-modal) intuitionistic ordered logic~\autocites{Lambek:AMM58}{Polakow+Pfenning:MFPS99}---a restriction of linear logic in which the context of hypotheses forms a list rather than a multiset or bag---and defend the thesis in this restricted setting.
% The proposed research is to extend the argument to linear logic.
The proposed thesis research is to relax the restrictions and expand the ideas in this document to intuitionistic linear logic.

Specifically, this document describes ... as depicted in \cref{fig:outline}.
First, \cref{?} reviews a string rewriting interpretation of proof construction in a [non-modal] fragment of intuitionistic ordered logic~\autocite{Simmons:CMU12}.
String rewriting specifications in this fragment are equipped with a natural notion of concurrency based on treating  as equivalent the different interleavings of independent rewriting steps.
[equivalence classes of proofs.]

Despite being concurrent, these string rewriting specifications lack an immediate notion of \emph{process} or \emph{process identity}.
Toward this end, \cref{?} introduces \vocab{choreographies}, a further restriction of string rewriting specifications obtained when [in which] atomic propositions are assigned roles as either process-like atoms or message-like atoms.
(Message-like atoms, such as $\inc[<-]$ in \cref{fig:outline}, are indicated with an arrow decoration.)
A specification may admit several choreographies, but, as described in \cref{?}, a well-formed choreography must be (lock-step) equivalent with the specification.
% \Cref{?} also describes a lock-step equivalence that must hold between a specification and its well-formed choreography.

Even with process-like atoms, choreographies remain string rewriting specifications, not actual distributed implementations of processes.
Nevertheless, choreographies are a stepping-stone to process implementations. 
In \cref{?}, we develop a session-typed process calculus from a Curry--Howard interpretation of a fragment of linear logic; 
\Cref{?} shows how choreographies can be compiled to 



Choreographies serve as a stepping stone to full-fledged process definitions.



\begin{figure}[!t]
  \centering
  \begin{tikzpicture}[node font=\small]
    \matrix [column sep=6em] {
      \node (srs) [align=center] {%
        $\bit{1} \fuse \inc \lrimp \monad{\inc \fuse \bit{0}}$\\[0.5ex]
        \textbf{String rewriting specifications (\cref{?})}\\[0.5ex]
        \textit{Proof construction in a fragment of}\\\textit{propositional ordered logic}%
      };
      &
      \node (ch) [align=center] {%
        $\bit{1} \fuse \inc[<-] \lrimp \monad{\inc[<-] \fuse \bit{0}}$\\[0.5ex]
        \textbf{Choreographies (\cref{?})}\\[0.5ex]
        \textit{Proof construction in a fragment of}\\\textit{propositional ordered logic}%
      };
      \\[8ex]
      \node (ssos) [align=center] {%
        % $\bit{1} = \caseR{\inc[<-] <= \selectL{\inc[<-]; \bit{0}}}$\\[0.5ex]
        \textbf{String rewriting specifications (\cref{?})}\\[0.5ex]
        \textit{Proof construction in a fragment of}\\\textit{first-order ordered logic}%
      };
      &
      \node (proc) [align=center] {%
        % $\exec{\bit{1}} \fuse \msg{\inc[<-]} \lrimp \monad{\selectL{\inc[<-]; \bit{0}}}$\\
        % $\exec{(\selectL{\inc[<-]; \bit{0}})} \lrimp \monad{\msg{\inc[<-]} \fuse \exec{\bit{0}}}$\\[0.5ex]
        \textbf{Session-typed processes (\cref{?})}\\[0.5ex]
        \textit{Proof reduction in}\\\textit{singleton linear logic}%
      };
      \\
    };

    \graph [edges={node font=\footnotesize}] {
      (srs) ->["role assignment"]
      (ch) ->["session types"]
      (proc) ->["SSOS"' align=center]
      (ssos) ->["specialization"]
      (srs);
    };
  \end{tikzpicture}
  \caption{Proof construction to proof reduction---there and back again\label{fig:outline}}
\end{figure}


% Then, by showing that session types bridge the notions of concurrency that arise in a proof-construction-as-computation interpretation of ordered logic and a proof-reduction-as-computation of a further restricted logic.

% a special case of the thesis is defended
% Then, by showing that session types bridge the notions of concurrency that arise in a proof-construction-as-computation interpretation of ordered logic and a proof-reduction-as-computation of a further restricted logic.

% Specifically, we show how session types bridge the notions of concurrency that arise in a proof-construction-as-computation interpretation of ordered logic and a proof-reduction-as-computation of a further restricted logic.

% Rather than considering linear logic from the outset, we restrict the logic to ordered logic
% In it, we show how session types bridge the notions of concurrency that arise in a proof-construction-as-computation interpretation of ordered logic and a proof-reduction-as-computation of a further restricted logic.

% The proposed research thesis is supported



% \section{}


% Similarly, from a proof-reduction perspective, 
% \textcite{Abramsky:TCS93} and later \textcite{Bellin+Scott:TCS94} gave correspondences between proofs in classical linear logic and concurrent processes, with proof reduction corresponding to concurrent process execution.
% These works fall short of a full Curry--Howard isomorphism in that propositions do not clearly correspond to a notion of type.
% More recently, \textcite{Caires+Pfenning:CONCUR10} with Toninho~\autocites*{Caires+:TLDI12}{Caires+:MSCS13} have developed a proof-as-processes corresponsdence for intuitionistic linear logic that indeed treats propositions as types---session types that describe a process's behavior.





% \vspace{\baselineskip}






% Under a proof-construction-as-computation viewpoint, computation arises from the act of searching, according to a fixed strategy, for a proof;
% it is the foundation of logic programming~\autocites{Miller+:PAL91}{Andreoli:JLC92}.
% Proof construction naturally lends itself to the specification and simulation of concurrent systems because
% % proof construction itself can proceed concurrently 
% independent subproofs can occur in any order, just as independant program steps can execute in any order.
% % be constructed concurrently.
% % Because independent parts of a proofcan be constructed concurrently, the proof-construction-as-computation naturally lends itself to specification and simulation of concurrent systems.
% As examples, \acifused{CLF}{}{the }\ac{CLF}~\autocite{Watkins+:CMU02} has been used to specify a variety of concurrent systems, ranging from the $\pi$-calculus to security protocols such as Needham--Schroeder~\autocite{Cervesato+:CMU02}.




% The computation-as-deduction principle comes in two flavors, \vocab{proof-reduction-as-computation} and \vocab{proof-construction-as-computation}, each of which has been (separately) applied to the problem of clearly specifying and correctly simulating or implementing concurrent systems.
% Under a proof-construction-as-computation viewpoint, computation arises from the act of searching, according to a fixed strategy, for a proof;
% it is the foundation of logic programming~\autocites{Miller+:PAL91}{Andreoli:JLC92}.
% Proof construction naturally lends itself to the specification and simulation of concurrent systems because
% % proof construction itself can proceed concurrently 
% independent subproofs can occur in any order, just as independant program steps can execute in any order.
% % be constructed concurrently.
% % Because independent parts of a proofcan be constructed concurrently, the proof-construction-as-computation naturally lends itself to specification and simulation of concurrent systems.
% As examples, \acifused{CLF}{}{the }\ac{CLF}~\autocite{Watkins+:CMU02} has been used to specify a variety of concurrent systems, ranging from the $\pi$-calculus to security protocols such as Needham--Schroeder~\autocite{Cervesato+:CMU02}.




% Computation-as-deduction comes in two flavors, \vocab{proof-reduction-as-computation} and \vocab{proof-construction-as-computation}, each of which has been (separately) applied to the problem of clearly specifying and correctly simulating or implementing concurrent systems.
% Proof-reduction-as-computation is the foundation for typed functional programming~\autocite{Martin-Lof:LMPS80}, such as the ML and Haskell languages, and revolves around a correspondence, known as the Curry--Howard Isomorphism~\autocite{Howard:Curry80}, between propositions and types, proofs and programs, and proof reduction, or simplification, and program evaluation.
% % originates from the {BHK} interpretation of intuitionistic logic.
% Proof-construction-as-computation is the foundation for logic programming~\autocites{Miller+:PAL91}{Andreoli:JLC92}, such as the Prolog and Datalog languages, and 



% From the proof-construction-as-computation viewpoint, computation arises from the act of searching, according to a fixed strategy, for a proof;
% it is the foundation of logic programming~\autocites{Miller+:PAL91}{Andreoli:JLC92}.
% Because independent parts of a proofcan be constructed concurrently, the proof-construction-as-computation naturally lends itself to specification and simulation of concurrent systems.
% For example, \acifused{CLF}{}{the }\ac{CLF}~\autocite{Watkins+:CMU02} has been used to specify a variety of concurrent systems, ranging from the $\pi$-calculus to security protocols such as Needham--Schroeder~\autocite{Cervesato+:CMU02}.





% Both the proof-reduction and proof-construction techniques have been separately applied to the problem of clearly specifying and correctly similating or implementing concurrent systems.
% Using proof-construction, for example, the {CLF}~\autocite{Watkins+:CMU02} has been used to specify a variety of concurrent systems, ranging from the $\pi$-calculus to the Needham--Schroder public key protocol~\autocite{Cervesato+:CMU02}.
% Those< same concurrent systems can be simulated using the Lollimon~\autocite{Lopez+:PPDP05} and Celf~\autocite{Schack-Nielsen:ITU11} logic programming interpreters.

% Using proof-reduction, SILL has been used to implement 

\end{document}


% arara: pdflatex
% arara: pdflatex
% arara: biber
% arara: pdflatex
% arara: pdflatex
\documentclass[
  class=../hdeyoung-proposal,
  crop=false
]{standalone}

\usepackage[subpreambles]{standalone}

\usepackage{ordered-logic}
\usepackage{binary-counter}

\addbibresource{../proposal.bib}

\begin{document}

\section{Background: Concurrent ordered logic programming}\label{sec:ordered-lp}

Viewed through a computational lens, proof search in a fragment of ordered logic becomes a forward-chaining logic programming language~\autocite{Pfenning+Simmons:LICS09}.
It can be seen as a logically motivated generalization of string rewriting~\autocite[see, \eg,][]{Book+Otto:SRS93}, an analogy which we will exploit to provide some intuition for this form of ordered logic programming.

From the perspective of string rewriting, an ordered logic program's atomic propositions are letters; ordered conjunctions (or ordered contexts) of these atoms are strings; and, under a focused proof search strategy~\autocite{Andreoli:JLC92}, the ordered implications that serve as program clauses are string rewriting rules.
An example will help to clarify.


% arara: pdflatex
% arara: biber
% arara: pdflatex
% arara: pdflatex
\documentclass[
  class=../hdeyoung-proposal,
  crop=false
]{standalone}

\usepackage{ordered-logic}
\usepackage{binary-counter}
\usepackage{proof}
\usepackage{tikz}

\NewDocumentCommand{\trans}{t* t+ o}{%
  \longrightarrow
  \IfBooleanT{#1}{^*}\IfBooleanT{#2}{^+}%
  \IfValueT{#3}{_{#3}}%
}
\NewDocumentCommand{\ntrans}{}{
  \longarrownot\trans
}

\DeclareAcronym{CLF}{
  short = CLF,
  long = the concurrent logical framework
}

\NewPredicate{\Cntr}[Cntr][font = \mathit]{0}
\NewPredicate{\cntr}[Cntr]{0}

\NewPredicate{\coin}{0}
\NewPredicate{\heads}{0}
\NewPredicate{\tails}{0}
\NewPredicate{\win}{0}
\NewPredicate{\loss}{0}

\begin{document}

\subsection{Example: Binary counter}\label{sec:exampl-binary-count-5}

Using an ordered logic program, we can specify an incrementable binary counter.
% Here we give an ordered logic program for an incrementable binary counter.
% We can implement an incrementable binary counter as an ordered logic program.
% From the perspective of string rewriting, atomic propositions are letters, and ordered conjunctions of these atoms are strings.
% 
% % In the string rewriting terminology, atomic propositions correspond to letters; ordered conjunctions of these atoms correspond to strings; and ordered implications correspond to string rewriting rules.
% In the string rewriting terminology, atomic propositions correspond to letters, and ordered conjunctions of these atoms correspond to strings.
% % Using a focused proof search strategy~\autocite{Andreoli:JLC92}, ordered implications correspond to string rewriting rules.
The counter is represented as a string of $\bit{0}$ and $\bit{1}$ atoms terminated at the most significant end by an $\eps$.
For instance, the ordered conjunction $\eps \fuse \bit{1} \fuse \bit{0}$ is a string that represents a counter with value $2$.
%
Increment instructions are represented by $\inc$ atoms at the counter's least significant end. 
% Also interspersed are $\inc$ atoms, each of which serves as an increment instruction sent to the counter given by the more significant bits.
Thus, $\eps \fuse \bit{1} \fuse \inc$ represents a counter with value $1$ that has been instructed to increment once.

Operationally, increments are described by the program's three clauses, the first of which is
\begin{equation*}
  \bit{1} \fuse \inc \lrimp \monad{\inc \fuse \bit{0}}
  \,.
\end{equation*}
From a string rewriting perspective, this implication is a rule for rewriting the (sub)string $\bit{1} \fuse \inc$ as $\inc \fuse \bit{0}$.%
% From a string rewriting perspective, this implication is interpreted as a rule for rewriting the (sub)string $\bit{1} \fuse \inc$ as $\inc \fuse \bit{0}$.%
\footnote{This interpretation is justified because, logically, implications transform one resource into another.  This will be explained in more detail in \cref{??}.}
%  in the sense that the following bottom-up rule
% % This clause allows the string $\bit{1} \fuse \inc$ to be rewritten as $\inc \fuse \bit{0}$, in the sense that the rule
% is admissible whenever this clause is part of the persistent context, $\uctx$.
% \begin{equation*}
%   \infer{\uctx ; \omatch{\bit{1} \fuse \inc} \seq C \lax}{
%     \uctx ; \ofill{\inc \fuse \bit{0}} \seq C \lax}
% \end{equation*}}
%
% Just as this implication transforms 
% When this proposition is part of the persistent context $\uctx$, the rule
% \begin{equation*}
%   \infer{\uctx ; \omatch{\bit{1} \fuse \inc} \seq C \lax}{
%     \uctx ; \ofill{\inc \fuse \bit{0}} \seq C \lax}
%   \,.
% \end{equation*}
% is admissible.
% Read bottom-up, this rule serves to rewrite the (sub)string $\bit{1} \fuse \inc$ as the string $\inc \fuse \bit{0}$.
%
By rewriting $\bit{1} \fuse \inc$ as $\inc \fuse \bit{0}$, this clause serves to carry the $\inc$ up past any $\bit{1}$s that may exist at the counter's least significant end.

Whenever the carried $\inc$ reaches the $\eps$ or right-most $\bit{0}$, the carry is resolved by one of the other two program clauses:
\begin{align*}
  &\eps \fuse \inc \lrimp \monad{\eps \fuse \bit{1}} \\
  &\bit{0} \fuse \inc \lrimp \monad{\bit{1}} \,.
\end{align*}
By rewriting $\eps \fuse \inc$ as $\eps \fuse \bit{1}$, this second clause ensures that the carry becomes a new most significant $\bit{1}$ in the $\eps$ case.
By rewriting $\bit{0} \fuse \inc$ as $\bit{1}$, the third clause ensures that the carry flips the $\bit{0}$ to $\bit{1}$ in that case.

For example, the counter $\eps \fuse \bit{1} \fuse \inc$ can be maximally rewritten as
\ExplSyntaxOn
\NewDocumentCommand{\mathul}{m}{
  \mathpalette\mathul:nn{#1}
}
\cs_new:Npn \mathul:nn #1#2 {
  \tikz [baseline] {
    \node (pr) [anchor = base, inner~sep = 0em] {$#1#2$};
    \draw [overlay, ultra~thick, gray]
      ([yshift=-0.3em]pr.base~west) -- ([yshift=-0.3em]pr.base~east);
  }
}
\ExplSyntaxOff
\begin{equation*}
  \eps \fuse \mathul{\bit{1} \fuse \inc}
    \trans \mathul{\eps \fuse \inc} \fuse \bit{0}
    \trans \eps \fuse \bit{1} \fuse \bit{0}
    \ntrans
  \text{\,,}
\end{equation*}
where at each step the sites available for rewriting are underlined.
This trace computes $1 + 1 = 2$ in binary representation.
More generally, the above clauses adequately specify an increment operation:
string $S$ represents a counter with value $N$ if and only if $S \fuse \inc \trans+ S' \ntrans$ for some string $S'$ that represents a counter of value $N + 1$, where $\trans+$ denotes the transitive closure of the rewriting relation $\trans$.

\subsection{Concurrency}\label{sec:concurrency-1}

Some strings contain more than one rewrite site.
% several disjoint substrings that are amenable to rewriting.
% If these sites are disjoint, the rewrites can be thought of as happening concurrently.
For instance, the following binary counter has two $\inc$s in flight, which give rise to two disjoint rewrite sites.
\begin{equation*}
  \mathul{\eps \fuse \inc} \fuse \mathul{\bit{0} \fuse \inc}
\end{equation*}
The rewrites in this example can be interleaved in two ways: either
% there are two interleavings of these rewrites: either
\begin{align*}
  &\mathul{\eps \fuse \inc} \fuse \mathul{\bit{0} \fuse \inc}
     \trans \eps \fuse \bit{1} \fuse \mathul{\bit{0} \fuse \inc}
     \trans \eps \fuse \bit{1} \fuse \bit{1}
     \ntrans \\
  %
  \shortintertext{or}
  %
  &\mathul{\eps \fuse \inc} \fuse \mathul{\bit{0} \fuse \inc}
     \trans \mathul{\eps \fuse \inc} \fuse \bit{1}
     \trans \eps \fuse \bit{1} \fuse \bit{1}
     \ntrans
   \,.
\end{align*}

However, because these two rewrites are independent, they should be considered morally concurrent.
Rather than giving a truly concurrent semantics for string rewriting, we can treat different interleavings of independent steps as indistinguishable.
Then, because we can't observe which rewrite occurred first, the two rewrites appear to happen concurrently.
This is the idea of \vocab{concurrent equality} from {CLF}~\autocite{Watkins+:CMU02} that gives a pretense to true concurrency, borrowed for ordered logic programming.


\subsubsection{Infinite traces and fairness}

\NewPredicate{\incs}{0}%
Thus far, all traces have been finite, but this is not necessarily so.
Consider adding an atom, $\incs$, that generates a stream of $\inc$ atoms:
\begin{equation*}
  \incs \lrimp \monad{\inc \fuse \incs}
  \,.
\end{equation*}
Among the infinite traces now possible is
% Beginning from $\eps \fuse \incs$, there are now infinite trace
\begin{equation*}
  \eps \fuse \mathul{\incs}
    \trans \mathul{\eps \fuse \inc} \fuse \mathul{\incs}
    \trans \dotsb
    \trans \mathul{\eps \fuse \inc} \fuse \inc \fuse \dotsb \fuse \inc \fuse \mathul{\incs}
    \trans \dotsb
  \,.
\end{equation*}

Notice that this trace never rewrites the infinitely available $\eps \fuse \inc$, instead always choosing to rewrite $\incs$.
Because of its\fxnote{\ \st{scheduling}} bias against rewriting $\eps \fuse \inc$, we say that the trace is \vocab{(weakly) process-unfair}.

Process unfairness is fundamentally at odds with true concurrency.
Because true concurrency allows multiple independent events to occur simultaneously, one event cannot preclude another independent event;
in the above example, for instance, rewriting the $\incs$ atom would not preclude rewriting the independent $\eps \fuse \inc$ substring.
Therefore, to maintain the pretense of true concurrency, we must require all traces to be (weakly) process-fair.


% Transition unfairness is fundamentally at odds with true concurrency
% because true concurrency allows multiple independent events to occur simultaneously;
% the occurrence of one event cannot preclude another independent event from occurring at the same time.
% To maintain the pretense of true concurrency, 

% In the above example, $\eps \fuse \inc$ and $\incs$ are independent rewrite sites

% In the above example, $\eps \fuse \inc$ and $\incs$ are independent rewrite sites.
% % ; if they are truly concurrent, the rewrite of $\eps \fuse \inc$ should eventually occur.
% To maintain any pretense of true concurrency, the rewrite of $\eps \fuse \inc$ should eventually occur since true concurrency would allow multiple independent events to occur simultaneously.



% However, by treating different interleavings of independent steps as indistinguishable, these two rewrites happen concurrently.




% For instance, notice that this binary counter specification allows multiple $\inc$s to be in flight at once, each of which is amenable to rewriting.



% Sometimes several disjoint substrings are amenable to rewriting.


% Different interleavings of independent steps are indistinguishable.

% String rewriting (and therefore ordered logic programming) gives rise to 


% The counter $\eps \fuse \inc \bit{0} \fuse \inc$ has two $\inc$s in flight, which can be rewritten independently.


% Rewrites of disjoint substrings can be thought of as happening concurrently.

% Notice that we can also allow multiple $\inc$s to be in flight at once, and that independent rewrites can thought of as happening concurrently.
% For instance, the counter $\eps \fuse \inc \fuse \bit{0} \fuse \inc$ has two $\inc$s in flight, and they give rise to independent rewrites.
% \begin{center}
%   \begin{tikzpicture}
%     \matrix [matrix of math nodes, column sep = 1.5em]
%     {
%       % First row
%       & |(inc-1-2)| \eps \fuse \bit{1} \fuse \mathul{\bit{0} \fuse \inc} & \\
%       % Second row
%       |(inc-2-1)| \mathul{\eps \fuse \inc} \fuse \mathul{\bit{0} \fuse \inc}
%         && |(inc-2-3)| \eps \fuse \bit{1} \fuse \bit{1} \\
%       % Third row
%       & |(inc-3-2)| \mathul{\eps \fuse \inc} \fuse \bit{1} & \\
%     };

%     \begin{scope}
%     [ start chain, every join/.style={->} ]
%       \chainin (inc-2-1);
%       \begin{scope}[start branch=inc-1-2]
%         \chainin (inc-1-2) [join];
%       \end{scope}
%       \begin{scope}[start branch=inc-3-2]
%         \chainin (inc-3-2) [join];
%       \end{scope}
%       \chainin (inc-2-3) [join = with inc-1-2, join = with inc-3-2];
%     \end{scope}
%   \end{tikzpicture}
% \end{center}


\subsection{Committed choice}\label{sec:committed-choice}

% In our forward-chaining ordered logic programming language, we assume a \vocab{committed-choice} semantics, meaning that when multiple rewritings are possible at a given site the choice is never reconsidered.
In forward-chaining ordered logic programming, we assume a \vocab{committed-choice} semantics, meaning that when several rewritings are possible at a given site, one is chosen and that choice is never reconsidered.

Committed choice does not clearly arise in the binary counter example;
instead, consider a single-player game in which the player wins if a
% each of two coin tosses land heads.
coin toss lands heads.
As an forward-chaining ordered logic program, this game is specified
\begin{align*}
  &\coin \lrimp \monad{\heads} \\
  &\coin \lrimp \monad{\tails} \\
  &\heads \lrimp \monad{\win} \\
  &\tails \lrimp \monad{\loss}
  \,.
\end{align*}
The first two clauses specify that a $\coin$ may land either $\heads$ or $\tails$, and the remaining clauses specify the winning condition.

% One way in which the starting string $\coin \fuse \coin$ of two coins can be maximally rewritten is
One way in which the starting string can be maximally rewritten is
\begin{equation*}
  \mathul{\coin}
    \trans \mathul{\tails}
    \trans \loss
    \ntrans
  \,.
\end{equation*}
Here the $\coin$ lands $\tails$, resulting in a $\loss$.
To get a $\win$, we'd like to somehow take back the toss and have it instead land $\heads$.
% Here the first $\coin$ lands $\tails$ and the second $\coin$ lands $\heads$, resulting in a $\loss$.
% To get a $\win$, we'd like to somehow take back the first toss and have it instead land $\heads$.
This violation of the game's rules is just what the committed-choice semantics proscribes: once the coin is tossed, we must commit to that toss's outcome.



\subsection{Example: Binary counter with decrements}\label{sec:exampl-binary-count-3}

Returning to the binary counter, its also possible \dots

To reiterate some of the points made above, \dots

It's also possible to add support for decrements to the above ordered logic program.
Like increments, a decrement instruction is represented by a $\dec$ atom at the counter's least significant end.
To perform the decrement, a $\dec$ begins propagating up the counter.
As it passes over any $\bit{0}$s at the least significant end, they are marked as $\bit[']{0}$s to indicate that they are waiting to borrow from their more significant neighbors:
\begin{equation*}
  \bit{0} \fuse \dec \lrimp \monad{\dec \fuse \bit[']{0}} \,.
\end{equation*}
Whenever it reaches the $\eps$ or right-most $\bit{1}$, the $\dec$ is replaced with either $\zero$ or $\suc$, respectively, to show whether the borrow was possible; in the case of $\bit{1}$, the borrow is also effected:
\begin{align*}
  &\eps \fuse \dec \lrimp \monad{\eps \fuse \zero} \\
  &\bit{1} \fuse \dec \lrimp \monad{\bit{0} \fuse \suc} \,.
\end{align*}
Then the $\zero$ or $\suc$ travels back over all of the $\bit[']{0}$s that were waiting to borrow.
In the case of $\zero$, the bits are returned to their original $\bit{0}$ state because no borrow was possible;
in the case of $\suc$, a borrow was performed and so the bits are set to $\bit{1}$:
\begin{align*}
  &\zero \fuse \bit[']{0} \lrimp \monad{\bit{0} \fuse \zero} \\
  &\suc \fuse \bit[']{0} \lrimp \monad{\bit{1} \fuse \suc} \,.
\end{align*}


% According to the following rewrite rules, a $\dec$ propogates up the counter past any $\bit{0}$s until it reaches an $\eps$ or the right-most $\bit{1}$.
% At this point, the $\dec$ is replaced with either $\zero$ or $\suc$, respectively.
% \begin{align*}
%   &\bit{0} \fuse \dec \lrimp \monad{\dec \fuse \bit[']{0}} \\
%   &\eps \fuse \dec \lrimp \monad{\eps \fuse \zero} \\
%   &\bit{1} \fuse \dec \lrimp \monad{\bit{0} \fuse \suc}
% \end{align*}
% Then the $\zero$ or $\suc$ traveles back down the counter
% \begin{align*}
%   &\zero \fuse \bit[']{0} \lrimp \monad{\bit{0} \fuse \zero} \\
%   &\suc \fuse \bit[']{0} \lrimp \monad{\bit{1} \fuse \suc}
% \end{align*}


For example, the counter $\eps \fuse \bit{1} \fuse \bit{0} \fuse \dec$ can be maximally rewritten as
\begin{align*}
  \MoveEqLeft[0.5]
  \eps \fuse \bit{1} \fuse \mathul{\bit{0} \fuse \dec} \\
    &\trans \eps \fuse \mathul{\bit{1} \fuse \dec} \fuse \bit[']{0} \\
    &\trans \eps \fuse \mathul{\bit{0} \fuse \suc} \fuse \bit[']{0} \\
    &\trans \eps \fuse \bit{0} \fuse \bit{1} \fuse \suc \\
    &\ntrans
\end{align*}
Once again, there are possibilities for concurrency.
For example, the following two traces are indistinguishable because they differ only in the order of independent rewrites:
\begin{align*}
  &\mathul{\eps \fuse \inc} \fuse \mathul{\bit{0} \fuse \dec} \trans \mathul{\eps \fuse \inc} \fuse \dec \fuse \bit[']{0} \trans \eps \fuse \mathul{\bit{1} \fuse \dec} \fuse \bit[']{0} \\
  %
  \shortintertext{and}
  %
  &\mathul{\eps \fuse \inc} \fuse \mathul{\bit{0} \fuse \dec} \trans \eps \fuse \bit{1} \fuse \mathul{\bit{0} \fuse \dec} \trans \eps \fuse \mathul{\bit{1} \fuse \dec} \fuse \bit[']{0}
\end{align*}
This justifies treating those two rewrites as concurrent.


% C ::= eps | C * bit0 | C * bit1 | C * inc


% c <- dec <- d =
% { case d of
%     eps => wait d;
%            d' <- eps;
%            c <- zero <- d'
%   | bit0 => d' <- dec <- d
%             c <- bit0' <- d'
%   | bit1 => d' <- bit0 <- d
%             c <- succ <- d' }

% c <- zero <- d =
% { case c of
%     bit0' => d' <- bit0 <- d
%              c <- zero <- d' }

% Cntr = +{ eps: 1 , bit0: Cntr , bit1: Cntr }
% Cntr' = &{ bit0': Cntr' }

% bit0 : {Cntr |- Cntr}
% dec : {Cntr |- Cntr'}
% zero : {Cntr |- Cntr'}
% bit0' : {Cntr' |- Cntr'}


\begin{align*}
  &\eps \fuse \dec \lrimp \monad{\eps \fuse \zero} \\
  &\bit{0} \fuse \dec \lrimp \monad[auto]{
                               \dec \fuse \parens[auto, align=c@{\,}l]{
                                                & (\zero \limp \monad{\bit{0} \fuse \zero}) \\
                                          \with & (\suc \limp \monad{\bit{1} \fuse \suc})}} \\
  &\bit{1} \fuse \dec \lrimp \monad{\bit{0} \fuse \suc}
\end{align*}



\subsection{Adequacy and generative invariants}\label{sec:gener-invar}

Thus far, we have used ordered logic programs to specify concurrent systems, whereas the non-modal fragment of ordered logic was originally developed by \textcite{Lambek:AMM58} to describe sentence structure.
However, these two modes of use of ordered logic are not as different as they might first appear.

In our running example of an incrementable binary counter, the counter is represented as a string of $\bit{0}$, $\bit{1}$, and $\inc$ atoms terminated at the most significant end by an $\eps$.
More precisely, a string is a well-formed binary counter if it can be generated from the $\Cntr$ nonterminal by the context-free grammar
\begin{gather*}
  \Cntr ::= \eps \mid \Cntr \fuse \bit{0} \mid \Cntr \fuse \bit{1} \mid \Cntr \fuse \inc
  \,,
%
\intertext{which is an abbreviated notation for four distinct productions:}
%
  \begin{aligned}
    &\Cntr \to \eps \\
    &\Cntr \to \Cntr \fuse \bit{0} \\
    &\Cntr \to \Cntr \fuse \bit{1} \\
    &\Cntr \to \Cntr \fuse \inc
    \,.
  \end{aligned}
\end{gather*}

Building on \citeauthor{Lambek:AMM58}'s work, the same context-free grammar can be described in ordered logic using \vocab{generative signatures}~\autocite{Simmons:CMU12}.
Each production in the grammar corresponds to a clause, with the $\Cntr$ nonterminal represented as the atomic proposition $\cntr$:
\begin{equation*}
  \begin{aligned}
    &\cntr \lrimp \monad{\eps} \\
    &\cntr \lrimp \monad{\cntr \fuse \bit{0}} \\
    &\cntr \lrimp \monad{\cntr \fuse \bit{1}} \\
    &\cntr \lrimp \monad{\cntr \fuse \inc}
    \,.
  \end{aligned}
\end{equation*}
Then, just as all well-formed binary counters are generated from the $\Cntr$ nonterminal according to the above productions, so are all binary counters generated as maximal rewritings of the $\cntr$ atom according to these clauses.
% Just as a string is well-formed binary counter if it can be generated from the $\Cntr$ nonterminal by the above context-free grammar, so too is a string well-formed if it can be generated by maximally rewriting the $\cntr$ atom.
For example, $\eps \fuse \bit{1} \fuse \inc$ is a well-formed binary counter because it is a maximal rewriting of $\cntr$:
\begin{equation*}
  \mathul{\cntr}
    \trans \mathul{\cntr} \fuse \inc
    \trans \mathul{\cntr} \fuse \bit{1} \fuse \inc
    \trans \eps \fuse \bit{1} \fuse \inc
    \ntrans
  \,.
\end{equation*}
Note that 

\begingroup
  \RenewPredicate{\cntr}[Cntr]{1}%
Generative signatures in fact generalize context-free grammars.
One example is to augment $\cntr{}$ with a natural number, effectively \wc{giving}[\st{creating}] a countably infinite family of nonterminals.
Thus, a binary counter is well-formed \emph{and} represents value $N$ if it is a maximal rewriting of $\cntr{N}$ according to the first-order ordered logic program
\begin{equation*}
  \begin{aligned}
    &\cntr{0} \lrimp \monad{\eps} \\
    &\cntr{(2N)} \lrimp \monad{\cntr{N} \fuse \bit{0}} \\
    &\cntr{(2N{+}1)} \lrimp \monad{\cntr{N} \fuse \bit{1}} \\
    &\cntr{(N{+}1)} \lrimp \monad{\cntr{N} \fuse \inc}
    \,.
  \end{aligned}
\end{equation*}

This generative signature allows us to formally state (and prove) adequacy of the incrementable binary counter program:
\begin{definition}
  A string is \vocab{quiescent} if $S \ntrans$.
  A string $S$ is a \vocab{well-formed counter} if $\cntr{} \trans+ S \ntrans$.
  A string $S$ \vocab{represents} natural number $N$ if $\cntr{N} \trans+ S \ntrans$.
\end{definition}

\begin{theorem}[Preservation]
  For every well-formed counter $S$ such that $S \trans S'$, the string $S'$ is also a well-formed counter.
\end{theorem}
\begin{theorem}[Adequacy of counters]
  \mbox{}
  \begin{enumerate}
  \item For every natural number $N$, there is a unique well-formed counter $S$ such that
%     \item $\cntr{} \trans+ S \ntrans$;
 $S$ represents $N$ and % $\cntr{N} \trans+ S \ntrans$
 $S$ is quiescent.% $S \ntrans$.
%  \item For every string $S$ such that $\cntr{} \trans+ S \ntrans$, there is a unique natural number $N$ such that $\cntr{N} \trans+ S \ntrans$.
  \item For every well-formed counter $S$, there is a unique natural number $N$ such that $S$ represents $N$.
  \end{enumerate}
\end{theorem}
\begin{theorem}[Adequacy of $\inc$]\mbox{}
  \begin{enumerate}
  \item If $N + 1 = N'$, then there exist well-formed counters $S$ and unique $S'$ such that string $S$ represents $N$, string $S'$ represents $N'$, and $S \fuse \inc \trans+ S' \ntrans$.
  \item For all well-formed counters $S$ and $S'$ such that $S \fuse \inc \trans+ S' \ntrans$, there exist unique natural numbers $N$ and $N'$ such that string $S$ represents $N$, string $S'$ represents $N'$, and $N + 1 = N'$.
  \end{enumerate}
\end{theorem}

\begin{theorem}[Adequacy of $\inc$]
  $\cntr{N} \trans+ S \ntrans$ if and only if $S \fuse \inc \trans+ S' \ntrans$ and $\cntr{(N{+}1)} \trans+ S' \ntrans$.
  \begin{itemize}
  \item If $\cntr{N} \trans+ S \ntrans$ and $S \fuse \inc \trans+ S' \ntrans$, then $\cntr{(N{+}1)} \trans+ S' \ntrans$.
  \item If $\cntr{N} \trans+ S \ntrans$, then $S \fuse \inc \trans+ S' \ntrans$.
  \end{itemize}
\end{theorem}
\endgroup

% stype C = &{ inc: C, dec: C', halt: X }
%   and C' = +{ zero: C, succ: C }
% eps : {C <- X}, bit0 : {C <- C}, bit1 : {C <- C}, inc : {C <- C},
% dec : {C' <- C}, zero : {C' <- C}, succ : {C' <- C}, bit0' : {C' <- C'},
% halt : {X <- C}
% 
%  C ::= X * eps | C * bit0 | C * bit1 | C * inc
% C' ::= C * dec | C * zero | C * succ | C' * bit0'
%  X ::= C * halt

% stype C = &{ inc: C, dec: C' }
%   and C' = +{ zero: C-, succ: C- }
% bit0 : {C- <- C-} /\ {C <- C}, ...
% 
% C- ::= eps | C- * bit0 | C- * bit1
%  C ::= eps | C * bit0 | C * bit1 | C * inc
% C' ::= C * dec | C- * zero | C- * succ | C' * bit0'




% \subsubsection{String rewriting rules as ordered implications.}

% Using a focused proof search strategy~\autocite{Andreoli:JLC92}, ordered implications correspond to string rewriting rules.


% \subsection{Example: Binary counter}\label{sec:exampl-binary-count-4}

% As a running example, we can implement a binary counter that supports increments.
% The counter is represented as a string of $\bit{0}$ and $\bit{1}$ letters terminated at the most significant end by an $\eps$.
% So, for instance, the ordered conjunction, or string, $\eps \fuse \bit{1} \fuse \bit{0}$ represents a counter with value $2$.
% Also interspersed are $\inc$ atoms, each of which serves as an increment instruction sent to the counter given by the more significant bits.
% Thus, $\eps \fuse \bit{1} \fuse \inc$ represents a counter with value $1$ that has been sent an increment instruction.

% Operationally, increments are described by three ordered implications that correspond to string rewriting rules; the first of these is
% \begin{equation*}
%   \bit{1} \fuse \inc \rimp \monad{\inc \fuse \bit{0}} \,.
% \end{equation*}
% By rewriting the string $\bit{1} \fuse \inc$ as $\inc \fuse \bit{0}$, this rule carries the $\inc$ up past any $\bit{1}$s at the counter's least significant end.
% Whenever the carried $\inc$ reaches the $\eps$ or right-most $\bit{0}$, the carry is resolved:
% \begin{align*}
%   &\eps \fuse \inc \lrimp \monad{\eps \fuse \bit{1}} \\
%   &\bit{0} \fuse \inc \lrimp \monad{\bit{1}} \,.
% \end{align*}
% By rewriting $\eps \fuse \inc$ as $\eps \fuse \bit{1}$, the second rule ensures that the carry becomes a new most significant $\bit{1}$ in the $\eps$ case.
% By rewriting $\bit{0} \fuse \inc$ as $\bit{1}$, the third rule ensures that the carry flips the $\bit{0}$ to $\bit{1}$ in that case.



% When this proposition is part of the persistent context $\uctx$, the following rule is derivable:
% \begin{equation*}
%   \infer{\uctx ; \omatch{\bit{1}, \inc} \seq J}{
%     \uctx ; \ofill{\inc, \bit{0}} \seq J}
%   \,.
% \end{equation*}
% Read bottom-up, this derived rule rewrites part of the ordered context so that $\bit{1}, \inc$ becomes $\inc, \bit{0}$.


% % \begin{equation*}
% %   \infer[\lab{copy}]{\uctx ; \omatch{\bit{1}, \inc} \seq J}{
% %     \infer[\llab{{\rimp}}]{\uctx ; \ofill{\bit{1}, \inc, (\bit{1} \fuse \inc \rimp \inc \fuse \bit{0})} \seq J}{
% %       \infer[\rlab{{\fuse}}]{\uctx ; \bit{1}, \inc \seq \bit{1} \fuse \inc}{
% %         \infer[\lab{id}]{\uctx ; \bit{1} \seq \bit{1}}{
% %           } &
% %         \infer[\lab{id}]{\uctx ; \inc \seq \inc}{
% %           }} &
% %       \infer[\llab{{\fuse}}]{\uctx ; \ofill{\inc \fuse \bit{0}} \seq J}{
% %         \uctx ; \ofill{\inc, \bit{0}} \seq J}}}
% % \end{equation*}




% \subsection{Example: Binary counter}\label{sec:exampl-binary-count-2}

% As an example of an ordered logic program, we can implement a binary counter that supports increments.
% Similarly to the process implementation from \cref{sec:exampl-binary-count}, the counter is represented as a list of $\bit{0}$ and $\bit{1}$s terminated at the most significant end by an $\eps$.
% Here, however, the $\bit{}$s and $\eps$ are not processes, but rather atomic propositions (or, in string rewriting terminology, letters).
% For instance, the ordered conjunction (or string) $\eps \fuse \bit{1} \fuse \bit{0}$ represents a counter with value $2$.

% % An increment instruction is represented by an $\inc$ atom at the counter's least significant end.
% % There are three rewrite rules that describe the increment operation:
% % \begin{align*}
% %   &\eps \fuse \inc \lrimp \monad{\eps \fuse \bit{1}} \\
% %   &\bit{0} \fuse \inc \lrimp \monad{\bit{1}} \\
% %   &\bit{1} \fuse \inc \lrimp \monad{\inc \fuse \bit{0}}
% % \end{align*}
% % By rewriting $\eps \fuse \inc$ as $\eps \fuse \bit{1}$, the first rule introduces $\bit{1}$ as a new most significant bit, and thereby serves to increment an $\eps$.
% % % By rewriting $\eps \fuse \inc$ as $\eps \fuse \bit{1}$ and thereby introducing $\bit{1}$ as a new most significant bit, the first rule serves to increment $\eps$s.
% % Likewise, the second rule serves to increment a counter whose least significant bit is $\bit{0}$, by rewriting $\bit{0} \fuse \inc$ as $\bit{1}$ and thereby flipping $\bit{0}$.
% % % Likewise, by rewriting $\bit{0} \fuse \inc$ as $\bit{1}$ and thereby flipping $\bit{0}$, the second rule serves to increment a counter whose least significant bit is $\bit{0}$.
% % Finally, by rewriting $\bit{1} \fuse \inc$ as $\inc \fuse \bit{0}$, the third rule flips $\bit{1}$ and propogates a carry to the more significant bits, thereby serving to increment a counter whose least significant bit is $\bit{1}$.

% An increment instruction is represented by an $\inc$ atom at the counter's least significant end.
% There are three rewrite rules that describe increments, the first of which is
% \begin{equation*}
%   \bit{1} \fuse \inc \lrimp \monad{\inc \fuse \bit{0}} \,.
% \end{equation*}
% By rewriting $\bit{1} \fuse \inc$ as $\inc \fuse \bit{0}$, this rule carries the $\inc$ up past any $\bit{1}$s at the counter's least significant end.
% Whenever the carried $\inc$ reaches the $\eps$ or right-most $\bit{0}$, the carry is resolved:
% \begin{align*}
%   &\eps \fuse \inc \lrimp \monad{\eps \fuse \bit{1}} \\
%   &\bit{0} \fuse \inc \lrimp \monad{\bit{1}} \,.
% \end{align*}
% By rewriting $\eps \fuse \inc$ as $\eps \fuse \bit{1}$, the second rule ensures that the carry becomes a new most significant $\bit{1}$ in the $\eps$ case.
% By rewriting $\bit{0} \fuse \inc$ as $\bit{1}$, the third rule ensures that the carry flips the $\bit{0}$ to $\bit{1}$ in that case.

\end{document}

%%% Local Variables:
%%% TeX-master: "ordered-lp"
%%% End:


% arara: pdflatex
% arara: biber
% arara: pdflatex
% arara: pdflatex
\documentclass[
  class=../hdeyoung-proposal,
  crop=false
]{standalone}

\usepackage{ordered-logic}
\usepackage{basic-atoms}
\usepackage{ordered-lp-terms}
\usepackage{proof}
\usepackage{mathpartir}

\NewDocumentCommand{\tctx}{}{\Psi}
\NewDocumentCommand{\tctxe}{}{\cdot}

% \NewDocumentCommand{\trans}{t* t+ o}{%
%   \longrightarrow
%   \IfBooleanT{#1}{^*}\IfBooleanT{#2}{^+}%
%   \IfValueT{#3}{_{#3}}%
% }
% \NewDocumentCommand{\ntrans}{}{
%   \longarrownot\trans
% }

\begin{document}

\subsection{Technical details}\label{sec:technical-details}

\begin{alignat*}{2}
  &\text{Negative propositions}\quad & A^- &::= \forall a{:}\tau. A^- \mid A^+ \rimp B^- \mid A^+ \limp B^- \mid A^-_1 \with A^-_2 \mid \monad{A^+} \\
  &\text{Positive propositions}      & A^+ &::= \p^+ \mid A^+ \fuse B^+ \mid \one \mid \exists a{:}\tau. A^+ \mid A^-
\end{alignat*}

\begin{alignat*}{2}
  &\text{Normal terms}\quad & N &::= \tlam{a.N} \mid \rlam{p.N} \mid \llam{p.N} \mid \pair{N_1, N_2} \mid \lett{T in V} \\
  &\text{Atomic terms}\quad & R &::= \atm{c . S} \mid \atm{x . S} \\
  &\text{Spines} & S &::= \tapp{t ; S} \mid \rapp{V ; S} \mid \lapp{V ; S} \mid \fst{S} \mid \snd{S} \mid \snil \\
  &\text{Values} & V &::= x \mid \vfuse{V_1}{V_2} \mid \vone \mid \vexists{t.V} \mid N \\
  &\text{Patterns} & p &::= x \mid \pfuse{p_1}{p_2} \mid \pone \mid \pexists{a.p} \\
  &\text{Traces} & T &::= \tstep{p <- R} \mid \tnil \mid \tseq{T_1 ; T_2}
\end{alignat*}

\begin{mathpar}
  \infer{\tctx ; \octx \seq \nof{\tlam{a.N} : \forall a{:}\tau.A^-}}{
    \tctx, a{:}\tau ; \octx \seq \nof{N : A^-}}
  \and
  \infer{\tctx ; \octx \seq \nof{\rlam{p.N} : A^+ \rimp B^-}}{
    \tctx_{A^+} ; \octx_{A^+} \pseq \pof{p : A^+} &
    \tctx, \tctx_{A^+} ; \octx, \octx_{A^+} \seq \nof{N : B^-}}
  \and
  \infer{\tctx ; \octx \seq \nof{\llam{p.N} : A^+ \limp B^-}}{
    \tctx_{A^+} ; \octx_{A^+} \pseq \pof{p : A^+} &
    \tctx, \tctx_{A^+} ; \octx_{A^+}, \octx \seq \nof{N : B^-}}
  \and
  \infer{\tctx ; \octx \seq \nof{\pair{N_1 , N_2} : A^-_1 \with A^-_2}}{
    \tctx ; \octx \seq \nof{N_1 : A^-_1} &
    \tctx ; \octx \seq \nof{N_2 : A^-_2}}
  \and
  \infer{\tctx ; \octx \seq \nof{\lett{T in V} : \monad{A^+}}}{
    \tof{T :: (\tctx ; \octx) \trans* (\tctx' ; \octx')} &
    \tctx' ; \octx' \seq \vof{V : [A^+]}}
\end{mathpar}

\begin{mathpar}
  \infer{\tctx ; \omatch{x{:}A^-} \seq \aof{\atm{x . S} : \susp-{C^-}}}{
    \tctx ; \ofill{\lfoc{A^-}} \seq \sof{S : \susp-{C^-}}}
  \and
  \infer{\tctx ; \omatch{\octxe} \seq \aof{\atm{c . S} : \susp-{C^-}}}{
    c{:}A^- \in \sig &
    \tctx ; \ofill{\lfoc{A^-}} \seq \sof{S : \susp-{C^-}}}
\end{mathpar}

\begin{mathpar}
  \infer{\tctx ; \omatch{\lfoc{A^-}} \seq \sof{\snil : \susp-{A^-}}}{
    }
  \and
  \infer{\tctx ; \omatch{\lfoc{\forall a{:}\tau.A^-}} \seq \sof{\tapp{t ; S} : \susp-{C^-}}}{
    \tctx \seq t : \tau &
    \tctx ; \ofill{\lfoc{\subst{t/a}{A^-}}} \seq \sof{S : \susp-{C^-}}}
  \and
  \infer{\tctx ; \omatch{\lfoc{A^+ \rimp B^-}, \octx} \seq \sof{\rapp{V ; S} : \susp-{C^-}}}{
    \tctx ; \octx \seq \vof{V : \rfoc{A^+}} &
    \tctx ; \ofill{\lfoc{B^-}} \seq \sof{S : \susp-{C^-}}}
  \and
  \infer{\tctx ; \omatch{\octx, \lfoc{A^+ \limp B^-}} \seq \sof{\lapp{V ; S} : \susp-{C^-}}}{
    \tctx ; \octx \seq \vof{V : \rfoc{A^+}} &
    \tctx ; \ofill{\lfoc{B^-}} \seq \sof{S : \susp-{C^-}}}
  \and
  \infer{\tctx ; \omatch{\lfoc{A^-_1 \with A^-_2}} \seq \sof{\fst{S} : \susp-{C^-}}}{
    \tctx ; \ofill{\lfoc{A^-_1}} \seq \sof{S : \susp-{C^-}}}
  \and
  \infer{\tctx ; \omatch{\lfoc{A^-_1 \with A^-_2}} \seq \sof{\snd{S} : \susp-{C^-}}}{
    \tctx ; \ofill{\lfoc{A^-_2}} \seq \sof{S : \susp-{C^-}}}
\end{mathpar}

\begin{mathpar}
  \infer{\tctxe ; x{:}\susp+{\p^+} \pseq \pof{x : \p^+}}{
    }
  \and
  \infer{\tctx_1, \tctx_2 ; \octx_1, \octx_2 \pseq \pof{\pfuse{p_1}{p_2} : A^+_1 \fuse A^+_2}}{
    \tctx_1 ; \octx_1 \pseq \pof{p_1 : A^+_1} &
    \tctx_2 ; \octx_2 \pseq \pof{p_2 : A^+_2}}
  \and
  \infer{\tctxe ; \octxe \pseq \pof{\pone : \one}}{
    }
  \and
  \infer{\tctx, a{:}\tau ; \octx \pseq \pof{\pexists{a.p} : \exists a{:}\tau.A^+}}{
    \tctx ; \octx \pseq \pof{p : A^+}}
  \and
  \infer{\tctxe ; x{:}A^- \pseq \pof{x : A^-}}{
    }
\end{mathpar}

\begin{mathpar}
  \infer{\tctx ; x{:}\susp+{\p^+} \seq \vof{x : \rfoc{\p^+}}}{
    }
  \and
  \infer{\tctx ; \octx_1, \octx_2 \seq \vof{\vfuse{V_1}{V_2} : \rfoc{A^+_1 \fuse A^+_2}}}{
    \tctx ; \octx_1 \seq \vof{V_1 : \rfoc{A^+_1}} &
    \tctx ; \octx_2 \seq \vof{V_2 : \rfoc{A^+_2}}}
  \and
  \infer{\tctx ; \octxe \seq \vof{\vone : \rfoc{\one}}}{
    }
  \and
  \infer{\tctx ; \octx \seq \vof{\vexists{t.V} : \rfoc{\exists a{:}\tau.A^+}}}{
    \tctx \seq t : \tau &
    \tctx ; \octx \seq \vof{V : \rfoc{\subst{t/a}{A^+}}}}
  \and
  \infer{\tctx ; \octx \seq \vof{N : \rfoc{A^-}}}{
    \tctx ; \octx \seq \nof{N : A^-}}
\end{mathpar}

\begin{mathpar}
  \infer{\tof{\tstep{p <- R} :: (\tctx ; \omatch{\octx}) \trans* (\tctx, \tctx_{A^+} ; \ofill{\octx_{A^+}})}}{
    \tctx ; \octx \seq \aof{R : \susp-{\monad{A^+}}} &
    \tctx_{A^+} ; \octx_{A^+} \pseq \pof{p : A^+}}
  \and
  \infer{\tof{\tnil :: (\tctx ; \octx) \trans* (\tctx ; \octx)}}{
    }
  \and
  \infer{\tof{\tseq{T_1 ; T_2} :: (\tctx ; \octx) \trans* (\tctx'' ; \octx'')}}{
    \tof{T_1 :: (\tctx ; \octx) \trans* (\tctx' ; \octx')} &
    \tof{T_2 :: (\tctx' ; \octx') \trans* (\tctx'' ; \octx'')}}
\end{mathpar}


\subsubsection{Fairness}\label{sec:fairness}

\begin{equation*}
  \octx \seq \aof{R : \susp-{\monad{A^+}}}
\end{equation*}
\end{document}

%%% Local Variables:
%%% TeX-master: "ordered-lp"
%%% End:


\end{document}


% arara: pdflatex
% arara: biber
% arara: pdflatex
% arara: pdflatex
\documentclass[
  class=../hdeyoung-proposal,
  crop=false
]{standalone}

\usepackage[subpreambles]{standalone}

\usepackage{ordered-logic}
\usepackage{binary-counter}

\addbibresource{../proposal.bib}

\begin{document}

\section{Choreographies}\label{sec:choreographies}

Traditionally, concurrency is phrased as the composition of interacting, \wc{locally executing}[\st{distributed}] processes.
As the binary counter example from \cref{sec:exampl-binary-count-2,sec:exampl-binary-count-3} demonstrates, a notion of concurrency based on indistinguishable interleavings of independent rewrites arises naturally in ordered logic programming.
% a notion of concurrency based on indestinguishable interleavings arises naturally in ordered logic programming.
% However, it is not as clear how to identify a notion of process
But where are the \wc{locally executing}[\st{distributed}] processes?

Taking a formula-as-process view \autocites{Miller:ELP92}{Cervesato+Scedrov:IC09}, the processes are the ordered logic program's atomic propositions.
% This thesis proposes that the atomic propositions in an ordered logic program are the processes.
% \fxnote{[How much of this is already implied by a formula-as-process interpretation?]}%
The program's clauses, accordingly, serve to specify the valid interactions among processes. 
In the binary counter,
% \wc{specification}[\st{program}], 
for example, $\eps$, $\bit{0}$, $\bit{1}$, and $\inc$, among others,
% $\dec$, $\bit[']{0}$, $\zero$, and $\succ$ 
are
% all 
atoms-as-processes,
% whose interactions are governed by the program's rules.
% In particular, the rule
and the clause
\begin{equation*}
  \bit{1} \fuse \inc \lrimp \monad{\inc \fuse \bit{0}}
\end{equation*}
% says that neighboring $\bit{1}$ and $\inc$ processes \fxnote{should be able to} interact to form neighboring $\inc$ and $\bit{0}$ processes.
says that one valid interaction is for neighboring $\bit{1}$ and $\inc$ processes to \wc{coordinate}[\st{cooperate}] to become neighboring $\inc$ and $\bit{0}$ processes.
% (with similar readings for the other clauses).

The program's clauses don't tell the full story, however:
the clauses specify\fxnote{\ globally} \emph{what} are valid interactions but not \emph{how} to realize \wc{them}[\st{those interactions}]\fxnote{\ \st{locally}}.
% The program is thus only a \vocab{specification}, with the how instead being supplied by the logic programming language's operational semantics.
In ordered logic programming, the how is\fxnote{\ \st{instead}} traditionally supplied by an operational semantics in which a\fxnote{\st{n omniscient}} central conductor, having the benefit of a global view of all atoms, directs the atoms' interactions according to the program's clauses.
% The how is instead supplied by the language's operational semantics.
% % The  language's operational semantics instead supplies the how.
% % The program is thus only a \vocab{specification}, with
% In the usual operational semantics for ordered logic programming, there is a central \enquote{conductor} who, having the benefit of a global view of all atoms, directs the atoms' interactions according to the program's clauses.
But because they rely so heavily on the central conductor, processes using this semantics are no more than superficially \wc{local}[\st{distributed}].
% Under this semantics, however, processes are only nominally distributed because they rely so heavily on the central conductor.


To be truly \wc{local}[\st{distributed}], the processes should instead communicate directly with their neighbors to identify which, if any, of the valid interactions are possible for them at that moment.
% For instance, by communicating directly with its left-hand neighbor, an $\inc$ process might learn that that neighbor is a $\bit{1}$ process and that the above clause therefore applies; further direct communication between the two processes would effect
For instance, by communicating directly with its left-hand neighbor, an $\inc$ process might learn that that neighbor is a $\bit{1}$ process; with further direct communication, the $\bit{1}$ and $\inc$ processes could coordinate to effect the above\fxnote{\ globally specified} interaction.
% How does $\bit{1}$, for example, learn that its right-hand neighbor is $\inc$ and that the above clause therefore applies?


So, the distinction being drawn here is one between a \vocab{specification} and its \vocab{choreography}---the what and the how.
% \fxnote{\st{The original program is only a specification of the valid process interactions, whereas a choreography is a pattern of communication that implements that specification.}}
A specification is the original program, which serves as a \emph{global} description of the valid process interactions; a choreography is a \emph{local}\fxnote{\ message-passing} implementation of that specification.%
% A specification is the original program used to describe the valid process interactions, whereas a choreography is a pattern of communication that implements that specification%
% \footnote{Notice that we always speak of a choreography relative to a specification, just as an implementation is always relative to an abstraction.}%
% , and which must be given by the programmer, at least implicitly.%
\footnote{We borrow the term \enquote*{choreography} from the literature on session-based concurrency.
The analogy is intended only as a loose one, however, and should not be taken to imply a precise, technical correspondence.}%
%, which must also be given by the programmer, at least implicitly.%
\footnote{Notice that a choreography is always relative to a given specification.}

% Designing a one-size-fits-all distributed operational semantics appears to be difficult, however.
% We could try to design a one-size-fits-all local operational semantics, but this appears to be difficult.
Ideally, an operational semantics would automatically generate a choreography from the specification supplied by the programmer, but designing such a semantics\fxnote{\ \st{unfortunately}} appears to be difficult.
% % Designing a distributed operational semantics that is uniformly suitable appears to be difficult, however.
% % Designing a uniformly suitable distributed operational semantics appears to be difficult, however.
Different specifications will often require different patterns of interprocess communication;
sometimes a specification will even admit several choreographies, and, more often than not, the programmer will want to exercise control in those cases.
% % Therefore, rather than relying on a one-size-fits-all operational semantics, the programmer must indicate the intended pattern of communication for each program.
% Not having a one-size-fits-all local operational semantics, the programmer himself must indicate the intended pattern of communication for each program.
Therefore, not having a one-size-fits-all local operational semantics, the programmer himself must supply the choregraphy, at least implicitly.

In the previous \lcnamecref{sec:??}, we saw several examples of specifications (characterized as forward-chaining ordered logic programs), including the aforementioned binary counter supporting increment and decrement operations.
We'll now describe what counts as a choreography, first with informal examples and then with formal definitions.



% % Designing a one-size-fits-all distributed operational semantics appears to be difficult, however.
% % We could try to design a one-size-fits-all local operational semantics, but this appears to be difficult.
% Ideally, the operational semantics would localize programs in this way, but, unfortunately, designing such a semantics\fxnote{\ \st{that is also uniformly suitable}} appears to be difficult.
% % % Designing a distributed operational semantics that is uniformly suitable appears to be difficult, however.
% % % Designing a uniformly suitable distributed operational semantics appears to be difficult, however.
% Different programs will often require different patterns of interprocess communication;
% sometimes a program will even admit several communication patterns, and, more often than not, the programmer will want to exercise control in those cases.
% % Therefore, rather than relying on a one-size-fits-all operational semantics, the programmer must indicate the intended pattern of communication for each program.
% Not having a one-size-fits-all local operational semantics, the programmer himself must indicate the intended pattern of communication for each program.

% % The distinction being drawn here is one between a \vocab{specification} and its \vocab{choreography}---the what and the how.
% % The original program serves only as a specification of what are valid process interactions, whereas the choregraphy is the programmer's intended 

% So, the distinction being drawn here is one between a \vocab{specification} and its \vocab{choreography}---the what and the how.
% % \fxnote{\st{The original program is only a specification of the valid process interactions, whereas a choreography is a pattern of communication that implements that specification.}}
% A specification is the original program, which serves as a global description of the valid process interactions; a choreography is a local\fxnote{, message-passing} implementation of that specification.%
% % A specification is the original program used to describe the valid process interactions, whereas a choreography is a pattern of communication that implements that specification%
% % \footnote{Notice that we always speak of a choreography relative to a specification, just as an implementation is always relative to an abstraction.}%
% % , and which must be given by the programmer, at least implicitly.%
% \footnote{We borrow the term \enquote*{choreography} from the literature on session-based concurrency.
% The analogy is intended only as a loose one, however, and should not be taken to imply a precise, technical correspondence.}%
% %, which must also be given by the programmer, at least implicitly.%
% \footnote{Notice that a choreography is always relative to a given specification.}

% In the previous \lcnamecref{sec:??}, we saw several examples of specifications (characterized as forward-chaining ordered logic programs), including a binary counter supporting increment and decrement operations.
% We'll now describe what counts as a choreography, first with informal examples and then with formal definitions.

% To build intuition, we'll now describe choregraphies by example


% , adapting terminology from the concurrency literature,


% So, unfortunately, 


% Unfortunately, because different programs will require different patterns of communication among processes, we won't be able to leave the how up to the operational semantics.
% The programmer will want control over the communication patterns.



% Borrowing terminology from the literature on sessions

% In session terminology, the logic program with a centralized operational semantics is known as an orchestration of processes, whereas the desired distributed semantics is known as a choreography.




% \mbox{}\\

% The program's clauses don't tell the full story, however:
% The clauses specify what are valid interactions but not \emph{how} to realize those interactions; the \enquote*{how} is instead supplied by the logic programming language's operational semantics.
% In the usual operational semantics, there is a central \enquote{conductor} who, having the benefit of a global view of all atoms, directs the atoms' interactions according to the program's clauses.

% However, because they rely so heavily on the central conductor, processes using this semantics are no more than superficially distributed.
% % Under this semantics, however, processes are only nominally distributed because they rely so heavily on the central conductor.
% To be truly distributed, the processes should instead communicate directly with their neighbors to identify which, if any, of the valid interactions are possible for them at that moment.

% It's difficult to argue that this centralized \enquote{how} is suitable for \emph{distributed} processes, however.
% The distributed processes should instead communicate directly with their neighbors to identify which, if any, of the valid interactions are possible for them at that moment.
% How does $\bit{1}$, for example, learn that its right-hand neighbor is $\inc$ and that the above clause therefore applies?

% In session terminology, the logic program with a centralized operational semantics is known as an orchestration of processes, whereas the desired distributed semantics is known as a choreography.




% The program's clauses do not tell the full story, however: the clauses specify what are valid interactions but not \emph{how} to realize those interactions.
% In the usual operational semantics, the \enquote{how} is supplied by providing a central \enquote{conductor} that, having the benefit of a global view\fxnote{\ \st{of all atoms}}, manipulates the atoms according to the program's clauses.
% But it's difficult to argue that this centralized \enquote{how} is suitable for \emph{distributed} processes.
% Distributed processes should instead communicate directly with their neighbors to identify which, if any, of the valid interactions are possible for them at that moment.
% How does $\bit{1}$, for example, learn that its right-hand neighbor is $\inc$ and that the above clause therefore applies?






% This isn't the full story, however:
% % The program's clauses specify \emph{what} are valid interactions but not \emph{how} to achieve those interactions.
% the program's clauses specify what are valid interactions but not \emph{how} to achieve them.
% The \enquote{how} is provided by the logic programming language's operational semantics.
% The usual operational semantics


% How do $\bit{1}$ and $\inc$, for example, learn that they are neighbors and that the above clause therefore applies?


% the \enquote{how}it is provided by the logic programming language's operational semantics.
% The usual operational semantics for logic programming 

% This isn't the full story, however.
% The usual operational semantics for ordered logic programming assumes a central \enquote{puppeteer} that has a global view of all atoms and manipulates them according to the program's clauses.
% It's difficult to argue that this centralization is appropriate for distributed processes, however.
% Instead, the processes should communicate directly to identify their neighbors and thereby deduce which, if any, of the valid interactions are possible for them at that moment.
% But this communication is left unspecified in the original logic program.
% How does $\bit{1}$, for example, learn that its right-hand neighbor is $\inc$ and that the above clause therefore applies?

% % What's left unspecified in the ordered logic program is how the distributed processes communicate to identify their neighbors and thereby deduce which, if any, of the valid interactions are possible for them at that moment.
% % % Using a communication protocol that is left unspecified in the program, the atoms deduce 
% % How does $\bit{1}$, for example, learn that its right-hand neighbor is $\inc$ and that the above clause therefore applies?

% Orchestration vs. choreography


% % arara: pdflatex
% % arara: pdflatex
% % arara: biber
% % arara: pdflatex
% % arara: pdflatex
% % \documentclass{../hdeyoung-proposal}
% \documentclass[
%   class=../hdeyoung-proposal,
%   crop=false
% ]{standalone}

% \usepackage{ordered-logic}
% \usepackage{basic-atoms}
% \usepackage{binary-counter}

% \crefname{choreography}{chor.}{chors.}
% \Crefname{choreography}{Chor.}{Chors.}

% \DeclareAcronym{BHK}{
  short = BHK,
  long  = Brouwer-Heyting-Kolmogorov
}

\DeclareAcronym{JILL}{
  short = JILL,
  long  = judgmental intuitionistic linear logic
}

\DeclareAcronym{ILL}{
  short = ILL,
  long  = intuitionistic linear logic
}

\DeclareAcronym{SML}{
  short = SML,
  long  = Standard ML
}

\DeclareAcronym{SOS}{
  short = SOS,
  short-indefinite = an,
  long = structural operational semantics
}

\DeclareAcronym{SSOS}{
  short = SSOS,
  short-indefinite = an,
  long = substructural operational semantics
}

\DeclareAcronym{SILL}{
  short = SILL,
  long = session-typed intuitionistic linear logic
}

\DeclareAcronym{SISLL}{
  short = singleton \acs{SILL},
  long = session-typed intuitionistic singleton linear logic
}

\DeclareAcronym{CLF}{
  short = CLF,
  long = the Concurrent Logical Framework
}


% \begin{document}

\subsection{Choreographies by example}\label{sec:chor-by-example}

\subsubsection{The binary counter}\label{sec:chor-example-counter}

In giving the intuition behind the binary counter specification (\cref{sec:olp-intuition:binary-counter}), we described the $\inc$ atoms % as moving --- moving past any $\bit{1}$s and eventually stopping at the $\eps$ or right-most $\bit{0}$.
as moving up the counter.
% % , a subliminal hint that $\inc$s are like messages.
% % This suggests a choreography in which $\inc$ processes take the active lead:
% This hints that $\inc$s are a bit like messages, and suggests a choreography in which $\inc$ processes initiate the interaction:
% First, each $\inc$ process sends a message, $\inc[<-]$, to its left-hand neighbor, thereby notifying that neighbor of its existence, and then the $\inc$ process terminates.
% If the neighbor is $\eps$, $\bit{0}$, or $\bit{1}$, then, upon receiving the $\inc$'s message, that neighbor takes full responsibility for completing the corresponding interaction.
This hints at a choreography in which $\inc$ atoms act as messages that trigger the increment action:
Whenever an $\inc$ message arrives at an $\eps$, $\bit{0}$, or $\bit{1}$ process, that process takes responsiblity for completing the increment action.
% First, each $\inc$ process sends a message, $\inc[<-]$, to its left-hand neighbor, thereby notifying that neighbor of its existence, and then the $\inc$ process terminates.
% If the neighbor is $\eps$, $\bit{0}$, or $\bit{1}$, then, upon receiving the $\inc$'s message, that neighbor takes full responsibility for completing the corresponding interaction.


% In giving the intuition behind the binary counter specification (\cref{sec:olp-intuition:binary-counter}), we described the $\inc$ atoms as moving up the counter.
% % This hints that $\inc$s are a bit like messages, and suggests a choreography in which $\inc$ processes initiate the interaction:
% % First, each $\inc$ process sends a message, $\inc[<-]$, to its left-hand neighbor, thereby notifying that neighbor of its existence, and then the $\inc$ process terminates.
% % If the neighbor is $\eps$, $\bit{0}$, or $\bit{1}$, then, upon receiving the $\inc$'s message, that neighbor takes full responsibility for completing the corresponding interaction.
% This hints at a choreography in which $\inc$ atoms are messages that trigger the increment action by $\eps$, $\bit{0}$, and $\bit{1}$ atoms that act as processes.
% When the $\eps$, $\bit{0}$, or $\bit{1}$, processes receive the $\inc[<-]$ message,  then, upon receiving the $\inc$'s message, that neighbor takes full responsibility for completing the corresponding interaction.

Expressed as an annotation of the original ordered logical specification, this choreography is:
\begin{equation}\label[choreography]{chor:oop-counter}
  \!\begin{aligned}
    &\eps \fuse \inc[<-] \lrimp \monad{\eps \fuse \bit{1}} \\
    &\bit{0} \fuse \inc[<-] \lrimp \monad{\bit{1}} \\
    &\bit{1} \fuse \inc[<-] \lrimp \monad{\inc[<-] \fuse \bit{0}}
    \text{\,,}
  \end{aligned}
\end{equation}
where the $\eps$, $\bit{0}$, and $\bit{1}$ atoms are viewed as processes, but the $\inc[<-]$ atoms are viewed as messages.
%
Two properties are crucial:
\begin{description}[font=\normalfont\itshape, leftmargin=\parindent, labelindent=\leftmargin, listparindent=\parindent, parsep=0pt]
\item[Locality.]
  Each clause's premise depends on exactly one process-like atom and (at most) one message-like atom.
  Consequently, each process's decisions are entirely local: the $\eps$, $\bit{0}$, and $\bit{1}$ processes act (independently) only after receiving an $\inc[<-]$ message.%
  \footnote{In {SSOS} terminology, processes that wait to receive a message, like $\eps$, $\bit{0}$, and $\bit{1}$ here, would be termed \vocab{latent} propositions; and messages, like $\inc[<-]$ here, would be termed \vocab{passive} propositions.}

  Locality serves to ensure that the choreography describes sensible message-passing behaviors.
  A clause such as $\inc[<-] \lrimp \monad{{\dots}}$, whose premise does not contain a process-like atom, is not message-passing because no process receives the $\inc[<-]$ message.
%
\item[Specification-preserving.]
% % % Second, notice that
% % The choreography exposes the same $\eps$, $\bit{}$, and $\inc$ processes as the original binary counter specification; the last three clauses of the choreography differ from the specification's clauses only in the substitution of $\inc[<-]$ for $\inc$ in their premises.
% The choreography exposes the same $\eps$, $\bit{}$, and $\inc$ processes as the original binary counter specification.
% Its clauses differ from those of the specification only in the substitution of $\inc[<-]$ for $\inc$ in their premises.
% (The choreography also includes an $\inc \lrimp \monad{\inc[<-]}$ clause to justify that substitution.)
% In this sense, there is a very strong equivalence between the two programs.
% The choreography does not fundamentally alter the specification---it only refines that specification by making the communication patterns explicit.
%
The choreography exposes the same behaviors for $\eps$, $\bit{}$, and $\inc$ as in the original specification.
Its clauses are exactly those of the specification, except that each $\inc$ atom in the specification has been annotated as an $\inc[<-]$ message-like atom in the choreography.

In this sense, there is a very strong, lock-step equivalence between the choreography and its specification.
The choreography does not fundamentally alter the specification---it only refines that specification by making the communication patterns explicit.
\end{description}
%
% In this sense, there is a strong equivalence between the 
% The choreography does not fundamentally alter the implementation given in the original program---it only refines that implementation by making the communication patterns explicit.
% In this sense, there is a strong equivalence between, which will be made precise in \cref{??}
%
% Notice that this choreography \wc{refactors} the original program so that each new clause depends on exactly one process atom and at most one message atom.
% In this way, each process's decisions are completely local: the $\inc$ process always sends $\inc[<-]$ regardless of its neighbors, and the $\eps$ and $\bit{}$ processes act only after receiving an $\inc[<-]$ message.%
% \footnote{In \ac{SSOS} terminology, processes that act regardless of their neighbors, like $\inc$, would be termed \vocab{active} propositions; processes that wait to receive a message, like $\eps$, $\bit{0}$, and $\bit{1}$, would be termed \vocab{latent} propositions; and messages, like $\inc[<-]$, would be termed \vocab{passive} propositions.}
%
It's convenient to think of the programmer as supplying this choreography in full, but in practice the programmer might only give the assignment of roles to atoms, \eg\ $\inc[<-]$ for $\inc$.

\subsubsection{Messages can flow in both directions}\label{sec:chor-binary-count}

In our binary counter specification with decrements (\cref{sec:olp-intuition:decrements}), $\dec$ atoms propagate up the counter similarly to $\inc$s, with the difference that each $\dec$ atom eventually gives rise to either a $\fail$ or $\suc$ atom that travels back down the counter.
Once again, this hints at a choreography in which $\dec$, $\fail$, and $\suc$ atoms are message-like:
\begin{itemize}
\item Whenever a $\dec[<-]$ message arrives at an $\eps$, $\bit{0}$, or $\bit{1}$ process's right-hand side, that process completes the local decrement action:
      the $\eps$ and $\bit{1}$ processes send a $\fail[->]$ or $\suc[->]$ message, respectively, to their right;
      the $\bit{0}$ process forwards the $\dec[<-]$ message to its left and continues as a $\bit[']{0}$ process.
\item Whenever a $\fail[->]$ or $\suc[->]$ message arrives at a $\bit[']{0}$ process's left-hand side, that process forwards the message to its right-hand neighbor and continues as a $\bit{0}$ or $\bit{1}$ process, respectively.
% \item Each $\dec$ process sends a message, $\dec[<-]$, to its left-hand neighbor and terminates.
%       If the neighbor is $\eps$, $\bit{0}$, or $\bit{1}$, then, upon receiving the message, that neighbor completes the corresponding interaction given in the specification.
% \item Each $\fail$ or $\suc$ process sends a message, $\fail[->]$ or $\suc[->]$, respectively, to its \emph{right-hand} neighbor and terminates.
%       If the neighbor is $\bit[']{0}$, then, upon receiving the message from $\fail$ or $\suc$, that neighbor completes the corresponding interaction.
\end{itemize}
To account for decrements, the binary counter's choreography is therefore extended with the following clauses:
\begin{equation}
  \!\begin{aligned}
    &\eps \fuse \dec[<-] \lrimp \monad{\eps \fuse \fail[->]} \\
    &\bit{0} \fuse \dec[<-] \lrimp \monad{\dec[<-] \fuse \bit[']{0}} \\
    &\bit{1} \fuse \dec[<-] \lrimp \monad{\bit{0} \fuse \suc[->]} \\[1.5\jot]
    % 
    &\fail[->] \fuse \bit[']{0} \lrimp \monad{\bit{0} \fuse \fail[->]} \\
    &\suc[->] \fuse \bit[']{0} \lrimp \monad{\bit{1} \fuse \suc[->]}
    \,.
  \end{aligned}
\end{equation}
Once again, these clauses are just an annotation of the original specification's clauses, with $\dec$, $\suc$, and $\fail$ annotated as $\dec[<-]$, $\suc[->]$, and $\fail[->]$.
% Once again, the atoms that are decorated with arrows are formally distinct from their undecorated counterparts.
% (As before, the atoms that are decorated with arrows are formally distinct from their undecorated counterparts.)
The extended choreography thus continues to be specification-preserving.

This extended choreography illustrates that message atoms may be either left-directed, like $\inc[<-]$ and $\dec[<-]$, or right-directed, like $\fail[->]$ and $\suc[->]$.
% Moreover, a message's direction determines the structure of premises in which it is received:
% a left-directed (right-directed) message must arrive at the receiving process's right (resp., left) side, otherwise the message would not be traveling from left to right (resp., right to left).
% 
% Because it is traveling left-to-right, a left-directed message must always arrive at the right-hand side of its recipient; dually, a right-directed message must always arrive at the left-hand side of its recipient.
Because a left-directed message travels from right to left, it must always arrive at the right-hand side of its recipient; dually, a right-directed message must always arrive at the left-hand side of its recipient.
This directionality is another aspect of locality, and it further constrains the structure of a choreography's premises.
For example, this choreography's premises are well-formed because each message flows toward its recipient, whereas premises of the forms $\matom[<-] \fuse \patom$ or $\patom \fuse \matom[->]$ are not well-formed because process $\patom$ doesn't receive the $\matom[<-]$ or $\matom[->]$ message.
% For instance, $\bit{1} \fuse \inc[<-]$ and $\fail[->] \fuse \bit[']{0}$ are well-formed premises because each message flows toward its recipient, whereas premises of the forms $\matom[<-] \fuse \p$ or $\p \fuse \matom[->]$ are not well-formed because process $\p$ doesn't receive the $\matom[<-]$ or $\matom[->]$ message.

% As well as retaining locality, notice that this extended choreography continues to be specification-preserving:
% the choreography's clauses differ from those of the specification only in using the decorated forms
% % the substitution of
% $\dec[<-]$, $\fail[->]$, and $\suc[->]$
% in place of
% % for
% their undecorated counterparts.


\subsubsection{Choreographies are not always unique}\label{sec:mult-chor-are}

As alluded to previously, multiple choreographies are possible for some specifications.

This is true of our binary counter specification, for instance.
(To simplify the example, we'll ignore decrements for now.)
In the $\inc[<-]$-choreography (\cref{sec:chor-example-counter}),
% the $\inc$ processes initiate the interaction but leave all remaining work to the $\eps$, $\bit{0}$, and $\bit{1}$ processes alone.
the counter's value is represented by a chain of $\eps$, $\bit{0}$, and $\bit{1}$ processes that are acted upon by $\inc[<-]$ messages.
%
% Alternatively, the $\inc$ processes could wait for $\eps$, $\bit{0}$, or $\bit{1}$ to initiate the interaction, but thereafter take full responsibility for its completion.
Alternatively, the counter's value could be represented by a sequence of $\eps[->]$, $\bit{0}[->]$, and $\bit{1}[->]$ messages; when fed such a message sequence, an $\inc$ process would emit another sequence that represents the result:
%
% Specifically, each $\eps$, $\bit{0}$, and $\bit{1}$ process sends an identifying message, $\eps[->]$, $\bit{0}[->]$, or $\bit{1}[->]$, to its right-hand neighbor and then terminates.
% If the neighbor is $\inc$, then, upon receiving the message, that $\inc$ completes the corresponding interaction.
% % takes responsibility for carrying out the corresponding clause of the specification.
\begin{equation}
  \!\begin{aligned}
    &\eps[->] \fuse \inc \lrimp \monad{\eps[->] \fuse \bit{1}[->]} \\
    &\bit{0}[->] \fuse \inc \lrimp \monad{\bit{1}[->]} \\
    &\bit{1}[->] \fuse \inc \lrimp \monad{\inc \fuse \bit{0}[->]}
      \,.
  \end{aligned}
\end{equation}
Once again, this choreography possesses the locality and specification-preserving properties.

% Owing to the difference in roles held by, these two choreographies have distinct flavors.
% These two choreographies
These two choreographies
% presented thus far
have distinct flavors, owing to the different process and message roles that they assign to the $\inc$ and $\eps$, $\bit{0}$, and $\bit{1}$ atoms.
The $\inc[<-]$-choreography has an object-oriented character: by sending an $\inc[<-]$ message, the increment method dispatches on the receiving object's class---either $\eps$, $\bit{0}$, or $\bit{1}$.
In contrast, this new $\bit{}[->]$-choreography has a functional character: $\inc$ is a function that receives its argument as a sequence of messages---either $\eps[->]$, $\bit{0}[->]$, or $\bit{1}[->]$.

% The increment method dispatches on 
% $\inc$ invokes the increment method on the neighboring object by sending an $\inc[<-]$ message
% There, the $\inc$ method sends an $\inc[<-]$ message like a method that dispatches on the class of the recipient object---either $\eps$, $\bit{0}$, or $\bit{1}$.
% Our first choreography has an object-oriented flavor, with $\inc$ like a method that dispatches on the class of the recipient object---either $\eps$, $\bit{0}$, or $\bit{1}$.
% In contrast, this second choreography has a more functional flavor, with 

% This alternate choreography has a funcitonal flavor: $\inc$ can be viewed as a function on the $\eps$-and-$\bit{}$ representation of data.
% In contrast, the previous choreography has a more object-oriented flavor

% The difference in sender and recipient between this alternate choreography and the previous one gives the two choreographies different flavors.
% In this alternate choreography
% In constrast, the previous choreography has a more object-oriented flavor, with $\inc$ being a method that dispatches on the class of the recipient---either $\eps$, $\bit{0}$, or $\bit{1}$.

% If we view the $\eps$ and $\bit{}$ processes as data, then this alternate choreography has a functional flavor.
% In contrast, the previous choreography has an object-oriented flavor, with a dynamic dispatch of $\inc[<-]$ on the recipient.

\subsubsection{Two non-choreographies}\label{sec:non-choreographies}

Another, slightly more complex reformulation of the binary counter specification chooses to treat the $\inc$ atom as a simple process, not a message:
\begin{equation}
  \!\begin{aligned}
    &\inc \lrimp \monad{\incm[<-]} \\[1.5\jot]
    % 
    &\eps \fuse \incm[<-] \lrimp \monad{\eps \fuse \bit{1}} \\
    &\bit{0} \fuse \incm[<-] \lrimp \monad{\bit{1}} \\
    &\bit{1} \fuse \incm[<-] \lrimp \monad{\inc \fuse \bit{0}}
      \,.
  \end{aligned}
\end{equation}
In fact, the $\inc$ process does nothing but send an $\incm[<-]$ message.

This signature is equivalent to the binary counter specification in that it ultimately exposes the same $\eps$, $\bit{}$, and $\inc$ behaviors.
However, it is \emph{not} specification-preserving under the informal definition that we have used thus far.
% In contrast with the choreographies, this implementation's main clauses are not a simple decoration of the specification's clauses: each of the specification's clauses is spread across several clauses here.
In contrast with the choreographies, this formulation does more than simply refine the specification by making the communication explicit: it introduces a new message-like atom, $\incm[<-]$, a new clause, $\inc \lrimp \monad{\incm[<-]}$, and modifies the existing clauses.






% Another, more complex implementation of the binary counter specification divides the work of completing the interaction among the $\eps$ and $\bit{}$ and (the continuation of) the $\inc$ processes:
% \begin{equation}
%   \!\begin{aligned}
%     &\inc \lrimp \monad{\inc[^l, <-] \fuse \inc[^r]} \\[1.5\jot]
%     % 
%     &\eps \fuse \inc[^l, <-] \lrimp \monad{\eps \fuse \eps[^r, ->]} \\
%     &\bit{0} \fuse \inc[^l, <-] \lrimp \monad{\bit{0}[^r, ->]} \\
%     &\bit{1} \fuse \inc[^l, <-] \lrimp \monad{\inc \fuse \bit{1}[^r, ->]} \\[1.5\jot]
%     % 
%     &\eps[^r, ->] \fuse \inc[^r] \lrimp \monad{\bit{1}} \\
%     &\bit{0}[^r, ->] \fuse \inc[^r] \lrimp \monad{\bit{1}} \\
%     &\bit{1}[^r, ->] \fuse \inc[^r] \lrimp \monad{\bit{0}} \,.
%   \end{aligned}
% \end{equation}
% In this implementation, $\inc$ first sends an $\inc[^l, <-]$ message to its left-hand neighbor and then waits for a response as process $\inc[^r]$.
% Upon receiving an $\inc[^l, <-]$ message, the recipient process, either $\eps$, $\bit{0}$, or $\bit{1}$, partially completes the interaction and sends an identifying message to its right-hand neighbor, which is necessarily an $\inc[^r]$ process.
% The $\inc[^r]$ process finishes the interaction once it receives the identifying message.

% This implementation is equivalent to the binary counter specification in that it ultimately exposes the same $\eps$, $\bit{}$, and $\inc$ processes.
% However, it is not specification-preserving under the informal definition that we have used thus far.
% % In contrast with the choreographies, this implementation's main clauses are not in one-to-one correspondence with those of the specification.
% % In contrast with the choreographies, this implementation's main clauses are not a simple decoration of the specification's clauses: each of the specification's clauses is spread across several clauses here.
% In contrast with the choreographies, this implementation does more than simply refine the specification by making the communication explicit: using a temporary $\inc[^r]$ process, it spreads each of the specification's clauses across several clauses.

So, this signature is not specification-preserving, and therefore not a choreography, for the binary counter specification.
But that doesn't mean that the programmer cannot achieve the same behavior anyway: the programmer is free to rewrite the \emph{specification} to incorporate the behavior at the specification level.
% Although this implementation is not specification-preserving for the binary counter specification, and therefore not a choreography, the programmer can nevertheless achieve the same behavior by changing the \emph{specification}.
If the specification is changed to be
\begin{equation}
  \!\begin{aligned}
    &\inc \lrimp \monad{\incm} \\[1.5\jot]
    % 
    &\eps \fuse \incm \lrimp \monad{\eps \fuse \bit{1}} \\
    &\bit{0} \fuse \incm \lrimp \monad{\bit{1}} \\
    &\bit{1} \fuse \incm \lrimp \monad{\inc \fuse \bit{0}}
      \,,
  \end{aligned}
\end{equation}
then the above signature is indeed a choreography for \emph{this} specification.

% Although this implimentation is not specification-preserving, and therefore not a choreography, for the binary counter specification (\cref{??}), the programmer can nevertheless acheive the same behavior by changing the \emph{specification}.
% Instead of using the binary counter specification from \cref{??}, the following specification could be used because one of its choreographies, which uses $\inc[^l, <-]$, $\eps[^r, ->]$, $\bit{0}[^r, ->]$, and $\bit{1}[^r, ->]$ messages, is nearly identical the disallowed implimentation.
% \begin{equation*}
%   % \!\begin{aligned}[t]
%   %   &\inc[^l] \lrimp \monad{\inc[^l, <-]} \\
%   %   &\eps[^r] \lrimp \monad{\eps[^r, ->]} \\
%   %   &\bit{0}[^r] \lrimp \monad{\bit{0}[^r, ->]} \\
%   %   &\bit{1}[^r] \lrimp \monad{\bit{1}[^r, ->]}
%   % \end{aligned}
%   % \qquad
%   \!\begin{aligned}[t]
%     &\inc \lrimp \monad{\inc[^l] \fuse \inc[^r]} \\
%     &\eps \fuse \inc[^l] \lrimp \monad{\eps \fuse \eps[^r]} \\
%     &\bit{0} \fuse \inc[^l] \lrimp \monad{\bit{0}[^r]} \\
%     &\bit{1} \fuse \inc[^l] \lrimp \monad{\inc \fuse \bit{1}[^r]} \\
%     &\eps[^r] \fuse \inc[^r] \lrimp \monad{\bit{1}} \\
%     &\bit{0}[^r] \fuse \inc[^r] \lrimp \monad{\bit{1}} \\
%     &\bit{1}[^r] \fuse \inc[^r] \lrimp \monad{\bit{0}} \,.
%   \end{aligned}
% \end{equation*}


% Although this implementation ultimately exposes the same $\eps$, $\bit{}$, and $\inc$ processes

% Even so\fxnote{\ \st{Even though this choreography introduces and auxiliary process atom}}, we still consider it to be a valid choreography: $\inc[']$ is only temporary, leaving the underlying specification fundamentally unchanged.

Another signature that is equivalent to the binary counter specification, in the sense that the two track the same value, is
\begin{equation}
  \!\begin{aligned}
    &\num{N} \fuse \inc[<-] \lrimp \monad{\num{(N{+}1)}} \,.
  \end{aligned}
\end{equation}
Nevertheless, we wouldn't consider this to be a choreography of the binary counter specification because, by using a single number held by $\num{}$ instead of a string of $\bit{}$s, it fundamentally alters the specification.
We would, however, consider this signature to be a choreography of a different, simple counter specification, namely $\num{N} \fuse \inc \lrimp \monad{\num{(N{+}1)}}$.

% % Although it is equivalent to the binary counter in the sense that it tracks the same value, we wouldn't consider the following program to be a choreography of the binary counter specification because it fundamentally alters the implementation by using a single $\num{}$ instead of a string of $\bit{}$s.
% The following program is also equivalent to the binary counter specification, in the sense that the two track the same value.
% Nevertheless, we wouldn't consider it to be a choreography of the binary counter because it fundamentally alters the specification by using a single number held by $\num{}$ instead of a string of $\bit{}$s.
% \begin{align*}
%   &\inc \lrimp \monad{\inc[<-]} \\
%   &\num{N} \fuse \inc[<-] \lrimp \monad{\num{(N{+}1)}}
% \end{align*}
% We would, however, consider it to be a choreography of a different, simple counter specification: $\num{N} \fuse \inc \lrimp \monad{\num{(N{+}1)}}$.



% \end{document}

%%% Local Variables:
%%% TeX-master: "choreographies"
%%% End:


% arara: pdflatex
% arara: biber
% arara: pdflatex
% arara: pdflatex
\documentclass[
  class=../hdeyoung-proposal,
  crop=false
]{standalone}

\usepackage{ordered-logic}
\usepackage{basic-atoms}

\NewDocumentCommand{\chor}{}{X}
\NewDocumentCommand{\spec}{}{\Sigma}

\NewDocumentCommand{\erasemsg}{m}{#1^{e}}

\NewDocumentCommand{\trans}{t* t+ o}{%
  \longrightarrow
  \IfBooleanT{#1}{^*}\IfBooleanT{#2}{^+}%
  \IfValueT{#3}{_{#3}}%
}

% \cs_new_protected:Nn \trans: {
%   \peek_meaning:nTF {*}
%     { \@@_trans_star: }
%     { \peek_meaning:nTF {+}
%         { \@@_trans_plus: }
%    \longrightarrow
%   \IfBooleanT{#1}{^*}\IfBooleanT{#2}{^+}%
%   \IfValueT{#3}{_{#3}}%
% }

\begin{document}

\subsection{Choreographies, formally}\label{sec:chor-formal}

Hopefully the preceding examples have given some intuition for what counts as a choreography.
To make the definition precise, we need only formalize the locality and specification-preserving properties.

We use the forward-chaining ordered logic programming language described in \cref{??}, with a few restrictions.


\subsubsection{Locality}\label{sec:locality}

Each clause must have the form $U^+ \lrimp \monad{A^+}$, where $U^+$ is an \vocab{uncurried local premise} that adheres to the following grammar:
\begin{alignat*}{2}
  A^- &::= L^+ \limp \monad{A^+} \mid R^+ \rimp \monad{A^+} \mid A^-_1 \with A^-_2 \\
  A^+ &::= \p^+ \mid \p[->]^+ \mid \p[<-]^+ \mid A^+_1 \fuse A^+_2 \mid \one \mid A^- \\
  U^+ &::= L^+ \fuse U^+ \mid \p^+ \mid U^+ \fuse R^+ \\
  L^+ &::= \p[->]^+ \mid L^+_1 \fuse L^+_2 \mid \one \\
  R^+ &::= \p[<-]^+ \mid R^+_1 \fuse R^+_2 \mid \one
\end{alignat*}
The grammar is a bit complicated, but the idea behind it is simple and matches the intuition behind locality:
an uncurried local premise $U^+$ contains exactly one process atom $\p^+$ that receives right-directed messages, $\p[->]^+$, from its left and left-directed messages, $\p[<-]^+$, from its right.

At the expense of a more complicated grammar, we could allow curried clauses, such as $\p[_1, ->]^+ \lrimp \p[_2, ->]^+ \limp \p^+ \rimp \p[_3, <-]^+ \rimp \monad{A^+}$.
Uncurrying clauses would not seem to place a large burden on the programmer, for it is easy enough to write $\q[->]^+ \fuse \p[->]^+ \fuse \rr^+ \fuse \s[<-]^+ \lrimp \monad{A^+}$, and this detail is anyway orthogonal to what follows.

\subsubsection{Specification-preserving}\label{sec:spec-pres}

\begin{definition}[Specification-preserving]
  An ordered logic program $\chor$ is \vocab{specification-preserving} for specification $\spec$ if:
  \begin{enumerate}
  \item for each step $\octx \trans[\spec] \octx'$ in the specification $\spec$, there is a non-empty trace $\octx \trans+[\chor] \octx'$ in the program $\chor$; and
  \item for each step $\octx \trans[\chor] \octx'$ in the program $\chor$, either $\erasemsg{(\octx)} = \erasemsg{(\octx')}$ or there is a step $\erasemsg{(\octx)} \trans[\spec] \erasemsg{(\octx')}$ in the specification $\spec$.
  \end{enumerate}
\end{definition}

\end{document}

%%% Local Variables:
%%% TeX-master: "choreographies"
%%% End:


\end{document}

% % arara: lualatex
% % arara: lualatex
% % arara: biber
% % arara: lualatex
% % arara: lualatex
% % \documentclass{../hdeyoung-proposal}
% \documentclass[
%   class=../hdeyoung-proposal,
%   crop=false
% ]{standalone}


% \usepackage{linear-logic}
% \usepackage{ordered-logic}
% \usepackage{proof}
% \usepackage{mathpartir}

% \usepackage{tikz}
% \usetikzlibrary{shapes.misc,graphs,quotes,graphdrawing}
% \usegdlibrary{trees}

% \usepackage{scalerel}

% \ExplSyntaxOn

% % \DeclarePairedDelimiter \parens { \lparen } { \rparen }
% \DeclarePairedDelimiter \braced:wn { \lbrace } { \rbrace }
% \NewDocumentCommand{ \braced }{ s o m o }
%   {
%     \IfBooleanTF {#1}
%       { \braced:wn* {#3} }
%       {
%         \IfValueTF {#2}
%           { \braced:wn[#2] {#3} }
%           { \braced:wn {#3} }
%       }
%     \IfValueT {#4} { \sb{#4} }
%   }

% \DeclarePairedDelimiter \bagged:wn { \lbag } { \rbag }
% \NewDocumentCommand{ \bagged }{ s o m o }
%   {
%     \IfBooleanTF {#1}
%       { \bagged:wn* {#3} }
%       {
%         \IfValueTF {#2}
%           { \bagged:wn[#2] {#3} }
%           { \bagged:wn {#3} }
%       }
%     \IfValueT {#4} { \sb{#4} }
%   }


% \NewDocumentCommand \oseq { >{ \SplitArgument{1}{|-} } m }
%   { \oseq:nn #1 }
% \cs_new:Npn \oseq:nn #1#2 { \oseq_ctxs:n {#1} \vdash #2 }
% \cs_new:Npn \oseq_ctxs:n #1 {
%   \seq_set_split:Nnn \l_tmpa_seq {;} {#1}
%   \seq_use:Nn \l_tmpa_seq { \mathrel{;} }
% }

% \NewDocumentCommand \procof { m m } { #1 \dblcolon #2 }
% \NewDocumentCommand \hypof { m } { #1 }


% \NewDocumentCommand \cut { m } { \text{\textsc{\MakeLowercase{Cut}}}\sb{#1} }
% \NewDocumentCommand \id { m } { \text{\textsc{\MakeLowercase{Id}}}\sb{#1} }

% \NewDocumentCommand \comp { >{ \SplitArgument{1}{|} } m }
%   { \comp:nn #1 }
% \cs_new:Npn \comp:nn #1#2 { #1 \parallel #2 }

% \NewDocumentCommand \fwd {} { \mathord{\leftrightarrow} }


% \RenewDocumentCommand \with { s }
%   { \IfBooleanTF {#1} \with:n \with: }
% \cs_new:Npn \with:n #1 {
%   \mathord{\binampersand}
%   \braced {
%     \seq_set_split:Nnn \l_tmpa_seq {,} {#1}
%     \seq_use:Nn \l_tmpa_seq {,}
%   }
% }
% \cs_new:Npn \with: { \mathbin{\binampersand} }

% \NewDocumentCommand \ssor { s }
%   { \IfBooleanTF {#1} \ssor:n \ssor: }
% \cs_new:Npn \ssor:n #1 {
%   \mathord{\ssor:}
%   \braced {
%     \seq_set_split:Nnn \l_tmpa_seq {,} {#1}
%     \seq_use:Nn \l_tmpa_seq {,}
%   }
% }
% \cs_new:Npn \ssor: { \oplus }

% \NewDocumentCommand \caseR { s m o }
%   {
%     \IfBooleanTF {#1}
%       {
%         \IfValueTF {#3}
%           { \case:nNnn { \mathsf{caseR} } \parens {#2} { \sb{#3} } }
%           { \case:nNnn { \mathsf{caseR} } \parens {#2} {} }
%       }
%       { \case:nNnn { \mathsf{caseR} } \parens {#2} {} }
%   }
% \NewDocumentCommand \caseL { s m o }
%   {
%     \IfBooleanTF {#1}
%       {
%         \IfValueTF {#3}
%           { \case:nNnn { \mathsf{caseL} } \parens {#2} { \sb{#3} } }
%           { \case:nNnn { \mathsf{caseL} } \parens {#2} {} }
%       }
%       { \case:nNnn { \mathsf{caseL} } \parens {#2} {} }
%   }
% \cs_new:Npn \case:nNnn #1#2#3#4 {
%   #1#4 \mskip\thinmuskip
%   #2 {
%     \seq_set_split:Nnn \l_tmpa_seq {|} {#3}
%     \seq_clear:N \l_tmpb_seq
%     \seq_map_inline:Nn \l_tmpa_seq
%       { \seq_put_right:Nn \l_tmpb_seq { \case_branch:n {##1} } }
%     \seq_use:Nn \l_tmpb_seq { \talloblong }
%   }
% }
% \cs_new:Npn \case_branch:n #1 { \case_branch_aux:w #1 \q_stop }
% \cs_new:Npn \case_branch_aux:w #1 => #2 \q_stop {
%   #1 \Rightarrow #2
% }

% \NewDocumentCommand \selectL { >{ \SplitArgument{1}{;} } m }
%   { \select:nnn { \mathsf{selectL} } #1 }
% \NewDocumentCommand \selectR { >{ \SplitArgument{1}{;} } m }
%   { \select:nnn { \mathsf{selectR} } #1 }
% \cs_new:Npn \select:nnn #1#2#3 {
%   \!\mathord{}\mathop{#1} #2 ; #3
% }


% \NewDocumentCommand \inj { m } { \mathsf{in}\sb{#1} }

% \NewDocumentCommand \inl {} { \inj{ \mathsf{1} } }
% \NewDocumentCommand \inr {} { \inj{ \mathsf{2} } }


% \RenewDocumentCommand \one {} { \mathord { \mathbf{1} } }

% \NewDocumentCommand \closeR {} { \mathsf{closeR} }
% \NewDocumentCommand \waitL { m } { \mathsf{waitL} ; #1 }


% \NewDocumentCommand \rrule { o m } {
%   \IfValueTF {#1}
%     { \rrule:nn {#2} {#1} }
%     { \rrule:n {#2} }
% }
% \cs_new:Npn \rrule:nn #1#2 { {#1}\text{\textsc{\MakeLowercase{R}}}\sb{#2} }
% \cs_new:Npn \rrule:n #1 { {#1}\text{\textsc{\MakeLowercase{R}}} }

% \NewDocumentCommand \lrule { o m } {
%   \IfValueTF {#1}
%     { \lrule:nn {#2} {#1} }
%     { \lrule:n {#2} }
% }
% \cs_new:Npn \lrule:nn #1#2 { {#1}\text{\textsc{\MakeLowercase{L}}}\sb{#2} }
% \cs_new:Npn \lrule:n #1 { {#1}\text{\textsc{\MakeLowercase{L}}} }


% \NewDocumentCommand \bnd { >{\SplitArgument{1}{=}}m } { \bnd:nn #1 }
% \cs_new:Npn \bnd:nn #1#2 { \mathsf{bnd} \mskip\thinmuskip #1 \mskip\thinmuskip #2 }
% \NewDocumentCommand \exec { } { \mathsf{exec} \mskip\thinmuskip }
% \NewDocumentCommand \msgL { } { \mathsf{msgL} \mskip\thinmuskip }
% \NewDocumentCommand \msgR { } { \mathsf{msgR} \mskip\thinmuskip }
% \NewDocumentCommand \msgQ { } { \mathsf{msgQ} }

% \ExplSyntaxOff




% \addbibresource{../proposal.bib}

% \NewDocumentCommand{\ie}{}{i.e.}


% \usepackage{listings}
% \crefname{listing}{listing}{listings}
% \Crefname{listing}{Listing}{Listings}

% \newlength{\mywidth}
% \settowidth{\mywidth}{\ttfamily A}
% \lstset{basicstyle=\ttfamily, basewidth=\mywidth}

% \captionsetup[lstlisting]{%
%   box=colorbox, boxcolor=gray,
%   font={normalfont, sf, color=white},
%   labelfont=bf,
%   justification=justified, singlelinecheck=false
% }

% \lstnewenvironment{sillcode}[1][]
%   {\lstset{language={},frame=bottomline,framerule=0.8ex,rulecolor=\color{gray},float,#1}}%
%   {}

% \lstnewenvironment{sillcode*}[1][]
%   {\lstset{language={},#1}}%
%   {}

% \NewDocumentCommand{\sillinline}{o}{%
%   \IfValueTF{#1}{\lstinline[#1]}{\lstinline}%
% }


% \NewDocumentCommand{\pctx}{}{\Psi}
% \ExplSyntaxOn
% \NewDocumentCommand{\ctxmonad}{>{\SplitArgument{1}{<-}}m}{
%   \{\use_ii:nn #1 \vdash \use_i:nn #1\}
% }
% \NewDocumentCommand \spawn { >{ \SplitArgument{1}{;} } m } { \spawn:nn #1 }
% \cs_new:Npn \spawn:nn #1#2 {
%   \mathsf{spawn}
%   \tl_if_empty:nF {#1} {
%     \mskip\thinmuskip #1 ; #2
%   }
% }
% \NewDocumentCommand{\mbind}{>{\SplitArgument{1}{;}}m}{
%   \use_i:nn#1 ; \use_ii:nn#1
% }
% \NewDocumentCommand{\mletrec}{o m m}{
%   \mathsf{letrec}\IfValueT{#1}{\sb{#1}}
%   \mskip\thinmuskip
%   #2
%   \mskip\thinmuskip
%   \mathsf{in} \mskip\thinmuskip #3
% }
% \NewDocumentCommand{\mprocdef}{m}{
%   #1
% }
% \ExplSyntaxOff


% \newcommand*\kay{k}



% \DeclareAcronym{BHK}{
  short = BHK,
  long  = Brouwer-Heyting-Kolmogorov
}

\DeclareAcronym{JILL}{
  short = JILL,
  long  = judgmental intuitionistic linear logic
}

\DeclareAcronym{ILL}{
  short = ILL,
  long  = intuitionistic linear logic
}

\DeclareAcronym{SML}{
  short = SML,
  long  = Standard ML
}

\DeclareAcronym{SOS}{
  short = SOS,
  short-indefinite = an,
  long = structural operational semantics
}

\DeclareAcronym{SSOS}{
  short = SSOS,
  short-indefinite = an,
  long = substructural operational semantics
}

\DeclareAcronym{SILL}{
  short = SILL,
  long = session-typed intuitionistic linear logic
}

\DeclareAcronym{SISLL}{
  short = singleton \acs{SILL},
  long = session-typed intuitionistic singleton linear logic
}

\DeclareAcronym{CLF}{
  short = CLF,
  long = the Concurrent Logical Framework
}



% \begin{document}

\section{Session-typed processes from singleton linear logic}\label{sec:sill}

Thus far, we have followed a proof-construction approach to computation, having in \cref{sec:ordered-lp} reviewed an interpretation of ordered logical specifications as concurrent string rewriting and in \cref{sec:choreographies} identified a fragment in which those specifications have a message-passing character.
In this \lcnamecref{sec:sill}, we turn to a proof-reduction view of computation.

Recently, \textcite{Caires+Pfenning:CONCUR10} with Toninho~\autocite*{Caires+:MSCS13} have established a Curry--Howard isomorphism, dubbed \acs{SILL}, between the sequent calculus for intuitionistic linear logic and a session-typed $\pi$-calculus, in which propositions are session types, proofs are session-typed processes, and cut reductions are process reductions.\footnote{\Textcite{Wadler:JFP14} later developed a correspondence between classical linear logic and a session-typed $\pi$-calculus, but \citeauthor{Caires+:MSCS13}'s intuitionistic correspondence turns out to be better suited to our goals here, for reasons we will explain shortly.}
This gives a proof-reduction view of concurrency that differs, apparently substantially, from the proof-construction perspective.
But, by the end of this proposal document, we will have shown that the differences are not as substantial as they first appear.

In this section, we present a reformulation of \citeauthor{Caires+:MSCS13}'s \ac{SILL} for a restriction of intuitionistic linear logic, which we call singleton linear logic, that is a better fit for comparisons with ordered logical specifications.
% As hinted below and further justified in \cref{?}, \ac{SISLL} is a better fit for comparisons with ordered logical specifications.

% [[
% We begin our presentation of \ac{SISLL} with a review of the session-typing judgements of \citeauthor{Caires+:MSCS13}'s \ac{SILL}.
% ]]

\subsection{Toward singleton linear logic}

In a session-based model of concurrency, pairs of processes interact in well-defined sessions, with one process offering a service that its session partner uses.
Session types, pioneered by \textcite{Honda:CONCUR93}, describe the interaction protocol to which a process adheres when offering its service.
% the processes in that session must adhere.
When processes interact, the session type changes: one process now offers, and the other uses, the continuation of the initial service.
As shown by \textcite{Caires+:MSCS13}, the logical reading of session-based concurrency is linear logic exactly because it can express this change of state.

Because a process offers its service along a distinguished channel, the basic session-typing judgment of \citeauthor{Caires+:MSCS13}'s \ac{SILL} is $P :: x{:}A$, meaning \enquote{process $P$ offers a service of session type $A$ along channel $x$}.
However, $P$ itself may rely on services offered by yet other processes, and so, more generally, the \ac{SILL} session-typing judgment is a linear sequent annotated as
\begin{equation*}
  \underbrace{
    x_1{:}A_1 , x_2{:}A_2 , \dots , x_n{:}A_n
  }_{\textstyle \lctx}
  \vdash
  P :: x{:}A
  \quad
  \text{($n \geq 0$)}
  \,,
\end{equation*}
% where the channel names, $x_i$, are needed to unambiguously refer to hypotheses and the consequent.
meaning \enquote{Using services $A_i$ offered along channels $x_i$, the process $P$ offers service $A$ along channel $x$.}
(The channels $x_i$ and $x$ must all be distinct and are binding occurrences with scope over the process $P$.)

In \ac{SILL}, the linear sequent calculus's inference rules thus become session-typing rules for processes.
Just as the inference rules arrange sequents into a proof tree, so do the \ac{SILL} session-typing rules arrange processes into a tree-shaped network in which some processes are clients of more than one process (i.e., some nodes have more than one child).
The following is one such example.
% The way in which \ac{SILL} processes use and offer services along distinct channels thus arranges processes in a tree-shaped network, such as the following, in which some processes are clients of more than one process (i.e., some nodes have more than one child).
\begin{equation*}
  \begin{tikzpicture}[channel/.style = {text depth=0, midway, sloped, above}]
    \graph [
      tree layout, grow=left, math nodes, % empty nodes,
      nodes={
        rounded rectangle, rounded rectangle left arc=none,
        draw, minimum size=3ex,
      },
      edges={-},
    ] {
      / [draw=none] <-["$\scriptstyle x$"' channel]
      P <- { / [> {"$\scriptstyle x_1$"' channel}] <- / <- { / , / } ,
             / [> {"$\scriptstyle x_2$"' channel}] ,
             / [> {"$\scriptstyle x_3$"' channel}] <- { / , / <- / } };
    };
  \end{tikzpicture}
\end{equation*}
% Each edge in the above \ac{SILL} tree represents a top-level $\cut{}$, and each top-level $\cut{}$ corresponds to a hypothesis that serves as its principal formula.

% With this session-typing judgment, the inference rules of the linear sequent calculus become \ac{SILL} session-typing rules for concurrent processes.
% For instance, the cut rule of \ac{SILL} types a parallel composition of processes (shown in $\pi$-calculus syntax here):
% \begin{equation*}
%   \infer[\cut{A}]{\lctx , \lctx' \vdash (\nu x)(P \mid Q) :: z{:}C}{
%     \lctx \vdash P :: x{:}A &
%     \lctx' , x{:}A \vdash Q :: z{:}C}
% \end{equation*}
% Top-level $\cut{}$s thus arrange processes in a tree-shaped network, such as the following, in which some processes are clients of more than one process (i.e., some nodes have more than one child).
% \begin{equation*}
%   % \begin{tikzpicture}
%   %   \graph [
%   %     tree layout, grow=left, math nodes, % empty nodes,
%   %     nodes={
%   %       rounded rectangle, rounded rectangle left arc=none,
%   %       draw, minimum size=2ex,
%   %     }
%   %   ] {
%   %     / [draw=none] <-
%   %     Q <- { "P_1" <- / <- { / , / } ,
%   %            / ,
%   %            / <- { / , / <- / } };
%   %   };
%   % \end{tikzpicture}
% \end{equation*}
% Each edge in the above \ac{SILL} tree represents a top-level $\cut{}$, and each top-level $\cut{}$ corresponds to a hypothesis that serves as its principal formula.
% % Each hypothesis gives rise to a top-level $\cut{}$, represented as an edge in the above graph. 

% Recall that the goal of this thesis proposal is to establish a connection between ordered logical specifications and 

In this thesis proposal, we are interested in a restriction of \ac{SILL} that will match concurrent ordered logical specifications.
To match the ordered context of these specifications, we cannot directly use \ac{SILL}---at least not in its full generality.
Instead, we need a restriction of \ac{SILL} in which process networks are chains, not arbitrary trees.


One might expect that process chains should arise from \emph{ordered} logic.
But, taking into account the way that the structure of the \ac{SILL} session-typing judgment induces a tree-shape for process networks, it becomes apparent that process chains, in fact, arise when processes are restricted to use at most one service each---that is, when contexts $\lctx$ are restricted to be either empty or singletons.
The session-typing judgment becomes
\begin{equation*}
  \lctxe \vdash P :: x{:}A
  \quad\text{or}\quad
  x_1{:}A_1 \vdash P :: x{:}A
\end{equation*}
and the process networks are necessarily chains:
\begin{equation*}
  \begin{tikzpicture}[channel/.style = {text depth=0, midway, sloped, above}]
    \graph [
      tree layout, grow=left, math nodes, % empty nodes,
      nodes={
        rounded rectangle, rounded rectangle left arc=none,
        draw, minimum size=3ex,
      },
      edges={-},
    ] {
      / [draw=none] <-["$\scriptstyle x$"' channel]
      P <-
      / [> {"$\scriptstyle x_1$"' channel}] <-
      / <-
      / ;
    };
  \end{tikzpicture}
\end{equation*}

Having made this restriction, we can simplify the judgments: if there are at most two channels, they can always be unambiguously named \enquote{left} and \enquote{right}, rather than bothering with fresh names like $x_1$ and $x$.
Moreover, since the channel names are now fixed by position (rather like de~Bruijn indices), we may as well omit them altogether from the session-typing judgments:
\begin{equation*}
  \lctxe \vdash P :: A
  \quad\text{or}\quad
  A_1 \vdash P :: A
  \,.
\end{equation*}

The following \lcnamecrefs{sec:cut-as-composition} describe this restriction of \ac{SILL}.
% It's worth emphasizing that, even in the presence of the restriction to singleton antecedents, we still have a Curry--Howard isomorphism: although we choose to present the logical rules and process assignment simultaneously, singleton linear logic is indeed a well-defined logic in its own right.
It's worth emphasizing that although we choose to present the logical rules and process assignment simultaneously, singleton linear logic is indeed a well-defined logic in its own right: even in the presence of the restriction to singleton antecedents, we still have a Curry--Howard isomorphism.

% Finally, it's worth emphasizing that we still have a Curry--Howard isomorphism in the presence of the restriction to singleton antecedents: although we choose to present the logical rules and process assignment simultaneously, singleton linear logic is indeed a well-defined logic in its own right.


% This is a Curry--Howard isomophism in that the same restriction---contexts must be empty or singletons---still yields a well-defined logic when applied to linear logic.

% In this note, we are interested in developing a restriction of \ac{SILL} in which the process networks are linearly ordered:
% \begin{center}
%   % \begin{tikzpicture}
%   %   \graph [tree layout, grow=left, empty nodes, nodes={draw, circle}] {
%   %     / [draw=none] <-
%   %     x <- y <- z <- w;
%   %   };
%   % \end{tikzpicture}
% \end{center}

% Therefore, for the process network to be linearly ordered, contexts $\lctx$ must be either singletons or empty.
% In this special case, sequents have one of the simpler forms
% \begin{equation*}
%   x_0{:}A_0 \vdash P :: x_1{:}A_1
%   \quad\text{or}\quad
%   \lctxe \vdash P :: x_1{:}A_1
%   \,.
% \end{equation*}


% % exec(spawn P; Q) -> {exec P * exec Q}

\subsection{Cut as composition}\label{sec:cut-as-composition}

% Recall the $\cut{}$ rule for intuitionistic linear logic:
% \begin{equation*}
%   \infer[{\cut{A}}]{\oseq{\lctx, \lctx' |- \comp{x_0}{P_0 | P} :: C}}{
%     \oseq{\lctx |- \procof{P_0}{x_0{:}A}} &
%     \oseq{\lctx', x_0{:}A |- \procof{P}{x{:}C}}}
% \end{equation*}
% When restricted to empty or singleton antecedents, the $\cut{}$ rule becomes
% \begin{equation*}
%   \infer[\cut{A}]{\oseq{\lctx |- \comp{x_0}{P_0 | P} :: C}}{
%     \oseq{\lctx |- \procof{P_0}{x_0{:}A}} &
%     \oseq{x_0{:}A |- \procof{P}{x{:}C}}}
% \end{equation*}

The cut rule of intuitionistic linear logic composes a plan for obtaining resource $A$ with another plan that uses resource $A$:
\begin{equation*}
  \infer{\oseq{\lctx, \lctx' |- C}}{
    \oseq{\lctx |- A} &
    \oseq{\lctx', A |- C}}
  \,.
\end{equation*}

When antecedents are restricted to be either empty or singletons, that rule becomes the cut rule for singleton linear logic;
it retains the character of a composition:
% The cut rule of (intuitionistic) singleton linear logic restricts the antecedent of each sequent to be either empty or a singleton, but still retains the character of a composition:
\begin{equation*}
  \infer[\cut{A}]{\lseq{\lctx |- C}}{
    \lseq{\lctx |- A} &
    \lseq{A |- C}}
  \,.
\end{equation*}
Because proofs are to be processes, this suggests that the process interpretation of the cut rule should compose a process that offers service $A$ with another process that uses service $A$.
The $\cut{A}$ rule thus becomes a typing rule for process composition:
\begin{equation*}
  \infer[\cut{A}]{\lseq{\lctx |- \procof{\spawn{P; Q}}{C}}}{
    \lseq{\lctx |- \procof{P}{A}} &
    \lseq{A |- \procof{Q}{C}}}
  \,,
\end{equation*}
where $\spawn{P; Q}$ means \enquote{Spawn process $P$ to the left, and then continue as process $Q$.}
% The syntax is reminiscent of do notation for monadic computations, with the channel c being bound within c <- spawnP <- ;Qc.

To complete the description of $\spawn{}$, we must make its operational semantics precise.
Rather than using \iacl*{SOS}, it is convenient to describe the semantics as an ordered logical specification~\autocites{Pfenning:APLAS04}{Pfenning+Simmons:LICS09}, in the style known as \iacf{SSOS}.
In our \ac{SSOS}, we will use the atomic proposition $\exec{P}$ to represent an executing process $P$.
The rule for executing a $\spawn{}$ is
\begin{equation*}
  \exec{(\spawn{P; Q})} \lrimp \monad{\exec{P} \fuse \exec{Q}}
  \,.
\end{equation*}
Thus, to execute the $\spawn{}$, we execute the processes $P$ and $Q$ side-by-side, with process $P$ offering a service that $Q$ uses.


\subsection{Additive conjunction as branching}\label{sec:addit-conj-as-branch}

So far we have discussed only cut, a judgmental principle that applies to all services\footnote{The other judgmental  principle, identity, is postponed to \cref{sec:ident-as-forw} in the interest of presenting only what is necessary for a first, simple example.}; specific services are defined by the right and left rules of the logical connectives.

The singleton linear sequent calculus rules for the additive conjunction $A_1 \with A_2$ are as follows.
Other than the restriction to singleton or empty antecedents, these rules are the same as those from linear logic.
% Note that in this case the left rules' contexts are forced to be the singleton $A_1 \with A_2$.
\begin{mathpar}
  \infer[\rrule{\with}]{\lseq{\lctx |- A_1 \with A_2}}{
    \lseq{\lctx |- A_1} &
    \lseq{\lctx |- A_2}}
  \and
  \infer[{\lrule[1]{\with}}]{\lseq{A_1 \with A_2 |- C}}{
    \lseq{A_1 |- C}}
  \and
  \infer[{\lrule[2]{\with}}]{\lseq{A_1 \with A_2 |- C}}{
    \lseq{A_2 |- C}}
\end{mathpar}
The right rule, $\rrule{\with}$, says that to prove $A_1 \with A_2$ we must prove both $A_1$ and $A_2$ (using the same resource $\lctx$) so that we are prepared for whichever of the two resources, $A_1$ or $A_2$, is eventually chosen by a $\lrule[1]{\with}$ or $\lrule[2]{\with}$ left rule.

Correspondingly, a process that offers service $A_1 \with A_2$ gives its client a choice of services $A_1$ and $A_2$; the process must be prepared to offer whichever service the client chooses.
Based on this intuition, we interpret the $\rrule{\with}$ rule as typing a binary guarded choice:
\begin{equation*}
  \infer[\rrule{\with}]{\lseq{\lctx |- \procof{\caseR{\inl => P_1 | \inr => P_2}}{A_1 \with A_2}}}{
    \lseq{\lctx |- \procof{P_1}{A_1}} &
    \lseq{\lctx |- \procof{P_2}{A_2}}}
\end{equation*}
where $\caseR{\inl => P_1 | \inr => P_2}$ means \enquote{Input either $\inl$ or $\inr$ along the right-hand channel, and then continue as process $P_1$ or $P_2$, respectively.}

Conversely, the client sitting to the right that uses service $A_1 \with A_2$ must behave in a complementary way: the client should select either service $A_1$ or service $A_2$ and then, having notified the offering process of its choice (as $\inl$ or $\inr$), continue the session by using that service.
The left rules for type $A_1 \with A_2$ are thus:
\begin{mathpar}
  \infer[{\lrule[1]{\with}}]{\lseq{\hypof{A_1 \with A_2} |- \procof{\selectL{\inl; Q}}{C}}}{
    \lseq{\hypof{A_1} |- \procof{Q}{C}}}
  \and
  \infer[{\lrule[2]{\with}}]{\lseq{\hypof{A_1 \with A_2} |- \procof{\selectL{\inr; Q}}{C}}}{
    \lseq{\hypof{A_2} |- \procof{Q}{C}}}
\end{mathpar}
where $\selectL{\inj{\mathsf{1/2}} ; Q}$ means \enquote{Send label $\inj{\mathsf{1/2}}$ along the left-hand channel and then continue as process $Q$.}

Our intuition about the behavior of the guarded choice processes is made precise by their operational semantics.
First, the $\inl$ branch:
\begin{equation*}
  \begin{lgathered}
    \exec(\selectL{\inl ; Q}) \lrimp \monad{\msgL{\inl} \fuse \exec{Q}} \\
    \exec(\caseR{\inl => P_1 | \inr => P_2}) \fuse \msgL{\inl} \lrimp \monad{\exec{P_1}}
      \,.
  \end{lgathered}
\end{equation*}
To execute the selection process $\selectL{\inl ; Q}$, we asynchronously send to the left a message containing the label $\inl$, which is represented in the \ac{SSOS} as the proposition $\msgL{\inl}$, and then immediately continue the session by executing process $Q$.
When this message arrives, the destination process $\caseR{\inl => P_1 | \inr => P_2}$ resumes execution as $P_1$.

The operational semantics of the $\inr$ branch is symmetric to that of the $\inl$ branch:
\begin{equation*}
  \begin{lgathered}
    \exec(\selectL{\inr ; Q}) \lrimp \monad{\msgL{\inr} \fuse \exec{Q}} \\
    \exec(\caseR{\inl => P_1 | \inr => P_2}) \fuse \msgL{\inr} \lrimp \monad{\exec{P_2}}
      \,.
  \end{lgathered}
\end{equation*}

\paragraph{Practical considerations.}

% To make the language more palatable for the programmer, we diverge slightly from a pure propositions-as-types interpretation of one-dimensional linear logic by including $n$-ary labeled additive conjunctions $\with*{\ell_i: A_i}[i \in I]$ as a primitive.
% Formaly, the conjunction is over a mutiset of label-type pairs,
% % expressed as a parametric comprehension
% indexed by set $I$
% \autocite{Cervesato+Sans:FI14}

To make the language more palatable for the programmer, we diverge slightly from a pure propositions-as-types interpretation of singleton linear logic by including $n$-ary labeled additive conjunctions, $\with*{\ell: A_\ell}[\ell \in L]$, as a primitive.
% Formally, these types $\with*{\ell_i: A_i}[i \in I]$ are conjunctions over multiset comprehensions~\autocite{Cervesato+Sans:FI14} of label--type pairs.

By analogy with the binary conjunction, the typing rules and operational semantics for $\with*{\ell: A_\ell}[\ell \in L]$ are as follows.
Notice that the $\rrule{\with}$ has a set of premises, one for each label $\ell \in L$.
\begin{gather*}
  \infer[\rrule{\with}]{\lseq{\lctx |- \procof{\caseR[\ell \in L]{\ell => P_\ell}}{\with*{\ell: A_\ell}[\ell \in L]}}}{
    \forall \ell \in L \mathpunct{:}\enskip \lseq{\lctx |- \procof{P_\ell}{A_\ell}}}
  \qquad
  % \raisebox{0.5\baselineskip}{$\bagged{\,\raisebox{-0.5\baselineskip}{
  % \scaleleftright[0.4em]{\lbag}{
    \infer[{\lrule{\with}}]{\lseq{\hypof{\with*{\ell: A_\ell}[\ell \in L]} |- \procof{\selectL{\kay; Q}}{C}}}{
      \lseq{\hypof{A_{\kay}} |- \procof{Q}{C}} &
      \text{($\kay \in L$)}}
  % }{\rbag}_{k \in I}
  % }}[k \in I]$}
  \\[2\jot]
  % \bagged{\,
  % \scaleleftright[0.4em]{\lbag}{
    \begin{lgathered}
      \exec{(\selectL{\kay ; Q})} \lrimp \monad{\msgL{\kay} \fuse \exec{Q}} \\
      \exec{(\caseR[\ell \in L]{\ell => P_\ell})} \fuse \msgL{\kay} \lrimp \monad{\exec{P_{\kay}}} \mathrlap{\qquad\text{($\kay \in L$)}}
    \end{lgathered}
  % }{\rbag}_{k \in I}
  % \,}[k \in I]
\end{gather*}
% According to the multiset nature of the type, our presentation uses the multiset-oriented inference rules of \textcite{Cervesato+Sans:FI14}: the $\rrule{\with}$ right rule has a multiset of premises---one for each index $i \in I$---and there are multisets of $\lrule[k]{\with}$ left rules and \ac{SSOS} rules.

Another possibility would be to simply treat $n$-ary labeled conjunctions as syntactic sugar for nested binary conjunctions, but this would
% turn out to
introduce a communication overhead because we would be sending multiple $\inj{\mathsf{1/2}}$s separately rather than a single label $\kay$.


\subsection{Recursive session types and process definitions}\label{sec:recurs-sess-types}

Concurrent processes frequently exhibit unbounded or infinite, yet well-defined, behavior;
for instance, we may wish to have a counter that offers an increment service indefinitely.
\Textcite{Toninho+:TGC14} have proposed an extension of their \ac{SILL} type theory that incorporates inductive and coinductive session types.
However, to keep matters simpler, we instead rely on general recursion here and choose to be content with the departure from a pure Curry--Howard isomorphism.

Session types thus include general recursive types, $\mu t.A$, and type variables, $t$.
The type $\mu t.A$ is interpreted equi-recursively, being identified with the unfolding $\subst{(\mu t.A)/t}{A}$.
Processes correspondingly include mutually recursive process definitions, via \sillinline`letrec`, and process variables, $X$.

We extend the session-typing judgment with a context, $\pctx$, of process-variable typings.
% Because a process is typed according to the services that it uses and offers, process variables are typed as $X : \ctxmonad{A <- \vec{B}}$, meaning that process $X$ can offer service $A$ if provided with channels along which services $\vec{B}$ are offered.
Because a process is typed according with a sequent of services that it uses and offers, process variables are typed as ${X : \ctxmonad{A <- \lctx}}$ if process $X$ can offer service $A$ 
% when provided with channels along which services $\vec{B}$ are offered.
by using services $\lctx$.
When channels of appropriate types are available, the process $X$ can be called:
\begin{equation*}
  \infer[\text{\scshape call}]{\lseq{\pctx, X{:}\ctxmonad{A <- \lctx} ; \lctx |- \procof{X}{A}}}{
    }
  \:.
\end{equation*}
A common idiom is $\spawn{X ; Q}$, which spawns a call to $X$ that is run in parallel with some process $Q$.
We will frequently abbreviate this with the syntactic sugar $\mbind{X ; Q}$.

% Mutually recursive process definitions add process variables to the context.
% Process variables are added to the context to allow for mutual recursion.
In mutually recursive process definitions,
% , the programmer declares each process with a type that must be checked.
the process bodies may refer to any of the mutually recursive processes via process variables.
The typing rule is
\begin{equation*}
  \infer[\text{\scshape letrec}]
  {\lseq{\pctx ; \lctx |- \procof{\mletrec[X \in \mathcal{X}]{(\mprocdef{X{:}\ctxmonad{A_X <- \lctx_X} = P_X})}{Q}}{C}}}{
    \text{($\pctx' = \braced{X{:}\ctxmonad{A_X <- \lctx_X}}[X \in \mathcal{X}]$)}
    &
    % \bagged{
    \forall X \in \mathcal{X}\mathpunct{:}\enskip
      \lseq{\pctx, \pctx' ; \lctx_X |- \procof{P_X}{A_X}}
    % }[j \in I]
    &
    \lseq{\pctx, \pctx' ; \lctx |- \procof{Q}{C}}}
  \,.
\end{equation*}

The operational semantics of these constructs are as follows:
\begin{equation*}
  \begin{lgathered}
    \exec{(\mletrec[X \in \mathcal{X}]{(\mprocdef{X = P_X})}{Q})} \lrimp \monad{\braced{\bang \bnd{X = P_X}}[X \in \mathcal{X}] \fuse \exec{Q}}
    \\
    \bang \bnd{X = P} \fuse \exec{X} \lrimp \monad{\exec{P}}
  \end{lgathered}
\end{equation*}
The environment of bindings of process variables to process expressions is represented in the \ac{SSOS} as a collection of $\bnd{X = P}$ hypotheses.
To execute a group of mutually recursive process definitions, $\mletrec[X \in \mathcal{X}]{(\mprocdef{X = P_X})}{Q}$, bindings are introduced for each of the process variables and then the body $Q$ is executed.
To execute a free process variable $X$, instead execute the process expression to which $X$ is bound.


%
Now, having presented recursion, we can finally give a simple example program.


\subsection{Example: Binary counter}\label{sec:exampl-binary-count}

We can implement a simple session-typed counter on natural numbers as shown in \cref{lst:counter-inc}.%
\footnote{This example is adapted from one by \textcite{Toninho+:ESOP13}.}
%
\begin{sillcode}[
  caption={A simple binary counter supporting an increment operation},
  label={lst:counter-inc},
  floatplacement=tb,
  gobble=2
]
  stype Cntr = &{ inc: Cntr }
  
  eps : { |- Cntr } =
  { caseR of
      inc => eps; bit1 }
  
  bit0 : { Cntr |- Cntr } =
  { caseR of
      inc => bit1 }
  
  bit1 : { Cntr |- Cntr } =
  { caseR of
      inc => selectL inc; bit0 }
\end{sillcode}
%
The counter is a chain of \sillinline`bit0` and \sillinline`bit1` processes, one for each bit in the binary representation of the counter's value, and is terminated at the most significant end with an \sillinline`eps` process.
For instance, the process chain \sillinline`eps; bit1; bit0` represents a counter with value $2$.

The counter offers a very simple service: the client may only choose to increment the counter, with the same service being offered recursively after the increment.
This service, \sillinline`Cntr`, is therefore a recursive (unary) additive conjunction, declared in the concrete syntax as \sillinline`stype Cntr = &{ inc: Cntr }`.
The \sillinline`eps` process offers this service outright, and thus has type \sillinline`{ |- Cntr }`.
The \sillinline`bit0` and \sillinline`bit1` processes, on the other hand, use the service offered by their more significant neighbors, and thus have type \sillinline`{ Cntr |- Cntr }`.

The process definitions of \sillinline`eps`, \sillinline`bit0`, and \sillinline`bit1` are mutually recursive.
When an \sillinline`eps` process receives an \sillinline`inc` message, it creates a new most significant bit by spawning a new \sillinline`eps` process and then making a recursive call to a \sillinline`bit1` process.
% that uses the service offered by the new \sillinline`eps`.
When a \sillinline`bit0` process receives an \sillinline`inc`, the bit is flipped by virtue of a recursive call to a \sillinline`bit1` process.
Lastly, when a \sillinline`bit1` process receives an \sillinline`inc`, the bit is flipped and a carry is propagated; this is accomplished by first sending \sillinline`inc` along the left to the \sillinline`Cntr` offered by the next more significant bit and then making a recursive call to a \sillinline`bit0` process.

Informally, we can see that, as implemented, the \sillinline`inc` operation respects a counter's denotation: whenever a counter representing natural number $N$ is incremented, the resulting counter represents $N+1$.
Note, however, that this adequacy property is not enforced by the type \sillinline`Cntr`.
An appropriate dependent session type could enforce increment adequacy, but, for simplicity of exposition, we prefer the simple type here.

\subsection{Additive disjunction as choice}\label{sec:addit-disj-as}

In the singleton linear sequent calculus, additive disjunction, $A \ssor B$, is dual to additive conjunction, $A \with B$.
\begin{mathpar}
  \infer[{\rrule[1]{\ssor}}]{\lseq{\lctx |- A_1 \ssor A_2}}{
    \lseq{\lctx |- A_1}}
  \and
  \infer[{\rrule[2]{\ssor}}]{\lseq{\lctx |- A_1 \ssor A_2}}{
    \lseq{\lctx |- A_2}}
  \and
  \infer[\lrule{\ssor}]{\lseq{A_1 \ssor A_2 |- C}}{
    \lseq{A_1 |- C} &
    \lseq{A_2 |- C}}
\end{mathpar}

We should expect this duality to also appear in the process assignment.
Whereas a process of type $A \with B$ offers its client at the right a choice of services $A$ and $B$, a process of type $A \ssor B$ chooses between offering service $A$ or service $B$ to its client at the right.
The client waits to be notified of the offering process's choice and then uses that service.
\begin{mathpar}
  \infer[{\rrule[1]{\ssor}}]{\lseq{\lctx |- \procof{\selectR{\inl; P}}{A_1 \ssor A_2}}}{
    \lseq{\lctx |- \procof{P}{A_1}}}
  \and
  \infer[{\rrule[2]{\ssor}}]{\lseq{\lctx |- \procof{\selectR{\inr; P}}{A_1 \ssor A_2}}}{
    \lseq{\lctx |- \procof{P}{A_2}}}
  \and
  \infer[\lrule{\ssor}]{\lseq{A_1 \ssor A_2 |- \procof{\caseL{\inl => Q_1 | \inr => Q_2}}{C}}}{
    \lseq{A_1 |- \procof{Q_1}{C}} &
    \lseq{A_2 |- \procof{Q_2}{C}}}
\end{mathpar}
Confirming the intuition that $\selectR{\inj{\mathsf{1/2}}; P}$ sends along the right-hand channel and $\caseL{\inl => Q_1 | \inr => Q_2}$ receives along the left-hand channel are the \ac{SSOS} rules:
\begin{equation*}
  \begin{lgathered}
    \exec{(\selectR{\inl ; P})} \lrimp \monad{\exec{P} \fuse \msgR{\inl}} \\
    \msgR{\inl} \fuse \exec{(\caseL{\inl => Q_1 | \inr => Q_2})} \lrimp \monad{\exec{Q_1}}
    %
    \\[\jot]
    %
    \exec{(\selectR{\inr ; P})} \lrimp \monad{\exec{P} \fuse \msgR{\inr}} \\
    \msgR{\inr} \fuse \exec{(\caseL{\inl => Q_1 | \inr => Q_2})} \lrimp \monad{\exec{Q_2}}
      \,.
  \end{lgathered}
\end{equation*}
Because the operational semantics is asynchronous, it's important to distinguish the $\msgR{}$ predicate, which represents messages that flow to the right, from the $\msgL{}$ predicate, which represents messages that flow to the left.
Otherwise, a selection process's continuation could mistakenly capture the message that was just sent, as might happen in executing $\selectR{\inl ; \caseR{\inl => P_1 | \inr => P_2}}$, for example.

Once again, to make the language more convenient for the programmer, we include $n$-ary labeled additive disjunctions $\ssor*{\ell: A_\ell}[\ell \in L]$.
The typing rules and operational semantics are thus more generally
\begin{gather*}
  % \bagged{
    \infer[{\rrule{\ssor}}]{\lseq{\lctx |- \procof{\selectR{\kay; P}}{\ssor*{\ell: A_\ell}[\ell \in L]}}}{
      \lseq{\lctx |- \procof{P}{A_{\kay}}} &
      \text{($\kay \in L$)}}
%   }[k \in I]
  \qquad
  \infer[\lrule{\ssor}]{\lseq{\ssor*{\ell: A_\ell}[\ell \in L] |- \procof{\caseL[\ell \in L]{\ell => Q_\ell}}{C}}}{
    \forall \ell \in L\mathpunct{:}\enskip \lseq{A_\ell |- \procof{Q_\ell}{C}}}
  \\[2\jot]
  % \bagged{
    \begin{lgathered}
      \exec{(\selectR{\kay; P})} \lrimp \monad{\exec{P} \fuse \msgR{\kay}} \\
      \msgR{\kay} \fuse \exec{(\caseL[\ell \in L]{\ell => Q_\ell})} \lrimp \monad{\exec{Q_{\kay}}} \mathrlap{\qquad\text{($\kay \in L$)}}
    \end{lgathered}
  % }[k \in I]
\end{gather*}


\subsection{Example: Binary counter with decrements}\label{sec:exampl-binary-count-1}

\begin{sillcode}[
  caption={A binary counter supporting increments and decrements},
  label={lst:counter-dec},
  floatplacement=tb,
  gobble=2
]
  stype Cntr = &{ inc: Cntr , dec: Cntr' }
    and Cntr' = +{ ok: Cntr , fail: Cntr }

  eps : { |- Cntr } =
  { caseR of
      inc => eps; bit1
    | dec => selectR fail; eps }
  
  bit0 : { Cntr |- Cntr } =
  { caseR of
      inc => bit1
    | dec => selectL dec; bit0' }
  
  bit0' : { Cntr' |- Cntr' } =
  { caseL of
      ok => selectR ok; bit1
    | fail => selectR fail; bit0 }

  bit1 : { Cntr |- Cntr } =
  { caseR of
      inc => selectL inc; bit0
    | dec => selectR ok; bit1 }
\end{sillcode}

\Cref{lst:counter-dec} shows a counter that takes advantage of additive disjunction to support a truncated decrement operation. 
According to the type declaration, a process offering the \sillinline`Cntr` service gives its client a choice of increment or decrement services.
If the client chooses to decrement, the offering process will choose to reply with either \sillinline`fail` or \sillinline`ok` and then recursively offer the \sillinline`Cntr` service.

As implemented, decrementing the counter gives \sillinline`fail` and leaves the process network unchanged if the counter represents $0$; if it represents some $N > 0$, then decrementing the counter gives \sillinline`ok` after decrementing to $N - 1$.
Once again, these adequacy properties are not enforced by the type \sillinline`Cntr`, although they could be with an appropriate dependent session type.


\subsection{Identity as forwarding}\label{sec:ident-as-forw}

In singleton linear logic, in addition to the cut principle, there is an identity principle that states that one way to obtain a resource is to directly use an existing resource:
\begin{equation*}
  \infer[\id{A}]{\lseq{A |- A}}{
    }
  \,.
\end{equation*}
Under the process interpretation, a process can offer service $A$ by acting as a forwarding intermediary between its clients and another process that offers service $A$.
The $\id{A}$ rule thus types a forwarding process between two channels:
\begin{equation*}
  \infer[\id{A}]{\lseq{A |- \procof{\fwd}{A}}}{
    }
  \,.
\end{equation*}
Rather than making the forwarding explicit in the operational semantics, we can simply eliminate the middleman, adjoining the neighboring processes:
\begin{equation*}
  \exec{(\fwd)} \lrimp \monad{\one}
  \,.
\end{equation*}

% \begin{sillcode}[
%   caption={A binary counter as a stream transformer},
%   label={lst:counter-bit},
%   floatplacement=tb,
%   gobble=2
% ]
%   stype Bin = +{ eps: 1 , bit0: Bin , bit1: Bin }
  
%   inc : {Bin |- Bin} =
%   { caseL of
%       eps => selectR bit1; selectR eps; <->
%     | bit0 => selectR bit1; <->
%     | bit1 => selectR bit0; inc }
% \end{sillcode}



\subsection{Other session types}\label{sec:other-session-types}

In addition to the those already mentioned, the other connectives of singleton linear logic correspond to session types.

\paragraph{Multiplicative unit.}
Like its \ac{SILL} cousin, the multiplicative unit $\one$ is the service that terminates without any interaction.
Its right rule, $\rrule{\one}$, types a process that immediately terminates; its left rule, $\lrule{\one}$, types a process that waits for the left-hand side to terminate:
\begin{mathpar}
  \infer[\rrule{\one}]{\lseq{\lctxe |- \procof{\closeR}{\one}}}{
    }
  \and
  \infer[\lrule{\one}]{\lseq{\hypof{\one} |- \procof{\waitL{Q}}{C}}}{
    \lseq{\lctxe |- \procof{Q}{C}}}
\end{mathpar}
The operational semantics is asynchronous, with the $\closeR$ process sending a quit message, $\msgQ$:
\begin{equation*}
  \begin{lgathered}
    \exec{\closeR} \lrimp \monad{\msgQ} \\
    \msgQ \fuse \exec{(\waitL{Q})} \lrimp \monad{\exec{Q}}
  \end{lgathered}
\end{equation*}


\paragraph{First-order universal and existential quantification.}
The first-order quantifiers type processes that exchange functional values.
A process offering service $\forall x{:}\tau. A_x$ (or using service $\exists x{:}\tau. A_x$) first inputs a value $x$ of functional type $\tau$ and then offers (resp., uses) service $A_x$.
Dually, a process offering service $\exists x{:}\tau. A_x$ (or using service $\forall x{:}\tau. A_x$) asynchronously outputs the value of some functional term $M$ of type $\tau$ and then offers (resp., uses) service $\subst{M/x}{A_x}$. 
The first-order quantifiers are thus dependent session types; in the non-dependent case, we write the types as $\tau \vimp A$ and $A \vand \tau$.
% The syntax for value inputs and outputs is $\recv x <- inputc;Pxand outputc M;Q.

Since value inputs and outputs are not critical to the remainder of this proposal, the reader who is interested in further details of their static and dynamic semantics in \ac{SILL} should refer to the papers by \textcites{Toninho+:ESOP13}{Toninho+:PPDP11}; we leave the extrapolation to singleton linear logic as an exercise for the reader.

\paragraph{Multiplicative conjunction and linear implication.}
The restriction to sequents with singleton antecedents proves fatal to attempts to include multiplicative conjunction ($A \tensor B$) and linear implication ($A \lolli B$) as connectives in singleton linear logic.
For multiplicative conjunction, the left rule is problematic because it breaks down one hypothesis into two; for linear implication, the right rule is problematic because it introduces a new hypothesis even if one is already there.
Fortunately, for the examples in which we are interested, the absence of $\tensor$ and $\lolli$ is not an issue.

% \end{document}


% arara: pdflatex
% arara: pdflatex
% arara: biber
% arara: pdflatex
% arara: pdflatex
% \documentclass{../hdeyoung-proposal}
\documentclass[
  class=../hdeyoung-proposal,
  crop=false
]{standalone}


\usepackage{linear-logic}
\usepackage{ordered-logic}
\usepackage{basic-atoms}
\usepackage{proof}
\usepackage{mathpartir}

\usepackage{tikz}
% \usetikzlibrary{shapes.misc,graphs,graphdrawing}
% \usegdlibrary{trees}

\ExplSyntaxOn

% \DeclarePairedDelimiter \parens { \lparen } { \rparen }
\DeclarePairedDelimiter \bagged:wn { \lbag } { \rbag }
\NewDocumentCommand{ \bagged }{ s o m o }
  {
    \IfBooleanTF {#1}
      { \bagged:wn* {#3} }
      {
        \IfValueTF {#2}
          { \bagged:wn[#2] {#3} }
          { \bagged:wn {#3} }
      }
    \IfValueT {#4} { \sb{#4} }
  }


\NewDocumentCommand \oseq { >{ \SplitArgument{1}{|-} } m }
  { \oseq:nn #1 }
\cs_new:Npn \oseq:nn #1#2 { \oseq_ctxs:n {#1} \vdash #2 }
\cs_new:Npn \oseq_ctxs:n #1 {
  \seq_set_split:Nnn \l_tmpa_seq {;} {#1}
  \seq_use:Nn \l_tmpa_seq { \mathrel{;} }
}

\NewDocumentCommand \procof { m m } { #1 \dblcolon #2 }
\NewDocumentCommand \hypof { m } { #1 }


\NewDocumentCommand \cut { m } { \text{\textsc{\MakeLowercase{Cut}}}\sb{#1} }
\NewDocumentCommand \id { m } { \text{\textsc{\MakeLowercase{Id}}}\sb{#1} }

\NewDocumentCommand \comp { >{ \SplitArgument{1}{|} } m }
  { \comp:nn #1 }
\cs_new:Npn \comp:nn #1#2 { #1 \parallel #2 }

\NewDocumentCommand \fwd {} { \mathord{\leftrightarrow} }


\RenewDocumentCommand \with { s }
  { \IfBooleanTF {#1} \with:n \with: }
\cs_new:Npn \with:n #1 {
  \mathord{\binampersand}
  \bagged {
    \seq_set_split:Nnn \l_tmpa_seq {,} {#1}
    \seq_use:Nn \l_tmpa_seq {,}
  }
}
\cs_new:Npn \with: { \mathbin{\binampersand} }

\NewDocumentCommand \ssor { s }
  { \IfBooleanTF {#1} \with:n \with: }
\cs_new:Npn \ssor:n #1 {
  \mathord{\ssor:}
  \bagged {
    \seq_set_split:Nnn \l_tmpa_seq {,} {#1}
    \seq_use:Nn \l_tmpa_seq {,}
  }
}
\cs_new:Npn \ssor: { \oplus }

\NewDocumentCommand \caseR { s m }
  {
    \IfBooleanTF {#1}
      { \case:nNn { \mathsf{caseR} } \bagged {#2} }
      { \case:nNn { \mathsf{caseR} } \parens {#2} }
  }
\NewDocumentCommand \caseL { s m }
  {
    \IfBooleanTF {#1}
      { \case:nNn { \mathsf{caseL} } \bagged {#2} }
      { \case:nNn { \mathsf{caseL} } \parens {#2} }
  }
\cs_new:Npn \case:nNn #1#2#3 {
  #1 \mskip\thinmuskip
  #2 {
    \seq_set_split:Nnn \l_tmpa_seq {|} {#3}
    \seq_clear:N \l_tmpb_seq
    \seq_map_inline:Nn \l_tmpa_seq
      { \seq_put_right:Nn \l_tmpb_seq { \case_branch:n {##1} } }
    \seq_use:Nn \l_tmpb_seq { \mid }
  }
}
\cs_new:Npn \case_branch:n #1 { \case_branch_aux:w #1 \q_stop }
\cs_new:Npn \case_branch_aux:w #1 => #2 \q_stop {
  #1 \Rightarrow #2
}

\NewDocumentCommand \selectL { >{ \SplitArgument{1}{;} } m }
  { \select:nnn { \mathsf{selectL} } #1 }
\NewDocumentCommand \selectR { >{ \SplitArgument{1}{;} } m }
  { \select:nnn { \mathsf{selectR} } #1 }
\cs_new:Npn \select:nnn #1#2#3 {
  \!\mathord{}\mathop{#1} #2 ; #3
}


\NewDocumentCommand \inj { m } { \mathsf{in}\sb{#1} }

\NewDocumentCommand \inl {} { \inj{ \mathsf{1} } }
\NewDocumentCommand \inr {} { \inj{ \mathsf{2} } }


\RenewDocumentCommand \one {} { \mathord { \mathbf{1} } }

\NewDocumentCommand \quitR {} { \mathsf{quitR} }
\NewDocumentCommand \waitL { m } { \mathsf{waitL} ; #1 }


\NewDocumentCommand \rrule { o m } {
  \IfValueTF {#1}
    { \rrule:nn {#2} {#1} }
    { \rrule:n {#2} }
}
\cs_new:Npn \rrule:nn #1#2 { {#1}\text{\textsc{\MakeLowercase{R}}}\sb{#2} }
\cs_new:Npn \rrule:n #1 { {#1}\text{\textsc{\MakeLowercase{R}}} }

\NewDocumentCommand \lrule { o m } {
  \IfValueTF {#1}
    { \lrule:nn {#2} {#1} }
    { \lrule:n {#2} }
}
\cs_new:Npn \lrule:nn #1#2 { {#1}\text{\textsc{\MakeLowercase{L}}}\sb{#2} }
\cs_new:Npn \lrule:n #1 { {#1}\text{\textsc{\MakeLowercase{L}}} }


\NewDocumentCommand \exec { } { \mathsf{exec} \mskip\thinmuskip }
\NewDocumentCommand \msg { } { \mathsf{msg} \mskip\thinmuskip }

\ExplSyntaxOff


\DeclareAcronym{SSOS}{
  short = SSOS,
  long = substructural operational semantics,
  short-format = \scshape\MakeLowercase
}
\DeclareAcronym{SILL}{
  short = \MakeLowercase{SILL},
  long = session-typed intuitionistic linear logic,
  short-format = \scshape
}

\addbibresource{../proposal.bib}


\ExplSyntaxOn

\NewDocumentCommand \spawn { >{ \SplitArgument{1}{;} }m } { \spawn:nn #1 }
\cs_new:Npn \spawn:nn #1#2 { \mathsf{spawn} \mskip\thinmuskip #1 ; #2 }

\NewDocumentCommand \call { m } { \call:n {#1} }
\cs_new:Npn \call:n #1 { \mathsf{call} \mskip\thinmuskip #1 }

\NewDocumentCommand \compile { m m } { \compile:nn {#1} {#2} }
\cs_new:Npn \compile:nn #1#2 { \llbracket #1 \rrbracket = #2 }

\ExplSyntaxOff


\begin{document}

\section{}

\begin{mathpar}
  \infer{\compile{A^+_1 \fuse A^+_2}{\spawn{P_1 ; P_2}}}{
    \compile{A^+_1}{P_1} &
    \compile{A^+_2}{P_2}}
  \and
  \infer{\compile{\one}{\fwd}}{
    }
  \\
  \infer{\compile{\patom}{\call{\patom}}}{
    }
  \and
  \infer{\compile{\matom[->]}{\selectR{\matom[->] ; \fwd}}}{
    }
  \and
  \infer{\compile{\matom[<-]}{\selectL{\matom[<-] ; \fwd}}}{
    }
\end{mathpar}

\begin{mathpar}
  \infer{\compile{\with*{\matom[->]_i \limp \monad{A^+_i}}[i \in I]}{\caseL*{\matom[->]_i => P_i}[i \in I]}}{
    \bagged{\compile{A^+_i}{P_i}}[i \in I]}
  \and
  \infer{\compile{\with*{\matom[<-]_i \rimp \monad{A^+_i}}[i \in I]}{\caseR*{\matom[<-]_i => P_i}[i \in I]}}{
    \bagged{\compile{A^+_i}{P_i}}[i \in I]}
  \and
  \infer{\compile{\monad{A^+}}{P}}{
    \compile{A^+}{P}}
\end{mathpar}

\begin{mathpar}
  \infer{\compile{\octx_1, \octx_2}{\octx'_1, \octx'_2}}{
    \compile{\octx_1}{\octx'_1} &
    \compile{\octx_2}{\octx'_2}}
  \and
  \infer{\compile{\octxe}{\octxe}}{
    }
  \and
  \infer{\compile{A^+}{\exec{P}}}{
    \compile{A^+}{P}}
  \\
  \infer{\compile{\susp+{\patom}}{\exec{P}}}{
    \patom \lrimp A^- \in \sig &
    \compile{A^-}{P}}
  \and
  \infer{\compile{\susp+{\matom[->]}}{\msg{\matom[->]}}}{
    }
  \and
  \infer{\compile{\susp+{\matom[<-]}}{\msg{\matom[<-]}}}{
    }
\end{mathpar}

\end{document}


% % arara: lualatex
% % arara: lualatex
% % arara: biber
% % arara: lualatex
% % arara: lualatex
% % \documentclass{../hdeyoung-proposal}
% \documentclass[
%   class=../hdeyoung-proposal,
%   crop=false
% ]{standalone}


% \usepackage{linear-logic}
% \usepackage{ordered-logic}
% \usepackage{proof}
% \usepackage{mathpartir}

% \usepackage{tikz}
% \usetikzlibrary{shapes.misc,graphs,quotes,graphdrawing}
% \usegdlibrary{trees}

% \usepackage{scalerel}

% \ExplSyntaxOn

% % \DeclarePairedDelimiter \parens { \lparen } { \rparen }
% \DeclarePairedDelimiter \bagged:wn { \lbag } { \rbag }
% \NewDocumentCommand{ \bagged }{ s o m o }
%   {
%     \IfBooleanTF {#1}
%       { \bagged:wn* {#3} }
%       {
%         \IfValueTF {#2}
%           { \bagged:wn[#2] {#3} }
%           { \bagged:wn {#3} }
%       }
%     \IfValueT {#4} { \sb{#4} }
%   }


% \NewDocumentCommand \oseq { >{ \SplitArgument{1}{|-} } m }
%   { \oseq:nn #1 }
% \cs_new:Npn \oseq:nn #1#2 { \oseq_ctxs:n {#1} \vdash #2 }
% \cs_new:Npn \oseq_ctxs:n #1 {
%   \seq_set_split:Nnn \l_tmpa_seq {;} {#1}
%   \seq_use:Nn \l_tmpa_seq { \mathrel{;} }
% }

% \NewDocumentCommand \procof { m m } { #1 \dblcolon #2 }
% \NewDocumentCommand \hypof { m } { #1 }


% \NewDocumentCommand \cut { m } { \text{\textsc{\MakeLowercase{Cut}}}\sb{#1} }
% \NewDocumentCommand \id { m } { \text{\textsc{\MakeLowercase{Id}}}\sb{#1} }

% \NewDocumentCommand \comp { >{ \SplitArgument{1}{|} } m }
%   { \comp:nn #1 }
% \cs_new:Npn \comp:nn #1#2 { #1 \parallel #2 }

% \NewDocumentCommand \fwd {} { \mathord{\leftrightarrow} }


% \RenewDocumentCommand \with { s }
%   { \IfBooleanTF {#1} \with:n \with: }
% \cs_new:Npn \with:n #1 {
%   \mathord{\binampersand}
%   \bagged {
%     \seq_set_split:Nnn \l_tmpa_seq {,} {#1}
%     \seq_use:Nn \l_tmpa_seq {,}
%   }
% }
% \cs_new:Npn \with: { \mathbin{\binampersand} }

% \NewDocumentCommand \ssor { s }
%   { \IfBooleanTF {#1} \ssor:n \ssor: }
% \cs_new:Npn \ssor:n #1 {
%   \mathord{\ssor:}
%   \bagged {
%     \seq_set_split:Nnn \l_tmpa_seq {,} {#1}
%     \seq_use:Nn \l_tmpa_seq {,}
%   }
% }
% \cs_new:Npn \ssor: { \oplus }

% \NewDocumentCommand \caseR { s m }
%   {
%     \IfBooleanTF {#1}
%       { \case:nNn { \mathsf{caseR} } \bagged {#2} }
%       { \case:nNn { \mathsf{caseR} } \parens {#2} }
%   }
% \NewDocumentCommand \caseL { s m }
%   {
%     \IfBooleanTF {#1}
%       { \case:nNn { \mathsf{caseL} } \bagged {#2} }
%       { \case:nNn { \mathsf{caseL} } \parens {#2} }
%   }
% \cs_new:Npn \case:nNn #1#2#3 {
%   #1 \mskip\thinmuskip
%   #2 {
%     \seq_set_split:Nnn \l_tmpa_seq {|} {#3}
%     \seq_clear:N \l_tmpb_seq
%     \seq_map_inline:Nn \l_tmpa_seq
%       { \seq_put_right:Nn \l_tmpb_seq { \case_branch:n {##1} } }
%     \seq_use:Nn \l_tmpb_seq { \talloblong }
%   }
% }
% \cs_new:Npn \case_branch:n #1 { \case_branch_aux:w #1 \q_stop }
% \cs_new:Npn \case_branch_aux:w #1 => #2 \q_stop {
%   #1 \Rightarrow #2
% }

% \NewDocumentCommand \selectL { >{ \SplitArgument{1}{;} } m }
%   { \select:nnn { \mathsf{selectL} } #1 }
% \NewDocumentCommand \selectR { >{ \SplitArgument{1}{;} } m }
%   { \select:nnn { \mathsf{selectR} } #1 }
% \cs_new:Npn \select:nnn #1#2#3 {
%   \!\mathord{}\mathop{#1} #2 ; #3
% }


% \NewDocumentCommand \inj { m } { \mathsf{in}\sb{#1} }

% \NewDocumentCommand \inl {} { \inj{ \mathsf{1} } }
% \NewDocumentCommand \inr {} { \inj{ \mathsf{2} } }


% \RenewDocumentCommand \one {} { \mathord { \mathbf{1} } }

% \NewDocumentCommand \quitR {} { \mathsf{quitR} }
% \NewDocumentCommand \waitL { m } { \mathsf{waitL} ; #1 }


% \NewDocumentCommand \rrule { o m } {
%   \IfValueTF {#1}
%     { \rrule:nn {#2} {#1} }
%     { \rrule:n {#2} }
% }
% \cs_new:Npn \rrule:nn #1#2 { {#1}\text{\textsc{\MakeLowercase{R}}}\sb{#2} }
% \cs_new:Npn \rrule:n #1 { {#1}\text{\textsc{\MakeLowercase{R}}} }

% \NewDocumentCommand \lrule { o m } {
%   \IfValueTF {#1}
%     { \lrule:nn {#2} {#1} }
%     { \lrule:n {#2} }
% }
% \cs_new:Npn \lrule:nn #1#2 { {#1}\text{\textsc{\MakeLowercase{L}}}\sb{#2} }
% \cs_new:Npn \lrule:n #1 { {#1}\text{\textsc{\MakeLowercase{L}}} }


% \NewDocumentCommand \exec { } { \mathsf{exec} \mskip\thinmuskip }
% \NewDocumentCommand \msg { } { \mathsf{msg} \mskip\thinmuskip }

% \ExplSyntaxOff




% \addbibresource{../proposal.bib}

% \NewDocumentCommand{\ie}{}{i.e.}


% \usepackage{listings}
% \crefname{listing}{listing}{listings}
% \Crefname{listing}{Listing}{Listings}

% \newlength{\mywidth}
% \settowidth{\mywidth}{\ttfamily A}
% \lstset{basicstyle=\ttfamily, basewidth=\mywidth}

% \captionsetup[lstlisting]{%
%   box=colorbox, boxcolor=gray,
%   font={normalfont, sf, color=white},
%   labelfont=bf,
%   justification=justified, singlelinecheck=false
% }

% \lstnewenvironment{sillcode}[1][]
%   {\lstset{language={},frame=bottomline,framerule=0.8ex,rulecolor=\color{gray},float,#1}}%
%   {}

% \lstnewenvironment{sillcode*}[1][]
%   {\lstset{language={},#1}}%
%   {}

% \NewDocumentCommand{\sillinline}{o}{%
%   \IfValueTF{#1}{\lstinline[#1]}{\lstinline}%
% }


% \NewDocumentCommand{\pctx}{}{\Psi}
% \ExplSyntaxOn
% \NewDocumentCommand{\ctxmonad}{>{\SplitArgument{1}{<-}}m}{
%   \{\use_ii:nn #1 \vdash \use_i:nn #1\}
% }
% \NewDocumentCommand \spawn { >{ \SplitArgument{1}{;} } m } { \spawn:nn #1 }
% \cs_new:Npn \spawn:nn #1#2 {
%   \mathsf{spawn}
%   \tl_if_empty:nF {#1} {
%     \mskip\thinmuskip #1 ; #2
%   }
% }
% \NewDocumentCommand{\mbind}{>{\SplitArgument{1}{;}}m}{
%   \use_i:nn#1 ; \use_ii:nn#1
% }
% \NewDocumentCommand{\mletrec}{m m}{
%   \mathsf{letrec} \mskip\thinmuskip #1 \mskip\thinmuskip \mathsf{in} \mskip\thinmuskip #2
% }
% \NewDocumentCommand{\mprocdef}{m}{
%   #1
% }
% \ExplSyntaxOff




% \DeclareAcronym{BHK}{
  short = BHK,
  long  = Brouwer-Heyting-Kolmogorov
}

\DeclareAcronym{JILL}{
  short = JILL,
  long  = judgmental intuitionistic linear logic
}

\DeclareAcronym{ILL}{
  short = ILL,
  long  = intuitionistic linear logic
}

\DeclareAcronym{SML}{
  short = SML,
  long  = Standard ML
}

\DeclareAcronym{SOS}{
  short = SOS,
  short-indefinite = an,
  long = structural operational semantics
}

\DeclareAcronym{SSOS}{
  short = SSOS,
  short-indefinite = an,
  long = substructural operational semantics
}

\DeclareAcronym{SILL}{
  short = SILL,
  long = session-typed intuitionistic linear logic
}

\DeclareAcronym{SISLL}{
  short = singleton \acs{SILL},
  long = session-typed intuitionistic singleton linear logic
}

\DeclareAcronym{CLF}{
  short = CLF,
  long = the Concurrent Logical Framework
}



% \begin{document}

\section{Proposed work}\label{sec:proposed-work}

In this document, we have shown how the session types that arise from singleton linear logic form a bridge between a class of ordered logical specifications and well-typed processes---between proof-construction-as-computation and proof-reduction-as-computation.
% There are three areas of proposed work.
Most of the proposed work involves generalizing this connection along several dimensions: 
\begin{enumerate*}[label=\emph{\roman*}), itemjoin={{; }}, itemjoin*={{; and }}]
\item a more expressive logic for specifications
% ---relating \emph{linear logical}, not just ordered logical, specifications to well-typed processes;
\item a more expansive translation that covers generative invariants
\item a more permissive session-type system
% the strength of the session-type system---relating a larger class of logical specifications to untyped or more weakly typed processes.
% \item a coarser equivalence for the specification-preserving property
\end{enumerate*}.
We now outline that proposed work.
% In this \lcnamecref{sec:proposed-work},


% The primary area of proposed work is to generalize this connection:
% instead of showing only how certain ordered logical specifications can translate to process chains that are typed by singleton linear logic, I propose to show how certain \emph{linear logical} specifications can translate to process trees that are typed by 

%  to (a class of) linear logical specifications and \ac{SILL} process definitions typed in linear logic.
% % To defend the proposed thesis, this connection must be extended to (a class of) linear logical specifications and \ac{SILL} process definitions typed in linear logic.

\subsection{From ordered logical to linear logical specifications}\label{sec:from-ordered-logical}

The primary area of proposed work is to generalize the logic used for specifications from ordered logic to the more expressive linear logic.
The process chains used in this proposal will be correspondingly generalized to \citeauthor{Caires+:MSCS13}'s~\autocite*{Caires+:MSCS13} \ac{SILL} process trees.
We'll motivate this generalization with an example: addition of binary representations.

\paragraph{Logical specification.}
By adapting ideas from Turing machines, it is possible---though undoubtedly awkward---to give an ordered logical specification for adding two binary numbers.
First, the numbers are arranged end-to-end, separated by a $\plus$ atom and terminated by an $\equals$ atom.
For instance, the string
\begin{equation*}
  \eps \fuse \bit{1} \fuse \bit{0} \fuse \plus \fuse \bit{1} \fuse \bit{0} \fuse \equals
\end{equation*}
represents a request to evaluate $2+2$.
Next, repeatedly decrement the second number and increment the first number.
When the second number reaches $0$, the first number holds the desired sum.
%
\begin{figure}
  \begin{equation*}
    \begin{alignedat}{2}
      &\equals \lrimp \monad{\dec \fuse \equals[']} \\[1.5\jot]
      % 
      &\bit{0} \fuse \dec \lrimp \monad{\dec \fuse \bit[']{0}} &\quad\enskip& \bit{0} \fuse \skp \lrimp \monad{\skp \fuse \bit{0}} \\
      &\bit{1} \fuse \dec \lrimp \monad{\skp \fuse \bit{0} \fuse \ok} && \bit{1} \fuse \skp \lrimp \monad{\skp \fuse \bit{1}} \\
      &\plus \fuse \dec \lrimp \monad{\fail} && \plus \fuse \skp \lrimp \monad{\inc \fuse \plus} \\[1.5\jot]
      % 
      &\ok \fuse \smash{\bit[']{0}} \lrimp \monad{\bit{1} \fuse \ok} && \fail \fuse \smash{\bit[']{0}} \lrimp \monad{\fail} \\
      &\ok \fuse \smash{\equals[']} \lrimp \monad{\equals} && \fail \fuse \smash{\equals[']} \lrimp \monad{\one}
    \end{alignedat}
  \end{equation*}
  \caption{An ordered logical specification of Turing-machine--like binary addition\label{fig:turing-binary-add}}
\end{figure}
%
The ordered logical specification
% and corresponding well-typed process definitions
of this addition algorithm is shown in \cref{fig:turing-binary-add}.


% \begin{sillcode}
%   plus =
%   { caseR of
%       dec => selectR fail; <->
%     | skip_inc => selectL inc; plus }

%   bit0 =
%   { caseR of
%       dec => selectL dec; bit0'
%     | skip_inc => selectL skip_inc; bit0 }

%   bit0' =
%   { caseL of
%       ok => selectR ok; bit1
%     | fail => selectR fail; <-> }

%   bit1 =
%   { caseR of
%       dec => selectR ok; selectL skip_inc; bit0
%     | skip_inc => selectL skip_inc; bit1 }

%   equals =
%   { selectL dec; equals' }

%   equals' =
%   { caseL of
%       ok => equals
%     | fail => <-> }
% \end{sillcode}


Unfortunately, this algorithm is not especially efficient: it takes $\Omega(N\log N)$ work to compute $M+N$.
It would be better to add the two binary representations bit-by-bit using the usual grade-school algorithm.
However, bit-by-bit addition demands that we can \emph{locally} access the least significant bit of each number and, separately, produce output bits---which is not possible in an ordered logical specification.
% A better algorithm would add the two numbers bit-by-bit.

It is possible, however, in a \emph{destination-passing} linear logical specification~\autocite{Cervesato+:CMU02}.
Even without the ordering constraint, a tree structure can be recovered via destinations that thread the $\bit{}$ atoms together with a $\plus$ parent atom.
% into linked-list--like structures that are joined at a $\plus$ parent.
Pictorially, the request to compute $2+2$ would be expressed as the state
\begin{equation*}
  \!\begin{aligned}[c]
    \eps(c_2) \tensor \bit{1}(c_2 , c_1) \tensor \bit{0}(c_1 , c_0) & \\
    \eps(d_2) \tensor \bit{1}(d_2 , d_1) \tensor \bit{0}(d_1 , d_0) &
  \end{aligned}
  \tensor \plus(c_0 , d_0 , c)
\end{equation*}
where $c$ and the $c_i$s and $d_j$s are all destinations and where the sum will be output at destination $c$.
Thus, the destination-passing rule for adding two numbers that both end in $\bit{0}$ is 
\begin{equation*}
  \bit{0}(C_1 , C_0) \tensor \bit{0}(D_1 , D_0) \tensor \plus(C_0 , D_0 , C)
    \lolli \monad{\exists c'_0.\, \plus(C_1 , D_1 , c'_0) \tensor \bit{0}(c'_0 , C)}
  \,.
\end{equation*}
It says that if both inputs end in $\bit{0}$, then their sum also ends in $\bit{0}$, with the more significant bits obtained by inductively adding the more significant bits of the two inputs.
When this rule is applied to the above state, the state changes and the first bit of output is produced:
\begin{equation*}
  \!\begin{aligned}[c]
    \eps(c_2) \tensor \bit{1}(c_2 , c_1) & \\
    \eps(d_2) \tensor \bit{1}(d_2 , d_1) &
  \end{aligned}
  \tensor \plus(c_1 , d_1 , c_0') \tensor \bit{0}(c_0' , c)
  \,.
\end{equation*}


% \begin{figure}
%   \begin{equation*}
%     \begin{lgathered}
%       \eps(C_0) \tensor \inc(C_0 , C) \lolli \monad{\eps(C)} \\
%       \bit{0}(C_1 , C_0) \tensor \inc(C_0 , C) \lolli \monad{\bit{1}(C_1 , C)} \\
%       \bit{1}(C_1 , C_0) \tensor \inc(C_0 , C) \lolli \monad{\exists c'_0.\, \inc(C_1 , c'_0) \tensor \bit{0}(c'_0 , C)}
%     \end{lgathered}
%   \end{equation*}

%   \begin{equation*}
%     \begin{lgathered}
%       \eps(C_0) \tensor \eps(D_0) \tensor \plus(C_0 , D_0 , C) \lolli \monad{\eps(C)} \\
%       \eps(C_0) \tensor \bit{0}(D_1 , D_0) \tensor \plus(C_0 , D_0 , C) \lolli \monad{\bit{0}(D_1 , C)} \\
%       \eps(C_0) \tensor \bit{1}(D_1 , D_0) \tensor \plus(C_0 , D_0 , C) \lolli \monad{\bit{1}(D_1 , C)} \\
%       %
%       \bit{0}(C_1 , C_0) \tensor \bit{0}(D_1 , D_0) \tensor \plus(C_0 , D_0 , C) \lolli \monad{\exists c'_0.\, \plus(C_1 , D_1 , c'_0) \tensor \bit{0}(c'_0 , C)} \\
%       \bit{0}(C_1 , C_0) \tensor \bit{1}(D_1 , D_0) \tensor \plus(C_0 , D_0 , C) \lolli \monad{\exists c'_0.\, \plus(C_1 , D_1 , c'_0) \tensor \bit{1}(c'_0 , C)} \\
%       \bit{1}(C_1 , C_0) \tensor \bit{0}(D_1 , D_0) \tensor \plus(C_0 , D_0 , C) \lolli \monad{\exists c'_0.\, \plus(C_1 , D_1 , c'_0) \tensor \bit{1}(c'_0 , C)} \\
%       %
%       \bit{1}(C_1 , C_0) \tensor \bit{1}(D_1 , D_0) \tensor \plus(C_0 , D_0 , C) \lolli \monad{\exists c'_0, c''_0.\, \plus(C_1 , D_1 , c'_0) \tensor \inc(c'_0 , c''_0) \tensor \bit{0}(c''_0 , C)}
%     \end{lgathered}
%   \end{equation*}
% \end{figure}


\paragraph{Concurrent processes.}
Now, let's consider how we might add two binary numbers using \ac{SILL} process trees.
Suppose that we represent each number as before---each number is a chain of $\bit{}$ processes (a degenerate process subtree if you will)---and that we include a $\plus$ parent process that uses the two numbers to offer the sum.
For example, the following process network represents a request to compute $2+2$.%
\footnote{The process names have been abbreviated to $\mathsf{e}$, $\mathsf{0}$, $\mathsf{1}$, and $\mathsf{+}$ for this picture.}
\begin{equation*}
  \begin{tikzpicture}[channel/.style = {text depth=0, midway, sloped, above}]
    \graph [
      tree layout, grow=left, typeset=$\mathsf{\tikzgraphnodetext}$, % empty nodes,
      nodes={
        rounded rectangle, rounded rectangle left arc=none,
        draw, minimum size=3ex,
      },
      edges={-},
      simple
    ] {
      / [draw=none] <-["$\scriptstyle c$"' channel]
      p / "\mathord{\mathclap{+}}" <- { [name separator=]
        { [name=c] 0 <-["$\scriptstyle c_1$"' channel] 1 <-["$\scriptstyle c_2$"' channel] e } ,
        { [name=d] 0 <-["$\scriptstyle d_1$"' channel] 1 <-["$\scriptstyle d_2$"' channel] e }
      };

      c0 ->["$\scriptstyle c_0$" channel] p;
      d0 ->["$\scriptstyle d_0$" channel] p;
    };
  \end{tikzpicture}
\end{equation*}
where the $c_i$s and $d_j$s are channels.
Notice the remarkable similarity of this network with the initial linear logical state shown above: destinations become channels and atoms become processes.

We would expect the $\plus$ process to be implemented in such a way that the above process network eventually transforms to the following network.
\begin{equation*}
  \begin{tikzpicture}[channel/.style = {text depth=0, midway, sloped, above}]
    \graph [
      tree layout, grow=left, typeset=$\mathsf{\tikzgraphnodetext}$, % empty nodes,
      nodes={
        rounded rectangle, rounded rectangle left arc=none,
        draw, minimum size=3ex,
      },
      edges={-},
      simple
    ] {
      / [draw=none] <-["$\scriptstyle c$"' channel]
      0 <-["$\scriptstyle c_0'$"' channel]
      p / "\mathord{\mathclap{+}}" <- { [name separator=]
        { [name=c] 1 <-["$\scriptstyle c_2$"' channel] e } ,
        { [name=d] 1 <-["$\scriptstyle d_2$"' channel] e }
      };

      c1 ->["$\scriptstyle c_1$" channel] p;
      d1 ->["$\scriptstyle d_1$" channel] p;
    };
  \end{tikzpicture}
\end{equation*}
Once again, there is a remarkable similarity between this network and the linear logical state after producing one output bit.

The proposed work is to make these similarities precise.
Just as in this document, the overall goal will be to identify a class of linear logical specifications that can be translated to \ac{SILL} process trees.
This item of proposed work is of primary importance.

\paragraph{Specific goals.}
This proposed work involves several components.
\begin{itemize}
\item \emph{Identify the class of linear logical specifications that act as choreographies.}
  Not all linear logical specifications will describe process-like behaviors.
  As in the ordered case, choreographies will need to be both local and specification-preserving.
  Now, however, locality depends not on adjacency in the ordered context, but on sharing a destination.

  The key challenge here, therefore, will be to ensure that destinations are used in a channel-like way within the choreography.
  Each atom should \enquote{offer} along one destination and \enquote{use} possibly several distinct destinations, and each destination should have one occurrence as an \enquote{offer} and one occurrence as a \enquote{use}.
  The machinery of destination uniqueness and index sets~\autocite{Simmons:CMU12} will likely be useful here.

\item \emph{Develop a translation of choreographies to \ac{SILL} processes.}
  In addition to translating destinations to channels, the main challenges here will be expanding the class of choreographies to allow translation to processes of $\tensor$, $\lolli$, and $\bang$ type.
  The $\tensor$ and $\lolli$ types are not possible in singleton linear logic (as mentioned in \cref{sec:other-session-types}) and so nothing similar was considered in the translation of ordered choreographies to process chains.  
  The $\bang$ type was, by choice, not considered in the ordered translation to keep the initial development simple.  
  
\item \emph{Give a type system for choreographies.}
  I expect to follow the same pattern as in the ordered case: derive the choreography typing rules from the process typing rules by looking at the process to which a choreography translates.
  While everything will be notationally more complex, I do not expect many surprises here.

\item \emph{Relate the results for ordered logic to those for linear logic.}
  \Textcite{Simmons+Pfenning:HOSC11} show how to encode ordered logical specifications in linear logic using destinations.
  For instance, under their destination-adding translation, the clause for incrementing $\bit{1}$ from our $\inc[<-]$-choreography becomes the following linear logical clause.
  \begin{equation*}
    \bit{1} \fuse \inc[<-] \lrimp \monad{\inc[<-] \fuse \bit{0}}
    \qquad\mathord{\leftrightsquigarrow}\qquad
    \!\begin{aligned}[t]
      \MoveEqLeft[1]
      \bit{1}(C_1, C_0) \tensor \inc[<-](C_0, C) \\[-0.5\jot]
        &\lolli \monad{\exists c'_0.\, \inc[<-](C_1, c'_0) \tensor \bit{0}(c'_0, C)}
    \end{aligned}
  \end{equation*}
  There should be a similar \enquote{channel-adding} translation from \ac{SISLL} process chains to \ac{SILL} processes.
  \begin{equation*}
    \bit{1} = \caseR{\inc[<-] => \selectL{\inc[<-]; \bit{0}}}
    \qquad\mathord{\leftrightsquigarrow}\qquad
 %   \!\begin{aligned}[t]
%      \MoveEqLeft[1]
      \begin{array}[t]{@{}l@{}}
      c \shortleftarrow \bit{1} \shortleftarrow d = {}\\
        \mathsf{case}\,c\,\mathsf{of}\\
        \quad\inc[<-] \Rightarrow \begin{array}[t]{@{}l@{}}
        \mathsf{select}\,d\,\inc[<-];\\
        c \shortleftarrow \bit{0} \shortleftarrow d
      \end{array}
      \end{array}
  %  \end{aligned}
  \end{equation*}
  Moreover, the translation from choreography to process should respect the destination-adding translation: adding destinations to an ordered choreography and then translating it to a process should give the same result as first translating the choreography to a process chain and then adding channels.
  This will serve as a sanity check on our design of the translation from linear choreographies to \ac{SILL} processes.
\end{itemize}


\subsection{Generative invariants as session types}

In this proposal, we have used the non-modal fragment of ordered logic to specify concurrent systems, whereas that fragment of ordered logic was originally developed by \textcite{Lambek:AMM58} to describe sentence structure.
However, these two modes of use of ordered logic are not as different as they might first appear.

Recall that, in our running example of an incrementable binary counter, the counter is represented as a string of $\bit{0}$, $\bit{1}$, and $\inc$ atoms terminated at the most significant end by an $\eps$.
More precisely, a string is a well-formed binary counter if it can be generated from the $\Cntr$ nonterminal by the context-free grammar
\begin{equation*}
  \Cntr ::= \eps \mid \Cntr \fuse \bit{0} \mid \Cntr \fuse \bit{1} \mid \Cntr \fuse \inc
  \,,
\end{equation*}
which is notation for four distinct productions.

Building on \citeauthor{Lambek:AMM58}'s work, the same context-free grammar can be described in ordered logic using \vocab{generative invariants}~\autocite{Simmons:CMU12}.
Each production in the grammar (below, left) becomes a clause (below, right), with the atomic proposition $\cntr$ acting as the nonterminal:
\begin{equation*}
  \left.
  \!\begin{aligned}
    &\Cntr \to \eps \\
    &\Cntr \to \Cntr \fuse \bit{0} \\
    &\Cntr \to \Cntr \fuse \bit{1} \\
    &\Cntr \to \Cntr \fuse \inc
  \end{aligned}
  \qquad\middle\vert\qquad
  \!\begin{aligned}
    &\cntr \lrimp \monad{\eps} \\
    &\cntr \lrimp \monad{\cntr \fuse \bit{0}} \\
    &\cntr \lrimp \monad{\cntr \fuse \bit{1}} \\
    &\cntr \lrimp \monad{\cntr \fuse \inc}
    \,.
  \end{aligned}
  \right.
\end{equation*}
Just as all binary counters are generated from the $\Cntr$ nonterminal according to the above productions, so too are all binary counters generated as maximal rewritings of the $\cntr$ atom according to these clauses:
%
\begin{definition}[Counter well-formedness]\label{def:counter-wf}
  String $S$ is a well-formed counter if $S$ is a maximal rewriting of $\cntr$ under the signature $\sig_{\cntr}$, that is, if $\cntr \trans+[\sig_{\cntr}] S \ntrans[\sig_{\cntr}]$.
\end{definition}
%
\noindent
For example, the maximal trace
\begin{equation*}
  \mathul{\cntr}
    \trans[\sig_{\cntr}] \mathul{\cntr} \fuse \inc
    \trans[\sig_{\cntr}] \mathul{\cntr} \fuse \bit{1} \fuse \inc
    \trans[\sig_{\cntr}] \eps \fuse \bit{1} \fuse \inc
    \ntrans[\sig_{\cntr}]
\end{equation*}
witnesses that $\eps \fuse \bit{1} \fuse \inc$ is a well-formed binary counter.

As observed by~\textcite{Simmons:CMU12}, generative invariants like $\sig_{\cntr}$ serve a similar purpose for ordered logical specifications as types do for functional programs: both describe the valid states and enable preservation and progress properties for their respective notions of computation.
%
% \begin{theorem*}[Safety of $\inc$]\leavevmode
%   \begin{itemize}[nosep]
%   \item If $S$ is a well-formed counter and $S \trans[\sig_{\inc}] S'$, then $S'$ is a well-formed counter.
%   \item If $S$ is a well-formed counter, then either $S$ is $\inc$-free or $S \trans[\sig_{\inc}] S'$.
%   \end{itemize}
% \end{theorem*}
%

% \noindent
Given the translation from choreographies (i.e., ordered logical specifications) to well-typed processes that was presented in \cref{sec:translation}, it's thus natural to ask how that translation interacts with a generative invariant.
Being the choreography's \enquote{type}, does the generative invariant become the process's session type?

It appears that the answer is likely yes.  Compare, for example, the generative invariant for the $\inc[<-]$-choreography with the types of the \sillinline`eps`, \sillinline`bit0`, and \sillinline`bit1` processes and the recursive type \sillinline`Cntr` from \cref{sec:exampl-binary-count}:
% It appears that the generative invariant for a choreography is closely connected to the session type of the process that results from translating the choreography.
% Compare the generative invariant for the $\inc[<-]$-choreography with the types of the \sillinline`eps`, \sillinline`bit0`, and \sillinline`bit1` processes and the recursive type \sillinline`Cntr` from \cref{?}:
\begin{equation*}
  \left.
  \!\begin{aligned}
    &\cntr \lrimp \monad{\eps} \\
    &\cntr \lrimp \monad{\cntr \fuse \bit{0}} \\
    &\cntr \lrimp \monad{\cntr \fuse \bit{1}} \\
    &\cntr \lrimp \monad{\cntr \fuse \inc[<-]}
  \end{aligned}
  \qquad\middle\vert\qquad
  \!\begin{aligned}
    &\text{\sillinline`eps : \{ |- Cntr \}`} \\
    &\text{\sillinline`bit0 : \{ Cntr |- Cntr \}`} \\
    &\text{\sillinline`bit1 : \{ Cntr |- Cntr \}`} \\
    &\text{\sillinline`stype Cntr = &\{ inc: Cntr \}`}
    \,.
  \end{aligned}
  \right.
\end{equation*}
Similar correspondences between generative invariants and session types exist for the $\dec[<-]$-choreography, the $\bit{}[->]$-choreography, and all other examples that we've considered.
It seems much too tantalizing to be pure coincidence.

Therefore, if all else goes smoothly, I propose to develop a translation of generative invariants to session types and prove that it is respected by the translation from choreographies to processes.
I plan to follow the pattern of this thesis proposal, first developing the translation for the special case of ordered generative invariants before extending the results to linear generative invariants.
%
This item of proposed work is of somewhat lesser importance than the generalization from ordered logic to linear logic for specifications, but has appeal in giving a more compelling explanation of the choreography types presented in \cref{sec:chor-types}.

% C(c) -> ?d. C(d) * 0(d,c)
% C(c) -> ?d0,d1. C(d0) * C(d1) * a(d0,d1,c)

% a : {c:C -| d0:C , d1:C}


% C -> e
% C -> C * 0
% C -> C * 1

% C = +{e: 1, 0: C, 1: C}
% i : C |- C


% The generative invariant describes the valid choreography states; the session type describes the valid process states.

% C -> {e}
% C -> {C * 0}
% C -> {C * 1}
% C -> {C * i}
% C' -> {C * d}
% C' -> {C * s}
% C' -> {C * z}
% C' -> {C' * 0'}
% C' -> {C' * 1'}

% C = &{i: C, d: C'}
% C' = +{s: C, z: C}






% % For example, the maximal trace
% % \begin{equation*}
% %   \mathul{\cntr}
% %     \trans[\sig_{\cntr}] \mathul{\cntr} \fuse \inc
% %     \trans[\sig_{\cntr}] \mathul{\cntr} \fuse \bit{1} \fuse \inc
% %     \trans[\sig_{\cntr}] \eps \fuse \bit{1} \fuse \inc
% %     \ntrans[\sig_{\cntr}]
% %   \,.
% % \end{equation*}
% % witnesses that $\eps \fuse \bit{1} \fuse \inc$ is a well-formed binary counter.
% % Note how important the choice of signature is: if we also allowed increment clauses from $\sig_{\inc}$ here, this trace would no longer be maximal.

% \begingroup
%   \RenewPredicate{\cntr}[Cntr]{1}%
% In fact, generative signatures generalize context-free grammars.
% For example, here $\cntr{}$ could be predicated on a natural number, effectively generalizing the context-free grammar to have a countably infinite family of nonterminals:
% % One slight generalization would be to predicate $\cntr{}$ on a natural number, effectively giving a countably infinite family of nonterminals:
% \begin{equation*}
%   \sig_{\cntr{}} =
%   \!\begin{aligned}[t]
%     &\cntr{0} \lrimp \monad{\eps} \,, \\
%     &\cntr{(2N)} \lrimp \monad{\cntr{N} \fuse \bit{0}} \,, \\
%     &\cntr{(2N{+}1)} \lrimp \monad{\cntr{N} \fuse \bit{1}} \,, \\
%     &\cntr{(N{+}1)} \lrimp \monad{\cntr{N} \fuse \inc}
%     \,.
%   \end{aligned}
% \end{equation*}

% This generative signature allows us to formally state and prove adequacy of the incrementable binary counter specification, $\sig_{\inc}$.
% %
% \begin{definition}[Counter well-formedness]\label{def:counter-wf}
%   String $S$ is a well-formed counter that represents natural number $N$ (or, more simply, $S$ represents $N$) if $S$ is a maximal rewriting of $\cntr{N}$ under the signature $\sig_{\cntr{}}$, that is, if $\cntr{N} \trans+[\sig_{\cntr{}}] S \ntrans[\sig_{\cntr{}}]$.
% \end{definition}
% %
% For example, the maximal trace
% \begin{equation*}
%   \mathul{\cntr{2}}
%     \trans[\sig_{\cntr{}}] \mathul{\cntr{1}} \fuse \inc
%     \trans[\sig_{\cntr{}}] \mathul{\cntr{0}} \fuse \bit{1} \fuse \inc
%     \trans[\sig_{\cntr{}}] \eps \fuse \bit{1} \fuse \inc
%     \ntrans[\sig_{\cntr{}}]
% \end{equation*}
% witnesses that $\eps \fuse \bit{1} \fuse \inc$ is a well-formed binary counter that represents $2$.
% Adequacy of $\inc$ can be stated as follows.
% (We elide the proof---it is straightforward but requires definitions and lemmas that would only obscure our main point.)
% %
% \begin{theorem}[Adequacy of $\inc$]
%   For all natural numbers $N$ and $N'$ and every well-formed counter $S$ that represents $N$, the equality $N + 1 = N'$ holds if and only if $S \fuse \inc \trans+[\sig_{\inc}] S' \ntrans[\sig_{\inc}]$ for some $S'$ that represents $N'$.
% \end{theorem}
% %
% As first observed by~\textcite{Simmons:CMU12}, generative signatures like $\sig_{\cntr{}}$ serve a similar purpose for ordered logical specifications as types do for functional programs: both enable preservation and progress properties for their respective notions of transition.

% Given the translation from ordered logical specifications (i.e., choreographies) to well-typed processes that was presented in \cref{?}, it's natural to ask how that translation interacts with a generative signature that acts as a specification's \enquote{type}.



\subsection{Translating untyped choreographies to untyped processes}

In this proposal, we have been concerned only with \emph{well-typed} processes and a corresponding class of well-typed choreographies.
The logically grounded session-type discipline ensures that well-typed processes (and, consequently, well-typed choreographies) enjoy communication safety, session fidelity, and deadlock freedom (i.e., global progress).
However, by demanding such a strong form of progress, the current session-type discipline forbids \emph{all} racy processes, even if the races are benign or non-critical.

% For example, consider the $\pi$-calculus process $x(y).z(w).P + z(w).x(y).P$, which waits to receive---in either order---both $y$ along channel $x$ and $w$ along channel $z$.
% This process is certainly racy because it's impossible, in general, to predict the order in which $y$ and $w$ will arrive.
% But, just as certainly, this race is benign because \emph{both} $y$ and $w$ must arrive before continuing with process $P$.
% This and other benign races should be permitted by the session-type discipline.

\NewPredicate{\okL}{0}
\NewPredicate{\okR}{0}

For example, consider the following process:
\begin{equation*}
  \caseL{\okL => \caseR{\okR => P}} + \caseR{\okR => \caseL{\okL => P}} \,,
\end{equation*}
which waits to receive---in either order---$\okL$ and $\okR$ labels from both its left- and right-hand neighbors, respectively.
(The process constructor $+$ denotes nondeterministic choice.)
This process is certainly racy because it's impossible, in general, to predict the order in which the $\okL$ and $\okR$ labels will arrive.
But, even so, this race is benign: execution continues with the process $P$ once, and only once, both labels arrive in either order.
% This and other benign races should be permitted by the session-type discipline.

Choreographies may serve as a stepping-stone toward a more permissive, yet still logically grounded, session-type discipline that allows this and other benign races.
The above example can be cast as the choreography
\begin{equation*}
  (\okL[->] \limp \monad{\okR[<-] \rimp \monad{A^+}}) \with (\okR[<-] \rimp \monad{\okL[->] \limp \monad{A^+}}) \,.
\end{equation*}
By considering how the proposed translation from generative invariants to session types might apply to a generative invariant for this choreography, we may gain insight into a session-type discipline that allows benign races.
I also propose to develop a translation of a broader class of choreographies to untyped processes, which may provide different insight than just looking at the existing session-type discipline.

% In this proposal, we have been concerned with translating only \emph{well-typed} choreographies to \emph{well-typed} processes.
% By concentrating on well-typed processes, we ensure that well-typed choreographies enjoy the same, strong progress and preservation properties as their process counterparts.
% At the same time, however, by demanding such a strong form of progress, we forbid all racy processes and choreographies, even if those races are benign or non-critical.



\subsection{Session-typed Turing machines}

Finally, as the example in \cref{sec:from-ordered-logical} shows, some Turing machines can be session-typed: by translating the ordered logical specification from \cref{fig:turing-binary-add}, we get a well-typed, Turing-machine--like process for adding two binary representations.
In particular, the chain structure of singleton linear logic suggests a fit with the one-way infinite tapes of Turing machines.

Although not directly related to my proposed thesis statement, if time permits, I would like to explore further the possible connections between singleton linear logic and Turing machines.
This is the most open-ended item of proposed work and the least related to my proposed thesis statement, but, if successful, may have some interest to researchers outside the programming languages community, e.g., those working in the theory of computation.


% \subsubsection{Earlier draft}

% First, we need a few \lcnamecrefs{def:counter-wf}.
% %
% \begin{definition}[Counter well-formedness and other properties]\label{def:counter-wf}
%   \mbox{}
%   \begin{itemize}[nosep]
%   \item String $S$ is a well-formed counter that represents natural number $N$ (or, more simply, $S$ represents $N$) if $S$ is a maximal rewriting of $\cntr{N}$ under the signature $\sig_{\cntr{}}$, that is, if $\cntr{N} \trans+[\sig_{\cntr{}}] S \ntrans[\sig_{\cntr{}}]$.
%   \item String $S$ is a well-formed counter if there is some $N$ for which $S$ represents $N$.
%   \item Well-formed counter $S$ is $\inc$-free if the well-formedness trace does not use the $\inc$ clause from the $\sig_{\cntr{}}$ signature.
%   \item Likewise, well-formed counter $S$ has no leading $\bit{0}$s if the well-formedness trace uses the $\bit{0}$ clause only when $N > 0$.
%   \end{itemize}
% \end{definition}
% %
% For example, the maximal trace
% \begin{equation*}
%   \mathul{\cntr{2}}
%     \trans[\sig_{\cntr{}}] \mathul{\cntr{1}} \fuse \inc
%     \trans[\sig_{\cntr{}}] \mathul{\cntr{0}} \fuse \bit{1} \fuse \inc
%     \trans[\sig_{\cntr{}}] \eps \fuse \bit{1} \fuse \inc
%     \ntrans[\sig_{\cntr{}}]
% \end{equation*}
% witnesses that $\eps \fuse \bit{1} \fuse \inc$ is a well-formed binary counter that represents $2$; it is not $\inc$-free, but it does have no leading $\bit{0}$s.
% Note how important the choice of signature is: if we also allowed increment clauses from $\sig_{\inc}$ here, this trace would no longer be maximal.

% With these \lcnamecrefs{def:counter-wf} in hand, we can establish a bijection between the natural numbers and equivalence classes of well-formed counters.
% \begin{theorem}[Adequacy of counters]\label{thm:counter-adequacy}
%   \mbox{}
%   \begin{subtheorems}{theorem}[nosep]
%   \item\label{thm:counter-adequacy:value}
%     For each natural number $N$, there is a unique well-formed counter $S$ that represents $N$, is $\inc$-free, and has no leading $\bit{0}$s.
%   \item\label{thm:counter-adequacy:counter}
%     For each well-formed counter $S$, there is a unique natural number $N$ such that $S$ represents $N$.
%   \end{subtheorems}
% \end{theorem}
% \begin{proof}
%   \Cref{thm:counter-adequacy:value} is by induction on $N$, and \cref{thm:counter-adequacy:counter} is by induction on the structure of the maximal trace that witnesses the well-formedness of $S$.
% \end{proof}
% %
% As the following \lcnamecrefs{thm:counter-adequacy} show, the represented value is invariant under $\sig_{\inc}$-rewriting and $\sig_{\inc}$-rewriting always terminates in an $\inc$-free counter.
% It follows that $\inc$s adequately specify increments.
% %
% % \begin{lemma}
% %   \mbox{}
% %   \begin{itemize}[nosep]
% %   \item If $S_0 \fuse \bit{0} \trans[\sig_{\inc}] S'$, then $S' = S'_0 \fuse \bit{0}$ and $S_0 \trans[\sig_{\inc}] S'_0$.
% %   \item If $S_0 \fuse \bit{1} \trans[\sig_{\inc}] S'$, then $S' = S'_0 \fuse \bit{1}$ and $S_0 \trans[\sig_{\inc}] S'_0$.
% %   \item If $S_0 \fuse \inc \trans[\sig_{\inc}] S'$, then either:
% %     \begin{itemize}[nosep]
% %     \item $S' = S'_0 \fuse \inc$ and $S_0 \trans[\sig_{\inc}] S'_0$;
% %     \item $S_0 = \eps$ and $S' = \eps \fuse \bit{1}$;
% %     \item $S_0 = S_{00} \fuse \bit{0}$ and $S' = S_{00} \fuse \bit{1}$; or
% %     \item $S_0 = S_{00} \fuse \bit{1}$ and $S' = S_{00} \fuse \inc \fuse \bit{0}$.
% %     \end{itemize}
% %   \end{itemize}
% % \end{lemma}
% % 
% \begin{theorem}[Preservation]\label{thm:counter-preservation}
%   For every well-formed counter $S$ that represents $N$, if $S \trans[\sig_{\inc}] S'$, then $S'$ is also a well-formed counter that represents $N$.
%   % \begin{enumerate}[nosep]
%   % \item For every well-formed counter $S$, if $S \trans[\sig_{\inc}] S'$, then $S'$ is a well-formed counter.
%   % \item For all well-formed counters $S$ and $S'$, if $S$ represents $N$ and $S \trans[\sig_{\inc}] S'$, then $S'$ also represents $N$.
%   % \end{enumerate}
% \end{theorem}
% \begin{proof}
%   By induction on the structure of the maximal trace that witnesses the well-formedness of $S$, relying on an inversion lemma for $\sig_{\inc}$-steps:
%   \begin{itemize}[nosep]
%   \item If $S_0 \fuse \bit[_{\mathit{b}}]{} \trans[\sig_{\inc}] S'$, then $S_0 \trans[\sig_{\inc}] S'_0$ and $S' = S'_0 \fuse \bit[_{\mathit{b}}]{}$.
%   \item If $S_0 \fuse \inc \trans[\sig_{\inc}] S'$, then either:
%     \begin{itemize}[nosep]
%     \item $S_0 \trans[\sig_{\inc}] S'_0$ and $S' = S'_0 \fuse \inc$;
%     \item $S_0 = \eps$ and $S' = \eps \fuse \bit{1}$;
%     \item $S_0 = S'_0 \fuse \bit{0}$ and $S' = S'_0 \fuse \bit{1}$; or
%     \item $S_0 = S'_0 \fuse \bit{1}$ and $S' = S'_0 \fuse \inc \fuse \bit{0}$.
%     \end{itemize}
%   \end{itemize}
% \end{proof}

% % \begin{theorem}[Progress]
% %   \mbox{}
% %   \begin{enumerate}[nosep]
% %   \item For every well-formed counter $S$, if $S$ is $\inc$-free, then $S \ntrans[\sig_{\inc}]$.
% %   \item For every well-formed counter $S$, either $S$ is $\inc$-free or $S \trans[\sig_{\inc}] S'$ for some $S'$.
% %   \end{enumerate}
% % \end{theorem}
% % \begin{proof}
% %   By induction on the structure of the maximal trace that generates $S$.
% % \end{proof}

% \begin{theorem}[Termination]\label{thm:inc-termination}
%   \mbox{}
%   \begin{subtheorems}{theorem}[nosep]
%   \item\label{thm:inc-termination:inc-free}
%     For every well-formed counter $S$, string $S$ is $\inc$-free if and only if $S \ntrans[\sig_{\inc}]$.
%   \item\label{thm:inc-termination:finite}
%     For every well-formed counter $S$, there is no infinite $\sig_{\inc}$-rewriting of $S$.
%   \end{subtheorems}
% \end{theorem}
% \begin{proof}
%   \Cref{thm:inc-termination:inc-free} is by induction on the structure of the maximal trace that witnesses the well-formedness of $S$.
%   %
%   \DeclarePairedDelimiter{\meas}{\lVert}{\rVert}%
%   \DeclarePairedDelimiter{\size}{\lvert}{\rvert}%
%   To prove \cref{thm:inc-termination:finite}, define $\meas{S}$ to be a measure in which each $\inc$ in $S$ contributes an amount equal to the length of its higher-order substring:%
%   \footnote{If desired, this measure can also be defined using a generative signature.}
%   \begin{equation*}
%     \!\begin{aligned}[t]
%       \meas{\eps} &= 0 \\
%       \meas{S \fuse \bit[_{\mathit{b}}]{}} &= %\meas{S} \\
%       % \meas{S \fuse \bit{1}} = 
%       \meas{S} \\
%       \meas{S \fuse \inc} &= \meas{S} + \size{S}
%     \end{aligned}
%     \qquad
%     \!\begin{aligned}[t]
%       \size{\eps} &= 1 \\
%       \size{S \fuse \bit[_{\mathit{b}}]{}} &=
%       % \size{S \fuse \bit{1}} =
%       \size{S \fuse \inc} = \size{S} + 1
%     \end{aligned}
%   \end{equation*}
%   % Define
%   % \begin{align*}
%   %   &\meas{0,0} \lrimp \monad{\eps} \\
%   %   &\meas{M,(L{+}1)} \lrimp \monad{\meas{M,L} \fuse \bit{0}} \\
%   %   &\meas{M,(L{+}1)} \lrimp \monad{\meas{M,L} \fuse \bit{1}} \\
%   %   &\meas{(M{+}L),(L{+}1)} \lrimp \monad{\meas{M,L} \fuse \inc}
%   % \end{align*}
%   One can show that $\meas{\mathord{-}}$ is strictly decreasing for each step $S \trans[\sig_{\inc}] S'$, from which \cref{thm:inc-termination:finite} follows.
% \end{proof}

% % Cntr 0 <- eps
% % Cntr N+1 <- Cntr N * bit0
% % Cntr 

% \begin{corollary}[Adequacy of $\inc$]
%   For all natural numbers $N$ and $N'$ and every well-formed counter $S$ that represents $N$, the equality $N + 1 = N'$ holds if and only if $S \fuse \inc \trans+[\sig_{\inc}] S' \ntrans[\sig_{\inc}]$ for some $S'$ that represents $N'$.
%   % For all natural numbers $N$ and $N'$ and every well-formed counter $S$, if $S$ represents $N$, then $N + 1 = N'$ if and only if $S \fuse \inc \trans+[\sig_{\inc}] S' \ntrans[\sig_{\inc}]$ for some $S'$ that represents $N'$.
%   % % \begin{enumerate}
%   % % \item For all natural numbers $N$ and $N'$, if $S$ represents $N$, $S'$ represents $N'$, and $N + 1 = N'$, then $S \fuse \inc \trans+ S' \ntrans$.
%   % % %
%   % % \item For all well-formed counters $S$ and $S'$, if $S$ represents $N$, $S'$ represents $N'$, and $S \fuse \inc \trans+ S' \ntrans$, then $N + 1 = N'$.
%   % % \end{enumerate}
% \end{corollary}

% % \begin{falseclaim}
% %   For all natural numbers $N$ and $N'$ and well-formed counters $S$ and $S'$, if $S$ represents $N$ and $S'$ represents $N'$ and is $\inc$-free, then $N + 1 = N'$ if and only if $S \fuse \inc \trans+[\sig_{\inc}] S' \ntrans[\sig_{\inc}]$.
% %   % \begin{enumerate}
% %   % \item For all natural numbers $N$ and $N'$, if $S$ represents $N$, $S'$ represents $N'$, and $N + 1 = N'$, then $S \fuse \inc \trans+ S' \ntrans$.
% %   % %
% %   % \item For all well-formed counters $S$ and $S'$, if $S$ represents $N$, $S'$ represents $N'$, and $S \fuse \inc \trans+ S' \ntrans$, then $N + 1 = N'$.
% %   % \end{enumerate}
% % \end{falseclaim}

% In this way, the generative signature $\sig_{\cntr{}}$ serves a similar purpose for logic programming as types do for functional programming~\autocite{Simmons:CMU12}: both enable preservation and progress properties for their respective notions of transition.
% We will return to this point in \cref{sec:proposed-work}.

% \endgroup


% \end{document}

% Functional logic languages, such as Curry\cite{Hanus:Ganzinger13}, view functional programs as logic programs.
This proposal takes the opposite approach, compiling a class of logic programs to functional programs.

\citeauthor{Simmons-Zerny:LICS13}'s correspondence between natural semantics (functional program) and abstract machines (logic program)\autocite{Simmons-Zerny:LICS13}.

%%% Local Variables:
%%% TeX-master: "proposal"
%%% End:


\appendix
\section{Turing-machine--like addition process}

In this appendix, we present the code for a well-typed, Turing-machine--like addition process.
It corresponds to the ordered logical specification shown in \cref{fig:turing-binary-add}.

\begin{sillcode*}[
  % caption={Turing-machine--like addition of binary numbers},
  % label={lst:turing-add},
  % floatplacement=tb,
  gobble=2
]
  stype Cntr = &{ inc: Cntr }
  
  eps : { |- Cntr } =
  { caseR of
      inc => eps; bit1 }
  
  bit0 : { Cntr |- Cntr } =
  { caseR of
      inc => bit1 }
  
  bit1 : { Cntr |- Cntr } =
  { caseR of
      inc => selectL inc; bit0 }

  stype Cntr_D = &{ dec: Cntr_D' , skip: Cntr_D }
    and Cntr_D' = +{ ok: Cntr_D , fail: Cntr }

  bit0_d : { Cntr_D |- Cntr_D } =
  { caseR of
      dec => selectL dec; bit0_d'
    | skip => selectL skip; bit0_d }

  bit1_d : { Cntr_D |- Cntr_D } =
  { caseR of
      dec => selectL skip; selectR ok; bit0_d
    | skip => selectL skip; bit1_d }

  bit0_d' : { Cntr_D' |- Cntr_D' } =
  { caseL of
      ok => selectR ok; bit1_d
    | fail => selectR fail; <-> }

  plus : { Cntr |- Cntr_D } =
  { caseR of
      dec => selectR fail; <->
    | skip => selectL inc; plus }

  equals : { Cntr_D |- Cntr } =
  { selectL dec; equals' }

  equals' : { Cntr_D' |- Cntr } =
  { caseL of
      ok => equals
    | fail => <-> }
\end{sillcode*}

%\nocite{*}

\printbibliography

\end{document}
