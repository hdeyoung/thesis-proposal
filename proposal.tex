% arara: lualatex
% arara: biber
% arara: lualatex
% arara: lualatex
% arara: lualatex
\documentclass{hdeyoung-proposal}

\usepackage{proposal-macros}
\usetikzlibrary{positioning,shapes.misc,graphs,quotes,graphdrawing}
\usegdlibrary{trees}

\DeclareAcronym{BHK}{
  short = BHK,
  long  = Brouwer-Heyting-Kolmogorov
}

\DeclareAcronym{JILL}{
  short = JILL,
  long  = judgmental intuitionistic linear logic
}

\DeclareAcronym{ILL}{
  short = ILL,
  long  = intuitionistic linear logic
}

\DeclareAcronym{SML}{
  short = SML,
  long  = Standard ML
}

\DeclareAcronym{SOS}{
  short = SOS,
  short-indefinite = an,
  long = structural operational semantics
}

\DeclareAcronym{SSOS}{
  short = SSOS,
  short-indefinite = an,
  long = substructural operational semantics
}

\DeclareAcronym{SILL}{
  short = SILL,
  long = session-typed intuitionistic linear logic
}

\DeclareAcronym{CLF}{
  short = CLF,
  long = the concurrent logical framework
}

\addbibresource{proposal.bib}

\begin{document}

\title{Session-Typed Concurrent\\Logical Specifications}
\author{Henry DeYoung\\\texttt{hdeyoung@cs.cmu.edu}}
\date{February 5, 2015}
\maketitle

\begin{abstract}
% Concurrency arises naturally in a proof-construction-as-computation setting: if all interleavings of independent [proof] steps in a logical specification are treated as indistinguishable, then those steps appear to be concurrent. 
% Concurrency also arises naturally in a proof-reduction-as-computation setting: there is a Curry--Howard isomorphism between sequent proofs in intuitionistic linear logic and session-typed processes in the $\pi$-calculus.
% 
% Using session types as a bridge, we propose to identify a fragment of intuitionistic linear logic in which these two notions of concurrency coincide.
% Specifically, we propose to show that, given an assignment of message and process roles to atomic propositions, a class of concurrent linear logical specifications can be translated to session-typed processes.
% In addition to the practical benefits of generating well-typed implementations from logical specifications, the proposed work can be seen as giving a proof-theoretic reconstruction of work on global session types for multiparty communication, and furthering an understanding of connections between proof construction and proof reduction, despite their apparent disparity.
% 
% This document aims to establish the plausibility of justifies the proposed work's feasibility by carrying out the program in the simpler case of concurrent ordered logical specifications.
% 
% 
% 
\noindent
% In computational interpretations of intuitionistic linear logic, two notions of concurrency arise.
Concurrency arises naturally in a proof-construction-as-computation interpretation of intuitionistic linear logic: if all interleavings of independent proof steps in a logical specification are treated as indistinguishable, then those steps appear to be concurrent. 
Concurrency also arises naturally in proof-reduction-as-computation: there is a Curry--Howard isomorphism, due to \citeauthor{Caires+Pfenning:CONCUR10}, between sequent proofs in intuitionistic linear logic and session-typed processes in the $\pi$-calculus.

In this proposal, we put forward the thesis that session types form a bridge between these two apparently disparate notions of concurrency.
Specifically, we propose to show that, given an assignment of process and message roles to atomic propositions, a class of concurrent linear logical specifications can be translated to session-typed processes.
In addition to the practical benefits of generating well-typed implementations from logical specifications, the proposed work can be seen as giving a proof-theoretic reconstruction of work on multiparty session types; as assigning behavioral types to a class of logical specifications, thereby ensuring deadlock freedom for those specifications; and as furthering an understanding of the relationship between proof construction and proof reduction.

This document aims to establish the thesis's plausibility by defending it in the restricted setting of intuitionistic ordered logic.
% We show that a class of concurrent ordered logical specifications can be translated to session-typed processes.
The primary area of proposed research will then be to relax that restriction, extending the ideas in this document to linear logic.


% Concurrency arises naturally in a proof-construction-as-computation interpretation of intuitionistic linear logic: if all interleavings of independent proof steps in a logical specification are treated as indistinguishable, then those steps appear to be concurrent. 
% Concurrency also arises naturally in proof-reduction-as-computation: there is a Curry--Howard isomorphism, due to \citeauthor{Caires+Pfenning:CONCUR10}, between sequent proofs in intuitionistic linear logic and session-typed processes in the $\pi$-calculus.

% In this proposal, we put forward the thesis that session types form a bridge between these two apparently disparate notions of concurrency in computational interpretations of intuitionistic linear logic.
% Specifically, we propose to show that, given an assignment of process and message roles to atomic propositions, a class of concurrent linear logical specifications can be translated to session-typed processes.
% In addition to the practical benefits of generating well-typed implementations from logical specifications, the proposed work can be seen as giving a proof-theoretic reconstruction of work on multiparty session types; assigning behavioral types to a class of logical specifications, thereby ensuring deadlock freedom for those specifications; and furthering an understanding of the relationship between proof construction and proof reduction.

% This document aims to establish the thesis's plausibility by defending it in the restricted setting of intuitionistic ordered logic.
% % We show that a class of concurrent ordered logical specifications can be translated to session-typed processes.
% The primary area of proposed research will then be to relax that restriction, extending the ideas in this document to linear logic.

\vspace{\baselineskip}
\noindent\textbf{Keywords:} substructural logics, proof reduction, proof construction, concurrency, session types
\end{abstract}

\tableofcontents
\clearpage

\begin{itemize}
\item Two paradigms for logic-based computation: proof search (logic programming) and proof reduction (functional programming).
      Understand the relationship between them.
\item Proof search allows very concise, expressive programs (e.g., [parallel] bubblesort in one line), but does not provide many (any?) guarantees.
      Proof reduction provides strong guarantees (e.g., preservation, progress, and termination), but requires the programmer to supply many details (e.g., bubblesort).
\end{itemize}

%%% Local Variables:
%%% TeX-master: "proposal"
%%% End:


% arara: pdflatex
% arara: pdflatex
% arara: biber
% arara: pdflatex
% arara: pdflatex
\documentclass[
  class=../hdeyoung-proposal,
  crop=false
]{standalone}

\usepackage[subpreambles]{standalone}

\usepackage{ordered-logic}
\usepackage{binary-counter}

\addbibresource{../proposal.bib}

\begin{document}

\section{Background: Concurrent ordered logic programming}\label{sec:ordered-lp}

Viewed through a computational lens, proof search in a fragment of ordered logic becomes a forward-chaining logic programming language~\autocite{Pfenning+Simmons:LICS09}.
It can be seen as a logically motivated generalization of string rewriting~\autocite[see, \eg,][]{Book+Otto:SRS93}, an analogy which we will exploit to provide some intuition for this form of ordered logic programming.

From the perspective of string rewriting, an ordered logic program's atomic propositions are letters; ordered conjunctions (or ordered contexts) of these atoms are strings; and, under a focused proof search strategy~\autocite{Andreoli:JLC92}, the ordered implications that serve as program clauses are string rewriting rules.
An example will help to clarify.


% arara: pdflatex
% arara: biber
% arara: pdflatex
% arara: pdflatex
\documentclass[
  class=../hdeyoung-proposal,
  crop=false
]{standalone}

\usepackage{ordered-logic}
\usepackage{binary-counter}
\usepackage{proof}
\usepackage{tikz}

\NewDocumentCommand{\trans}{t* t+ o}{%
  \longrightarrow
  \IfBooleanT{#1}{^*}\IfBooleanT{#2}{^+}%
  \IfValueT{#3}{_{#3}}%
}
\NewDocumentCommand{\ntrans}{}{
  \longarrownot\trans
}

\DeclareAcronym{CLF}{
  short = CLF,
  long = the concurrent logical framework
}

\NewPredicate{\Cntr}[Cntr][font = \mathit]{0}
\NewPredicate{\cntr}[Cntr]{0}

\NewPredicate{\coin}{0}
\NewPredicate{\heads}{0}
\NewPredicate{\tails}{0}
\NewPredicate{\win}{0}
\NewPredicate{\loss}{0}

\begin{document}

\subsection{Example: Binary counter}\label{sec:exampl-binary-count-5}

Using an ordered logic program, we can specify an incrementable binary counter.
% Here we give an ordered logic program for an incrementable binary counter.
% We can implement an incrementable binary counter as an ordered logic program.
% From the perspective of string rewriting, atomic propositions are letters, and ordered conjunctions of these atoms are strings.
% 
% % In the string rewriting terminology, atomic propositions correspond to letters; ordered conjunctions of these atoms correspond to strings; and ordered implications correspond to string rewriting rules.
% In the string rewriting terminology, atomic propositions correspond to letters, and ordered conjunctions of these atoms correspond to strings.
% % Using a focused proof search strategy~\autocite{Andreoli:JLC92}, ordered implications correspond to string rewriting rules.
The counter is represented as a string of $\bit{0}$ and $\bit{1}$ atoms terminated at the most significant end by an $\eps$.
For instance, the ordered conjunction $\eps \fuse \bit{1} \fuse \bit{0}$ is a string that represents a counter with value $2$.
%
Increment instructions are represented by $\inc$ atoms at the counter's least significant end. 
% Also interspersed are $\inc$ atoms, each of which serves as an increment instruction sent to the counter given by the more significant bits.
Thus, $\eps \fuse \bit{1} \fuse \inc$ represents a counter with value $1$ that has been instructed to increment once.

Operationally, increments are described by the program's three clauses, the first of which is
\begin{equation*}
  \bit{1} \fuse \inc \lrimp \monad{\inc \fuse \bit{0}}
  \,.
\end{equation*}
From a string rewriting perspective, this implication is a rule for rewriting the (sub)string $\bit{1} \fuse \inc$ as $\inc \fuse \bit{0}$.%
% From a string rewriting perspective, this implication is interpreted as a rule for rewriting the (sub)string $\bit{1} \fuse \inc$ as $\inc \fuse \bit{0}$.%
\footnote{This interpretation is justified because, logically, implications transform one resource into another.  This will be explained in more detail in \cref{??}.}
%  in the sense that the following bottom-up rule
% % This clause allows the string $\bit{1} \fuse \inc$ to be rewritten as $\inc \fuse \bit{0}$, in the sense that the rule
% is admissible whenever this clause is part of the persistent context, $\uctx$.
% \begin{equation*}
%   \infer{\uctx ; \omatch{\bit{1} \fuse \inc} \seq C \lax}{
%     \uctx ; \ofill{\inc \fuse \bit{0}} \seq C \lax}
% \end{equation*}}
%
% Just as this implication transforms 
% When this proposition is part of the persistent context $\uctx$, the rule
% \begin{equation*}
%   \infer{\uctx ; \omatch{\bit{1} \fuse \inc} \seq C \lax}{
%     \uctx ; \ofill{\inc \fuse \bit{0}} \seq C \lax}
%   \,.
% \end{equation*}
% is admissible.
% Read bottom-up, this rule serves to rewrite the (sub)string $\bit{1} \fuse \inc$ as the string $\inc \fuse \bit{0}$.
%
By rewriting $\bit{1} \fuse \inc$ as $\inc \fuse \bit{0}$, this clause serves to carry the $\inc$ up past any $\bit{1}$s that may exist at the counter's least significant end.

Whenever the carried $\inc$ reaches the $\eps$ or right-most $\bit{0}$, the carry is resolved by one of the other two program clauses:
\begin{align*}
  &\eps \fuse \inc \lrimp \monad{\eps \fuse \bit{1}} \\
  &\bit{0} \fuse \inc \lrimp \monad{\bit{1}} \,.
\end{align*}
By rewriting $\eps \fuse \inc$ as $\eps \fuse \bit{1}$, this second clause ensures that the carry becomes a new most significant $\bit{1}$ in the $\eps$ case.
By rewriting $\bit{0} \fuse \inc$ as $\bit{1}$, the third clause ensures that the carry flips the $\bit{0}$ to $\bit{1}$ in that case.

For example, the counter $\eps \fuse \bit{1} \fuse \inc$ can be maximally rewritten as
\ExplSyntaxOn
\NewDocumentCommand{\mathul}{m}{
  \mathpalette\mathul:nn{#1}
}
\cs_new:Npn \mathul:nn #1#2 {
  \tikz [baseline] {
    \node (pr) [anchor = base, inner~sep = 0em] {$#1#2$};
    \draw [overlay, ultra~thick, gray]
      ([yshift=-0.3em]pr.base~west) -- ([yshift=-0.3em]pr.base~east);
  }
}
\ExplSyntaxOff
\begin{equation*}
  \eps \fuse \mathul{\bit{1} \fuse \inc}
    \trans \mathul{\eps \fuse \inc} \fuse \bit{0}
    \trans \eps \fuse \bit{1} \fuse \bit{0}
    \ntrans
  \text{\,,}
\end{equation*}
where at each step the sites available for rewriting are underlined.
This trace computes $1 + 1 = 2$ in binary representation.
More generally, the above clauses adequately specify an increment operation:
string $S$ represents a counter with value $N$ if and only if $S \fuse \inc \trans+ S' \ntrans$ for some string $S'$ that represents a counter of value $N + 1$, where $\trans+$ denotes the transitive closure of the rewriting relation $\trans$.

\subsection{Concurrency}\label{sec:concurrency-1}

Some strings contain more than one rewrite site.
% several disjoint substrings that are amenable to rewriting.
% If these sites are disjoint, the rewrites can be thought of as happening concurrently.
For instance, the following binary counter has two $\inc$s in flight, which give rise to two disjoint rewrite sites.
\begin{equation*}
  \mathul{\eps \fuse \inc} \fuse \mathul{\bit{0} \fuse \inc}
\end{equation*}
The rewrites in this example can be interleaved in two ways: either
% there are two interleavings of these rewrites: either
\begin{align*}
  &\mathul{\eps \fuse \inc} \fuse \mathul{\bit{0} \fuse \inc}
     \trans \eps \fuse \bit{1} \fuse \mathul{\bit{0} \fuse \inc}
     \trans \eps \fuse \bit{1} \fuse \bit{1}
     \ntrans \\
  %
  \shortintertext{or}
  %
  &\mathul{\eps \fuse \inc} \fuse \mathul{\bit{0} \fuse \inc}
     \trans \mathul{\eps \fuse \inc} \fuse \bit{1}
     \trans \eps \fuse \bit{1} \fuse \bit{1}
     \ntrans
   \,.
\end{align*}

However, because these two rewrites are independent, they should be considered morally concurrent.
Rather than giving a truly concurrent semantics for string rewriting, we can treat different interleavings of independent steps as indistinguishable.
Then, because we can't observe which rewrite occurred first, the two rewrites appear to happen concurrently.
This is the idea of \vocab{concurrent equality} from {CLF}~\autocite{Watkins+:CMU02} that gives a pretense to true concurrency, borrowed for ordered logic programming.


\subsubsection{Infinite traces and fairness}

\NewPredicate{\incs}{0}%
Thus far, all traces have been finite, but this is not necessarily so.
Consider adding an atom, $\incs$, that generates a stream of $\inc$ atoms:
\begin{equation*}
  \incs \lrimp \monad{\inc \fuse \incs}
  \,.
\end{equation*}
Among the infinite traces now possible is
% Beginning from $\eps \fuse \incs$, there are now infinite trace
\begin{equation*}
  \eps \fuse \mathul{\incs}
    \trans \mathul{\eps \fuse \inc} \fuse \mathul{\incs}
    \trans \dotsb
    \trans \mathul{\eps \fuse \inc} \fuse \inc \fuse \dotsb \fuse \inc \fuse \mathul{\incs}
    \trans \dotsb
  \,.
\end{equation*}

Notice that this trace never rewrites the infinitely available $\eps \fuse \inc$, instead always choosing to rewrite $\incs$.
Because of its\fxnote{\ \st{scheduling}} bias against rewriting $\eps \fuse \inc$, we say that the trace is \vocab{(weakly) process-unfair}.

Process unfairness is fundamentally at odds with true concurrency.
Because true concurrency allows multiple independent events to occur simultaneously, one event cannot preclude another independent event;
in the above example, for instance, rewriting the $\incs$ atom would not preclude rewriting the independent $\eps \fuse \inc$ substring.
Therefore, to maintain the pretense of true concurrency, we must require all traces to be (weakly) process-fair.


% Transition unfairness is fundamentally at odds with true concurrency
% because true concurrency allows multiple independent events to occur simultaneously;
% the occurrence of one event cannot preclude another independent event from occurring at the same time.
% To maintain the pretense of true concurrency, 

% In the above example, $\eps \fuse \inc$ and $\incs$ are independent rewrite sites

% In the above example, $\eps \fuse \inc$ and $\incs$ are independent rewrite sites.
% % ; if they are truly concurrent, the rewrite of $\eps \fuse \inc$ should eventually occur.
% To maintain any pretense of true concurrency, the rewrite of $\eps \fuse \inc$ should eventually occur since true concurrency would allow multiple independent events to occur simultaneously.



% However, by treating different interleavings of independent steps as indistinguishable, these two rewrites happen concurrently.




% For instance, notice that this binary counter specification allows multiple $\inc$s to be in flight at once, each of which is amenable to rewriting.



% Sometimes several disjoint substrings are amenable to rewriting.


% Different interleavings of independent steps are indistinguishable.

% String rewriting (and therefore ordered logic programming) gives rise to 


% The counter $\eps \fuse \inc \bit{0} \fuse \inc$ has two $\inc$s in flight, which can be rewritten independently.


% Rewrites of disjoint substrings can be thought of as happening concurrently.

% Notice that we can also allow multiple $\inc$s to be in flight at once, and that independent rewrites can thought of as happening concurrently.
% For instance, the counter $\eps \fuse \inc \fuse \bit{0} \fuse \inc$ has two $\inc$s in flight, and they give rise to independent rewrites.
% \begin{center}
%   \begin{tikzpicture}
%     \matrix [matrix of math nodes, column sep = 1.5em]
%     {
%       % First row
%       & |(inc-1-2)| \eps \fuse \bit{1} \fuse \mathul{\bit{0} \fuse \inc} & \\
%       % Second row
%       |(inc-2-1)| \mathul{\eps \fuse \inc} \fuse \mathul{\bit{0} \fuse \inc}
%         && |(inc-2-3)| \eps \fuse \bit{1} \fuse \bit{1} \\
%       % Third row
%       & |(inc-3-2)| \mathul{\eps \fuse \inc} \fuse \bit{1} & \\
%     };

%     \begin{scope}
%     [ start chain, every join/.style={->} ]
%       \chainin (inc-2-1);
%       \begin{scope}[start branch=inc-1-2]
%         \chainin (inc-1-2) [join];
%       \end{scope}
%       \begin{scope}[start branch=inc-3-2]
%         \chainin (inc-3-2) [join];
%       \end{scope}
%       \chainin (inc-2-3) [join = with inc-1-2, join = with inc-3-2];
%     \end{scope}
%   \end{tikzpicture}
% \end{center}


\subsection{Committed choice}\label{sec:committed-choice}

% In our forward-chaining ordered logic programming language, we assume a \vocab{committed-choice} semantics, meaning that when multiple rewritings are possible at a given site the choice is never reconsidered.
In forward-chaining ordered logic programming, we assume a \vocab{committed-choice} semantics, meaning that when several rewritings are possible at a given site, one is chosen and that choice is never reconsidered.

Committed choice does not clearly arise in the binary counter example;
instead, consider a single-player game in which the player wins if a
% each of two coin tosses land heads.
coin toss lands heads.
As an forward-chaining ordered logic program, this game is specified
\begin{align*}
  &\coin \lrimp \monad{\heads} \\
  &\coin \lrimp \monad{\tails} \\
  &\heads \lrimp \monad{\win} \\
  &\tails \lrimp \monad{\loss}
  \,.
\end{align*}
The first two clauses specify that a $\coin$ may land either $\heads$ or $\tails$, and the remaining clauses specify the winning condition.

% One way in which the starting string $\coin \fuse \coin$ of two coins can be maximally rewritten is
One way in which the starting string can be maximally rewritten is
\begin{equation*}
  \mathul{\coin}
    \trans \mathul{\tails}
    \trans \loss
    \ntrans
  \,.
\end{equation*}
Here the $\coin$ lands $\tails$, resulting in a $\loss$.
To get a $\win$, we'd like to somehow take back the toss and have it instead land $\heads$.
% Here the first $\coin$ lands $\tails$ and the second $\coin$ lands $\heads$, resulting in a $\loss$.
% To get a $\win$, we'd like to somehow take back the first toss and have it instead land $\heads$.
This violation of the game's rules is just what the committed-choice semantics proscribes: once the coin is tossed, we must commit to that toss's outcome.



\subsection{Example: Binary counter with decrements}\label{sec:exampl-binary-count-3}

Returning to the binary counter, its also possible \dots

To reiterate some of the points made above, \dots

It's also possible to add support for decrements to the above ordered logic program.
Like increments, a decrement instruction is represented by a $\dec$ atom at the counter's least significant end.
To perform the decrement, a $\dec$ begins propagating up the counter.
As it passes over any $\bit{0}$s at the least significant end, they are marked as $\bit[']{0}$s to indicate that they are waiting to borrow from their more significant neighbors:
\begin{equation*}
  \bit{0} \fuse \dec \lrimp \monad{\dec \fuse \bit[']{0}} \,.
\end{equation*}
Whenever it reaches the $\eps$ or right-most $\bit{1}$, the $\dec$ is replaced with either $\zero$ or $\suc$, respectively, to show whether the borrow was possible; in the case of $\bit{1}$, the borrow is also effected:
\begin{align*}
  &\eps \fuse \dec \lrimp \monad{\eps \fuse \zero} \\
  &\bit{1} \fuse \dec \lrimp \monad{\bit{0} \fuse \suc} \,.
\end{align*}
Then the $\zero$ or $\suc$ travels back over all of the $\bit[']{0}$s that were waiting to borrow.
In the case of $\zero$, the bits are returned to their original $\bit{0}$ state because no borrow was possible;
in the case of $\suc$, a borrow was performed and so the bits are set to $\bit{1}$:
\begin{align*}
  &\zero \fuse \bit[']{0} \lrimp \monad{\bit{0} \fuse \zero} \\
  &\suc \fuse \bit[']{0} \lrimp \monad{\bit{1} \fuse \suc} \,.
\end{align*}


% According to the following rewrite rules, a $\dec$ propogates up the counter past any $\bit{0}$s until it reaches an $\eps$ or the right-most $\bit{1}$.
% At this point, the $\dec$ is replaced with either $\zero$ or $\suc$, respectively.
% \begin{align*}
%   &\bit{0} \fuse \dec \lrimp \monad{\dec \fuse \bit[']{0}} \\
%   &\eps \fuse \dec \lrimp \monad{\eps \fuse \zero} \\
%   &\bit{1} \fuse \dec \lrimp \monad{\bit{0} \fuse \suc}
% \end{align*}
% Then the $\zero$ or $\suc$ traveles back down the counter
% \begin{align*}
%   &\zero \fuse \bit[']{0} \lrimp \monad{\bit{0} \fuse \zero} \\
%   &\suc \fuse \bit[']{0} \lrimp \monad{\bit{1} \fuse \suc}
% \end{align*}


For example, the counter $\eps \fuse \bit{1} \fuse \bit{0} \fuse \dec$ can be maximally rewritten as
\begin{align*}
  \MoveEqLeft[0.5]
  \eps \fuse \bit{1} \fuse \mathul{\bit{0} \fuse \dec} \\
    &\trans \eps \fuse \mathul{\bit{1} \fuse \dec} \fuse \bit[']{0} \\
    &\trans \eps \fuse \mathul{\bit{0} \fuse \suc} \fuse \bit[']{0} \\
    &\trans \eps \fuse \bit{0} \fuse \bit{1} \fuse \suc \\
    &\ntrans
\end{align*}
Once again, there are possibilities for concurrency.
For example, the following two traces are indistinguishable because they differ only in the order of independent rewrites:
\begin{align*}
  &\mathul{\eps \fuse \inc} \fuse \mathul{\bit{0} \fuse \dec} \trans \mathul{\eps \fuse \inc} \fuse \dec \fuse \bit[']{0} \trans \eps \fuse \mathul{\bit{1} \fuse \dec} \fuse \bit[']{0} \\
  %
  \shortintertext{and}
  %
  &\mathul{\eps \fuse \inc} \fuse \mathul{\bit{0} \fuse \dec} \trans \eps \fuse \bit{1} \fuse \mathul{\bit{0} \fuse \dec} \trans \eps \fuse \mathul{\bit{1} \fuse \dec} \fuse \bit[']{0}
\end{align*}
This justifies treating those two rewrites as concurrent.


% C ::= eps | C * bit0 | C * bit1 | C * inc


% c <- dec <- d =
% { case d of
%     eps => wait d;
%            d' <- eps;
%            c <- zero <- d'
%   | bit0 => d' <- dec <- d
%             c <- bit0' <- d'
%   | bit1 => d' <- bit0 <- d
%             c <- succ <- d' }

% c <- zero <- d =
% { case c of
%     bit0' => d' <- bit0 <- d
%              c <- zero <- d' }

% Cntr = +{ eps: 1 , bit0: Cntr , bit1: Cntr }
% Cntr' = &{ bit0': Cntr' }

% bit0 : {Cntr |- Cntr}
% dec : {Cntr |- Cntr'}
% zero : {Cntr |- Cntr'}
% bit0' : {Cntr' |- Cntr'}


\begin{align*}
  &\eps \fuse \dec \lrimp \monad{\eps \fuse \zero} \\
  &\bit{0} \fuse \dec \lrimp \monad[auto]{
                               \dec \fuse \parens[auto, align=c@{\,}l]{
                                                & (\zero \limp \monad{\bit{0} \fuse \zero}) \\
                                          \with & (\suc \limp \monad{\bit{1} \fuse \suc})}} \\
  &\bit{1} \fuse \dec \lrimp \monad{\bit{0} \fuse \suc}
\end{align*}



\subsection{Adequacy and generative invariants}\label{sec:gener-invar}

Thus far, we have used ordered logic programs to specify concurrent systems, whereas the non-modal fragment of ordered logic was originally developed by \textcite{Lambek:AMM58} to describe sentence structure.
However, these two modes of use of ordered logic are not as different as they might first appear.

In our running example of an incrementable binary counter, the counter is represented as a string of $\bit{0}$, $\bit{1}$, and $\inc$ atoms terminated at the most significant end by an $\eps$.
More precisely, a string is a well-formed binary counter if it can be generated from the $\Cntr$ nonterminal by the context-free grammar
\begin{gather*}
  \Cntr ::= \eps \mid \Cntr \fuse \bit{0} \mid \Cntr \fuse \bit{1} \mid \Cntr \fuse \inc
  \,,
%
\intertext{which is an abbreviated notation for four distinct productions:}
%
  \begin{aligned}
    &\Cntr \to \eps \\
    &\Cntr \to \Cntr \fuse \bit{0} \\
    &\Cntr \to \Cntr \fuse \bit{1} \\
    &\Cntr \to \Cntr \fuse \inc
    \,.
  \end{aligned}
\end{gather*}

Building on \citeauthor{Lambek:AMM58}'s work, the same context-free grammar can be described in ordered logic using \vocab{generative signatures}~\autocite{Simmons:CMU12}.
Each production in the grammar corresponds to a clause, with the $\Cntr$ nonterminal represented as the atomic proposition $\cntr$:
\begin{equation*}
  \begin{aligned}
    &\cntr \lrimp \monad{\eps} \\
    &\cntr \lrimp \monad{\cntr \fuse \bit{0}} \\
    &\cntr \lrimp \monad{\cntr \fuse \bit{1}} \\
    &\cntr \lrimp \monad{\cntr \fuse \inc}
    \,.
  \end{aligned}
\end{equation*}
Then, just as all well-formed binary counters are generated from the $\Cntr$ nonterminal according to the above productions, so are all binary counters generated as maximal rewritings of the $\cntr$ atom according to these clauses.
% Just as a string is well-formed binary counter if it can be generated from the $\Cntr$ nonterminal by the above context-free grammar, so too is a string well-formed if it can be generated by maximally rewriting the $\cntr$ atom.
For example, $\eps \fuse \bit{1} \fuse \inc$ is a well-formed binary counter because it is a maximal rewriting of $\cntr$:
\begin{equation*}
  \mathul{\cntr}
    \trans \mathul{\cntr} \fuse \inc
    \trans \mathul{\cntr} \fuse \bit{1} \fuse \inc
    \trans \eps \fuse \bit{1} \fuse \inc
    \ntrans
  \,.
\end{equation*}
Note that 

\begingroup
  \RenewPredicate{\cntr}[Cntr]{1}%
Generative signatures in fact generalize context-free grammars.
One example is to augment $\cntr{}$ with a natural number, effectively \wc{giving}[\st{creating}] a countably infinite family of nonterminals.
Thus, a binary counter is well-formed \emph{and} represents value $N$ if it is a maximal rewriting of $\cntr{N}$ according to the first-order ordered logic program
\begin{equation*}
  \begin{aligned}
    &\cntr{0} \lrimp \monad{\eps} \\
    &\cntr{(2N)} \lrimp \monad{\cntr{N} \fuse \bit{0}} \\
    &\cntr{(2N{+}1)} \lrimp \monad{\cntr{N} \fuse \bit{1}} \\
    &\cntr{(N{+}1)} \lrimp \monad{\cntr{N} \fuse \inc}
    \,.
  \end{aligned}
\end{equation*}

This generative signature allows us to formally state (and prove) adequacy of the incrementable binary counter program:
\begin{definition}
  A string is \vocab{quiescent} if $S \ntrans$.
  A string $S$ is a \vocab{well-formed counter} if $\cntr{} \trans+ S \ntrans$.
  A string $S$ \vocab{represents} natural number $N$ if $\cntr{N} \trans+ S \ntrans$.
\end{definition}

\begin{theorem}[Preservation]
  For every well-formed counter $S$ such that $S \trans S'$, the string $S'$ is also a well-formed counter.
\end{theorem}
\begin{theorem}[Adequacy of counters]
  \mbox{}
  \begin{enumerate}
  \item For every natural number $N$, there is a unique well-formed counter $S$ such that
%     \item $\cntr{} \trans+ S \ntrans$;
 $S$ represents $N$ and % $\cntr{N} \trans+ S \ntrans$
 $S$ is quiescent.% $S \ntrans$.
%  \item For every string $S$ such that $\cntr{} \trans+ S \ntrans$, there is a unique natural number $N$ such that $\cntr{N} \trans+ S \ntrans$.
  \item For every well-formed counter $S$, there is a unique natural number $N$ such that $S$ represents $N$.
  \end{enumerate}
\end{theorem}
\begin{theorem}[Adequacy of $\inc$]\mbox{}
  \begin{enumerate}
  \item If $N + 1 = N'$, then there exist well-formed counters $S$ and unique $S'$ such that string $S$ represents $N$, string $S'$ represents $N'$, and $S \fuse \inc \trans+ S' \ntrans$.
  \item For all well-formed counters $S$ and $S'$ such that $S \fuse \inc \trans+ S' \ntrans$, there exist unique natural numbers $N$ and $N'$ such that string $S$ represents $N$, string $S'$ represents $N'$, and $N + 1 = N'$.
  \end{enumerate}
\end{theorem}

\begin{theorem}[Adequacy of $\inc$]
  $\cntr{N} \trans+ S \ntrans$ if and only if $S \fuse \inc \trans+ S' \ntrans$ and $\cntr{(N{+}1)} \trans+ S' \ntrans$.
  \begin{itemize}
  \item If $\cntr{N} \trans+ S \ntrans$ and $S \fuse \inc \trans+ S' \ntrans$, then $\cntr{(N{+}1)} \trans+ S' \ntrans$.
  \item If $\cntr{N} \trans+ S \ntrans$, then $S \fuse \inc \trans+ S' \ntrans$.
  \end{itemize}
\end{theorem}
\endgroup

% stype C = &{ inc: C, dec: C', halt: X }
%   and C' = +{ zero: C, succ: C }
% eps : {C <- X}, bit0 : {C <- C}, bit1 : {C <- C}, inc : {C <- C},
% dec : {C' <- C}, zero : {C' <- C}, succ : {C' <- C}, bit0' : {C' <- C'},
% halt : {X <- C}
% 
%  C ::= X * eps | C * bit0 | C * bit1 | C * inc
% C' ::= C * dec | C * zero | C * succ | C' * bit0'
%  X ::= C * halt

% stype C = &{ inc: C, dec: C' }
%   and C' = +{ zero: C-, succ: C- }
% bit0 : {C- <- C-} /\ {C <- C}, ...
% 
% C- ::= eps | C- * bit0 | C- * bit1
%  C ::= eps | C * bit0 | C * bit1 | C * inc
% C' ::= C * dec | C- * zero | C- * succ | C' * bit0'




% \subsubsection{String rewriting rules as ordered implications.}

% Using a focused proof search strategy~\autocite{Andreoli:JLC92}, ordered implications correspond to string rewriting rules.


% \subsection{Example: Binary counter}\label{sec:exampl-binary-count-4}

% As a running example, we can implement a binary counter that supports increments.
% The counter is represented as a string of $\bit{0}$ and $\bit{1}$ letters terminated at the most significant end by an $\eps$.
% So, for instance, the ordered conjunction, or string, $\eps \fuse \bit{1} \fuse \bit{0}$ represents a counter with value $2$.
% Also interspersed are $\inc$ atoms, each of which serves as an increment instruction sent to the counter given by the more significant bits.
% Thus, $\eps \fuse \bit{1} \fuse \inc$ represents a counter with value $1$ that has been sent an increment instruction.

% Operationally, increments are described by three ordered implications that correspond to string rewriting rules; the first of these is
% \begin{equation*}
%   \bit{1} \fuse \inc \rimp \monad{\inc \fuse \bit{0}} \,.
% \end{equation*}
% By rewriting the string $\bit{1} \fuse \inc$ as $\inc \fuse \bit{0}$, this rule carries the $\inc$ up past any $\bit{1}$s at the counter's least significant end.
% Whenever the carried $\inc$ reaches the $\eps$ or right-most $\bit{0}$, the carry is resolved:
% \begin{align*}
%   &\eps \fuse \inc \lrimp \monad{\eps \fuse \bit{1}} \\
%   &\bit{0} \fuse \inc \lrimp \monad{\bit{1}} \,.
% \end{align*}
% By rewriting $\eps \fuse \inc$ as $\eps \fuse \bit{1}$, the second rule ensures that the carry becomes a new most significant $\bit{1}$ in the $\eps$ case.
% By rewriting $\bit{0} \fuse \inc$ as $\bit{1}$, the third rule ensures that the carry flips the $\bit{0}$ to $\bit{1}$ in that case.



% When this proposition is part of the persistent context $\uctx$, the following rule is derivable:
% \begin{equation*}
%   \infer{\uctx ; \omatch{\bit{1}, \inc} \seq J}{
%     \uctx ; \ofill{\inc, \bit{0}} \seq J}
%   \,.
% \end{equation*}
% Read bottom-up, this derived rule rewrites part of the ordered context so that $\bit{1}, \inc$ becomes $\inc, \bit{0}$.


% % \begin{equation*}
% %   \infer[\lab{copy}]{\uctx ; \omatch{\bit{1}, \inc} \seq J}{
% %     \infer[\llab{{\rimp}}]{\uctx ; \ofill{\bit{1}, \inc, (\bit{1} \fuse \inc \rimp \inc \fuse \bit{0})} \seq J}{
% %       \infer[\rlab{{\fuse}}]{\uctx ; \bit{1}, \inc \seq \bit{1} \fuse \inc}{
% %         \infer[\lab{id}]{\uctx ; \bit{1} \seq \bit{1}}{
% %           } &
% %         \infer[\lab{id}]{\uctx ; \inc \seq \inc}{
% %           }} &
% %       \infer[\llab{{\fuse}}]{\uctx ; \ofill{\inc \fuse \bit{0}} \seq J}{
% %         \uctx ; \ofill{\inc, \bit{0}} \seq J}}}
% % \end{equation*}




% \subsection{Example: Binary counter}\label{sec:exampl-binary-count-2}

% As an example of an ordered logic program, we can implement a binary counter that supports increments.
% Similarly to the process implementation from \cref{sec:exampl-binary-count}, the counter is represented as a list of $\bit{0}$ and $\bit{1}$s terminated at the most significant end by an $\eps$.
% Here, however, the $\bit{}$s and $\eps$ are not processes, but rather atomic propositions (or, in string rewriting terminology, letters).
% For instance, the ordered conjunction (or string) $\eps \fuse \bit{1} \fuse \bit{0}$ represents a counter with value $2$.

% % An increment instruction is represented by an $\inc$ atom at the counter's least significant end.
% % There are three rewrite rules that describe the increment operation:
% % \begin{align*}
% %   &\eps \fuse \inc \lrimp \monad{\eps \fuse \bit{1}} \\
% %   &\bit{0} \fuse \inc \lrimp \monad{\bit{1}} \\
% %   &\bit{1} \fuse \inc \lrimp \monad{\inc \fuse \bit{0}}
% % \end{align*}
% % By rewriting $\eps \fuse \inc$ as $\eps \fuse \bit{1}$, the first rule introduces $\bit{1}$ as a new most significant bit, and thereby serves to increment an $\eps$.
% % % By rewriting $\eps \fuse \inc$ as $\eps \fuse \bit{1}$ and thereby introducing $\bit{1}$ as a new most significant bit, the first rule serves to increment $\eps$s.
% % Likewise, the second rule serves to increment a counter whose least significant bit is $\bit{0}$, by rewriting $\bit{0} \fuse \inc$ as $\bit{1}$ and thereby flipping $\bit{0}$.
% % % Likewise, by rewriting $\bit{0} \fuse \inc$ as $\bit{1}$ and thereby flipping $\bit{0}$, the second rule serves to increment a counter whose least significant bit is $\bit{0}$.
% % Finally, by rewriting $\bit{1} \fuse \inc$ as $\inc \fuse \bit{0}$, the third rule flips $\bit{1}$ and propogates a carry to the more significant bits, thereby serving to increment a counter whose least significant bit is $\bit{1}$.

% An increment instruction is represented by an $\inc$ atom at the counter's least significant end.
% There are three rewrite rules that describe increments, the first of which is
% \begin{equation*}
%   \bit{1} \fuse \inc \lrimp \monad{\inc \fuse \bit{0}} \,.
% \end{equation*}
% By rewriting $\bit{1} \fuse \inc$ as $\inc \fuse \bit{0}$, this rule carries the $\inc$ up past any $\bit{1}$s at the counter's least significant end.
% Whenever the carried $\inc$ reaches the $\eps$ or right-most $\bit{0}$, the carry is resolved:
% \begin{align*}
%   &\eps \fuse \inc \lrimp \monad{\eps \fuse \bit{1}} \\
%   &\bit{0} \fuse \inc \lrimp \monad{\bit{1}} \,.
% \end{align*}
% By rewriting $\eps \fuse \inc$ as $\eps \fuse \bit{1}$, the second rule ensures that the carry becomes a new most significant $\bit{1}$ in the $\eps$ case.
% By rewriting $\bit{0} \fuse \inc$ as $\bit{1}$, the third rule ensures that the carry flips the $\bit{0}$ to $\bit{1}$ in that case.

\end{document}

%%% Local Variables:
%%% TeX-master: "ordered-lp"
%%% End:


% arara: pdflatex
% arara: biber
% arara: pdflatex
% arara: pdflatex
\documentclass[
  class=../hdeyoung-proposal,
  crop=false
]{standalone}

\usepackage{ordered-logic}
\usepackage{basic-atoms}
\usepackage{ordered-lp-terms}
\usepackage{proof}
\usepackage{mathpartir}

\NewDocumentEnvironment{infers}{o O{}}%
  {%
    \noindent\IfValueT{#1}{\fbox{#1}#2}%
    \begin{mathpar}\ignorespaces
  }%
  {\end{mathpar}\ignorespacesafterend}

\NewDocumentCommand{\irlabel}{m m o}%
  {{#2}\text{\textsc{#1}}\IfValueT{#3}{_{#3}}}
\NewDocumentCommand{\rlab}{m o}%
  {\IfValueTF{#2}{\irlabel{r}{#1}[#2]}{\irlabel{r}{#1}}}
\NewDocumentCommand{\llab}{m o}%
  {\IfValueTF{#2}{\irlabel{l}{#1}[#2]}{\irlabel{l}{#1}}}

\NewDocumentCommand{\tctx}{}{\Psi}
\NewDocumentCommand{\tctxe}{}{\cdot}

% \NewDocumentCommand{\trans}{t* t+ o}{%
%   \longrightarrow
%   \IfBooleanT{#1}{^*}\IfBooleanT{#2}{^+}%
%   \IfValueT{#3}{_{#3}}%
% }
% \NewDocumentCommand{\ntrans}{}{
%   \longarrownot\trans
% }

\begin{document}

\subsection{Technical details}\label{sec:ordered-lp:technical}

The previous \lcnamecrefs{sec:??} have hopefully served to provide intuition for forward-chaining ordered logic programming.
In this \lcnamecref{sec:technical-details}, we review the technical details of this framework; 
we generally follow the lead of \citeauthor{Simmons:CMU12}'s SLS framework~\autocite*{Simmons:CMU12}.

\subsubsection{Propositions, terms, and traces}\label{sec:props-terms-traces}

\paragraph{Propositions.}\label{sec:propositions}

Propositions are polarized into \vocab{negative} and \vocab{positive} classes:
\begin{alignat*}{2}
  &\text{Negative propositions}\quad & A^- &::= \forall a{:}\tau. A^- \mid A^+ \rimp B^- \mid A^+ \limp B^- \mid A^- \with B^- \mid \monad{A^+} \\
  &\text{Positive propositions}      & A^+ &::= \exists a{:}\tau. A^+ \mid A^+ \fuse B^+ \mid \one \mid \p^+ \mid A^-
\end{alignat*}
The negative propositions, $A^-$, are those whose right rules are invertible, whereas the positive propositions, $A^+$, are those whose left rules are invertible.
As an example, the ordered implications $A^+ \rimp B^-$ and $A^+ \limp B^-$ are classified as negative because the $\rlab{\rimp}$ and $\rlab{\limp}$ rules are invertible.
% For the fragment in which we are interested, the grammar of polarized propositions is

A few additional comments on the polarized propositions are worthwhile.
First, a lax modality, $\monad{A^+}$, allows positive propositions to be treated as negative ones, while still maintaining an explicit separation of the classes.
% The lax modality gives logical force to 
The lax modality will be responsible for typing traces (\cref{??}).
Second, negative propositions can similiarly be treated as positive ones via the implicit inclusion $A^-$.
Positive atomic propositions $\p^+$ stand in for arbitrary positive propositions. 

\paragraph{Contexts.}\label{sec:contexts}

\begin{alignat*}{2}
  &\text{Ordered contexts}\quad & \octx &::= \octxe \mid x{:}\susp+{\p^+} \mid x{:}A^- \mid \octx_1, \octx_2
\end{alignat*}


\paragraph{Terms.}\label{sec:terms}

As previously alluded, a focused proof-search strategy~\autocite{Andreoli:JLC92} forms the basis of ordered logic programming.
In this focused calculus, there are several forms of sequent%
%---inversion sequents, right-focused sequents, and left-focused sequents---
, each corresponding to a syntactic class of proof terms.

\begin{figure}[!tbp]
  \begin{infers}[Inversion: $\tctx ; \octx \seq \nof{N : A^-}$][ with $\octx$ stable]
    \infer[\rlab{\forall}]{\tctx ; \octx \seq \nof{\tlam{a.N} : \forall a{:}\tau.A^-}}{
      \tctx, a{:}\tau ; \octx \seq \nof{N : A^-}}
    \and
    \infer[\rlab{\rimp}]{\tctx ; \octx \seq \nof{\rlam{p.N} : A^+ \rimp B^-}}{
      \tctx_{A^+} ; \octx_{A^+} \pseq \pof{p : A^+} &
      \tctx, \tctx_{A^+} ; \octx, \octx_{A^+} \seq \nof{N : B^-}}
    \and
    \infer[\rlab{\limp}]{\tctx ; \octx \seq \nof{\llam{p.N} : A^+ \limp B^-}}{
      \tctx_{A^+} ; \octx_{A^+} \pseq \pof{p : A^+} &
      \tctx, \tctx_{A^+} ; \octx_{A^+}, \octx \seq \nof{N : B^-}}
    \and
    \infer[\rlab{\with}]{\tctx ; \octx \seq \nof{\pair{N_1 , N_2} : A^-_1 \with A^-_2}}{
      \tctx ; \octx \seq \nof{N_1 : A^-_1} &
      \tctx ; \octx \seq \nof{N_2 : A^-_2}}
    \and
    \infer[\rlab{\monad{}}]{\tctx ; \octx \seq \nof{\lett{T in V} : \monad{A^+}}}{
      \tof{T :: (\tctx ; \octx) \trans* (\tctx' ; \octx')} &
      \tctx' ; \octx' \seq \vof{V : [A^+]}}
  \end{infers}
  \caption{Normal terms\label{fig:normal-terms}}
\end{figure}

Normal terms $N$ are typed by inversion sequents, $\tctx ; \octx \seq \nof{N : A^-}$ (\cref{fig:normal-terms}).
Inversion proceeds by eagerly applying right rules, which are invertible because the succedent $A^-$ is of negative polarity.
Rather than \enquote{inlining} invertible left rules along the derivation's trunk, the focused calculus maintains the invariant that the ordered context $\octx$ in an inversion sequent $\tctx ; \octx \seq \nof{N : A^-}$ is stable.
% the rules for inversion sequents maintain the invariant that the ordered context $\octx$ is stable.
It's for this reason that the lambda abstractions $\rlam{p.N}$ and $\llam{p.N}$ bind patterns $p$ instead of single variables---the patterns ensure that the argument is fully left-inverted.
% Among the normal terms are the lambda abstractions $\rlam{p.N}$ and $\llam{p.N}$, which are typed by the right rules for the ordered implications.
% These abstractions bind patterns, $p$, instead of single variables because, rather than \enquote{inlining} left inversion along the derivation's trunk, this judgment's rules maintain the invariant that the ordered context $\octx$ is stable.

Eventually, inversion terminates with a lax modality, $\monad{A^+}$, as the succedent.
Forward-chaining to construct a trace $T$ begins here.

\begin{figure}
  \begin{infers}[$\tof{T :: (\tctx ; \octx) \trans* (\tctx' ; \octx')}$]
    \infer{\tof{\tnil :: (\tctx ; \octx) \trans* (\tctx ; \octx)}}{
      }
    \and
    \infer{\tof{\tstep{p <- R; T} :: (\tctx ; \omatch{\octx}) \trans* (\tctx' ; \octx')}}{
      \tctx ; \octx \seq \aof{R : \susp-{\monad{A^+}}} &
      \tctx_{A^+} ; \octx_{A^+} \pseq \pof{p : A^+} &
      \tof{T :: (\tctx, \tctx_{A^+} ; \ofill{\octx_{A^+}}) \trans* (\tctx' ; \octx')}}
  \end{infers}
  \caption{Traces\label{fig:traces}}
\end{figure}

Traces $T$ are possibly empty sequences of steps $\tstep{p <- R}$.



\begin{figure}
  \begin{infers}[Stable: $\tctx ; \octx \seq \aof{R : \susp-{C^-}}$]
    \infer{\tctx ; \omatch{x{:}A^-} \seq \aof{\atm{x . S} : \susp-{C^-}}}{
      \tctx ; \ofill{\lfoc{A^-}} \seq \sof{S : \susp-{C^-}}}
    \and
    \infer{\tctx ; \omatch{\octxe} \seq \aof{\atm{c . S} : \susp-{C^-}}}{
      c{:}A^- \in \sig &
      \tctx ; \ofill{\lfoc{A^-}} \seq \sof{S : \susp-{C^-}}}    
  \end{infers}
  \caption{Atomic terms\label{fig:atomic-terms}}
\end{figure}

Rather than introducing terms for individual left-invertible rules

\begin{figure}[!t]
  \begin{infers}[Left-focus: $\tctx ; \omatch{\lfoc{A^-}} \seq \sof{S : \susp-{C^-}}$]
    \infer{\tctx ; \omatch{\lfoc{A^-}} \seq \sof{\snil : \susp-{A^-}}}{
    }
    \and
    \infer{\tctx ; \omatch{\lfoc{\forall a{:}\tau.A^-}} \seq \sof{\tapp{t ; S} : \susp-{C^-}}}{
      \tctx \seq t : \tau &
      \tctx ; \ofill{\lfoc{\subst{t/a}{A^-}}} \seq \sof{S : \susp-{C^-}}}
    \and
    \infer{\tctx ; \omatch{\lfoc{A^+ \rimp B^-}, \octx} \seq \sof{\rapp{V ; S} : \susp-{C^-}}}{
      \tctx ; \octx \seq \vof{V : \rfoc{A^+}} &
      \tctx ; \ofill{\lfoc{B^-}} \seq \sof{S : \susp-{C^-}}}
    \and
    \infer{\tctx ; \omatch{\octx, \lfoc{A^+ \limp B^-}} \seq \sof{\lapp{V ; S} : \susp-{C^-}}}{
      \tctx ; \octx \seq \vof{V : \rfoc{A^+}} &
      \tctx ; \ofill{\lfoc{B^-}} \seq \sof{S : \susp-{C^-}}}
    \and
    \infer{\tctx ; \omatch{\lfoc{A^-_1 \with A^-_2}} \seq \sof{\fst{S} : \susp-{C^-}}}{
      \tctx ; \ofill{\lfoc{A^-_1}} \seq \sof{S : \susp-{C^-}}}
    \and
    \infer{\tctx ; \omatch{\lfoc{A^-_1 \with A^-_2}} \seq \sof{\snd{S} : \susp-{C^-}}}{
      \tctx ; \ofill{\lfoc{A^-_2}} \seq \sof{S : \susp-{C^-}}}    
  \end{infers}
  \caption{Spines}
\end{figure}

\begin{figure}
  \begin{infers}[Right-focus: $\tctx ; \octx \seq \vof{V : \rfoc{A^+}}$]
    \infer{\tctx ; x{:}\susp+{\p^+} \seq \vof{x : \rfoc{\p^+}}}{
    }
    \and
    \infer{\tctx ; \octx_1, \octx_2 \seq \vof{\vfuse{V_1}{V_2} : \rfoc{A^+_1 \fuse A^+_2}}}{
      \tctx ; \octx_1 \seq \vof{V_1 : \rfoc{A^+_1}} &
      \tctx ; \octx_2 \seq \vof{V_2 : \rfoc{A^+_2}}}
    \and
    \infer{\tctx ; \octxe \seq \vof{\vone : \rfoc{\one}}}{
    }
    \and
    \infer{\tctx ; \octx \seq \vof{\vexists{t.V} : \rfoc{\exists a{:}\tau.A^+}}}{
      \tctx \seq t : \tau &
      \tctx ; \octx \seq \vof{V : \rfoc{\subst{t/a}{A^+}}}}
    \and
    \infer{\tctx ; \octx \seq \vof{N : \rfoc{A^-}}}{
      \tctx ; \octx \seq \nof{N : A^-}}
  \end{infers}
  \caption{Values}
\end{figure}

\begin{figure}
  \begin{infers}[Pattern: $\tctx ; \octx \pseq \pof{p : A^+}$]
    \infer{\tctxe ; x{:}\susp+{\p^+} \pseq \pof{x : \p^+}}{
    }
    \and
    \infer{\tctx_1, \tctx_2 ; \octx_1, \octx_2 \pseq \pof{\pfuse{p_1}{p_2} : A^+_1 \fuse A^+_2}}{
      \tctx_1 ; \octx_1 \pseq \pof{p_1 : A^+_1} &
      \tctx_2 ; \octx_2 \pseq \pof{p_2 : A^+_2}}
    \and
    \infer{\tctxe ; \octxe \pseq \pof{\pone : \one}}{
    }
    \and
    \infer{\tctx, a{:}\tau ; \octx \pseq \pof{\pexists{a.p} : \exists a{:}\tau.A^+}}{
      \tctx ; \octx \pseq \pof{p : A^+}}
    \and
    \infer{\tctxe ; x{:}A^- \pseq \pof{x : A^-}}{
    }
  \end{infers}
  \caption{Patterns}
\end{figure}

Traces $T$ are sequences of steps $\tstep{p <- R}$, where the atomic term $R$ specifies the transition's inputs and $p$ matchs the transition's outputs.   
\begin{alignat*}{2}
  &\text{Normal terms}\quad & N &::= \tlam{a.N} \mid \rlam{p.N} \mid \llam{p.N} \mid \pair{N_1, N_2} \mid \lett{T in V} \\
  &\text{Atomic terms}\quad & R &::= \atm{c . S} \mid \atm{x . S} \\
  &\text{Spines} & S &::= \tapp{t ; S} \mid \rapp{V ; S} \mid \lapp{V ; S} \mid \fst{S} \mid \snd{S} \mid \snil \\
  &\text{Values} & V &::= x \mid \vfuse{V_1}{V_2} \mid \vone \mid \vexists{t.V} \mid N \\
  &\text{Patterns} & p &::= x \mid \pfuse{p_1}{p_2} \mid \pone \mid \pexists{a.p} \\
  &\text{Traces} & T &::= \tnil \mid \tstep{p <- R; T}
\end{alignat*}



\subsubsection{Concurrent equality}\label{sec:concurrent-equality}






\subsubsection{Fairness}\label{sec:fairness}

\NewDocumentCommand{\headpath}{m}{(#1)^p}
A trace $\tof{T :: \octx_0 \trans[\omega][]}$ is unfair if there exist $i$ and $R'_i, R'_{i+1}, \dotsc$ such that:
\begin{enumerate}[label=\alph*., ref=\alph*]
\item for all $j \geq i$, there exist $p'_j$ and $\octx'_j$ such that $\tof{\tstep{p'_j <- R'_j} :: \octx_j \trans \octx'_j}$;
\item $\headpath{R'_j} = \headpath{R'_{j+1}}$ for all $j \geq i$;
\item $\headpath{R_j} \neq \headpath{R'_j}$ for all $j \geq i$.
\end{enumerate}


\begin{equation*}
  \octx_j = \omatch[_j]{\octx_{R_j}}
\end{equation*}

\begin{equation*}
  \octx_{R_j} \seq \aof{R_j : \susp-{\monad{A^+_j}}}
\end{equation*}
\end{document}

%%% Local Variables:
%%% TeX-master: "ordered-lp"
%%% End:


\end{document}


\subsection{What counts as a choreography?}\label{sec:what-counts-choreo}

\begin{align*}
  &\inc \lrimp \inc[<-] \\
  &\eps \lrimp (\inc[<-] \rimp \eps \fuse \bit{1}) \\
  &\bit{0} \lrimp (\inc[<-] \rimp \bit{1}) \\
  &\bit{1} \lrimp (\inc[<-] \rimp \inc \fuse \bit{0})
\end{align*}

\begin{align*}
  &\eps \lrimp \eps[->] \\
  &\bit{0} \lrimp \bit{0}[->] \\
  &\bit{1} \lrimp \bit{1}[->] \\
  &\inc \lrimp \parens[auto, align=c@{\,}l]{
                     & (\eps[->] \limp \eps \fuse \bit{1}) \\[2pt]
               \with & (\bit{0}[->] \limp \bit{1}) \\[2pt]
               \with & (\bit{1}[->] \limp \inc \fuse \bit{0})}
\end{align*}

\NewDocumentCommand{\fch}{o m}{\IfValueTF{#1}{\monad[#1]}{\monad}{#2}}

\begin{align*}
  &\elem{M} \lrimp \elem[->]{M} \\
  &\elem{N} \lrimp (\elem[->]{M} \limp ((M > N) \uimp \fch{\elem{N} \fuse \elem{M}}))
\end{align*}

\begin{itemize}
\item Grammar of choreographies.
\item Each choreography rule must be able to fire independently of the other processes (although it may depend on messages).
\end{itemize}

%%% Local Variables:
%%% TeX-master: "proposal"
%%% End:


% % arara: lualatex
% % arara: lualatex
% % arara: biber
% % arara: lualatex
% % arara: lualatex
% % \documentclass{../hdeyoung-proposal}
% \documentclass[
%   class=../hdeyoung-proposal,
%   crop=false
% ]{standalone}


% \usepackage{linear-logic}
% \usepackage{ordered-logic}
% \usepackage{proof}
% \usepackage{mathpartir}

% \usepackage{tikz}
% \usetikzlibrary{shapes.misc,graphs,quotes,graphdrawing}
% \usegdlibrary{trees}

% \usepackage{scalerel}

% \ExplSyntaxOn

% % \DeclarePairedDelimiter \parens { \lparen } { \rparen }
% \DeclarePairedDelimiter \braced:wn { \lbrace } { \rbrace }
% \NewDocumentCommand{ \braced }{ s o m o }
%   {
%     \IfBooleanTF {#1}
%       { \braced:wn* {#3} }
%       {
%         \IfValueTF {#2}
%           { \braced:wn[#2] {#3} }
%           { \braced:wn {#3} }
%       }
%     \IfValueT {#4} { \sb{#4} }
%   }

% \DeclarePairedDelimiter \bagged:wn { \lbag } { \rbag }
% \NewDocumentCommand{ \bagged }{ s o m o }
%   {
%     \IfBooleanTF {#1}
%       { \bagged:wn* {#3} }
%       {
%         \IfValueTF {#2}
%           { \bagged:wn[#2] {#3} }
%           { \bagged:wn {#3} }
%       }
%     \IfValueT {#4} { \sb{#4} }
%   }


% \NewDocumentCommand \oseq { >{ \SplitArgument{1}{|-} } m }
%   { \oseq:nn #1 }
% \cs_new:Npn \oseq:nn #1#2 { \oseq_ctxs:n {#1} \vdash #2 }
% \cs_new:Npn \oseq_ctxs:n #1 {
%   \seq_set_split:Nnn \l_tmpa_seq {;} {#1}
%   \seq_use:Nn \l_tmpa_seq { \mathrel{;} }
% }

% \NewDocumentCommand \procof { m m } { #1 \dblcolon #2 }
% \NewDocumentCommand \hypof { m } { #1 }


% \NewDocumentCommand \cut { m } { \text{\textsc{\MakeLowercase{Cut}}}\sb{#1} }
% \NewDocumentCommand \id { m } { \text{\textsc{\MakeLowercase{Id}}}\sb{#1} }

% \NewDocumentCommand \comp { >{ \SplitArgument{1}{|} } m }
%   { \comp:nn #1 }
% \cs_new:Npn \comp:nn #1#2 { #1 \parallel #2 }

% \NewDocumentCommand \fwd {} { \mathord{\leftrightarrow} }


% \RenewDocumentCommand \with { s }
%   { \IfBooleanTF {#1} \with:n \with: }
% \cs_new:Npn \with:n #1 {
%   \mathord{\binampersand}
%   \braced {
%     \seq_set_split:Nnn \l_tmpa_seq {,} {#1}
%     \seq_use:Nn \l_tmpa_seq {,}
%   }
% }
% \cs_new:Npn \with: { \mathbin{\binampersand} }

% \NewDocumentCommand \ssor { s }
%   { \IfBooleanTF {#1} \ssor:n \ssor: }
% \cs_new:Npn \ssor:n #1 {
%   \mathord{\ssor:}
%   \braced {
%     \seq_set_split:Nnn \l_tmpa_seq {,} {#1}
%     \seq_use:Nn \l_tmpa_seq {,}
%   }
% }
% \cs_new:Npn \ssor: { \oplus }

% \NewDocumentCommand \caseR { s m o }
%   {
%     \IfBooleanTF {#1}
%       {
%         \IfValueTF {#3}
%           { \case:nNnn { \mathsf{caseR} } \parens {#2} { \sb{#3} } }
%           { \case:nNnn { \mathsf{caseR} } \parens {#2} {} }
%       }
%       { \case:nNnn { \mathsf{caseR} } \parens {#2} {} }
%   }
% \NewDocumentCommand \caseL { s m o }
%   {
%     \IfBooleanTF {#1}
%       {
%         \IfValueTF {#3}
%           { \case:nNnn { \mathsf{caseL} } \parens {#2} { \sb{#3} } }
%           { \case:nNnn { \mathsf{caseL} } \parens {#2} {} }
%       }
%       { \case:nNnn { \mathsf{caseL} } \parens {#2} {} }
%   }
% \cs_new:Npn \case:nNnn #1#2#3#4 {
%   #1#4 \mskip\thinmuskip
%   #2 {
%     \seq_set_split:Nnn \l_tmpa_seq {|} {#3}
%     \seq_clear:N \l_tmpb_seq
%     \seq_map_inline:Nn \l_tmpa_seq
%       { \seq_put_right:Nn \l_tmpb_seq { \case_branch:n {##1} } }
%     \seq_use:Nn \l_tmpb_seq { \talloblong }
%   }
% }
% \cs_new:Npn \case_branch:n #1 { \case_branch_aux:w #1 \q_stop }
% \cs_new:Npn \case_branch_aux:w #1 => #2 \q_stop {
%   #1 \Rightarrow #2
% }

% \NewDocumentCommand \selectL { >{ \SplitArgument{1}{;} } m }
%   { \select:nnn { \mathsf{selectL} } #1 }
% \NewDocumentCommand \selectR { >{ \SplitArgument{1}{;} } m }
%   { \select:nnn { \mathsf{selectR} } #1 }
% \cs_new:Npn \select:nnn #1#2#3 {
%   \!\mathord{}\mathop{#1} #2 ; #3
% }


% \NewDocumentCommand \inj { m } { \mathsf{in}\sb{#1} }

% \NewDocumentCommand \inl {} { \inj{ \mathsf{1} } }
% \NewDocumentCommand \inr {} { \inj{ \mathsf{2} } }


% \RenewDocumentCommand \one {} { \mathord { \mathbf{1} } }

% \NewDocumentCommand \closeR {} { \mathsf{closeR} }
% \NewDocumentCommand \waitL { m } { \mathsf{waitL} ; #1 }


% \NewDocumentCommand \rrule { o m } {
%   \IfValueTF {#1}
%     { \rrule:nn {#2} {#1} }
%     { \rrule:n {#2} }
% }
% \cs_new:Npn \rrule:nn #1#2 { {#1}\text{\textsc{\MakeLowercase{R}}}\sb{#2} }
% \cs_new:Npn \rrule:n #1 { {#1}\text{\textsc{\MakeLowercase{R}}} }

% \NewDocumentCommand \lrule { o m } {
%   \IfValueTF {#1}
%     { \lrule:nn {#2} {#1} }
%     { \lrule:n {#2} }
% }
% \cs_new:Npn \lrule:nn #1#2 { {#1}\text{\textsc{\MakeLowercase{L}}}\sb{#2} }
% \cs_new:Npn \lrule:n #1 { {#1}\text{\textsc{\MakeLowercase{L}}} }


% \NewDocumentCommand \bnd { >{\SplitArgument{1}{=}}m } { \bnd:nn #1 }
% \cs_new:Npn \bnd:nn #1#2 { \mathsf{bnd} \mskip\thinmuskip #1 \mskip\thinmuskip #2 }
% \NewDocumentCommand \exec { } { \mathsf{exec} \mskip\thinmuskip }
% \NewDocumentCommand \msgL { } { \mathsf{msgL} \mskip\thinmuskip }
% \NewDocumentCommand \msgR { } { \mathsf{msgR} \mskip\thinmuskip }
% \NewDocumentCommand \msgQ { } { \mathsf{msgQ} }

% \ExplSyntaxOff




% \addbibresource{../proposal.bib}

% \NewDocumentCommand{\ie}{}{i.e.}


% \usepackage{listings}
% \crefname{listing}{listing}{listings}
% \Crefname{listing}{Listing}{Listings}

% \newlength{\mywidth}
% \settowidth{\mywidth}{\ttfamily A}
% \lstset{basicstyle=\ttfamily, basewidth=\mywidth}

% \captionsetup[lstlisting]{%
%   box=colorbox, boxcolor=gray,
%   font={normalfont, sf, color=white},
%   labelfont=bf,
%   justification=justified, singlelinecheck=false
% }

% \lstnewenvironment{sillcode}[1][]
%   {\lstset{language={},frame=bottomline,framerule=0.8ex,rulecolor=\color{gray},float,#1}}%
%   {}

% \lstnewenvironment{sillcode*}[1][]
%   {\lstset{language={},#1}}%
%   {}

% \NewDocumentCommand{\sillinline}{o}{%
%   \IfValueTF{#1}{\lstinline[#1]}{\lstinline}%
% }


% \NewDocumentCommand{\pctx}{}{\Psi}
% \ExplSyntaxOn
% \NewDocumentCommand{\ctxmonad}{>{\SplitArgument{1}{<-}}m}{
%   \{\use_ii:nn #1 \vdash \use_i:nn #1\}
% }
% \NewDocumentCommand \spawn { >{ \SplitArgument{1}{;} } m } { \spawn:nn #1 }
% \cs_new:Npn \spawn:nn #1#2 {
%   \mathsf{spawn}
%   \tl_if_empty:nF {#1} {
%     \mskip\thinmuskip #1 ; #2
%   }
% }
% \NewDocumentCommand{\mbind}{>{\SplitArgument{1}{;}}m}{
%   \use_i:nn#1 ; \use_ii:nn#1
% }
% \NewDocumentCommand{\mletrec}{o m m}{
%   \mathsf{letrec}\IfValueT{#1}{\sb{#1}}
%   \mskip\thinmuskip
%   #2
%   \mskip\thinmuskip
%   \mathsf{in} \mskip\thinmuskip #3
% }
% \NewDocumentCommand{\mprocdef}{m}{
%   #1
% }
% \ExplSyntaxOff


% \newcommand*\kay{k}



% \DeclareAcronym{BHK}{
  short = BHK,
  long  = Brouwer-Heyting-Kolmogorov
}

\DeclareAcronym{JILL}{
  short = JILL,
  long  = judgmental intuitionistic linear logic
}

\DeclareAcronym{ILL}{
  short = ILL,
  long  = intuitionistic linear logic
}

\DeclareAcronym{SML}{
  short = SML,
  long  = Standard ML
}

\DeclareAcronym{SOS}{
  short = SOS,
  short-indefinite = an,
  long = structural operational semantics
}

\DeclareAcronym{SSOS}{
  short = SSOS,
  short-indefinite = an,
  long = substructural operational semantics
}

\DeclareAcronym{SILL}{
  short = SILL,
  long = session-typed intuitionistic linear logic
}

\DeclareAcronym{CLF}{
  short = CLF,
  long = the concurrent logical framework
}



% \begin{document}

\section{Session-typed processes from singleton linear logic}\label{sec:sill}

Thus far, we have followed a proof-construction approach to computation, having in \cref{sec:ordered-lp} reviewed an interpretation of ordered logical specifications as concurrent string rewriting and in \cref{sec:choreographies} identified a fragment in which those specifications have a message-passing character.
In this \lcnamecref{sec:sill}, we turn to a proof-reduction view of computation.

Recently, \textcite{Caires+Pfenning:CONCUR10} with Toninho~\autocite*{Caires+:MSCS13} have established a Curry--Howard isomorphism, dubbed \acs{SILL}, between the sequent calculus for intuitionistic linear logic and a session-typed $\pi$-calculus, in which propositions are session types, proofs are session-typed processes, and cut reductions are process reductions.\footnote{\Textcite{Wadler:JFP14} later developed a correspondence between classical linear logic and a session-typed $\pi$-calculus, but \citeauthor{Caires+:MSCS13}'s intuitionistic correspondence turns out to be better suited to our goals here, for reasons we will explain shortly.}
This gives a proof-reduction view of concurrency that differs, apparently substantially, from the proof-construction perspective.
But, by the end of this proposal document, we will have shown that the differences are not as substantial as they first appear.

In this section, we present a reformulation of \citeauthor{Caires+:MSCS13}'s \ac{SILL} for a restriction of intuitionistic linear logic, which we call singleton linear logic, that is a better fit for comparisons with ordered logical specifications.
% As hinted below and further justified in \cref{?}, \ac{SISLL} is a better fit for comparisons with ordered logical specifications.

% [[
% We begin our presentation of \ac{SISLL} with a review of the session-typing judgements of \citeauthor{Caires+:MSCS13}'s \ac{SILL}.
% ]]

\subsection{Toward singleton linear logic}

In a session-based model of concurrency, pairs of processes interact in well-defined sessions, with one process offering a service that its session partner uses.
Session types, pioneered by \textcite{Honda:CONCUR93}, describe the interaction protocol to which a process adheres when offering its service.
% the processes in that session must adhere.
When processes interact, the session type changes: one process now offers, and the other uses, the continuation of the initial service.
As shown by \textcite{Caires+:MSCS13}, the logical reading of session-based concurrency is linear logic exactly because it can express this change of state.

Because a process offers its service along a distinguished channel, the basic session-typing judgment of \citeauthor{Caires+:MSCS13}'s \ac{SILL} is $P :: x{:}A$, meaning \enquote{process $P$ offers a service of session type $A$ along channel $x$}.
However, $P$ itself may rely on services offered by yet other processes, and so, more generally, the \ac{SILL} session-typing judgment is a linear sequent annotated as
\begin{equation*}
  \underbrace{
    x_1{:}A_1 , x_2{:}A_2 , \dots , x_n{:}A_n
  }_{\textstyle \lctx}
  \vdash
  P :: x{:}A
  \quad
  \text{($n \geq 0$)}
  \,,
\end{equation*}
% where the channel names, $x_i$, are needed to unambiguously refer to hypotheses and the consequent.
meaning \enquote{Using services $A_i$ offered along channels $x_i$, the process $P$ offers service $A$ along channel $x$.}
(The channels $x_i$ and $x$ must all be distinct and are binding occurrences with scope over the process $P$.)

In \ac{SILL}, the linear sequent calculus's inference rules thus become session-typing rules for processes.
Just as the inference rules arrange sequents into a proof tree, so do the \ac{SILL} session-typing rules arrange processes into a tree-shaped network in which some processes are clients of more than one process (i.e., some nodes have more than one child).
The following is one such example.
% The way in which \ac{SILL} processes use and offer services along distinct channels thus arranges processes in a tree-shaped network, such as the following, in which some processes are clients of more than one process (i.e., some nodes have more than one child).
\begin{equation*}
  \begin{tikzpicture}[channel/.style = {text depth=0, midway, sloped, above}]
    \graph [
      tree layout, grow=left, math nodes, % empty nodes,
      nodes={
        rounded rectangle, rounded rectangle left arc=none,
        draw, minimum size=3ex,
      },
      edges={-},
    ] {
      / [draw=none] <-["$\scriptstyle x$"' channel]
      P <- { / [> {"$\scriptstyle x_1$"' channel}] <- / <- { / , / } ,
             / [> {"$\scriptstyle x_2$"' channel}] ,
             / [> {"$\scriptstyle x_3$"' channel}] <- { / , / <- / } };
    };
  \end{tikzpicture}
\end{equation*}
% Each edge in the above \ac{SILL} tree represents a top-level $\cut{}$, and each top-level $\cut{}$ corresponds to a hypothesis that serves as its principal formula.

% With this session-typing judgment, the inference rules of the linear sequent calculus become \ac{SILL} session-typing rules for concurrent processes.
% For instance, the cut rule of \ac{SILL} types a parallel composition of processes (shown in $\pi$-calculus syntax here):
% \begin{equation*}
%   \infer[\cut{A}]{\lctx , \lctx' \vdash (\nu x)(P \mid Q) :: z{:}C}{
%     \lctx \vdash P :: x{:}A &
%     \lctx' , x{:}A \vdash Q :: z{:}C}
% \end{equation*}
% Top-level $\cut{}$s thus arrange processes in a tree-shaped network, such as the following, in which some processes are clients of more than one process (i.e., some nodes have more than one child).
% \begin{equation*}
%   % \begin{tikzpicture}
%   %   \graph [
%   %     tree layout, grow=left, math nodes, % empty nodes,
%   %     nodes={
%   %       rounded rectangle, rounded rectangle left arc=none,
%   %       draw, minimum size=2ex,
%   %     }
%   %   ] {
%   %     / [draw=none] <-
%   %     Q <- { "P_1" <- / <- { / , / } ,
%   %            / ,
%   %            / <- { / , / <- / } };
%   %   };
%   % \end{tikzpicture}
% \end{equation*}
% Each edge in the above \ac{SILL} tree represents a top-level $\cut{}$, and each top-level $\cut{}$ corresponds to a hypothesis that serves as its principal formula.
% % Each hypothesis gives rise to a top-level $\cut{}$, represented as an edge in the above graph. 

% Recall that the goal of this thesis proposal is to establish a connection between ordered logical specifications and 

In this thesis proposal, we are interested in a restriction of \ac{SILL} that will match concurrent ordered logical specifications.
To match the ordered context of these specifications, we cannot directly use \ac{SILL}---at least not in its full generality.
Instead, we need a restriction of \ac{SILL} in which process networks are chains, not arbitrary trees.


One might expect that process chains should arise from \emph{ordered} logic.
But, taking into account the way that the structure of the \ac{SILL} session-typing judgment induces a tree-shape for process networks, it becomes apparent that process chains, in fact, arise when processes are restricted to use at most one service each---that is, when contexts $\lctx$ are restricted to be either empty or singletons.
The session-typing judgment becomes
\begin{equation*}
  \lctxe \vdash P :: x{:}A
  \quad\text{or}\quad
  x_1{:}A_1 \vdash P :: x{:}A
\end{equation*}
and the process networks are necessarily chains:
\begin{equation*}
  \begin{tikzpicture}[channel/.style = {text depth=0, midway, sloped, above}]
    \graph [
      tree layout, grow=left, math nodes, % empty nodes,
      nodes={
        rounded rectangle, rounded rectangle left arc=none,
        draw, minimum size=3ex,
      },
      edges={-},
    ] {
      / [draw=none] <-["$\scriptstyle x$"' channel]
      P <-
      / [> {"$\scriptstyle x_1$"' channel}] <-
      / <-
      / ;
    };
  \end{tikzpicture}
\end{equation*}

Having made this restriction, we can simplify the judgments: if there are at most two channels, they can always be unambiguously named \enquote{left} and \enquote{right}, rather than bothering with fresh names like $x_1$ and $x$.
Moreover, since the channel names are now fixed by position (rather like de~Bruijn indices), we may as well omit them altogether from the session-typing judgments:
\begin{equation*}
  \lctxe \vdash P :: A
  \quad\text{or}\quad
  A_1 \vdash P :: A
  \,.
\end{equation*}

The following \lcnamecrefs{sec:cut-as-composition} describe this restriction of \ac{SILL}.
% It's worth emphasizing that, even in the presence of the restriction to singleton antecedents, we still have a Curry--Howard isomorphism: although we choose to present the logical rules and process assignment simultaneously, singleton linear logic is indeed a well-defined logic in its own right.
It's worth emphasizing that although we choose to present the logical rules and process assignment simultaneously, singleton linear logic is indeed a well-defined logic in its own right: even in the presence of the restriction to singleton antecedents, we still have a Curry--Howard isomorphism.

% Finally, it's worth emphasizing that we still have a Curry--Howard isomorphism in the presence of the restriction to singleton antecedents: although we choose to present the logical rules and process assignment simultaneously, singleton linear logic is indeed a well-defined logic in its own right.


% This is a Curry--Howard isomophism in that the same restriction---contexts must be empty or singletons---still yields a well-defined logic when applied to linear logic.

% In this note, we are interested in developing a restriction of \ac{SILL} in which the process networks are linearly ordered:
% \begin{center}
%   % \begin{tikzpicture}
%   %   \graph [tree layout, grow=left, empty nodes, nodes={draw, circle}] {
%   %     / [draw=none] <-
%   %     x <- y <- z <- w;
%   %   };
%   % \end{tikzpicture}
% \end{center}

% Therefore, for the process network to be linearly ordered, contexts $\lctx$ must be either singletons or empty.
% In this special case, sequents have one of the simpler forms
% \begin{equation*}
%   x_0{:}A_0 \vdash P :: x_1{:}A_1
%   \quad\text{or}\quad
%   \lctxe \vdash P :: x_1{:}A_1
%   \,.
% \end{equation*}


% % exec(spawn P; Q) -> {exec P * exec Q}

\subsection{Cut as composition}\label{sec:cut-as-composition}

% Recall the $\cut{}$ rule for intuitionistic linear logic:
% \begin{equation*}
%   \infer[{\cut{A}}]{\oseq{\lctx, \lctx' |- \comp{x_0}{P_0 | P} :: C}}{
%     \oseq{\lctx |- \procof{P_0}{x_0{:}A}} &
%     \oseq{\lctx', x_0{:}A |- \procof{P}{x{:}C}}}
% \end{equation*}
% When restricted to empty or singleton antecedents, the $\cut{}$ rule becomes
% \begin{equation*}
%   \infer[\cut{A}]{\oseq{\lctx |- \comp{x_0}{P_0 | P} :: C}}{
%     \oseq{\lctx |- \procof{P_0}{x_0{:}A}} &
%     \oseq{x_0{:}A |- \procof{P}{x{:}C}}}
% \end{equation*}

The cut rule of intuitionistic linear logic composes a plan for obtaining resource $A$ with another plan that uses resource $A$:
\begin{equation*}
  \infer{\oseq{\lctx, \lctx' |- C}}{
    \oseq{\lctx |- A} &
    \oseq{\lctx', A |- C}}
  \,.
\end{equation*}

When antecedents are restricted to be either empty or singletons, that rule becomes the cut rule for singleton linear logic;
it retains the character of a composition:
% The cut rule of (intuitionistic) singleton linear logic restricts the antecedent of each sequent to be either empty or a singleton, but still retains the character of a composition:
\begin{equation*}
  \infer[\cut{A}]{\lseq{\lctx |- C}}{
    \lseq{\lctx |- A} &
    \lseq{A |- C}}
  \,.
\end{equation*}
Because proofs are to be processes, this suggests that the process interpretation of the cut rule should compose a process that offers service $A$ with another process that uses service $A$.
The $\cut{A}$ rule thus becomes a typing rule for process composition:
\begin{equation*}
  \infer[\cut{A}]{\lseq{\lctx |- \procof{\spawn{P; Q}}{C}}}{
    \lseq{\lctx |- \procof{P}{A}} &
    \lseq{A |- \procof{Q}{C}}}
  \,,
\end{equation*}
where $\spawn{P; Q}$ means \enquote{Spawn process $P$ to the left, and then continue as process $Q$.}
% The syntax is reminiscent of do notation for monadic computations, with the channel c being bound within c <- spawnP <- ;Qc.

To complete the description of $\spawn{}$, we must make its operational semantics precise.
Rather than using \iacl*{SOS}, it is convenient to describe the semantics as an ordered logical specification~\autocites{Pfenning:APLAS04}{Pfenning+Simmons:LICS09}, in the style known as \iacf{SSOS}.
In our \ac{SSOS}, we will use the atomic proposition $\exec{P}$ to represent an executing process $P$.
The rule for executing a $\spawn{}$ is
\begin{equation*}
  \exec{(\spawn{P; Q})} \lrimp \monad{\exec{P} \fuse \exec{Q}}
  \,.
\end{equation*}
Thus, to execute the $\spawn{}$, we execute the processes $P$ and $Q$ side-by-side, with process $P$ offering a service that $Q$ uses.


\subsection{Additive conjunction as branching}\label{sec:addit-conj-as-branch}

So far we have discussed only cut, a judgmental principle that applies to all services\footnote{The other judgmental  principle, identity, is postponed to \cref{sec:ident-as-forw} in the interest of presenting only what is necessary for a first, simple example.}; specific services are defined by the right and left rules of the logical connectives.

The singleton linear sequent calculus rules for the additive conjunction $A_1 \with A_2$ are as follows.
Other than the restriction to singleton or empty antecedents, these rules are the same as those from linear logic.
% Note that in this case the left rules' contexts are forced to be the singleton $A_1 \with A_2$.
\begin{mathpar}
  \infer[\rrule{\with}]{\lseq{\lctx |- A_1 \with A_2}}{
    \lseq{\lctx |- A_1} &
    \lseq{\lctx |- A_2}}
  \and
  \infer[{\lrule[1]{\with}}]{\lseq{A_1 \with A_2 |- C}}{
    \lseq{A_1 |- C}}
  \and
  \infer[{\lrule[2]{\with}}]{\lseq{A_1 \with A_2 |- C}}{
    \lseq{A_2 |- C}}
\end{mathpar}
The right rule, $\rrule{\with}$, says that to prove $A_1 \with A_2$ we must prove both $A_1$ and $A_2$ (using the same resource $\lctx$) so that we are prepared for whichever of the two resources, $A_1$ or $A_2$, is eventually chosen by a $\lrule[1]{\with}$ or $\lrule[2]{\with}$ left rule.

Correspondingly, a process that offers service $A_1 \with A_2$ gives its client a choice of services $A_1$ and $A_2$; the process must be prepared to offer whichever service the client chooses.
Based on this intuition, we interpret the $\rrule{\with}$ rule as typing a binary guarded choice:
\begin{equation*}
  \infer[\rrule{\with}]{\lseq{\lctx |- \procof{\caseR{\inl => P_1 | \inr => P_2}}{A_1 \with A_2}}}{
    \lseq{\lctx |- \procof{P_1}{A_1}} &
    \lseq{\lctx |- \procof{P_2}{A_2}}}
\end{equation*}
where $\caseR{\inl => P_1 | \inr => P_2}$ means \enquote{Input either $\inl$ or $\inr$ along the right-hand channel, and then continue as process $P_1$ or $P_2$, respectively.}

Conversely, the client sitting to the right that uses service $A_1 \with A_2$ must behave in a complementary way: the client should select either service $A_1$ or service $A_2$ and then, having notified the offering process of its choice (as $\inl$ or $\inr$), continue the session by using that service.
The left rules for type $A_1 \with A_2$ are thus:
\begin{mathpar}
  \infer[{\lrule[1]{\with}}]{\lseq{\hypof{A_1 \with A_2} |- \procof{\selectL{\inl; Q}}{C}}}{
    \lseq{\hypof{A_1} |- \procof{Q}{C}}}
  \and
  \infer[{\lrule[2]{\with}}]{\lseq{\hypof{A_1 \with A_2} |- \procof{\selectL{\inr; Q}}{C}}}{
    \lseq{\hypof{A_2} |- \procof{Q}{C}}}
\end{mathpar}
where $\selectL{\inj{\mathsf{1/2}} ; Q}$ means \enquote{Send label $\inj{\mathsf{1/2}}$ along the left-hand channel and then continue as process $Q$.}

Our intuition about the behavior of the guarded choice processes is made precise by their operational semantics.
First, the $\inl$ branch:
\begin{equation*}
  \begin{lgathered}
    \exec(\selectL{\inl ; Q}) \lrimp \monad{\msgL{\inl} \fuse \exec{Q}} \\
    \exec(\caseR{\inl => P_1 | \inr => P_2}) \fuse \msgL{\inl} \lrimp \monad{\exec{P_1}}
      \,.
  \end{lgathered}
\end{equation*}
To execute the selection process $\selectL{\inl ; Q}$, we asynchronously send to the left a message containing the label $\inl$, which is represented in the \ac{SSOS} as the proposition $\msgL{\inl}$, and then immediately continue the session by executing process $Q$.
When this message arrives, the destination process $\caseR{\inl => P_1 | \inr => P_2}$ resumes execution as $P_1$.

The operational semantics of the $\inr$ branch is symmetric to that of the $\inl$ branch:
\begin{equation*}
  \begin{lgathered}
    \exec(\selectL{\inr ; Q}) \lrimp \monad{\msgL{\inr} \fuse \exec{Q}} \\
    \exec(\caseR{\inl => P_1 | \inr => P_2}) \fuse \msgL{\inr} \lrimp \monad{\exec{P_2}}
      \,.
  \end{lgathered}
\end{equation*}

\paragraph{Practical considerations.}

% To make the language more palatable for the programmer, we diverge slightly from a pure propositions-as-types interpretation of one-dimensional linear logic by including $n$-ary labeled additive conjunctions $\with*{\ell_i: A_i}[i \in I]$ as a primitive.
% Formaly, the conjunction is over a mutiset of label-type pairs,
% % expressed as a parametric comprehension
% indexed by set $I$
% \autocite{Cervesato+Sans:FI14}

To make the language more palatable for the programmer, we diverge slightly from a pure propositions-as-types interpretation of singleton linear logic by including $n$-ary labeled additive conjunctions, $\with*{\ell: A_\ell}[\ell \in L]$, as a primitive.
% Formally, these types $\with*{\ell_i: A_i}[i \in I]$ are conjunctions over multiset comprehensions~\autocite{Cervesato+Sans:FI14} of label--type pairs.

By analogy with the binary conjunction, the typing rules and operational semantics for $\with*{\ell: A_\ell}[\ell \in L]$ are as follows.
Notice that the $\rrule{\with}$ has a set of premises, one for each label $\ell \in L$.
\begin{gather*}
  \infer[\rrule{\with}]{\lseq{\lctx |- \procof{\caseR[\ell \in L]{\ell => P_\ell}}{\with*{\ell: A_\ell}[\ell \in L]}}}{
    \forall \ell \in L \mathpunct{:}\enskip \lseq{\lctx |- \procof{P_\ell}{A_\ell}}}
  \qquad
  % \raisebox{0.5\baselineskip}{$\bagged{\,\raisebox{-0.5\baselineskip}{
  % \scaleleftright[0.4em]{\lbag}{
    \infer[{\lrule{\with}}]{\lseq{\hypof{\with*{\ell: A_\ell}[\ell \in L]} |- \procof{\selectL{\kay; Q}}{C}}}{
      \lseq{\hypof{A_{\kay}} |- \procof{Q}{C}} &
      \text{($\kay \in L$)}}
  % }{\rbag}_{k \in I}
  % }}[k \in I]$}
  \\[2\jot]
  % \bagged{\,
  % \scaleleftright[0.4em]{\lbag}{
    \begin{lgathered}
      \exec{(\selectL{\kay ; Q})} \lrimp \monad{\msgL{\kay} \fuse \exec{Q}} \\
      \exec{(\caseR[\ell \in L]{\ell => P_\ell})} \fuse \msgL{\kay} \lrimp \monad{\exec{P_{\kay}}} \mathrlap{\qquad\text{($\kay \in L$)}}
    \end{lgathered}
  % }{\rbag}_{k \in I}
  % \,}[k \in I]
\end{gather*}
% According to the multiset nature of the type, our presentation uses the multiset-oriented inference rules of \textcite{Cervesato+Sans:FI14}: the $\rrule{\with}$ right rule has a multiset of premises---one for each index $i \in I$---and there are multisets of $\lrule[k]{\with}$ left rules and \ac{SSOS} rules.

Another possibility would be to simply treat $n$-ary labeled conjunctions as syntactic sugar for nested binary conjunctions, but this would
% turn out to
introduce a communication overhead because we would be sending multiple $\inj{\mathsf{1/2}}$s separately rather than a single label $\kay$.


\subsection{Recursive session types and process definitions}\label{sec:recurs-sess-types}

Concurrent processes frequently exhibit unbounded or infinite, yet well-defined, behavior;
for instance, we may wish to have a counter that offers an increment service indefinitely.
\Textcite{Toninho+:TGC14} have proposed an extension of their \ac{SILL} type theory that incorporates inductive and coinductive session types.
However, to keep matters simpler, we instead rely on general recursion here and choose to be content with the departure from a pure Curry--Howard isomorphism.

Session types thus include general recursive types, $\mu t.A$, and type variables, $t$.
The type $\mu t.A$ is interpreted equi-recursively, being identified with the unfolding $\subst{(\mu t.A)/t}{A}$.
Processes correspondingly include mutually recursive process definitions, via \sillinline`letrec`, and process variables, $X$.

We extend the session-typing judgment with a context, $\pctx$, of process-variable typings.
% Because a process is typed according to the services that it uses and offers, process variables are typed as $X : \ctxmonad{A <- \vec{B}}$, meaning that process $X$ can offer service $A$ if provided with channels along which services $\vec{B}$ are offered.
Because a process is typed according with a sequent of services that it uses and offers, process variables are typed as ${X : \ctxmonad{A <- \lctx}}$ if process $X$ can offer service $A$ 
% when provided with channels along which services $\vec{B}$ are offered.
by using services $\lctx$.
When channels of appropriate types are available, the process $X$ can be called:
\begin{equation*}
  \infer[\text{\scshape call}]{\lseq{\pctx, X{:}\ctxmonad{A <- \lctx} ; \lctx |- \procof{X}{A}}}{
    }
  \:.
\end{equation*}
A common idiom is $\spawn{X ; Q}$, which spawns a call to $X$ that is run in parallel with some process $Q$.
We will frequently abbreviate this with the syntactic sugar $\mbind{X ; Q}$.

% Mutually recursive process definitions add process variables to the context.
% Process variables are added to the context to allow for mutual recursion.
In mutually recursive process definitions,
% , the programmer declares each process with a type that must be checked.
the process bodies may refer to any of the mutually recursive processes via process variables.
The typing rule is
\begin{equation*}
  \infer[\text{\scshape letrec}]
  {\lseq{\pctx ; \lctx |- \procof{\mletrec[X \in \mathcal{X}]{(\mprocdef{X{:}\ctxmonad{A_X <- \lctx_X} = P_X})}{Q}}{C}}}{
    \text{($\pctx' = \braced{X{:}\ctxmonad{A_X <- \lctx_X}}[X \in \mathcal{X}]$)}
    &
    % \bagged{
    \forall X \in \mathcal{X}\mathpunct{:}\enskip
      \lseq{\pctx, \pctx' ; \lctx_X |- \procof{P_X}{A_X}}
    % }[j \in I]
    &
    \lseq{\pctx, \pctx' ; \lctx |- \procof{Q}{C}}}
  \,.
\end{equation*}

The operational semantics of these constructs are as follows:
\begin{equation*}
  \begin{lgathered}
    \exec{(\mletrec[X \in \mathcal{X}]{(\mprocdef{X = P_X})}{Q})} \lrimp \monad{\braced{\bang \bnd{X = P_X}}[X \in \mathcal{X}] \fuse \exec{Q}}
    \\
    \bang \bnd{X = P} \fuse \exec{X} \lrimp \monad{\exec{P}}
  \end{lgathered}
\end{equation*}
The environment of bindings of process variables to process expressions is represented in the \ac{SSOS} as a collection of $\bnd{X = P}$ hypotheses.
To execute a group of mutually recursive process definitions, $\mletrec[X \in \mathcal{X}]{(\mprocdef{X = P_X})}{Q}$, bindings are introduced for each of the process variables and then the body $Q$ is executed.
To execute a free process variable $X$, instead execute the process expression to which $X$ is bound.


%
Now, having presented recursion, we can finally give a simple example program.


\subsection{Example: Binary counter}\label{sec:exampl-binary-count}

We can implement a simple session-typed counter on natural numbers as shown in \cref{lst:counter-inc}.%
\footnote{This example is adapted from one by \textcite{Toninho+:ESOP13}.}
%
\begin{sillcode}[
  caption={A simple binary counter supporting an increment operation},
  label={lst:counter-inc},
  floatplacement=tb,
  gobble=2
]
  stype Cntr = &{ inc: Cntr }
  
  eps : { |- Cntr } =
  { caseR of
      inc => eps; bit1 }
  
  bit0 : { Cntr |- Cntr } =
  { caseR of
      inc => bit1 }
  
  bit1 : { Cntr |- Cntr } =
  { caseR of
      inc => selectL inc; bit0 }
\end{sillcode}
%
The counter is a chain of \sillinline`bit0` and \sillinline`bit1` processes, one for each bit in the binary representation of the counter's value, and is terminated at the most significant end with an \sillinline`eps` process.
For instance, the process chain \sillinline`eps; bit1; bit0` represents a counter with value $2$.

The counter offers a very simple service: the client may only choose to increment the counter, with the same service being offered recursively after the increment.
This service, \sillinline`Cntr`, is therefore a recursive (unary) additive conjunction, declared in the concrete syntax as \sillinline`stype Cntr = &{ inc: Cntr }`.
The \sillinline`eps` process offers this service outright, and thus has type \sillinline`{ |- Cntr }`.
The \sillinline`bit0` and \sillinline`bit1` processes, on the other hand, use the service offered by their more significant neighbors, and thus have type \sillinline`{ Cntr |- Cntr }`.

The process definitions of \sillinline`eps`, \sillinline`bit0`, and \sillinline`bit1` are mutually recursive.
When an \sillinline`eps` process receives an \sillinline`inc` message, it creates a new most significant bit by spawning a new \sillinline`eps` process and then making a recursive call to a \sillinline`bit1` process.
% that uses the service offered by the new \sillinline`eps`.
When a \sillinline`bit0` process receives an \sillinline`inc`, the bit is flipped by virtue of a recursive call to a \sillinline`bit1` process.
Lastly, when a \sillinline`bit1` process receives an \sillinline`inc`, the bit is flipped and a carry is propagated; this is accomplished by first sending \sillinline`inc` along the left to the \sillinline`Cntr` offered by the next more significant bit and then making a recursive call to a \sillinline`bit0` process.

Informally, we can see that, as implemented, the \sillinline`inc` operation respects a counter's denotation: whenever a counter representing natural number $N$ is incremented, the resulting counter represents $N+1$.
Note, however, that this adequacy property is not enforced by the type \sillinline`Cntr`.
An appropriate dependent session type could enforce increment adequacy, but, for simplicity of exposition, we prefer the simple type here.

\subsection{Additive disjunction as choice}\label{sec:addit-disj-as}

In the singleton linear sequent calculus, additive disjunction, $A \ssor B$, is dual to additive conjunction, $A \with B$.
\begin{mathpar}
  \infer[{\rrule[1]{\ssor}}]{\lseq{\lctx |- A_1 \ssor A_2}}{
    \lseq{\lctx |- A_1}}
  \and
  \infer[{\rrule[2]{\ssor}}]{\lseq{\lctx |- A_1 \ssor A_2}}{
    \lseq{\lctx |- A_2}}
  \and
  \infer[\lrule{\ssor}]{\lseq{A_1 \ssor A_2 |- C}}{
    \lseq{A_1 |- C} &
    \lseq{A_2 |- C}}
\end{mathpar}

We should expect this duality to also appear in the process assignment.
Whereas a process of type $A \with B$ offers its client at the right a choice of services $A$ and $B$, a process of type $A \ssor B$ chooses between offering service $A$ or service $B$ to its client at the right.
The client waits to be notified of the offering process's choice and then uses that service.
\begin{mathpar}
  \infer[{\rrule[1]{\ssor}}]{\lseq{\lctx |- \procof{\selectR{\inl; P}}{A_1 \ssor A_2}}}{
    \lseq{\lctx |- \procof{P}{A_1}}}
  \and
  \infer[{\rrule[2]{\ssor}}]{\lseq{\lctx |- \procof{\selectR{\inr; P}}{A_1 \ssor A_2}}}{
    \lseq{\lctx |- \procof{P}{A_2}}}
  \and
  \infer[\lrule{\ssor}]{\lseq{A_1 \ssor A_2 |- \procof{\caseL{\inl => Q_1 | \inr => Q_2}}{C}}}{
    \lseq{A_1 |- \procof{Q_1}{C}} &
    \lseq{A_2 |- \procof{Q_2}{C}}}
\end{mathpar}
Confirming the intuition that $\selectR{\inj{\mathsf{1/2}}; P}$ sends along the right-hand channel and $\caseL{\inl => Q_1 | \inr => Q_2}$ receives along the left-hand channel are the \ac{SSOS} rules:
\begin{equation*}
  \begin{lgathered}
    \exec{(\selectR{\inl ; P})} \lrimp \monad{\exec{P} \fuse \msgR{\inl}} \\
    \msgR{\inl} \fuse \exec{(\caseL{\inl => Q_1 | \inr => Q_2})} \lrimp \monad{\exec{Q_1}}
    %
    \\[\jot]
    %
    \exec{(\selectR{\inr ; P})} \lrimp \monad{\exec{P} \fuse \msgR{\inr}} \\
    \msgR{\inr} \fuse \exec{(\caseL{\inl => Q_1 | \inr => Q_2})} \lrimp \monad{\exec{Q_2}}
      \,.
  \end{lgathered}
\end{equation*}
Because the operational semantics is asynchronous, it's important to distinguish the $\msgR{}$ predicate, which represents messages that flow to the right, from the $\msgL{}$ predicate, which represents messages that flow to the left.
Otherwise, a selection process's continuation could mistakenly capture the message that was just sent, as might happen in executing $\selectR{\inl ; \caseR{\inl => P_1 | \inr => P_2}}$, for example.

Once again, to make the language more convenient for the programmer, we include $n$-ary labeled additive disjunctions $\ssor*{\ell: A_\ell}[\ell \in L]$.
The typing rules and operational semantics are thus more generally
\begin{gather*}
  % \bagged{
    \infer[{\rrule{\ssor}}]{\lseq{\lctx |- \procof{\selectR{\kay; P}}{\ssor*{\ell: A_\ell}[\ell \in L]}}}{
      \lseq{\lctx |- \procof{P}{A_{\kay}}} &
      \text{($\kay \in L$)}}
%   }[k \in I]
  \qquad
  \infer[\lrule{\ssor}]{\lseq{\ssor*{\ell: A_\ell}[\ell \in L] |- \procof{\caseL[\ell \in L]{\ell => Q_\ell}}{C}}}{
    \forall \ell \in L\mathpunct{:}\enskip \lseq{A_\ell |- \procof{Q_\ell}{C}}}
  \\[2\jot]
  % \bagged{
    \begin{lgathered}
      \exec{(\selectR{\kay; P})} \lrimp \monad{\exec{P} \fuse \msgR{\kay}} \\
      \msgR{\kay} \fuse \exec{(\caseL[\ell \in L]{\ell => Q_\ell})} \lrimp \monad{\exec{Q_{\kay}}} \mathrlap{\qquad\text{($\kay \in L$)}}
    \end{lgathered}
  % }[k \in I]
\end{gather*}


\subsection{Example: Binary counter with decrements}\label{sec:exampl-binary-count-1}

\begin{sillcode}[
  caption={A binary counter supporting increments and decrements},
  label={lst:counter-dec},
  floatplacement=tb,
  gobble=2
]
  stype Cntr = &{ inc: Cntr , dec: Cntr' }
    and Cntr' = +{ ok: Cntr , fail: Cntr }

  eps : { |- Cntr } =
  { caseR of
      inc => eps; bit1
    | dec => selectR fail; eps }
  
  bit0 : { Cntr |- Cntr } =
  { caseR of
      inc => bit1
    | dec => selectL dec; bit0' }
  
  bit0' : { Cntr' |- Cntr' } =
  { caseL of
      ok => selectR ok; bit1
    | fail => selectR fail; bit0 }

  bit1 : { Cntr |- Cntr } =
  { caseR of
      inc => selectL inc; bit0
    | dec => selectR ok; bit1 }
\end{sillcode}

\Cref{lst:counter-dec} shows a counter that takes advantage of additive disjunction to support a truncated decrement operation. 
According to the type declaration, a process offering the \sillinline`Cntr` service gives its client a choice of increment or decrement services.
If the client chooses to decrement, the offering process will choose to reply with either \sillinline`fail` or \sillinline`ok` and then recursively offer the \sillinline`Cntr` service.

As implemented, decrementing the counter gives \sillinline`fail` and leaves the process network unchanged if the counter represents $0$; if it represents some $N > 0$, then decrementing the counter gives \sillinline`ok` after decrementing to $N - 1$.
Once again, these adequacy properties are not enforced by the type \sillinline`Cntr`, although they could be with an appropriate dependent session type.


\subsection{Identity as forwarding}\label{sec:ident-as-forw}

In singleton linear logic, in addition to the cut principle, there is an identity principle that states that one way to obtain a resource is to directly use an existing resource:
\begin{equation*}
  \infer[\id{A}]{\lseq{A |- A}}{
    }
  \,.
\end{equation*}
Under the process interpretation, a process can offer service $A$ by acting as a forwarding intermediary between its clients and another process that offers service $A$.
The $\id{A}$ rule thus types a forwarding process between two channels:
\begin{equation*}
  \infer[\id{A}]{\lseq{A |- \procof{\fwd}{A}}}{
    }
  \,.
\end{equation*}
Rather than making the forwarding explicit in the operational semantics, we can simply eliminate the middleman, adjoining the neighboring processes:
\begin{equation*}
  \exec{(\fwd)} \lrimp \monad{\one}
  \,.
\end{equation*}

% \begin{sillcode}[
%   caption={A binary counter as a stream transformer},
%   label={lst:counter-bit},
%   floatplacement=tb,
%   gobble=2
% ]
%   stype Bin = +{ eps: 1 , bit0: Bin , bit1: Bin }
  
%   inc : {Bin |- Bin} =
%   { caseL of
%       eps => selectR bit1; selectR eps; <->
%     | bit0 => selectR bit1; <->
%     | bit1 => selectR bit0; inc }
% \end{sillcode}



\subsection{Other session types}\label{sec:other-session-types}

In addition to the those already mentioned, the other connectives of singleton linear logic correspond to session types.

\paragraph{Multiplicative unit.}
Like its \ac{SILL} cousin, the multiplicative unit $\one$ is the service that terminates without any interaction.
Its right rule, $\rrule{\one}$, types a process that immediately terminates; its left rule, $\lrule{\one}$, types a process that waits for the left-hand side to terminate:
\begin{mathpar}
  \infer[\rrule{\one}]{\lseq{\lctxe |- \procof{\closeR}{\one}}}{
    }
  \and
  \infer[\lrule{\one}]{\lseq{\hypof{\one} |- \procof{\waitL{Q}}{C}}}{
    \lseq{\lctxe |- \procof{Q}{C}}}
\end{mathpar}
The operational semantics is asynchronous, with the $\closeR$ process sending a quit message, $\msgQ$:
\begin{equation*}
  \begin{lgathered}
    \exec{\closeR} \lrimp \monad{\msgQ} \\
    \msgQ \fuse \exec{(\waitL{Q})} \lrimp \monad{\exec{Q}}
  \end{lgathered}
\end{equation*}


\paragraph{First-order universal and existential quantification.}
The first-order quantifiers type processes that exchange functional values.
A process offering service $\forall x{:}\tau. A_x$ (or using service $\exists x{:}\tau. A_x$) first inputs a value $x$ of functional type $\tau$ and then offers (resp., uses) service $A_x$.
Dually, a process offering service $\exists x{:}\tau. A_x$ (or using service $\forall x{:}\tau. A_x$) asynchronously outputs the value of some functional term $M$ of type $\tau$ and then offers (resp., uses) service $\subst{M/x}{A_x}$. 
The first-order quantifiers are thus dependent session types; in the non-dependent case, we write the types as $\tau \vimp A$ and $A \vand \tau$.
% The syntax for value inputs and outputs is $\recv x <- inputc;Pxand outputc M;Q.

Since value inputs and outputs are not critical to the remainder of this proposal, the reader who is interested in further details of their static and dynamic semantics in \ac{SILL} should refer to the papers by \textcites{Toninho+:ESOP13}{Toninho+:PPDP11}; we leave the extrapolation to singleton linear logic as an exercise for the reader.

\paragraph{Multiplicative conjunction and linear implication.}
The restriction to sequents with singleton antecedents proves fatal to attempts to include multiplicative conjunction ($A \tensor B$) and linear implication ($A \lolli B$) as connectives in singleton linear logic.
For multiplicative conjunction, the left rule is problematic because it breaks down one hypothesis into two; for linear implication, the right rule is problematic because it introduces a new hypothesis even if one is already there.
Fortunately, for the examples in which we are interested, the absence of $\tensor$ and $\lolli$ is not an issue.

% \end{document}


% arara: pdflatex
% arara: pdflatex
% arara: biber
% arara: pdflatex
% arara: pdflatex
% \documentclass{../hdeyoung-proposal}
\documentclass[
  class=../hdeyoung-proposal,
  crop=false
]{standalone}


\usepackage{linear-logic}
\usepackage{ordered-logic}
\usepackage{basic-atoms}
\usepackage{proof}
\usepackage{mathpartir}

\usepackage{tikz}
% \usetikzlibrary{shapes.misc,graphs,graphdrawing}
% \usegdlibrary{trees}

\ExplSyntaxOn

% \DeclarePairedDelimiter \parens { \lparen } { \rparen }
\DeclarePairedDelimiter \bagged:wn { \lbag } { \rbag }
\NewDocumentCommand{ \bagged }{ s o m o }
  {
    \IfBooleanTF {#1}
      { \bagged:wn* {#3} }
      {
        \IfValueTF {#2}
          { \bagged:wn[#2] {#3} }
          { \bagged:wn {#3} }
      }
    \IfValueT {#4} { \sb{#4} }
  }


\NewDocumentCommand \oseq { >{ \SplitArgument{1}{|-} } m }
  { \oseq:nn #1 }
\cs_new:Npn \oseq:nn #1#2 { \oseq_ctxs:n {#1} \vdash #2 }
\cs_new:Npn \oseq_ctxs:n #1 {
  \seq_set_split:Nnn \l_tmpa_seq {;} {#1}
  \seq_use:Nn \l_tmpa_seq { \mathrel{;} }
}

\NewDocumentCommand \procof { m m } { #1 \dblcolon #2 }
\NewDocumentCommand \hypof { m } { #1 }


\NewDocumentCommand \cut { m } { \text{\textsc{\MakeLowercase{Cut}}}\sb{#1} }
\NewDocumentCommand \id { m } { \text{\textsc{\MakeLowercase{Id}}}\sb{#1} }

\NewDocumentCommand \comp { >{ \SplitArgument{1}{|} } m }
  { \comp:nn #1 }
\cs_new:Npn \comp:nn #1#2 { #1 \parallel #2 }

\NewDocumentCommand \fwd {} { \mathord{\leftrightarrow} }


\RenewDocumentCommand \with { s }
  { \IfBooleanTF {#1} \with:n \with: }
\cs_new:Npn \with:n #1 {
  \mathord{\binampersand}
  \bagged {
    \seq_set_split:Nnn \l_tmpa_seq {,} {#1}
    \seq_use:Nn \l_tmpa_seq {,}
  }
}
\cs_new:Npn \with: { \mathbin{\binampersand} }

\NewDocumentCommand \ssor { s }
  { \IfBooleanTF {#1} \with:n \with: }
\cs_new:Npn \ssor:n #1 {
  \mathord{\ssor:}
  \bagged {
    \seq_set_split:Nnn \l_tmpa_seq {,} {#1}
    \seq_use:Nn \l_tmpa_seq {,}
  }
}
\cs_new:Npn \ssor: { \oplus }

\NewDocumentCommand \caseR { s m }
  {
    \IfBooleanTF {#1}
      { \case:nNn { \mathsf{caseR} } \bagged {#2} }
      { \case:nNn { \mathsf{caseR} } \parens {#2} }
  }
\NewDocumentCommand \caseL { s m }
  {
    \IfBooleanTF {#1}
      { \case:nNn { \mathsf{caseL} } \bagged {#2} }
      { \case:nNn { \mathsf{caseL} } \parens {#2} }
  }
\cs_new:Npn \case:nNn #1#2#3 {
  #1 \mskip\thinmuskip
  #2 {
    \seq_set_split:Nnn \l_tmpa_seq {|} {#3}
    \seq_clear:N \l_tmpb_seq
    \seq_map_inline:Nn \l_tmpa_seq
      { \seq_put_right:Nn \l_tmpb_seq { \case_branch:n {##1} } }
    \seq_use:Nn \l_tmpb_seq { \mid }
  }
}
\cs_new:Npn \case_branch:n #1 { \case_branch_aux:w #1 \q_stop }
\cs_new:Npn \case_branch_aux:w #1 => #2 \q_stop {
  #1 \Rightarrow #2
}

\NewDocumentCommand \selectL { >{ \SplitArgument{1}{;} } m }
  { \select:nnn { \mathsf{selectL} } #1 }
\NewDocumentCommand \selectR { >{ \SplitArgument{1}{;} } m }
  { \select:nnn { \mathsf{selectR} } #1 }
\cs_new:Npn \select:nnn #1#2#3 {
  \!\mathord{}\mathop{#1} #2 ; #3
}


\NewDocumentCommand \inj { m } { \mathsf{in}\sb{#1} }

\NewDocumentCommand \inl {} { \inj{ \mathsf{1} } }
\NewDocumentCommand \inr {} { \inj{ \mathsf{2} } }


\RenewDocumentCommand \one {} { \mathord { \mathbf{1} } }

\NewDocumentCommand \quitR {} { \mathsf{quitR} }
\NewDocumentCommand \waitL { m } { \mathsf{waitL} ; #1 }


\NewDocumentCommand \rrule { o m } {
  \IfValueTF {#1}
    { \rrule:nn {#2} {#1} }
    { \rrule:n {#2} }
}
\cs_new:Npn \rrule:nn #1#2 { {#1}\text{\textsc{\MakeLowercase{R}}}\sb{#2} }
\cs_new:Npn \rrule:n #1 { {#1}\text{\textsc{\MakeLowercase{R}}} }

\NewDocumentCommand \lrule { o m } {
  \IfValueTF {#1}
    { \lrule:nn {#2} {#1} }
    { \lrule:n {#2} }
}
\cs_new:Npn \lrule:nn #1#2 { {#1}\text{\textsc{\MakeLowercase{L}}}\sb{#2} }
\cs_new:Npn \lrule:n #1 { {#1}\text{\textsc{\MakeLowercase{L}}} }


\NewDocumentCommand \exec { } { \mathsf{exec} \mskip\thinmuskip }
\NewDocumentCommand \msg { } { \mathsf{msg} \mskip\thinmuskip }

\ExplSyntaxOff


\DeclareAcronym{SSOS}{
  short = SSOS,
  long = substructural operational semantics,
  short-format = \scshape\MakeLowercase
}
\DeclareAcronym{SILL}{
  short = \MakeLowercase{SILL},
  long = session-typed intuitionistic linear logic,
  short-format = \scshape
}

\addbibresource{../proposal.bib}


\ExplSyntaxOn

\NewDocumentCommand \spawn { >{ \SplitArgument{1}{;} }m } { \spawn:nn #1 }
\cs_new:Npn \spawn:nn #1#2 { \mathsf{spawn} \mskip\thinmuskip #1 ; #2 }

\NewDocumentCommand \call { m } { \call:n {#1} }
\cs_new:Npn \call:n #1 { \mathsf{call} \mskip\thinmuskip #1 }

\NewDocumentCommand \compile { m m } { \compile:nn {#1} {#2} }
\cs_new:Npn \compile:nn #1#2 { \llbracket #1 \rrbracket = #2 }

\ExplSyntaxOff


\begin{document}

\section{}

\begin{mathpar}
  \infer{\compile{A^+_1 \fuse A^+_2}{\spawn{P_1 ; P_2}}}{
    \compile{A^+_1}{P_1} &
    \compile{A^+_2}{P_2}}
  \and
  \infer{\compile{\one}{\fwd}}{
    }
  \\
  \infer{\compile{\patom}{\call{\patom}}}{
    }
  \and
  \infer{\compile{\matom[->]}{\selectR{\matom[->] ; \fwd}}}{
    }
  \and
  \infer{\compile{\matom[<-]}{\selectL{\matom[<-] ; \fwd}}}{
    }
\end{mathpar}

\begin{mathpar}
  \infer{\compile{\with*{\matom[->]_i \limp \monad{A^+_i}}[i \in I]}{\caseL*{\matom[->]_i => P_i}[i \in I]}}{
    \bagged{\compile{A^+_i}{P_i}}[i \in I]}
  \and
  \infer{\compile{\with*{\matom[<-]_i \rimp \monad{A^+_i}}[i \in I]}{\caseR*{\matom[<-]_i => P_i}[i \in I]}}{
    \bagged{\compile{A^+_i}{P_i}}[i \in I]}
  \and
  \infer{\compile{\monad{A^+}}{P}}{
    \compile{A^+}{P}}
\end{mathpar}

\begin{mathpar}
  \infer{\compile{\octx_1, \octx_2}{\octx'_1, \octx'_2}}{
    \compile{\octx_1}{\octx'_1} &
    \compile{\octx_2}{\octx'_2}}
  \and
  \infer{\compile{\octxe}{\octxe}}{
    }
  \and
  \infer{\compile{A^+}{\exec{P}}}{
    \compile{A^+}{P}}
  \\
  \infer{\compile{\susp+{\patom}}{\exec{P}}}{
    \patom \lrimp A^- \in \sig &
    \compile{A^-}{P}}
  \and
  \infer{\compile{\susp+{\matom[->]}}{\msg{\matom[->]}}}{
    }
  \and
  \infer{\compile{\susp+{\matom[<-]}}{\msg{\matom[<-]}}}{
    }
\end{mathpar}

\end{document}


% arara: lualatex
% arara: lualatex
% arara: biber
% arara: lualatex
% arara: lualatex
% \documentclass{../hdeyoung-proposal}
\documentclass[
  class=../hdeyoung-proposal,
  crop=false
]{standalone}


\usepackage{linear-logic}
\usepackage{ordered-logic}
\usepackage{proof}
\usepackage{mathpartir}

\usepackage{tikz}
\usetikzlibrary{shapes.misc,graphs,quotes,graphdrawing}
\usegdlibrary{trees}

\usepackage{scalerel}

\ExplSyntaxOn

% \DeclarePairedDelimiter \parens { \lparen } { \rparen }
\DeclarePairedDelimiter \bagged:wn { \lbag } { \rbag }
\NewDocumentCommand{ \bagged }{ s o m o }
  {
    \IfBooleanTF {#1}
      { \bagged:wn* {#3} }
      {
        \IfValueTF {#2}
          { \bagged:wn[#2] {#3} }
          { \bagged:wn {#3} }
      }
    \IfValueT {#4} { \sb{#4} }
  }


\NewDocumentCommand \oseq { >{ \SplitArgument{1}{|-} } m }
  { \oseq:nn #1 }
\cs_new:Npn \oseq:nn #1#2 { \oseq_ctxs:n {#1} \vdash #2 }
\cs_new:Npn \oseq_ctxs:n #1 {
  \seq_set_split:Nnn \l_tmpa_seq {;} {#1}
  \seq_use:Nn \l_tmpa_seq { \mathrel{;} }
}

\NewDocumentCommand \procof { m m } { #1 \dblcolon #2 }
\NewDocumentCommand \hypof { m } { #1 }


\NewDocumentCommand \cut { m } { \text{\textsc{\MakeLowercase{Cut}}}\sb{#1} }
\NewDocumentCommand \id { m } { \text{\textsc{\MakeLowercase{Id}}}\sb{#1} }

\NewDocumentCommand \comp { >{ \SplitArgument{1}{|} } m }
  { \comp:nn #1 }
\cs_new:Npn \comp:nn #1#2 { #1 \parallel #2 }

\NewDocumentCommand \fwd {} { \mathord{\leftrightarrow} }


\RenewDocumentCommand \with { s }
  { \IfBooleanTF {#1} \with:n \with: }
\cs_new:Npn \with:n #1 {
  \mathord{\binampersand}
  \bagged {
    \seq_set_split:Nnn \l_tmpa_seq {,} {#1}
    \seq_use:Nn \l_tmpa_seq {,}
  }
}
\cs_new:Npn \with: { \mathbin{\binampersand} }

\NewDocumentCommand \ssor { s }
  { \IfBooleanTF {#1} \ssor:n \ssor: }
\cs_new:Npn \ssor:n #1 {
  \mathord{\ssor:}
  \bagged {
    \seq_set_split:Nnn \l_tmpa_seq {,} {#1}
    \seq_use:Nn \l_tmpa_seq {,}
  }
}
\cs_new:Npn \ssor: { \oplus }

\NewDocumentCommand \caseR { s m }
  {
    \IfBooleanTF {#1}
      { \case:nNn { \mathsf{caseR} } \bagged {#2} }
      { \case:nNn { \mathsf{caseR} } \parens {#2} }
  }
\NewDocumentCommand \caseL { s m }
  {
    \IfBooleanTF {#1}
      { \case:nNn { \mathsf{caseL} } \bagged {#2} }
      { \case:nNn { \mathsf{caseL} } \parens {#2} }
  }
\cs_new:Npn \case:nNn #1#2#3 {
  #1 \mskip\thinmuskip
  #2 {
    \seq_set_split:Nnn \l_tmpa_seq {|} {#3}
    \seq_clear:N \l_tmpb_seq
    \seq_map_inline:Nn \l_tmpa_seq
      { \seq_put_right:Nn \l_tmpb_seq { \case_branch:n {##1} } }
    \seq_use:Nn \l_tmpb_seq { \talloblong }
  }
}
\cs_new:Npn \case_branch:n #1 { \case_branch_aux:w #1 \q_stop }
\cs_new:Npn \case_branch_aux:w #1 => #2 \q_stop {
  #1 \Rightarrow #2
}

\NewDocumentCommand \selectL { >{ \SplitArgument{1}{;} } m }
  { \select:nnn { \mathsf{selectL} } #1 }
\NewDocumentCommand \selectR { >{ \SplitArgument{1}{;} } m }
  { \select:nnn { \mathsf{selectR} } #1 }
\cs_new:Npn \select:nnn #1#2#3 {
  \!\mathord{}\mathop{#1} #2 ; #3
}


\NewDocumentCommand \inj { m } { \mathsf{in}\sb{#1} }

\NewDocumentCommand \inl {} { \inj{ \mathsf{1} } }
\NewDocumentCommand \inr {} { \inj{ \mathsf{2} } }


\RenewDocumentCommand \one {} { \mathord { \mathbf{1} } }

\NewDocumentCommand \quitR {} { \mathsf{quitR} }
\NewDocumentCommand \waitL { m } { \mathsf{waitL} ; #1 }


\NewDocumentCommand \rrule { o m } {
  \IfValueTF {#1}
    { \rrule:nn {#2} {#1} }
    { \rrule:n {#2} }
}
\cs_new:Npn \rrule:nn #1#2 { {#1}\text{\textsc{\MakeLowercase{R}}}\sb{#2} }
\cs_new:Npn \rrule:n #1 { {#1}\text{\textsc{\MakeLowercase{R}}} }

\NewDocumentCommand \lrule { o m } {
  \IfValueTF {#1}
    { \lrule:nn {#2} {#1} }
    { \lrule:n {#2} }
}
\cs_new:Npn \lrule:nn #1#2 { {#1}\text{\textsc{\MakeLowercase{L}}}\sb{#2} }
\cs_new:Npn \lrule:n #1 { {#1}\text{\textsc{\MakeLowercase{L}}} }


\NewDocumentCommand \exec { } { \mathsf{exec} \mskip\thinmuskip }
\NewDocumentCommand \msg { } { \mathsf{msg} \mskip\thinmuskip }

\ExplSyntaxOff




\addbibresource{../proposal.bib}

\NewDocumentCommand{\ie}{}{i.e.}


\usepackage{listings}
\crefname{listing}{listing}{listings}
\Crefname{listing}{Listing}{Listings}

\newlength{\mywidth}
\settowidth{\mywidth}{\ttfamily A}
\lstset{basicstyle=\ttfamily, basewidth=\mywidth}

\captionsetup[lstlisting]{%
  box=colorbox, boxcolor=gray,
  font={normalfont, sf, color=white},
  labelfont=bf,
  justification=justified, singlelinecheck=false
}

\lstnewenvironment{sillcode}[1][]
  {\lstset{language={},frame=bottomline,framerule=0.8ex,rulecolor=\color{gray},float,#1}}%
  {}

\lstnewenvironment{sillcode*}[1][]
  {\lstset{language={},#1}}%
  {}

\NewDocumentCommand{\sillinline}{o}{%
  \IfValueTF{#1}{\lstinline[#1]}{\lstinline}%
}


\NewDocumentCommand{\pctx}{}{\Psi}
\ExplSyntaxOn
\NewDocumentCommand{\ctxmonad}{>{\SplitArgument{1}{<-}}m}{
  \{\use_ii:nn #1 \vdash \use_i:nn #1\}
}
\NewDocumentCommand \spawn { >{ \SplitArgument{1}{;} } m } { \spawn:nn #1 }
\cs_new:Npn \spawn:nn #1#2 {
  \mathsf{spawn}
  \tl_if_empty:nF {#1} {
    \mskip\thinmuskip #1 ; #2
  }
}
\NewDocumentCommand{\mbind}{>{\SplitArgument{1}{;}}m}{
  \use_i:nn#1 ; \use_ii:nn#1
}
\NewDocumentCommand{\mletrec}{m m}{
  \mathsf{letrec} \mskip\thinmuskip #1 \mskip\thinmuskip \mathsf{in} \mskip\thinmuskip #2
}
\NewDocumentCommand{\mprocdef}{m}{
  #1
}
\ExplSyntaxOff




\DeclareAcronym{BHK}{
  short = BHK,
  long  = Brouwer-Heyting-Kolmogorov
}

\DeclareAcronym{JILL}{
  short = JILL,
  long  = judgmental intuitionistic linear logic
}

\DeclareAcronym{ILL}{
  short = ILL,
  long  = intuitionistic linear logic
}

\DeclareAcronym{SML}{
  short = SML,
  long  = Standard ML
}

\DeclareAcronym{SOS}{
  short = SOS,
  short-indefinite = an,
  long = structural operational semantics
}

\DeclareAcronym{SSOS}{
  short = SSOS,
  short-indefinite = an,
  long = substructural operational semantics
}

\DeclareAcronym{SILL}{
  short = SILL,
  long = session-typed intuitionistic linear logic
}

\DeclareAcronym{CLF}{
  short = CLF,
  long = the concurrent logical framework
}



\begin{document}

\section{Proposed work}\label{sec:proposed-work}

In this document, we have shown how session types form a bridge between a class of ordered logical specifications and process definitions typed in singleton linear logic---between proof-construction-as-computation and proof-reduction-as-computation.
The primary area of proposed work is to extend this connection to (a class of) linear logical specifications and \ac{SILL} process definitions typed in linear logic.
% To defend the proposed thesis, this connection must be extended to (a class of) linear logical specifications and \ac{SILL} process definitions typed in linear logic.

\subsection{From ordered logical specifications to linear logical specifications}\label{sec:from-ordered-logical}

Although awkward, it is possible to give an ordered logical specification for addition of two binary numbers.
The idea is to arrange the numbers end-to-end, but separated by a $\plus$ marker and terminated by a $\equals$ marker.
For instance, the string
\begin{equation*}
  \eps \fuse \bit{1} \fuse \bit{0} \fuse \plus \fuse \bit{1} \fuse \bit{0} \fuse \equals
\end{equation*}
would represent a request to add $2+2$.
By decrementing the second number to $0$ and incrementing

\begin{equation*}
  \begin{lgathered}
    \bit{0} \fuse \dec \lrimp \monad{\dec \fuse \bit[']{0}} \\
    \bit{1} \fuse \dec \lrimp \monad{\skip \fuse \bit{0} \fuse \ok} \\
    \plus \fuse \dec \lrimp \monad{\fail} \\
    %
    \bit{0} \fuse \skip \lrimp \monad{\skip \fuse \bit{0}} \\
    \bit{1} \fuse \skip \lrimp \monad{\skip \fuse \bit{1}} \\
    \plus \fuse \skip \lrimp \monad{\inc \fuse \plus} \\
    %
    \ok \fuse \bit[']{0} \lrimp \monad{\bit{1} \fuse \ok} \\
    \fail \fuse \bit[']{0} \lrimp \monad{\fail} \\
    %
    \ok \fuse \equals['] \lrimp \monad{\equals} \\
    \fail \fuse \equals['] \lrimp \monad{\one}
  \end{lgathered}
\end{equation*}


\begin{sillcode}
  plus =
  { caseR of
      dec => selectR fail; <->
    | skip_inc => selectL inc; plus }

  bit0 =
  { caseR of
      dec => selectL dec; bit0'
    | skip_inc => selectL skip_inc; bit0 }

  bit0' =
  { caseL of
      ok => selectR ok; bit1
    | fail => selectR fail; <-> }

  bit1 =
  { caseR of
      dec => selectR ok; selectL skip_inc; bit0
    | skip_inc => selectL skip_inc; bit1 }

  equals =
  { selectL dec; equals' }

  equals' =
  { caseL of
      ok => equals
    | fail => <-> }
\end{sillcode}


A better algorithm would add the two numbers bit-by-bit.
Pictorially, 
\begin{equation*}
  \begin{tikzpicture}
    \graph [tree layout, math nodes] {
      + <- { / 0 <- / 1 <- / e ,
/ 0 <- / 1 <- / e };
};
  \end{tikzpicture}
\end{equation*}

\end{document}

% Functional logic languages, such as Curry~\autocite{Hanus:Ganzinger13}, view functional programs as logic programs.
This proposal takes the opposite approach, compiling a class of logic programs to functional programs.

\Citeauthor{Simmons+Zerny:LICS13}'s correspondence between natural semantics (functional program) and abstract machines (logic program)~\autocite*{Simmons+Zerny:LICS13}.

%%% Local Variables:
%%% TeX-master: "proposal"
%%% End:


\appendix
\section{Turing-machine--like addition process}

In this appendix, we present the code for a well-typed, Turing-machine--like addition process.
It corresponds to the ordered logical specification shown in \cref{fig:turing-binary-add}.

\begin{sillcode*}[
  % caption={Turing-machine--like addition of binary numbers},
  % label={lst:turing-add},
  % floatplacement=tb,
  gobble=2
]
  stype Cntr = &{ inc: Cntr }
  
  eps : { |- Cntr } =
  { caseR of
      inc => eps; bit1 }
  
  bit0 : { Cntr |- Cntr } =
  { caseR of
      inc => bit1 }
  
  bit1 : { Cntr |- Cntr } =
  { caseR of
      inc => selectL inc; bit0 }

  stype Cntr_D = &{ dec: Cntr_D' , skip: Cntr_D }
    and Cntr_D' = +{ ok: Cntr_D , fail: Cntr }

  bit0_d : { Cntr_D |- Cntr_D } =
  { caseR of
      dec => selectL dec; bit0_d'
    | skip => selectL skip; bit0_d }

  bit1_d : { Cntr_D |- Cntr_D } =
  { caseR of
      dec => selectL skip; selectR ok; bit0_d
    | skip => selectL skip; bit1_d }

  bit0_d' : { Cntr_D' |- Cntr_D' } =
  { caseL of
      ok => selectR ok; bit1_d
    | fail => selectR fail; <-> }

  plus : { Cntr |- Cntr_D } =
  { caseR of
      dec => selectR fail; <->
    | skip => selectL inc; plus }

  equals : { Cntr_D |- Cntr } =
  { selectL dec; equals' }

  equals' : { Cntr_D' |- Cntr } =
  { caseL of
      ok => equals
    | fail => <-> }
\end{sillcode*}

%\nocite{*}

\printbibliography

\end{document}
