% arara: pdflatex: { shell: yes }
% arara: biber
% arara: pdflatex: { shell: yes }
% arara: pdflatex: { shell: yes }
\documentclass{article}

\usepackage{fixltx2e}

\usepackage[T1]{fontenc}
\usepackage[sc,osf]{mathpazo}
\linespread{1.05}
\usepackage{microtype}

\usepackage[american]{babel}
\usepackage{csquotes}

\usepackage{enumitem}

\usepackage[capitalize]{cleveref}

\usepackage{xparse}

\usepackage[
  backend=biber,
  style=authoryear-comp,
  maxcitenames=2, uniquelist=false
]{biblatex}
\renewcommand*{\subtitlepunct}{:\ }
\addbibresource{proposal.bib}

\usepackage{acro}
\DeclareAcronym{BHK}{
  short = BHK,
  long  = Brouwer-Heyting-Kolmogorov
}
\DeclareAcronym{JILL}{
  short = JILL,
  long  = judgmental intuitionistic linear logic
}
\DeclareAcronym{SML}{
  short = SML,
  long  = Standard ML
}

\usepackage{booktabs}

\usepackage{logic}
\usepackage{predicates}

\usepackage{minted}
\usepackage{verbments}
% \NewDocumentEnvironment{smlcode}{}{\begin{pyglist}[language=sml,gobble=2]}{\end{pyglist}}
\usepackage{listings}
\newlength{\mywidth}
\settowidth{\mywidth}{\ttfamily A}
\lstset{basicstyle=\ttfamily, basewidth=\mywidth}
\NewDocumentCommand{\sml}{o}{%
  \IfValueTF{#1}{\lstinline[#1]}{\lstinline}%
}

\usepackage{fixme}
\fxsetup{inlineface=\color{blue}, nomargin, inline, status=draft}
\makeatletter
\renewcommand*\FXLayoutInline[3]{%
  {\@fxuseface{inline}#2}}
\makeatother
\usepackage{soul}


\NewDocumentCommand{\latin}{m}{\textit{#1}}
\NewDocumentCommand{\ie}{}{\latin{i.e.}}
\NewDocumentCommand{\eg}{}{\latin{e.g.}}
\NewDocumentCommand{\vocab}{m}{\emph{#1}}
\NewDocumentCommand{\wc}{m o}{%
  #1%
  \IfValueTF{#2}%
    {\fxnote{\ #2\textsuperscript{?}}}%
    {\fxnote{\textsuperscript{?}}}%
}

\NewDocumentCommand{\rulename}{m}{\text{\normalfont\ttfamily #1}}



% \includeonly{background}

\begin{document}

\title{Session-Typed Concurrent\\Logic Programming}
\author{Henry DeYoung\\\texttt{hdeyoung@cs.cmu.edu}}
\date{\today}
\maketitle

\begin{abstract}
  % Although both logic programming and functional programming are rooted in proof theory, they each have distinct advantages and disadvantages.
  % Logic programs are remarkably concise and expressive, whereas well-typed functional programs provide guarantees about their behavior.
  % This thesis proposes to identify a class of (bottom-up) logic programs that do, in fact, enjoy the same kinds of behavioral guarantees as functional programs, thereby affording programmers the best of both worlds.
\end{abstract}

\begin{itemize}
\item Two paradigms for logic-based computation: proof search (logic programming) and proof reduction (functional programming).
      Understand the relationship between them.
\item Proof search allows very concise, expressive programs (e.g., [parallel] bubblesort in one line), but does not provide many (any?) guarantees.
      Proof reduction provides strong guarantees (e.g., preservation, progress, and termination), but requires the programmer to supply many details (e.g., bubblesort).
\end{itemize}

%%% Local Variables:
%%% TeX-master: "proposal"
%%% End:


\subsection{Linear logic}\label{sec:linear-logic}

\begin{figure}
  \begin{mathpar}
    \infer[\lab{id}]{\uctx ; P \seq P}{
      }
    \and
    \infer[\lab{copy}]{\uctx, A ; \lctx' \seq J}{
      \uctx, A ; \lctx', A \seq J}
    \\
    \infer[\rlab{{\lolli}}]{\uctx ; \lctx \seq A \lolli B}{
      \uctx ; \lctx, A \seq B}
    \and
    \infer[\llab{{\lolli}}]{\uctx ; \lctx'_1, \lctx'_2, A \lolli B \seq J}{
      \uctx ; \lctx'_1 \seq A &
      \uctx ; \lctx'_2, B \seq J}
    \\
    \infer[\rlab{{\with}}]{\uctx ; \lctx \seq A \with B}{
      \uctx ; \lctx \seq A &
      \uctx ; \lctx \seq B}
    \and
    \infer[{\llab{{\with}}[1]}]{\uctx ; \lctx', A \with B \seq J}{
      \uctx ; \lctx', A \seq J}
    \and
    \infer[{\llab{{\with}}[2]}]{\uctx ; \lctx', A \with B \seq J}{
      \uctx ; \lctx', B \seq J}
    \\
    \infer[\rlab{{\tensor}}]{\uctx ; \lctx_1, \lctx_2 \seq A \tensor B}{
      \uctx ; \lctx_1 \seq A &
      \uctx ; \lctx_2 \seq B}
    \and
    \infer[\llab{{\tensor}}]{\uctx ; \lctx', A \tensor B \seq J}{
      \uctx ; \lctx', A, B \seq J}
    \\
    \infer[\rlab{\one}]{\uctx ; \lctxe \seq \one}{
      }
    \and
    \infer[\llab{\one}]{\uctx ; \lctx', \one \seq J}{
      \uctx ; \lctx' \seq J}
    \\
    \infer[\rlab{\bang}]{\uctx ; \lctxe \seq \bang A}{
      \uctx ; \lctxe \seq A}
    \and
    \infer[\llab{\bang}]{\uctx ; \lctx', \bang A \seq J}{
      \uctx, A ; \lctx' \seq J}
  \end{mathpar}
  \begin{mathpar}
    \infer-[\lab{id$_A$}]{\uctx ; A \seq A}{
      }
    \and
    \infer-[\lab{cut$_A$}]{\uctx ; \lctx, \lctx' \seq J}{
      \uctx ; \lctx \seq A &
      \uctx ; \lctx', A \seq J}
    \and
    \infer-[\lab{cut$^{\bang}_A$}]{\uctx ; \lctx' \seq J}{
      \uctx ; \lctxe \seq A &
      \uctx, A ; \lctx' \seq J}
  \end{mathpar}
  \caption{Sequent calculus rules for \acf{JILL}~\autocite{Chang+:CMU03}.}
\end{figure}

\begin{itemize}
\item Should I give a brief introduction to linear logic here?  Or, is this overkill for committee members?
\item Present ordered logic in its own right here, or present ordered logic only in the context of bottom-up logic programming?
\end{itemize}

\subsection{Proof search as computation: Bottom-up linear logic programming}\label{sec:linear-lp}

\begin{itemize}
\item Notion of transition obtained by reading sequent calculus left rules  (bipoles) bottom-up.
\end{itemize}

\subsubsection{Bottom-up ordered logic programming}\label{sec:ordered-lp}

\begin{itemize}
\item Example: binary counter with increment.
\end{itemize}

\subsection{Proof reduction as computation: Session-typed linear logic}\label{sec:async-sill}

\begin{itemize}
\item Present monadic language only?  Disadvantage is that proof reduction is not quite as clear or simple to see since it involves run-time typing.
\end{itemize}

%%% Local Variables:
%%% TeX-master: "proposal"
%%% End:


\subsection{What counts as a choreography?}\label{sec:what-counts-choreo}

\begin{align*}
  &\inc \lrimp \inc[<-] \\
  &\eps \lrimp (\inc[<-] \rimp \eps \fuse \bit{1}) \\
  &\bit{0} \lrimp (\inc[<-] \rimp \bit{1}) \\
  &\bit{1} \lrimp (\inc[<-] \rimp \inc \fuse \bit{0})
\end{align*}

\begin{align*}
  &\eps \lrimp \eps[->] \\
  &\bit{0} \lrimp \bit{0}[->] \\
  &\bit{1} \lrimp \bit{1}[->] \\
  &\inc \lrimp \parens[auto, align=c@{\,}l]{
                     & (\eps[->] \limp \eps \fuse \bit{1}) \\[2pt]
               \with & (\bit{0}[->] \limp \bit{1}) \\[2pt]
               \with & (\bit{1}[->] \limp \inc \fuse \bit{0})}
\end{align*}

\NewDocumentCommand{\fch}{o m}{\IfValueTF{#1}{\monad[#1]}{\monad}{#2}}

\begin{align*}
  &\elem{M} \lrimp \elem[->]{M} \\
  &\elem{N} \lrimp (\elem[->]{M} \limp ((M > N) \uimp \fch{\elem{N} \fuse \elem{M}}))
\end{align*}

\begin{itemize}
\item Grammar of choreographies.
\item Each choreography rule must be able to fire independently of the other processes (although it may depend on messages).
\end{itemize}

%%% Local Variables:
%%% TeX-master: "proposal"
%%% End:


\section{Compiling choreographies}\label{sec:compile-choreo}

\NewDocumentCommand{\compile}{m m m m}{%
  \llbracket #2\rrbracket^{#1}_{#3} = #4%
}

\begin{mathpar}
  \infer{\compile{d}{H_1 \fuse H_2}{c}{\mbind{\bv{c'} <- \mspawn{P_1} <- d; P_2}}}{
    \compile{d}{H_1}{c'}{P_1} &
    \compile{c'}{H_2}{c}{P_2}}
  \and
  \infer{\compile{d}{\one}{c}{\mfwd{c <- d}}}{
    }
  \and
  \infer{\compile{d}{p}{c}{\mbind{c <- p <- d}}}{
    }
\end{mathpar}

\NewDocumentCommand{\compilectx}{m m m m}{%
  \compile{#1}{#2}{#3}{#4}%
}

\begin{mathpar}
  \infer{\compilectx{d}{\octx_1, \octx_2}{c}{\existq c'. A^+_1 \tensor A^+_2}}{
    \compilectx{d}{\octx_1}{c'}{A^+_1} &
    \compilectx{c'}{\octx_2}{c}{A^+_2}}
  \and
  \infer{\compilectx{d}{\octxe}{c}{c \eq d}}{
    }
  \\
  \infer{\compilectx{d}{H}{c}{\exec{P}}}{
    \compile{d}{H}{c}{P}}
\end{mathpar}


\subsection{Correctness}\label{sec:correctness}

\begin{itemize}
\item Well-typed choreographies compile to well-typed processes.
\item Executions of a choreography and its compiled form are bisimilar.
\end{itemize}

%%% Local Variables:
%%% TeX-master: "proposal"
%%% End:


% arara: lualatex
% arara: lualatex
% arara: biber
% arara: lualatex
% arara: lualatex
% \documentclass{../hdeyoung-proposal}
\documentclass[
  class=../hdeyoung-proposal,
  crop=false
]{standalone}


\usepackage{linear-logic}
\usepackage{ordered-logic}
\usepackage{proof}
\usepackage{mathpartir}

\usepackage{tikz}
\usetikzlibrary{shapes.misc,graphs,quotes,graphdrawing}
\usegdlibrary{trees}

\usepackage{scalerel}

\ExplSyntaxOn

% \DeclarePairedDelimiter \parens { \lparen } { \rparen }
\DeclarePairedDelimiter \bagged:wn { \lbag } { \rbag }
\NewDocumentCommand{ \bagged }{ s o m o }
  {
    \IfBooleanTF {#1}
      { \bagged:wn* {#3} }
      {
        \IfValueTF {#2}
          { \bagged:wn[#2] {#3} }
          { \bagged:wn {#3} }
      }
    \IfValueT {#4} { \sb{#4} }
  }


\NewDocumentCommand \oseq { >{ \SplitArgument{1}{|-} } m }
  { \oseq:nn #1 }
\cs_new:Npn \oseq:nn #1#2 { \oseq_ctxs:n {#1} \vdash #2 }
\cs_new:Npn \oseq_ctxs:n #1 {
  \seq_set_split:Nnn \l_tmpa_seq {;} {#1}
  \seq_use:Nn \l_tmpa_seq { \mathrel{;} }
}

\NewDocumentCommand \procof { m m } { #1 \dblcolon #2 }
\NewDocumentCommand \hypof { m } { #1 }


\NewDocumentCommand \cut { m } { \text{\textsc{\MakeLowercase{Cut}}}\sb{#1} }
\NewDocumentCommand \id { m } { \text{\textsc{\MakeLowercase{Id}}}\sb{#1} }

\NewDocumentCommand \comp { >{ \SplitArgument{1}{|} } m }
  { \comp:nn #1 }
\cs_new:Npn \comp:nn #1#2 { #1 \parallel #2 }

\NewDocumentCommand \fwd {} { \mathord{\leftrightarrow} }


\RenewDocumentCommand \with { s }
  { \IfBooleanTF {#1} \with:n \with: }
\cs_new:Npn \with:n #1 {
  \mathord{\binampersand}
  \bagged {
    \seq_set_split:Nnn \l_tmpa_seq {,} {#1}
    \seq_use:Nn \l_tmpa_seq {,}
  }
}
\cs_new:Npn \with: { \mathbin{\binampersand} }

\NewDocumentCommand \ssor { s }
  { \IfBooleanTF {#1} \ssor:n \ssor: }
\cs_new:Npn \ssor:n #1 {
  \mathord{\ssor:}
  \bagged {
    \seq_set_split:Nnn \l_tmpa_seq {,} {#1}
    \seq_use:Nn \l_tmpa_seq {,}
  }
}
\cs_new:Npn \ssor: { \oplus }

\NewDocumentCommand \caseR { s m }
  {
    \IfBooleanTF {#1}
      { \case:nNn { \mathsf{caseR} } \bagged {#2} }
      { \case:nNn { \mathsf{caseR} } \parens {#2} }
  }
\NewDocumentCommand \caseL { s m }
  {
    \IfBooleanTF {#1}
      { \case:nNn { \mathsf{caseL} } \bagged {#2} }
      { \case:nNn { \mathsf{caseL} } \parens {#2} }
  }
\cs_new:Npn \case:nNn #1#2#3 {
  #1 \mskip\thinmuskip
  #2 {
    \seq_set_split:Nnn \l_tmpa_seq {|} {#3}
    \seq_clear:N \l_tmpb_seq
    \seq_map_inline:Nn \l_tmpa_seq
      { \seq_put_right:Nn \l_tmpb_seq { \case_branch:n {##1} } }
    \seq_use:Nn \l_tmpb_seq { \talloblong }
  }
}
\cs_new:Npn \case_branch:n #1 { \case_branch_aux:w #1 \q_stop }
\cs_new:Npn \case_branch_aux:w #1 => #2 \q_stop {
  #1 \Rightarrow #2
}

\NewDocumentCommand \selectL { >{ \SplitArgument{1}{;} } m }
  { \select:nnn { \mathsf{selectL} } #1 }
\NewDocumentCommand \selectR { >{ \SplitArgument{1}{;} } m }
  { \select:nnn { \mathsf{selectR} } #1 }
\cs_new:Npn \select:nnn #1#2#3 {
  \!\mathord{}\mathop{#1} #2 ; #3
}


\NewDocumentCommand \inj { m } { \mathsf{in}\sb{#1} }

\NewDocumentCommand \inl {} { \inj{ \mathsf{1} } }
\NewDocumentCommand \inr {} { \inj{ \mathsf{2} } }


\RenewDocumentCommand \one {} { \mathord { \mathbf{1} } }

\NewDocumentCommand \quitR {} { \mathsf{quitR} }
\NewDocumentCommand \waitL { m } { \mathsf{waitL} ; #1 }


\NewDocumentCommand \rrule { o m } {
  \IfValueTF {#1}
    { \rrule:nn {#2} {#1} }
    { \rrule:n {#2} }
}
\cs_new:Npn \rrule:nn #1#2 { {#1}\text{\textsc{\MakeLowercase{R}}}\sb{#2} }
\cs_new:Npn \rrule:n #1 { {#1}\text{\textsc{\MakeLowercase{R}}} }

\NewDocumentCommand \lrule { o m } {
  \IfValueTF {#1}
    { \lrule:nn {#2} {#1} }
    { \lrule:n {#2} }
}
\cs_new:Npn \lrule:nn #1#2 { {#1}\text{\textsc{\MakeLowercase{L}}}\sb{#2} }
\cs_new:Npn \lrule:n #1 { {#1}\text{\textsc{\MakeLowercase{L}}} }


\NewDocumentCommand \exec { } { \mathsf{exec} \mskip\thinmuskip }
\NewDocumentCommand \msg { } { \mathsf{msg} \mskip\thinmuskip }

\ExplSyntaxOff




\addbibresource{../proposal.bib}

\NewDocumentCommand{\ie}{}{i.e.}


\usepackage{listings}
\crefname{listing}{listing}{listings}
\Crefname{listing}{Listing}{Listings}

\newlength{\mywidth}
\settowidth{\mywidth}{\ttfamily A}
\lstset{basicstyle=\ttfamily, basewidth=\mywidth}

\captionsetup[lstlisting]{%
  box=colorbox, boxcolor=gray,
  font={normalfont, sf, color=white},
  labelfont=bf,
  justification=justified, singlelinecheck=false
}

\lstnewenvironment{sillcode}[1][]
  {\lstset{language={},frame=bottomline,framerule=0.8ex,rulecolor=\color{gray},float,#1}}%
  {}

\lstnewenvironment{sillcode*}[1][]
  {\lstset{language={},#1}}%
  {}

\NewDocumentCommand{\sillinline}{o}{%
  \IfValueTF{#1}{\lstinline[#1]}{\lstinline}%
}


\NewDocumentCommand{\pctx}{}{\Psi}
\ExplSyntaxOn
\NewDocumentCommand{\ctxmonad}{>{\SplitArgument{1}{<-}}m}{
  \{\use_ii:nn #1 \vdash \use_i:nn #1\}
}
\NewDocumentCommand \spawn { >{ \SplitArgument{1}{;} } m } { \spawn:nn #1 }
\cs_new:Npn \spawn:nn #1#2 {
  \mathsf{spawn}
  \tl_if_empty:nF {#1} {
    \mskip\thinmuskip #1 ; #2
  }
}
\NewDocumentCommand{\mbind}{>{\SplitArgument{1}{;}}m}{
  \use_i:nn#1 ; \use_ii:nn#1
}
\NewDocumentCommand{\mletrec}{m m}{
  \mathsf{letrec} \mskip\thinmuskip #1 \mskip\thinmuskip \mathsf{in} \mskip\thinmuskip #2
}
\NewDocumentCommand{\mprocdef}{m}{
  #1
}
\ExplSyntaxOff




\DeclareAcronym{BHK}{
  short = BHK,
  long  = Brouwer-Heyting-Kolmogorov
}

\DeclareAcronym{JILL}{
  short = JILL,
  long  = judgmental intuitionistic linear logic
}

\DeclareAcronym{ILL}{
  short = ILL,
  long  = intuitionistic linear logic
}

\DeclareAcronym{SML}{
  short = SML,
  long  = Standard ML
}

\DeclareAcronym{SOS}{
  short = SOS,
  short-indefinite = an,
  long = structural operational semantics
}

\DeclareAcronym{SSOS}{
  short = SSOS,
  short-indefinite = an,
  long = substructural operational semantics
}

\DeclareAcronym{SILL}{
  short = SILL,
  long = session-typed intuitionistic linear logic
}

\DeclareAcronym{CLF}{
  short = CLF,
  long = the concurrent logical framework
}



\begin{document}

\section{Proposed work}\label{sec:proposed-work}

In this document, we have shown how session types form a bridge between a class of ordered logical specifications and process definitions typed in singleton linear logic---between proof-construction-as-computation and proof-reduction-as-computation.
The primary area of proposed work is to extend this connection to (a class of) linear logical specifications and \ac{SILL} process definitions typed in linear logic.
% To defend the proposed thesis, this connection must be extended to (a class of) linear logical specifications and \ac{SILL} process definitions typed in linear logic.

\subsection{From ordered logical specifications to linear logical specifications}\label{sec:from-ordered-logical}

Although awkward, it is possible to give an ordered logical specification for addition of two binary numbers.
The idea is to arrange the numbers end-to-end, but separated by a $\plus$ marker and terminated by a $\equals$ marker.
For instance, the string
\begin{equation*}
  \eps \fuse \bit{1} \fuse \bit{0} \fuse \plus \fuse \bit{1} \fuse \bit{0} \fuse \equals
\end{equation*}
would represent a request to add $2+2$.
By decrementing the second number to $0$ and incrementing

\begin{equation*}
  \begin{lgathered}
    \bit{0} \fuse \dec \lrimp \monad{\dec \fuse \bit[']{0}} \\
    \bit{1} \fuse \dec \lrimp \monad{\skip \fuse \bit{0} \fuse \ok} \\
    \plus \fuse \dec \lrimp \monad{\fail} \\
    %
    \bit{0} \fuse \skip \lrimp \monad{\skip \fuse \bit{0}} \\
    \bit{1} \fuse \skip \lrimp \monad{\skip \fuse \bit{1}} \\
    \plus \fuse \skip \lrimp \monad{\inc \fuse \plus} \\
    %
    \ok \fuse \bit[']{0} \lrimp \monad{\bit{1} \fuse \ok} \\
    \fail \fuse \bit[']{0} \lrimp \monad{\fail} \\
    %
    \ok \fuse \equals['] \lrimp \monad{\equals} \\
    \fail \fuse \equals['] \lrimp \monad{\one}
  \end{lgathered}
\end{equation*}


\begin{sillcode}
  plus =
  { caseR of
      dec => selectR fail; <->
    | skip_inc => selectL inc; plus }

  bit0 =
  { caseR of
      dec => selectL dec; bit0'
    | skip_inc => selectL skip_inc; bit0 }

  bit0' =
  { caseL of
      ok => selectR ok; bit1
    | fail => selectR fail; <-> }

  bit1 =
  { caseR of
      dec => selectR ok; selectL skip_inc; bit0
    | skip_inc => selectL skip_inc; bit1 }

  equals =
  { selectL dec; equals' }

  equals' =
  { caseL of
      ok => equals
    | fail => <-> }
\end{sillcode}


A better algorithm would add the two numbers bit-by-bit.
Pictorially, 
\begin{equation*}
  \begin{tikzpicture}
    \graph [tree layout, math nodes] {
      + <- { / 0 <- / 1 <- / e ,
/ 0 <- / 1 <- / e };
};
  \end{tikzpicture}
\end{equation*}

\end{document}

Functional logic languages, such as Curry~\autocite{Hanus:Ganzinger13}, view functional programs as logic programs.
This proposal takes the opposite approach, compiling a class of logic programs to functional programs.

\Citeauthor{Simmons+Zerny:LICS13}'s correspondence between natural semantics (functional program) and abstract machines (logic program)~\autocite*{Simmons+Zerny:LICS13}.

%%% Local Variables:
%%% TeX-master: "proposal"
%%% End:


%\nocite{*}

\printbibliography

\end{document}
