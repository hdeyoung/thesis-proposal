% arara: pdflatex
% arara: pdflatex
% arara: biber
% arara: pdflatex
% arara: pdflatex
% \documentclass{../hdeyoung-proposal}
\documentclass[
  class=../hdeyoung-proposal,
  crop=false
]{standalone}

\usepackage{ordered-logic}
\usepackage{basic-atoms}
\usepackage{binary-counter}

\crefname{choreography}{chor.}{chors.}
\Crefname{choreography}{Chor.}{Chors.}

\DeclareAcronym{BHK}{
  short = BHK,
  long  = Brouwer-Heyting-Kolmogorov
}

\DeclareAcronym{JILL}{
  short = JILL,
  long  = judgmental intuitionistic linear logic
}

\DeclareAcronym{ILL}{
  short = ILL,
  long  = intuitionistic linear logic
}

\DeclareAcronym{SML}{
  short = SML,
  long  = Standard ML
}

\DeclareAcronym{SOS}{
  short = SOS,
  short-indefinite = an,
  long = structural operational semantics
}

\DeclareAcronym{SSOS}{
  short = SSOS,
  short-indefinite = an,
  long = substructural operational semantics
}

\DeclareAcronym{SILL}{
  short = SILL,
  long = session-typed intuitionistic linear logic
}

\DeclareAcronym{CLF}{
  short = CLF,
  long = the concurrent logical framework
}


\begin{document}

\subsection{Choreographies by example}\label{sec:chor-by-example}

\subsubsection{The binary counter}\label{sec:chor-example-counter}

In giving the intuition behind the binary counter specification (\cref{sec:olp-intuition:binary-counter}), we described the $\inc$ atoms % as moving --- moving past any $\bit{1}$s and eventually stopping at the $\eps$ or right-most $\bit{0}$.
as moving up the counter.
% % , a subliminal hint that $\inc$s are like messages.
% % This suggests a choreography in which $\inc$ processes take the active lead:
% This hints that $\inc$s are a bit like messages, and suggests a choreography in which $\inc$ processes initiate the interaction:
% First, each $\inc$ process sends a message, $\inc[<-]$, to its left-hand neighbor, thereby notifying that neighbor of its existence, and then the $\inc$ process terminates.
% If the neighbor is $\eps$, $\bit{0}$, or $\bit{1}$, then, upon receiving the $\inc$'s message, that neighbor takes full responsibility for completing the corresponding interaction.
This hints at a choreography in which $\inc$ atoms act as messages that trigger the increment action:
Whenever an $\inc$ message arrives at an $\eps$, $\bit{0}$, or $\bit{1}$ process, that process takes responsiblity for completing the increment action.
% First, each $\inc$ process sends a message, $\inc[<-]$, to its left-hand neighbor, thereby notifying that neighbor of its existence, and then the $\inc$ process terminates.
% If the neighbor is $\eps$, $\bit{0}$, or $\bit{1}$, then, upon receiving the $\inc$'s message, that neighbor takes full responsibility for completing the corresponding interaction.


% In giving the intuition behind the binary counter specification (\cref{sec:olp-intuition:binary-counter}), we described the $\inc$ atoms as moving up the counter.
% % This hints that $\inc$s are a bit like messages, and suggests a choreography in which $\inc$ processes initiate the interaction:
% % First, each $\inc$ process sends a message, $\inc[<-]$, to its left-hand neighbor, thereby notifying that neighbor of its existence, and then the $\inc$ process terminates.
% % If the neighbor is $\eps$, $\bit{0}$, or $\bit{1}$, then, upon receiving the $\inc$'s message, that neighbor takes full responsibility for completing the corresponding interaction.
% This hints at a choreography in which $\inc$ atoms are messages that trigger the increment action by $\eps$, $\bit{0}$, and $\bit{1}$ atoms that act as processes.
% When the $\eps$, $\bit{0}$, or $\bit{1}$, processes receive the $\inc[<-]$ message,  then, upon receiving the $\inc$'s message, that neighbor takes full responsibility for completing the corresponding interaction.

Expressed as an ordered logic program in its own right, this choreography is:
\begin{equation}\label[choreography]{chor:oop-counter}
  \!\begin{aligned}
    &\eps \fuse \inc[<-] \lrimp \monad{\eps \fuse \bit{1}} \\
    &\bit{0} \fuse \inc[<-] \lrimp \monad{\bit{1}} \\
    &\bit{1} \fuse \inc[<-] \lrimp \monad{\inc[<-] \fuse \bit{0}}
    \text{\,,}
  \end{aligned}
\end{equation}
where the $\eps$, $\bit{0}$, and $\bit{1}$ atoms are viewed as processes, but the $\inc[<-]$ atoms are viewed as messages.
%
Two properties are noteworthy:
\begin{description}[font=\normalfont\itshape, leftmargin=\parindent, labelindent=\leftmargin]
\item[Locality.]
  \fxnote{\st{In this choreography,\ }}Each clause's premise depends on exactly one process atom and (at most) one message atom.
  Consequently, each process's decisions are entirely local: the $\eps$, $\bit{0}$, and $\bit{1}$ processes act (independently) only after receiving an $\inc[<-]$ message.%
  \footnote{In {SSOS} terminology, processes that wait to receive a message, like $\eps$, $\bit{0}$, and $\bit{1}$ here, would be termed \vocab{latent} propositions; and messages, like $\inc[<-]$ here, would be termed \vocab{passive} propositions.}

  Locality serves to ensure that the choreography may be read as a sensible message-passing implementation.
  A clause such as $\inc[<-] \lrimp \monad{{\dots}}$, whose premise does not contain a process atom, is not message-passing because no process receives the $\inc[<-]$ message.
%
\item[Specification-preserving.]
% % % Second, notice that
% % The choreography exposes the same $\eps$, $\bit{}$, and $\inc$ processes as the original binary counter specification; the last three clauses of the choreography differ from the specification's clauses only in the substitution of $\inc[<-]$ for $\inc$ in their premises.
% The choreography exposes the same $\eps$, $\bit{}$, and $\inc$ processes as the original binary counter specification.
% Its clauses differ from those of the specification only in the substitution of $\inc[<-]$ for $\inc$ in their premises.
% (The choreography also includes an $\inc \lrimp \monad{\inc[<-]}$ clause to justify that substitution.)
% In this sense, there is a very strong equivalence between the two programs.
% The choreography does not fundamentally alter the specification---it only refines that specification by making the communication patterns explicit.
%
The choreography exposes the same $\eps$, $\bit{}$, and $\inc$ behaviors as the original specification.
Its clauses are exactly those of the specification, except that each $\inc$ atom in the specification has been annotated as an $\inc[<-]$ message in the choreography.
In this sense, there is a very strong, lock-step equivalence between the choreography and its specification.
The choreography does not fundamentally alter the specification---it only refines that specification by making the communication patterns explicit.
\end{description}
%
% In this sense, there is a strong equivalence between the 
% The choreography does not fundamentally alter the implementation given in the original program---it only refines that implementation by making the communication patterns explicit.
% In this sense, there is a strong equivalence between, which will be made precise in \cref{??}
%
% Notice that this choreography \wc{refactors} the original program so that each new clause depends on exactly one process atom and at most one message atom.
% In this way, each process's decisions are completely local: the $\inc$ process always sends $\inc[<-]$ regardless of its neighbors, and the $\eps$ and $\bit{}$ processes act only after receiving an $\inc[<-]$ message.%
% \footnote{In \ac{SSOS} terminology, processes that act regardless of their neighbors, like $\inc$, would be termed \vocab{active} propositions; processes that wait to receive a message, like $\eps$, $\bit{0}$, and $\bit{1}$, would be termed \vocab{latent} propositions; and messages, like $\inc[<-]$, would be termed \vocab{passive} propositions.}
%
It's convenient to think of the programmer as supplying this choreography in full, but in practice the programmer might only give the substitution, \eg\ $\inc[<-]$ for $\inc$.

\subsubsection{Messages can flow in both directions}\label{sec:chor-binary-count}

In our binary counter specification with decrements (\cref{sec:olp-intuition:decrements}), $\dec$ atoms propagate up the counter similarly to $\inc$s, with the difference that each $\dec$ atom eventually gives rise to either a $\fail$ or $\suc$ atom that travels back down the counter.
Once again, this hints at a choreography in which $\dec$, $\fail$, and $\suc$ atoms are messages:
\begin{itemize}
\item Whenever a $\dec[<-]$ message arrives at an $\eps$, $\bit{0}$, or $\bit{1}$ process's right-hand side, that process completes the local decrement action:
      the $\eps$ and $\bit{1}$ processes send a $\fail[->]$ or $\suc[->]$ message, respectively, to their right;
      the $\bit{0}$ process forwards the $\dec[<-]$ message to its left and continues as a $\bit[']{0}$ process.
\item Whenever a $\fail[->]$ or $\suc[->]$ message arrives at a $\bit[']{0}$ process's left-hand side, that process forwards the message to its right-hand neighbor and continues as a $\bit{0}$ or $\bit{1}$ process, respectively.
% \item Each $\dec$ process sends a message, $\dec[<-]$, to its left-hand neighbor and terminates.
%       If the neighbor is $\eps$, $\bit{0}$, or $\bit{1}$, then, upon receiving the message, that neighbor completes the corresponding interaction given in the specification.
% \item Each $\fail$ or $\suc$ process sends a message, $\fail[->]$ or $\suc[->]$, respectively, to its \emph{right-hand} neighbor and terminates.
%       If the neighbor is $\bit[']{0}$, then, upon receiving the message from $\fail$ or $\suc$, that neighbor completes the corresponding interaction.
\end{itemize}
To account for decrements, the binary counter's choreography is therefore extended with the following clauses:
\begin{equation}
  \!\begin{aligned}
    &\eps \fuse \dec[<-] \lrimp \monad{\eps \fuse \fail[->]} \\
    &\bit{0} \fuse \dec[<-] \lrimp \monad{\dec[<-] \fuse \bit[']{0}} \\
    &\bit{1} \fuse \dec[<-] \lrimp \monad{\bit{0} \fuse \suc[->]} \\[1.5\jot]
    % 
    &\fail[->] \fuse \bit[']{0} \lrimp \monad{\bit{0} \fuse \fail[->]} \\
    &\suc[->] \fuse \bit[']{0} \lrimp \monad{\bit{1} \fuse \suc[->]}
    \,.
  \end{aligned}
\end{equation}
% Once again, the atoms that are decorated with arrows are formally distinct from their undecorated counterparts.
(As before, the atoms that are decorated with arrows are formally distinct from their undecorated counterparts.)

This extended choreography illustrates that message atoms may be either left-directed, like $\inc[<-]$ and $\dec[<-]$, or right-directed, like $\zero[->]$ and $\suc[->]$.
% Moreover, a message's direction determines the structure of premises in which it is received:
% a left-directed (right-directed) message must arrive at the receiving process's right (resp., left) side, otherwise the message would not be traveling from left to right (resp., right to left).
% 
% Because it is traveling left-to-right, a left-directed message must always arrive at the right-hand side of its recipient; dually, a right-directed message must always arrive at the left-hand side of its recipient.
Because a left-directed message travels from right to left, it must always arrive at the right-hand side of its recipient; dually, a right-directed message must always arrive at the left-hand side of its recipient.
This directionality is another aspect of locality, and it further constrains the structure of a choreography's premises.  For instance, $\bit{1} \fuse \inc[<-]$ and $\zero[->] \fuse \bit[']{0}$ are well-formed premises because each message flows toward its recipient, whereas premises of the forms $\matom[<-] \fuse \p$ or $\p \fuse \matom[->]$ are not well-formed because process $\p$ doesn't receive the $\matom[<-]$ or $\matom[->]$ message.

As well as retaining locality, notice that this extended choreography continues to be specification-preserving:
the choreography's clauses differ from those of the specification only in the substitution of $\dec[<-]$, $\zero[->]$, and $\suc[->]$ for their undecorated counterparts.


\subsubsection{Choreographies are not always unique}\label{sec:mult-chor-are}

As alluded to previously, multiple choreographies are possible for some specifications.

This is true of our binary counter specification, for instance.
(To simplify the example, we'll ignore decrements for now.)
In the first choreography (\cref{chor:oop-counter}),
% the $\inc$ processes initiate the interaction but leave all remaining work to the $\eps$, $\bit{0}$, and $\bit{1}$ processes alone.
the counter's value is represented by a chain of $\eps$, $\bit{0}$, and $\bit{1}$ processes that are acted upon by $\inc[<-]$ messages.
%
% Alternatively, the $\inc$ processes could wait for $\eps$, $\bit{0}$, or $\bit{1}$ to initiate the interaction, but thereafter take full responsibility for its completion.
Alternatively, the counter's value could be represented by a sequence of $\eps[->]$, $\bit{0}[->]$, and $\bit{1}[->]$ messages; when fed such a message sequence, an $\inc$ process emits another sequence that represents the result:
%
% Specifically, each $\eps$, $\bit{0}$, and $\bit{1}$ process sends an identifying message, $\eps[->]$, $\bit{0}[->]$, or $\bit{1}[->]$, to its right-hand neighbor and then terminates.
% If the neighbor is $\inc$, then, upon receiving the message, that $\inc$ completes the corresponding interaction.
% % takes responsibility for carrying out the corresponding clause of the specification.
\begin{equation}
  \!\begin{aligned}
    &\eps[->] \fuse \inc \lrimp \monad{\eps[->] \fuse \bit{1}[->]} \\
    &\bit{0}[->] \fuse \inc \lrimp \monad{\bit{1}[->]} \\
    &\bit{1}[->] \fuse \inc \lrimp \monad{\inc \fuse \bit{0}[->]}
      \,.
  \end{aligned}
\end{equation}
Once again, this choreography possesses the locality and specification-preserving properties.

% Owing to the difference in roles held by, these two choreographies have distinct flavors.
% These two choreographies
These two choreographies
% presented thus far
have distinct flavors, owing to the different process and message roles that they assign to the $\inc$ and $\eps$, $\bit{0}$, and $\bit{1}$ atoms.
The first choreography has an object-oriented character: by sending an $\inc[<-]$ message, the increment method dispatches on the receiving object's class---either $\eps$, $\bit{0}$, or $\bit{1}$.
In contrast, this new choreography has a functional character: $\inc$ is a function that receives its argument as a sequence of messages---either $\eps[->]$, $\bit{0}[->]$, or $\bit{1}[->]$.

% The increment method dispatches on 
% $\inc$ invokes the increment method on the neighboring object by sending an $\inc[<-]$ message
% There, the $\inc$ method sends an $\inc[<-]$ message like a method that dispatches on the class of the recipient object---either $\eps$, $\bit{0}$, or $\bit{1}$.
% Our first choreography has an object-oriented flavor, with $\inc$ like a method that dispatches on the class of the recipient object---either $\eps$, $\bit{0}$, or $\bit{1}$.
% In contrast, this second choreography has a more functional flavor, with 

% This alternate choreography has a funcitonal flavor: $\inc$ can be viewed as a function on the $\eps$-and-$\bit{}$ representation of data.
% In contrast, the previous choreography has a more object-oriented flavor

% The difference in sender and recipient between this alternate choreography and the previous one gives the two choreographies different flavors.
% In this alternate choreography
% In constrast, the previous choreography has a more object-oriented flavor, with $\inc$ being a method that dispatches on the class of the recipient---either $\eps$, $\bit{0}$, or $\bit{1}$.

% If we view the $\eps$ and $\bit{}$ processes as data, then this alternate choreography has a functional flavor.
% In contrast, the previous choreography has an object-oriented flavor, with a dynamic dispatch of $\inc[<-]$ on the recipient.

\subsubsection{Two non-choreographies}\label{sec:non-choreographies}

Another, slightly more complex implementation of the binary counter specification chooses to treat the $\inc$ atom as a simple process, not a message:
\begin{equation}
  \!\begin{aligned}
    &\inc \lrimp \monad{\incm[<-]} \\[1.5\jot]
    % 
    &\eps \fuse \incm[<-] \lrimp \monad{\eps \fuse \bit{1}} \\
    &\bit{0} \fuse \incm[<-] \lrimp \monad{\bit{1}} \\
    &\bit{1} \fuse \incm[<-] \lrimp \monad{\inc \fuse \bit{0}}
      \,.
  \end{aligned}
\end{equation}
In fact, the $\inc$ process does nothing but send an $\incm[<-]$ message.

This implementation is equivalent to the binary counter specification in that it ultimately exposes the same $\eps$, $\bit{}$, and $\inc$ processes.
However, it is not specification-preserving under the informal definition that we have used thus far.
% In contrast with the choreographies, this implementation's main clauses are not a simple decoration of the specification's clauses: each of the specification's clauses is spread across several clauses here.
In contrast with the choreographies, this implementation does more than simply refine the specification by making the communication explicit: using a temporary $\inc[^r]$ process, it spreads each of the specification's clauses across several clauses.






Another, more complex implementation of the binary counter specification divides the work of completing the interaction among the $\eps$ and $\bit{}$ and (the continuation of) the $\inc$ processes:
\begin{equation}
  \!\begin{aligned}
    &\inc \lrimp \monad{\inc[^l, <-] \fuse \inc[^r]} \\[1.5\jot]
    % 
    &\eps \fuse \inc[^l, <-] \lrimp \monad{\eps \fuse \eps[^r, ->]} \\
    &\bit{0} \fuse \inc[^l, <-] \lrimp \monad{\bit{0}[^r, ->]} \\
    &\bit{1} \fuse \inc[^l, <-] \lrimp \monad{\inc \fuse \bit{1}[^r, ->]} \\[1.5\jot]
    % 
    &\eps[^r, ->] \fuse \inc[^r] \lrimp \monad{\bit{1}} \\
    &\bit{0}[^r, ->] \fuse \inc[^r] \lrimp \monad{\bit{1}} \\
    &\bit{1}[^r, ->] \fuse \inc[^r] \lrimp \monad{\bit{0}} \,.
  \end{aligned}
\end{equation}
In this implementation, $\inc$ first sends an $\inc[^l, <-]$ message to its left-hand neighbor and then waits for a response as process $\inc[^r]$.
Upon receiving an $\inc[^l, <-]$ message, the recipient process, either $\eps$, $\bit{0}$, or $\bit{1}$, partially completes the interaction and sends an identifying message to its right-hand neighbor, which is necessarily an $\inc[^r]$ process.
The $\inc[^r]$ process finishes the interaction once it receives the identifying message.

This implementation is equivalent to the binary counter specification in that it ultimately exposes the same $\eps$, $\bit{}$, and $\inc$ processes.
However, it is not specification-preserving under the informal definition that we have used thus far.
% In contrast with the choreographies, this implementation's main clauses are not in one-to-one correspondence with those of the specification.
% In contrast with the choreographies, this implementation's main clauses are not a simple decoration of the specification's clauses: each of the specification's clauses is spread across several clauses here.
In contrast with the choreographies, this implementation does more than simply refine the specification by making the communication explicit: using a temporary $\inc[^r]$ process, it spreads each of the specification's clauses across several clauses.

Although this implementation is not specification-preserving for the binary counter specification, and therefore not a choreography, the programmer can nevertheless achieve the same behavior by changing the \emph{specification}.
If the specification is changed to be
\begin{equation}
  % \!\begin{aligned}[t]
  %   &\inc[^l] \lrimp \monad{\inc[^l, <-]} \\
  %   &\eps[^r] \lrimp \monad{\eps[^r, ->]} \\
  %   &\bit{0}[^r] \lrimp \monad{\bit{0}[^r, ->]} \\
  %   &\bit{1}[^r] \lrimp \monad{\bit{1}[^r, ->]}
  % \end{aligned}
  % \qquad
  \!\begin{aligned}
    &\inc \lrimp \monad{\inc[^l] \fuse \inc[^r]} \\
    &\eps \fuse \inc[^l] \lrimp \monad{\eps \fuse \eps[^r]} \\
    &\bit{0} \fuse \inc[^l] \lrimp \monad{\bit{0}[^r]} \\
    &\bit{1} \fuse \inc[^l] \lrimp \monad{\inc \fuse \bit{1}[^r]} \\
    &\eps[^r] \fuse \inc[^r] \lrimp \monad{\bit{1}} \\
    &\bit{0}[^r] \fuse \inc[^r] \lrimp \monad{\bit{1}} \\
    &\bit{1}[^r] \fuse \inc[^r] \lrimp \monad{\bit{0}} \,,
  \end{aligned}
\end{equation}
then the above implementation is a choreography for \emph{this} specification.

% Although this implimentation is not specification-preserving, and therefore not a choreography, for the binary counter specification (\cref{??}), the programmer can nevertheless acheive the same behavior by changing the \emph{specification}.
% Instead of using the binary counter specification from \cref{??}, the following specification could be used because one of its choreographies, which uses $\inc[^l, <-]$, $\eps[^r, ->]$, $\bit{0}[^r, ->]$, and $\bit{1}[^r, ->]$ messages, is nearly identical the disallowed implimentation.
% \begin{equation*}
%   % \!\begin{aligned}[t]
%   %   &\inc[^l] \lrimp \monad{\inc[^l, <-]} \\
%   %   &\eps[^r] \lrimp \monad{\eps[^r, ->]} \\
%   %   &\bit{0}[^r] \lrimp \monad{\bit{0}[^r, ->]} \\
%   %   &\bit{1}[^r] \lrimp \monad{\bit{1}[^r, ->]}
%   % \end{aligned}
%   % \qquad
%   \!\begin{aligned}[t]
%     &\inc \lrimp \monad{\inc[^l] \fuse \inc[^r]} \\
%     &\eps \fuse \inc[^l] \lrimp \monad{\eps \fuse \eps[^r]} \\
%     &\bit{0} \fuse \inc[^l] \lrimp \monad{\bit{0}[^r]} \\
%     &\bit{1} \fuse \inc[^l] \lrimp \monad{\inc \fuse \bit{1}[^r]} \\
%     &\eps[^r] \fuse \inc[^r] \lrimp \monad{\bit{1}} \\
%     &\bit{0}[^r] \fuse \inc[^r] \lrimp \monad{\bit{1}} \\
%     &\bit{1}[^r] \fuse \inc[^r] \lrimp \monad{\bit{0}} \,.
%   \end{aligned}
% \end{equation*}


% Although this implementation ultimately exposes the same $\eps$, $\bit{}$, and $\inc$ processes

% Even so\fxnote{\ \st{Even though this choreography introduces and auxiliary process atom}}, we still consider it to be a valid choreography: $\inc[']$ is only temporary, leaving the underlying specification fundamentally unchanged.

Another program that is equivalent to the binary counter specification, in the sense that the two track the same value, is
\begin{equation}
  \!\begin{aligned}
    &\num{N} \fuse \inc[<-] \lrimp \monad{\num{(N{+}1)}} \,.
  \end{aligned}
\end{equation}
Nevertheless, we wouldn't consider this program to be a \emph{choreography} of the binary counter specification because, by using a single number held by $\num{}$ instead of a string of $\bit{}$s, it fundamentally alters the specification.
We would, however, consider this program to be a choreography of a different, simple counter specification, namely $\num{N} \fuse \inc \lrimp \monad{\num{(N{+}1)}}$.

% % Although it is equivalent to the binary counter in the sense that it tracks the same value, we wouldn't consider the following program to be a choreography of the binary counter specification because it fundamentally alters the implementation by using a single $\num{}$ instead of a string of $\bit{}$s.
% The following program is also equivalent to the binary counter specification, in the sense that the two track the same value.
% Nevertheless, we wouldn't consider it to be a choreography of the binary counter because it fundamentally alters the specification by using a single number held by $\num{}$ instead of a string of $\bit{}$s.
% \begin{align*}
%   &\inc \lrimp \monad{\inc[<-]} \\
%   &\num{N} \fuse \inc[<-] \lrimp \monad{\num{(N{+}1)}}
% \end{align*}
% We would, however, consider it to be a choreography of a different, simple counter specification: $\num{N} \fuse \inc \lrimp \monad{\num{(N{+}1)}}$.



\end{document}

%%% Local Variables:
%%% TeX-master: "choreographies"
%%% End:
