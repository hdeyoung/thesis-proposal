% arara: pdflatex
% arara: biber
% arara: pdflatex
% arara: pdflatex
\documentclass[
  class=../hdeyoung-proposal,
  crop=false
]{standalone}

\usepackage{ordered-logic}
\usepackage{binary-counter}
\usepackage{proof}
\usepackage{tikz}
\usepackage{tikz-cd}
\usepackage{subcaption}

\NewDocumentEnvironment{subtheorems}{m O{}}%
  {%
    \begin{enumerate}%
      [label=\alph*.,ref=\csname the#1\endcsname\alph*,#2]%
    \crefalias{enumi}{#1}%
    \ignorespaces
  }%
  {\end{enumerate}\ignorespacesafterend}

\NewDocumentCommand{\trans}{t* t+ o o}{%
  \longrightarrow
  \IfBooleanT{#1}{^*}\IfBooleanT{#2}{^+}%
  \IfNoValueTF{#4}{%
    \IfValueT{#3}{_{#3}}%
  }{%
    ^{#3}_{#4}%
  }%
}
\NewDocumentCommand{\ntrans}{}{
  \longarrownot\trans
}

\DeclareAcronym{CLF}{
  short = CLF,
  long = the concurrent logical framework
}

\NewPredicate{\Cntr}[Cntr][font = \mathit]{0}
\NewPredicate{\cntr}[Cntr]{0}

\NewPredicate{\coin}{0}
\NewPredicate{\heads}{0}
\NewPredicate{\tails}{0}
\NewPredicate{\win}{0}
\NewPredicate{\loss}{0}


\begin{document}

\NewDocumentEnvironment{ lpsig }{ }%
  { \begin{aligned} }%
  { \end{aligned} }

\ExplSyntaxOn
\NewDocumentCommand{\mathul}{m}{
  \mathpalette\mathul:Nn{#1}
}
\cs_new:Npn \mathul:Nn #1#2 {
  \tikz [baseline] {
    \node (pr) [anchor = base, inner~sep = 0em] {$#1#2$};
    \draw [overlay, ultra~thick, gray]
      ([yshift=-0.3em]pr.base~west) -- ([yshift=-0.3em]pr.base~east);
  }
}
\ExplSyntaxOff

\subsection{Example: Binary counter}\label{sec:olp-intuition:binary-counter}

Using an ordered logic program, we can specify an incrementable binary counter.
The counter is represented as a string of $\bit{0}$ and $\bit{1}$ atoms terminated at the most significant end by an $\eps$.
For instance, the ordered conjunction $\eps \fuse \bit{1} \fuse \bit{0}$ is a string that represents a counter with value $2$.
%
Increment instructions are represented by $\inc$ atoms at the counter's least significant end. 
% Also interspersed are $\inc$ atoms, each of which serves as an increment instruction sent to the counter given by the more significant bits.
Thus, $\eps \fuse \bit{1} \fuse \inc$ represents a counter with value $1$ that has been instructed to increment once.

Operationally, increments are described by three clauses, which together constitute the program, or \vocab{signature}, $\sig_{\inc}$.
The first of these clauses is
\begin{equation*}
  % \begin{lpsig}
  %   \inc/1 &
    \bit{1} \fuse \inc \lrimp \monad{\inc \fuse \bit{0}}
    \,.
  % \end{lpsig}
\end{equation*}
From a string rewriting perspective, this implication is a rule for rewriting the (sub)string $\bit{1} \fuse \inc$ as $\inc \fuse \bit{0}$, an interpretation which is justified logically because implications are transformations.%
\footnote{See \cref{??} for technical justification.}
% % From a string rewriting perspective, this implication is interpreted as a rule for rewriting the (sub)string $\bit{1} \fuse \inc$ as $\inc \fuse \bit{0}$.%
%  in the sense that the following bottom-up rule
% % This clause allows the string $\bit{1} \fuse \inc$ to be rewritten as $\inc \fuse \bit{0}$, in the sense that the rule
% is admissible whenever this clause is part of the persistent context, $\uctx$.
% \begin{equation*}
%   \infer{\uctx ; \omatch{\bit{1} \fuse \inc} \seq C \lax}{
%     \uctx ; \ofill{\inc \fuse \bit{0}} \seq C \lax}
% \end{equation*}}
%
% Just as this implication transforms 
% When this proposition is part of the persistent context $\uctx$, the rule
% \begin{equation*}
%   \infer{\uctx ; \omatch{\bit{1} \fuse \inc} \seq C \lax}{
%     \uctx ; \ofill{\inc \fuse \bit{0}} \seq C \lax}
%   \,.
% \end{equation*}
% is admissible.
% Read bottom-up, this rule serves to rewrite the (sub)string $\bit{1} \fuse \inc$ as the string $\inc \fuse \bit{0}$.
%
By rewriting $\bit{1} \fuse \inc$ as $\inc \fuse \bit{0}$, this clause serves to carry the $\inc$ up past any $\bit{1}$s that may exist at the counter's least significant end.

Whenever the carried $\inc$ reaches the $\eps$ or right-most $\bit{0}$, the carry is resolved by one of the other two clauses:
\begin{equation*}
  \begin{lgathered} % \begin{lpsig}
    % \inc/e &
    \eps \fuse \inc \lrimp \monad{\eps \fuse \bit{1}} \\
    % \inc/0 &
    \bit{0} \fuse \inc \lrimp \monad{\bit{1}}
    \,.
  \end{lgathered} % \end{lpsig}
\end{equation*}
By rewriting $\eps \fuse \inc$ as $\eps \fuse \bit{1}$, this second clause ensures that, in the $\eps$ case, the carry becomes a new most significant $\bit{1}$;
by rewriting $\bit{0} \fuse \inc$ as $\bit{1}$, the third clause ensures that, in the $\bit{0}$ case, the carry flips the $\bit{0}$ to $\bit{1}$.

For example, under the $\sig_{\inc}$ signature\rlap{,}%
\footnote{For precision, the rewriting relation, $\trans$, is indexed by the signature of clauses that may contribute to the rewriting, such as $\sig_{\inc}$ in $\trans[\sig_{\inc}]$ here.
We frequently omit the index when it is clear from the context.}
 the counter $\eps \fuse \bit{1} \fuse \inc$ can be maximally rewritten as in the trace
\begin{equation*}
  \eps \fuse \mathul{\bit{1} \fuse \inc}
    \trans[\sig_{\inc}] \mathul{\eps \fuse \inc} \fuse \bit{0}
    \trans[\sig_{\inc}] \eps \fuse \bit{1} \fuse \bit{0}
    \ntrans[\sig_{\inc}]
  \text{\,,}
\end{equation*}
where at each step the sites amenable to rewriting have been underlined.
This trace computes $1 + 1 = 2$ in binary representation.
More generally, the clauses adequately specify an increment operation, as will be described further in \cref{sec:gener-invar}.

\subsection{Concurrency}\label{sec:olp-intuition:concurrency}

Some strings contain more than one site that is amenable to rewriting.
% several disjoint substrings that are amenable to rewriting.
% If these sites are disjoint, the rewrites can be thought of as happening concurrently.
For instance, the binary counter $\eps \fuse \bit{1} \fuse \inc \fuse \inc$ has two $\inc$s in flight, which, after one step, give rise to two disjoint rewrite sites:
\begin{equation*}
  % \eps \fuse \mathul{\bit{1} \fuse \inc} \fuse \inc \trans
  \mathul{\eps \fuse \inc} \fuse \mathul{\bit{0} \fuse \inc}
  \,.
\end{equation*}
The rewritings at these sites can be interleaved in two ways: either
% \begin{align*}
%   &\mathul{\eps \fuse \inc} \fuse \mathul{\bit{0} \fuse \inc}
%      \trans \eps \fuse \bit{1} \fuse \mathul{\bit{0} \fuse \inc}
%      \trans \eps \fuse \bit{1} \fuse \bit{1}
%      \ntrans \\
%   %
%   \shortintertext{or}
%   %
%   &\mathul{\eps \fuse \inc} \fuse \mathul{\bit{0} \fuse \inc}
%      \trans \mathul{\eps \fuse \inc} \fuse \bit{1}
%      \trans \eps \fuse \bit{1} \fuse \bit{1}
%      \ntrans
%    \,.
% \end{align*}
according to the upper path or the lower path in the following diagram.
\begin{equation*}
  \begin{tikzcd}[row sep=1ex, arrow style=math font]
    &
    \eps \fuse \bit{1} \fuse \mathul{\bit{0} \fuse \inc}
      \drar
    \\
    \mathul{\eps \fuse \inc} \fuse \mathul{\bit{0} \fuse \inc}
      \urar \drar
    &&
    \eps \fuse \bit{1} \fuse \bit{1}
      \ntrans
    \\
    &
    \mathul{\eps \fuse \inc} \fuse \bit{1}
      \urar
  \end{tikzcd}
\end{equation*}

However, because these two rewritings are independent, they should be considered morally concurrent.
Rather than giving a truly concurrent semantics for string rewriting, we can treat different interleavings of independent steps as indistinguishable.
Then, because we can't observe which rewriting occurred first, the two rewritings appear to happen concurrently.

In the above example, indistinguishability of the two different interleavings is tantamount to requiring the diagram to commute.

This is the idea of \vocab{concurrent equality} from {CLF}~\autocite{Watkins+:CMU02}, borrowed for ordered logic programming.
Concurrent equality justifies adopting true concurrency as an abstraction.


\subsection{Example: Binary counter with decrements}\label{sec:olp-intuition:decrements}

As a further example, it's also possible to add support for decrements to the binary counter.
% above ordered logic program.
Like increments, a decrement instruction is represented by a $\dec$ atom at the counter's least significant end.
To perform the decrement, a $\dec$ begins propagating up the counter.
As it passes over any $\bit{0}$s at the least significant end, they are marked as $\bit[']{0}$s to indicate that they are waiting to borrow from their more significant neighbors:
\begin{equation*}
  \bit{0} \fuse \dec \lrimp \monad{\dec \fuse \bit[']{0}} \,.
\end{equation*}
Whenever it reaches the $\eps$ or right-most $\bit{1}$, the $\dec$ is replaced with either $\fail$ or $\suc$, respectively, to show whether the borrow was possible; in the case of $\bit{1}$, the borrow is also effected:
\begin{equation*}
  \begin{lgathered}
    \eps \fuse \dec \lrimp \monad{\eps \fuse \fail} \\
    \bit{1} \fuse \dec \lrimp \monad{\bit{0} \fuse \suc} \,.
  \end{lgathered}
\end{equation*}
Then the $\fail$ or $\suc$ travels back over all of the $\bit[']{0}$s that were waiting to borrow.
In the case of $\fail$, the bits are returned to their original $\bit{0}$ state because no borrow was possible;
in the case of $\suc$, a borrow was performed and so the bits are set to $\bit{1}$:
\begin{equation*}
  \begin{lgathered}
    \fail \fuse \bit[']{0} \lrimp \monad{\bit{0} \fuse \fail} \\
    \suc \fuse \bit[']{0} \lrimp \monad{\bit{1} \fuse \suc} \,.
  \end{lgathered}
\end{equation*}

% According to the following rewrite rules, a $\dec$ propogates up the counter past any $\bit{0}$s until it reaches an $\eps$ or the right-most $\bit{1}$.
% At this point, the $\dec$ is replaced with either $\zero$ or $\suc$, respectively.
% \begin{align*}
%   &\bit{0} \fuse \dec \lrimp \monad{\dec \fuse \bit[']{0}} \\
%   &\eps \fuse \dec \lrimp \monad{\eps \fuse \zero} \\
%   &\bit{1} \fuse \dec \lrimp \monad{\bit{0} \fuse \suc}
% \end{align*}
% Then the $\zero$ or $\suc$ traveles back down the counter
% \begin{align*}
%   &\zero \fuse \bit[']{0} \lrimp \monad{\bit{0} \fuse \zero} \\
%   &\suc \fuse \bit[']{0} \lrimp \monad{\bit{1} \fuse \suc}
% \end{align*}


For example, the counter $\eps \fuse \bit{1} \fuse \bit{0} \fuse \dec$ can be maximally rewritten as
\begin{align*}
  \MoveEqLeft[0.5]
  \eps \fuse \bit{1} \fuse \mathul{\bit{0} \fuse \dec} \\
    &\trans \eps \fuse \mathul{\bit{1} \fuse \dec} \fuse \bit[']{0} \\
    &\trans \eps \fuse \bit{0} \fuse \mathul{\suc \fuse \bit[']{0}} \\
    &\trans \eps \fuse \bit{0} \fuse \bit{1} \fuse \suc \\
    &\ntrans
\end{align*}
Once again, there are possibilities for concurrency.
For example, the following two traces are indistinguishable because they differ only in the order of independent rewritings:
\begin{align*}
  &\mathul{\eps \fuse \inc} \fuse \mathul{\bit{0} \fuse \dec} \trans \mathul{\eps \fuse \inc} \fuse \dec \fuse \bit[']{0} \trans \eps \fuse \mathul{\bit{1} \fuse \dec} \fuse \bit[']{0} \\
  %
  \shortintertext{and}
  %
  &\mathul{\eps \fuse \inc} \fuse \mathul{\bit{0} \fuse \dec} \trans \eps \fuse \bit{1} \fuse \mathul{\bit{0} \fuse \dec} \trans \eps \fuse \mathul{\bit{1} \fuse \dec} \fuse \bit[']{0}
\end{align*}
This justifies treating those two rewritings as concurrent.


\begin{figure}[!t]
  \centering
  \subcaptionbox{Increments}{$\displaystyle
    \sig_{\inc} =
    \!\begin{aligned}[t]
      &\bit{1} \fuse \inc \lrimp \monad{\inc \fuse \bit{0}} \,, \\
      &\eps \fuse \inc \lrimp \monad{\eps \fuse \bit{1}} \,, \\
      &\bit{0} \fuse \inc \lrimp \monad{\bit{1}} \\
      &{}
    \end{aligned}
  $}%
  \qquad\qquad
  \subcaptionbox{Decrements}{$\displaystyle
    \mathllap{\sig_{\dec}} =
    \!\begin{aligned}[t]
      &\bit{0} \fuse \dec \lrimp \monad{\dec \fuse \bit[']{0}} \,, \\
      &\eps \fuse \dec \lrimp \monad{\eps \fuse \fail} \,, \\
      &\bit{1} \fuse \dec \lrimp \monad{\bit{0} \fuse \suc} \,, \\
      &\fail \fuse \bit[']{0} \lrimp \monad{\bit{0} \fuse \fail} \,, \\
      &\suc \fuse \bit[']{0} \lrimp \monad{\bit{1} \fuse \suc}
    \end{aligned}
    $}
  \caption{Summary of binary counter example\label{fig:binary-counter-summary}}
\end{figure}

\Cref{fig:binary-counter-summary} provides a summary of the increment and decrement programs.


% C ::= eps | C * bit0 | C * bit1 | C * inc


% c <- dec <- d =
% { case d of
%     eps => wait d;
%            d' <- eps;
%            c <- zero <- d'
%   | bit0 => d' <- dec <- d
%             c <- bit0' <- d'
%   | bit1 => d' <- bit0 <- d
%             c <- succ <- d' }

% c <- zero <- d =
% { case c of
%     bit0' => d' <- bit0 <- d
%              c <- zero <- d' }

% Cntr = +{ eps: 1 , bit0: Cntr , bit1: Cntr }
% Cntr' = &{ bit0': Cntr' }

% bit0 : {Cntr |- Cntr}
% dec : {Cntr |- Cntr'}
% zero : {Cntr |- Cntr'}
% bit0' : {Cntr' |- Cntr'}




\subsection{Infinite traces and fairness}\label{sec:olp-intuition:infinite-traces}

\NewPredicate{\incs}{0}%
Thus far, all traces have been finite, but this is not necessarily so.
Consider adding an atom, $\incs$, that generates a stream of $\inc$ atoms:
\begin{equation*}
  \incs \lrimp \monad{\inc \fuse \incs}
  \,.
\end{equation*}
Among the infinite traces now possible is
% Beginning from $\eps \fuse \incs$, there are now infinite trace
\begin{equation*}
  \eps \fuse \mathul{\incs}
    \trans \mathul{\eps \fuse \inc} \fuse \mathul{\incs}
    \trans \dotsb
    \trans \mathul{\eps \fuse \inc} \fuse \inc \fuse \dotsb \fuse \inc \fuse \mathul{\incs}
    \trans \dotsb
  \,.
\end{equation*}

Notice that this trace never rewrites the infinitely available $\eps \fuse \inc$ substring, instead always choosing to rewrite $\incs$.
Because of its scheduling bias against rewriting $\eps \fuse \inc$, we say that the trace is \vocab{(weakly) transition-unfair}.

Transition unfairness is fundamentally at odds with\fxnote{\ \st{true}} concurrency.
Because\fxnote{\ \st{true}} concurrency allows multiple independent events to occur simultaneously, one event cannot preclude another independent event;
in the above example, for instance, rewriting the $\incs$ atom would not preclude rewriting the independent $\eps \fuse \inc$ substring.
Therefore, to maintain the pretense of true concurrency, we must require all traces to be transition-fair.


% Transition unfairness is fundamentally at odds with true concurrency
% because true concurrency allows multiple independent events to occur simultaneously;
% the occurrence of one event cannot preclude another independent event from occurring at the same time.
% To maintain the pretense of true concurrency, 

% In the above example, $\eps \fuse \inc$ and $\incs$ are independent rewrite sites

% In the above example, $\eps \fuse \inc$ and $\incs$ are independent rewrite sites.
% % ; if they are truly concurrent, the rewrite of $\eps \fuse \inc$ should eventually occur.
% To maintain any pretense of true concurrency, the rewrite of $\eps \fuse \inc$ should eventually occur since true concurrency would allow multiple independent events to occur simultaneously.



% However, by treating different interleavings of independent steps as indistinguishable, these two rewrites happen concurrently.




% For instance, notice that this binary counter specification allows multiple $\inc$s to be in flight at once, each of which is amenable to rewriting.



% Sometimes several disjoint substrings are amenable to rewriting.


% Different interleavings of independent steps are indistinguishable.

% String rewriting (and therefore ordered logic programming) gives rise to 


% The counter $\eps \fuse \inc \bit{0} \fuse \inc$ has two $\inc$s in flight, which can be rewritten independently.


% Rewrites of disjoint substrings can be thought of as happening concurrently.

% Notice that we can also allow multiple $\inc$s to be in flight at once, and that independent rewrites can thought of as happening concurrently.
% For instance, the counter $\eps \fuse \inc \fuse \bit{0} \fuse \inc$ has two $\inc$s in flight, and they give rise to independent rewrites.
% \begin{center}
%   \begin{tikzpicture}
%     \matrix [matrix of math nodes, column sep = 1.5em]
%     {
%       % First row
%       & |(inc-1-2)| \eps \fuse \bit{1} \fuse \mathul{\bit{0} \fuse \inc} & \\
%       % Second row
%       |(inc-2-1)| \mathul{\eps \fuse \inc} \fuse \mathul{\bit{0} \fuse \inc}
%         && |(inc-2-3)| \eps \fuse \bit{1} \fuse \bit{1} \\
%       % Third row
%       & |(inc-3-2)| \mathul{\eps \fuse \inc} \fuse \bit{1} & \\
%     };

%     \begin{scope}
%     [ start chain, every join/.style={->} ]
%       \chainin (inc-2-1);
%       \begin{scope}[start branch=inc-1-2]
%         \chainin (inc-1-2) [join];
%       \end{scope}
%       \begin{scope}[start branch=inc-3-2]
%         \chainin (inc-3-2) [join];
%       \end{scope}
%       \chainin (inc-2-3) [join = with inc-1-2, join = with inc-3-2];
%     \end{scope}
%   \end{tikzpicture}
% \end{center}


\subsection{Committed choice}\label{sec:olp-intuition:committed-choice}

% In our forward-chaining ordered logic programming language, we assume a \vocab{committed-choice} semantics, meaning that when multiple rewritings are possible at a given site the choice is never reconsidered.
In forward-chaining logic programming, one often assumes a \vocab{committed-choice} semantics, meaning that when several rewritings are possible at a given site, one is chosen and that choice is never reconsidered.
For our purpose, this semantics is reasonable because we are interested in \emph{simulating} concurrent systems, not in generating all possible executions for model-checking.


Committed choice is not readily apparent in the binary counter example;
instead, consider a single-player game in which the player wins if a
% each of two coin tosses land heads.
coin toss lands heads.
As a forward-chaining ordered logic program, this game is specified by the signature $\sig_{\coin}$, consisting of four clauses:
\begin{equation*}
  \sig_{\coin} =
  \!\begin{aligned}[t]
    &\coin \lrimp \monad{\heads} \,, \\
    &\coin \lrimp \monad{\tails} \,, \\
    &\heads \lrimp \monad{\win} \,, \\
    &\tails \lrimp \monad{\loss}
    \,.
  \end{aligned}
\end{equation*}
The first two clauses specify that a $\coin$ may land either $\heads$ or $\tails$, and the remaining clauses specify the winning condition.

% One way in which the starting string $\coin \fuse \coin$ of two coins can be maximally rewritten is
One way in which the initial string, $\coin$, can be maximally rewritten is
\begin{equation*}
  \mathul{\coin}
    \trans \mathul{\tails}
    \trans \loss
    \ntrans
  \,.
\end{equation*}
In other words, here the $\coin$ toss lands $\tails$, resulting in a $\loss$.
To get a $\win$, we'd like to somehow take back the toss and have it instead land $\heads$.
% Here the first $\coin$ lands $\tails$ and the second $\coin$ lands $\heads$, resulting in a $\loss$.
% To get a $\win$, we'd like to somehow take back the first toss and have it instead land $\heads$.
This violation of the game's rules is exactly what the committed-choice semantics proscribes: once the coin is tossed, we must commit to that toss's outcome.


% \subsection{Example: Binary counter with zero-tests}\label{sec:olp-intuition:zero-tests}

% It's also possible to add support for zero-tests to the counters.
% Like increment instructions, a zero-test instruction is represented by a $\zeroq$ atom at the counter's least significant end.
% To perform the test, the $\zeroq$ query begins propagating up the counter.
% As it passes over any $\bit{0}$s at the least significant end, they are marked as $\bit[']{0}$s to indicate that they are waiting to borrow from their more significant neighbors:
% \begin{equation*}
%   \bit{0} \fuse \zeroq \lrimp \monad{\zeroq \fuse \bit[']{0}} \,.
% \end{equation*}
% Whenever it reaches the $\eps$ or right-most $\bit{1}$, the $\zeroq$ is replaced with either $\zero$ or $\pos$, respectively, to show whether the counter has a value that is $0$ or positive:
% \begin{align*}
%   &\eps \fuse \zeroq \lrimp \monad{\eps \fuse \zero} \\
%   &\bit{1} \fuse \zeroq \lrimp \monad{\bit{1} \fuse \pos} \,.
% \end{align*}
% Then the $\zero$ or $\pos$ travels back over all of the $\bit[']{0}$s that were waiting to borrow.
% In the case of $\zero$, the bits are returned to their original $\bit{0}$ state because no borrow was possible;
% in the case of $\suc$, a borrow was performed and so the bits are set to $\bit{1}$:
% \begin{align*}
%   &\zero \fuse \bit[']{0} \lrimp \monad{\bit{0} \fuse \zero} \\
%   &\suc \fuse \bit[']{0} \lrimp \monad{\bit{0} \fuse \suc} \,.
% \end{align*}

% For example, the counter $\eps \fuse \bit{1} \fuse \bit{0} \fuse \dec$ can be maximally rewritten as
% \begin{align*}
%   \MoveEqLeft[0.5]
%   \eps \fuse \bit{1} \fuse \mathul{\bit{0} \fuse \dec} \\
%     &\trans \eps \fuse \mathul{\bit{1} \fuse \dec} \fuse \bit[']{0} \\
%     &\trans \eps \fuse \bit{0} \fuse \mathul{\suc \fuse \bit[']{0}} \\
%     &\trans \eps \fuse \bit{0} \fuse \bit{1} \fuse \suc \\
%     &\ntrans
% \end{align*}
% Once again, there are possibilities for concurrency.
% For example, the following two traces are indistinguishable because they differ only in the order of independent rewrites:
% \begin{align*}
%   &\mathul{\eps \fuse \inc} \fuse \mathul{\bit{0} \fuse \dec} \trans \mathul{\eps \fuse \inc} \fuse \dec \fuse \bit[']{0} \trans \eps \fuse \mathul{\bit{1} \fuse \dec} \fuse \bit[']{0} \\
%   %
%   \shortintertext{and}
%   %
%   &\mathul{\eps \fuse \inc} \fuse \mathul{\bit{0} \fuse \dec} \trans \eps \fuse \bit{1} \fuse \mathul{\bit{0} \fuse \dec} \trans \eps \fuse \mathul{\bit{1} \fuse \dec} \fuse \bit[']{0}
% \end{align*}
% This justifies treating those two rewrites as concurrent.


% \begin{figure}[t]
%   \centering
%   \subcaptionbox{Increments}{$\displaystyle
%     \sig_{\inc} =
%     \!\begin{aligned}[t]
%       &\bit{1} \fuse \inc \lrimp \monad{\inc \fuse \bit{0}} \,, \\
%       &\eps \fuse \inc \lrimp \monad{\eps \fuse \bit{1}} \,, \\
%       &\bit{0} \fuse \inc \lrimp \monad{\bit{1}} \\
%       &{}
%     \end{aligned}
%   $}%
%   \qquad\qquad
%   \subcaptionbox{Decrements}{$\displaystyle
%     \mathllap{\sig_{\zeroq}} =
%     \!\begin{aligned}[t]
%       &\bit{0} \fuse \zeroq \lrimp \monad{\zeroq \fuse \bit[']{0}} \,, \\
%       &\eps \fuse \zeroq \lrimp \monad{\eps \fuse \zero} \,, \\
%       &\bit{1} \fuse \zeroq \lrimp \monad{\bit{1} \fuse \pos} \,, \\
%       &\zero \fuse \bit[']{0} \lrimp \monad{\bit{0} \fuse \zero} \,, \\
%       &\pos \fuse \bit[']{0} \lrimp \monad{\bit{0} \fuse \pos}
%     \end{aligned}
%     $}
%   \caption{Summary of binary counter example\label{fig:binary-counter-summary}}
% \end{figure}


\subsection{Generalizing string rewriting}\label{sec:olp-intuition:gener-string-rewr}

In a rewriting setting, strings are comprised of letters, and strings, as we've seen, correspond to ordered conjunctions of atomic propositions.
Forward-chaining ordered logic programming generalizes string rewriting by allowing rewriting rules themselves to appear as \enquote{letters}.
Logically, ordered conjunctions may include implications as conjuncts.
\begin{equation*}
  \begin{lgathered}
    \eps \fuse \dec \lrimp \monad{\eps \fuse \zero} \\
    \bit{0} \fuse \dec \lrimp \monad[auto]{
                                \dec \fuse \parens[auto, align=c@{\,}l]{
                                                 & (\zero \limp \monad{\bit{0} \fuse \zero}) \\
                                           \with & (\suc \limp \monad{\bit{1} \fuse \suc})}} \\
    \bit{1} \fuse \dec \lrimp \monad{\bit{0} \fuse \suc}
  \end{lgathered}
\end{equation*}

\begin{align*}
  \MoveEqLeft[0.5]
  \eps \fuse \bit{1} \fuse \mathul{\bit{0} \fuse \dec} \\
    &\trans \eps \fuse \mathul{\bit{1} \fuse \dec} \fuse \parens[auto, align=c@{\,}l]{
                                                               & (\zero \limp \monad{\bit{0} \fuse \zero}) \\
                                                         \with & (\suc \limp \monad{\bit{1} \fuse \suc})} \\
    &\trans \eps \fuse \bit{0} \fuse \mathul{\suc \fuse \parens[auto, align=c@{\,}l]{
                                                              & (\zero \limp \monad{\bit{0} \fuse \zero}) \\
                                                        \with & (\suc \limp \monad{\bit{1} \fuse \suc})}} \\
    &\trans \eps \fuse \bit{0} \fuse \bit{1} \fuse \suc \\
    &\ntrans
\end{align*}



\subsection{Adequacy and generative invariants}\label{sec:olp-intuition:gener-invar}

Thus far, we have used ordered logic programs to specify concurrent systems, whereas the non-modal fragment of ordered logic was originally developed by \textcite{Lambek:AMM58} to describe sentence structure.
However, these two modes of use of ordered logic are not as different as they might first appear.

Recall that, in our running example of an incrementable binary counter, the counter is represented as a string of $\bit{0}$, $\bit{1}$, and $\inc$ atoms terminated at the most significant end by an $\eps$.
More precisely, a string is a well-formed binary counter if it can be generated from the $\Cntr$ nonterminal by the context-free grammar
\begin{equation*}
  \Cntr ::= \eps \mid \Cntr \fuse \bit{0} \mid \Cntr \fuse \bit{1} \mid \Cntr \fuse \inc
  \,,
\end{equation*}
which is notation for four distinct productions.

Building on \citeauthor{Lambek:AMM58}'s work, the same context-free grammar can be described in ordered logic using \vocab{generative signatures}~\autocite{Simmons:CMU12}.
Each production in the grammar (below, left) becomes a clause (below, right), with the atomic proposition $\cntr$ acting as the nonterminal:
\begin{equation*}
  \left.
  \!\begin{aligned}
    &\Cntr \to \eps \\
    &\Cntr \to \Cntr \fuse \bit{0} \\
    &\Cntr \to \Cntr \fuse \bit{1} \\
    &\Cntr \to \Cntr \fuse \inc
  \end{aligned}
  \qquad\middle\vert\qquad
  \!\begin{aligned}
    &\cntr \lrimp \monad{\eps} \\
    &\cntr \lrimp \monad{\cntr \fuse \bit{0}} \\
    &\cntr \lrimp \monad{\cntr \fuse \bit{1}} \\
    &\cntr \lrimp \monad{\cntr \fuse \inc}
    \,.
  \end{aligned}
  \right.
\end{equation*}
Just as all binary counters are generated from the $\Cntr$ nonterminal according to the above productions, so too are all binary counters generated as maximal rewritings of the $\cntr$ atom according to these clauses.

% For example, the maximal trace
% \begin{equation*}
%   \mathul{\cntr}
%     \trans[\sig_{\cntr}] \mathul{\cntr} \fuse \inc
%     \trans[\sig_{\cntr}] \mathul{\cntr} \fuse \bit{1} \fuse \inc
%     \trans[\sig_{\cntr}] \eps \fuse \bit{1} \fuse \inc
%     \ntrans[\sig_{\cntr}]
%   \,.
% \end{equation*}
% witnesses that $\eps \fuse \bit{1} \fuse \inc$ is a well-formed binary counter.
% Note how important the choice of signature is: if we also allowed increment clauses from $\sig_{\inc}$ here, this trace would no longer be maximal.

\begingroup
  \RenewPredicate{\cntr}[Cntr]{1}%
In fact, generative signatures generalize context-free grammars.
For example, here $\cntr{}$ could be predicated on a natural number, effectively generalizing the context-free grammar to have a countably infinite family of nonterminals:
% One slight generalization would be to predicate $\cntr{}$ on a natural number, effectively giving a countably infinite family of nonterminals:
\begin{equation*}
  \sig_{\cntr{}} =
  \!\begin{aligned}[t]
    &\cntr{0} \lrimp \monad{\eps} \,, \\
    &\cntr{(2N)} \lrimp \monad{\cntr{N} \fuse \bit{0}} \,, \\
    &\cntr{(2N{+}1)} \lrimp \monad{\cntr{N} \fuse \bit{1}} \,, \\
    &\cntr{(N{+}1)} \lrimp \monad{\cntr{N} \fuse \inc}
    \,.
  \end{aligned}
\end{equation*}

This generative signature allows us to formally state and prove adequacy of the incrementable binary counter program, $\sig_{\inc}$.
First, we need a few \lcnamecrefs{def:counter-wf}.
%
\begin{definition}[Counter well-formedness and other properties]\label{def:counter-wf}
  \mbox{}
  \begin{itemize}[nosep]
  \item String $S$ is a well-formed counter that represents natural number $N$ (or, more simply, $S$ represents $N$) if $S$ is a maximal rewriting of $\cntr{N}$ under the signature $\sig_{\cntr{}}$, that is, if $\cntr{N} \trans+[\sig_{\cntr{}}] S \ntrans[\sig_{\cntr{}}]$.
  \item String $S$ is a well-formed counter if there is some $N$ for which $S$ represents $N$.
  \item Well-formed counter $S$ is $\inc$-free if the well-formedness trace does not use the $\inc$ clause from the $\sig_{\cntr{}}$ signature.
  \item Likewise, well-formed counter $S$ has no leading $\bit{0}$s if the well-formedness trace uses the $\bit{0}$ clause only when $N > 0$.
  \end{itemize}
\end{definition}
%
For example, the maximal trace
\begin{equation*}
  \mathul{\cntr{2}}
    \trans[\sig_{\cntr{}}] \mathul{\cntr{1}} \fuse \inc
    \trans[\sig_{\cntr{}}] \mathul{\cntr{0}} \fuse \bit{1} \fuse \inc
    \trans[\sig_{\cntr{}}] \eps \fuse \bit{1} \fuse \inc
    \ntrans[\sig_{\cntr{}}]
\end{equation*}
witnesses that $\eps \fuse \bit{1} \fuse \inc$ is a well-formed binary counter that represents $2$; it is not $\inc$-free, but it does have no leading $\bit{0}$s.
Note how important the choice of signature is: if we also allowed increment clauses from $\sig_{\inc}$ here, this trace would no longer be maximal.

With these \lcnamecrefs{def:counter-wf} in hand, we can establish a bijection between the natural numbers and equivalence classes of well-formed counters.
\begin{theorem}[Adequacy of counters]\label{thm:counter-adequacy}
  \mbox{}
  \begin{subtheorems}{theorem}[nosep]
  \item\label{thm:counter-adequacy:value}
    For each natural number $N$, there is a unique well-formed counter $S$ that represents $N$, is $\inc$-free, and has no leading $\bit{0}$s.
  \item\label{thm:counter-adequacy:counter}
    For each well-formed counter $S$, there is a unique natural number $N$ such that $S$ represents $N$.
  \end{subtheorems}
\end{theorem}
\begin{proof}
  \Cref{thm:counter-adequacy:value} is by induction on $N$, and \cref{thm:counter-adequacy:counter} is by induction on the structure of the maximal trace that witnesses the well-formedness of $S$.
\end{proof}
%
As the following \lcnamecrefs{thm:counter-adequacy} show, the represented value is invariant under $\sig_{\inc}$-rewriting and $\sig_{\inc}$-rewriting always terminates in an $\inc$-free counter.
It follows that $\inc$s adequately specify increments.
%
% \begin{lemma}
%   \mbox{}
%   \begin{itemize}[nosep]
%   \item If $S_0 \fuse \bit{0} \trans[\sig_{\inc}] S'$, then $S' = S'_0 \fuse \bit{0}$ and $S_0 \trans[\sig_{\inc}] S'_0$.
%   \item If $S_0 \fuse \bit{1} \trans[\sig_{\inc}] S'$, then $S' = S'_0 \fuse \bit{1}$ and $S_0 \trans[\sig_{\inc}] S'_0$.
%   \item If $S_0 \fuse \inc \trans[\sig_{\inc}] S'$, then either:
%     \begin{itemize}[nosep]
%     \item $S' = S'_0 \fuse \inc$ and $S_0 \trans[\sig_{\inc}] S'_0$;
%     \item $S_0 = \eps$ and $S' = \eps \fuse \bit{1}$;
%     \item $S_0 = S_{00} \fuse \bit{0}$ and $S' = S_{00} \fuse \bit{1}$; or
%     \item $S_0 = S_{00} \fuse \bit{1}$ and $S' = S_{00} \fuse \inc \fuse \bit{0}$.
%     \end{itemize}
%   \end{itemize}
% \end{lemma}
% 
\begin{theorem}[Preservation]\label{thm:counter-preservation}
  For every well-formed counter $S$ that represents $N$, if $S \trans[\sig_{\inc}] S'$, then $S'$ is also a well-formed counter that represents $N$.
  % \begin{enumerate}[nosep]
  % \item For every well-formed counter $S$, if $S \trans[\sig_{\inc}] S'$, then $S'$ is a well-formed counter.
  % \item For all well-formed counters $S$ and $S'$, if $S$ represents $N$ and $S \trans[\sig_{\inc}] S'$, then $S'$ also represents $N$.
  % \end{enumerate}
\end{theorem}
\begin{proof}
  By induction on the structure of the maximal trace that witnesses the well-formedness of $S$, relying on an inversion lemma for $\sig_{\inc}$-steps:
  \begin{itemize}[nosep]
  \item If $S_0 \fuse \bit[_{\mathit{b}}]{} \trans[\sig_{\inc}] S'$, then $S_0 \trans[\sig_{\inc}] S'_0$ and $S' = S'_0 \fuse \bit[_{\mathit{b}}]{}$.
  \item If $S_0 \fuse \inc \trans[\sig_{\inc}] S'$, then either:
    \begin{itemize}[nosep]
    \item $S_0 \trans[\sig_{\inc}] S'_0$ and $S' = S'_0 \fuse \inc$;
    \item $S_0 = \eps$ and $S' = \eps \fuse \bit{1}$;
    \item $S_0 = S'_0 \fuse \bit{0}$ and $S' = S'_0 \fuse \bit{1}$; or
    \item $S_0 = S'_0 \fuse \bit{1}$ and $S' = S'_0 \fuse \inc \fuse \bit{0}$.
    \end{itemize}
  \end{itemize}
\end{proof}

% \begin{theorem}[Progress]
%   \mbox{}
%   \begin{enumerate}[nosep]
%   \item For every well-formed counter $S$, if $S$ is $\inc$-free, then $S \ntrans[\sig_{\inc}]$.
%   \item For every well-formed counter $S$, either $S$ is $\inc$-free or $S \trans[\sig_{\inc}] S'$ for some $S'$.
%   \end{enumerate}
% \end{theorem}
% \begin{proof}
%   By induction on the structure of the maximal trace that generates $S$.
% \end{proof}

\begin{theorem}[Termination]\label{thm:inc-termination}
  \mbox{}
  \begin{subtheorems}{theorem}[nosep]
  \item\label{thm:inc-termination:inc-free}
    For every well-formed counter $S$, string $S$ is $\inc$-free if and only if $S \ntrans[\sig_{\inc}]$.
  \item\label{thm:inc-termination:finite}
    For every well-formed counter $S$, there is no infinite $\sig_{\inc}$-rewriting of $S$.
  \end{subtheorems}
\end{theorem}
\begin{proof}
  \Cref{thm:inc-termination:inc-free} is by induction on the structure of the maximal trace that witnesses the well-formedness of $S$.
  %
  \DeclarePairedDelimiter{\meas}{\lVert}{\rVert}%
  \DeclarePairedDelimiter{\size}{\lvert}{\rvert}%
  To prove \cref{thm:inc-termination:finite}, define $\meas{S}$ to be a measure in which each $\inc$ in $S$ contributes an amount equal to the length of its higher-order substring:%
  \footnote{If desired, this measure can also be defined using a generative signature.}
  \begin{equation*}
    \!\begin{aligned}[t]
      \meas{\eps} &= 0 \\
      \meas{S \fuse \bit[_{\mathit{b}}]{}} &= %\meas{S} \\
      % \meas{S \fuse \bit{1}} = 
      \meas{S} \\
      \meas{S \fuse \inc} &= \meas{S} + \size{S}
    \end{aligned}
    \qquad
    \!\begin{aligned}[t]
      \size{\eps} &= 1 \\
      \size{S \fuse \bit[_{\mathit{b}}]{}} &=
      % \size{S \fuse \bit{1}} =
      \size{S \fuse \inc} = \size{S} + 1
    \end{aligned}
  \end{equation*}
  % Define
  % \begin{align*}
  %   &\meas{0,0} \lrimp \monad{\eps} \\
  %   &\meas{M,(L{+}1)} \lrimp \monad{\meas{M,L} \fuse \bit{0}} \\
  %   &\meas{M,(L{+}1)} \lrimp \monad{\meas{M,L} \fuse \bit{1}} \\
  %   &\meas{(M{+}L),(L{+}1)} \lrimp \monad{\meas{M,L} \fuse \inc}
  % \end{align*}
  One can show that $\meas{\mathord{-}}$ is strictly decreasing for each step $S \trans[\sig_{\inc}] S'$, from which \cref{thm:inc-termination:finite} follows.
\end{proof}

% Cntr 0 <- eps
% Cntr N+1 <- Cntr N * bit0
% Cntr 

\begin{corollary}[Adequacy of $\inc$]
  For all natural numbers $N$ and $N'$ and every well-formed counter $S$ that represents $N$, the equality $N + 1 = N'$ holds if and only if $S \fuse \inc \trans+[\sig_{\inc}] S' \ntrans[\sig_{\inc}]$ for some $S'$ that represents $N'$.
  % For all natural numbers $N$ and $N'$ and every well-formed counter $S$, if $S$ represents $N$, then $N + 1 = N'$ if and only if $S \fuse \inc \trans+[\sig_{\inc}] S' \ntrans[\sig_{\inc}]$ for some $S'$ that represents $N'$.
  % % \begin{enumerate}
  % % \item For all natural numbers $N$ and $N'$, if $S$ represents $N$, $S'$ represents $N'$, and $N + 1 = N'$, then $S \fuse \inc \trans+ S' \ntrans$.
  % % %
  % % \item For all well-formed counters $S$ and $S'$, if $S$ represents $N$, $S'$ represents $N'$, and $S \fuse \inc \trans+ S' \ntrans$, then $N + 1 = N'$.
  % % \end{enumerate}
\end{corollary}

% \begin{falseclaim}
%   For all natural numbers $N$ and $N'$ and well-formed counters $S$ and $S'$, if $S$ represents $N$ and $S'$ represents $N'$ and is $\inc$-free, then $N + 1 = N'$ if and only if $S \fuse \inc \trans+[\sig_{\inc}] S' \ntrans[\sig_{\inc}]$.
%   % \begin{enumerate}
%   % \item For all natural numbers $N$ and $N'$, if $S$ represents $N$, $S'$ represents $N'$, and $N + 1 = N'$, then $S \fuse \inc \trans+ S' \ntrans$.
%   % %
%   % \item For all well-formed counters $S$ and $S'$, if $S$ represents $N$, $S'$ represents $N'$, and $S \fuse \inc \trans+ S' \ntrans$, then $N + 1 = N'$.
%   % \end{enumerate}
% \end{falseclaim}

In this way, the generative signature $\sig_{\cntr{}}$ serves a similar purpose for logic programming as types do for functional programming~\autocite{Simmons:CMU12}: both enable preservation and progress properties for their respective notions of transition.
We will return to this point in \cref{sec:proposed-work}.

\endgroup





% stype C = &{ inc: C, dec: C', halt: X }
%   and C' = +{ zero: C, succ: C }
% eps : {C <- X}, bit0 : {C <- C}, bit1 : {C <- C}, inc : {C <- C},
% dec : {C' <- C}, zero : {C' <- C}, succ : {C' <- C}, bit0' : {C' <- C'},
% halt : {X <- C}
% 
%  C ::= X * eps | C * bit0 | C * bit1 | C * inc
% C' ::= C * dec | C * zero | C * succ | C' * bit0'
%  X ::= C * halt

% stype C = &{ inc: C, dec: C' }
%   and C' = +{ zero: C-, succ: C- }
% bit0 : {C- <- C-} /\ {C <- C}, ...
% 
% C- ::= eps | C- * bit0 | C- * bit1
%  C ::= eps | C * bit0 | C * bit1 | C * inc
% C' ::= C * dec | C- * zero | C- * succ | C' * bit0'




% \subsubsection{String rewriting rules as ordered implications.}

% Using a focused proof search strategy~\autocite{Andreoli:JLC92}, ordered implications correspond to string rewriting rules.


% \subsection{Example: Binary counter}\label{sec:exampl-binary-count-4}

% As a running example, we can implement a binary counter that supports increments.
% The counter is represented as a string of $\bit{0}$ and $\bit{1}$ letters terminated at the most significant end by an $\eps$.
% So, for instance, the ordered conjunction, or string, $\eps \fuse \bit{1} \fuse \bit{0}$ represents a counter with value $2$.
% Also interspersed are $\inc$ atoms, each of which serves as an increment instruction sent to the counter given by the more significant bits.
% Thus, $\eps \fuse \bit{1} \fuse \inc$ represents a counter with value $1$ that has been sent an increment instruction.

% Operationally, increments are described by three ordered implications that correspond to string rewriting rules; the first of these is
% \begin{equation*}
%   \bit{1} \fuse \inc \rimp \monad{\inc \fuse \bit{0}} \,.
% \end{equation*}
% By rewriting the string $\bit{1} \fuse \inc$ as $\inc \fuse \bit{0}$, this rule carries the $\inc$ up past any $\bit{1}$s at the counter's least significant end.
% Whenever the carried $\inc$ reaches the $\eps$ or right-most $\bit{0}$, the carry is resolved:
% \begin{align*}
%   &\eps \fuse \inc \lrimp \monad{\eps \fuse \bit{1}} \\
%   &\bit{0} \fuse \inc \lrimp \monad{\bit{1}} \,.
% \end{align*}
% By rewriting $\eps \fuse \inc$ as $\eps \fuse \bit{1}$, the second rule ensures that the carry becomes a new most significant $\bit{1}$ in the $\eps$ case.
% By rewriting $\bit{0} \fuse \inc$ as $\bit{1}$, the third rule ensures that the carry flips the $\bit{0}$ to $\bit{1}$ in that case.



% When this proposition is part of the persistent context $\uctx$, the following rule is derivable:
% \begin{equation*}
%   \infer{\uctx ; \omatch{\bit{1}, \inc} \seq J}{
%     \uctx ; \ofill{\inc, \bit{0}} \seq J}
%   \,.
% \end{equation*}
% Read bottom-up, this derived rule rewrites part of the ordered context so that $\bit{1}, \inc$ becomes $\inc, \bit{0}$.


% % \begin{equation*}
% %   \infer[\lab{copy}]{\uctx ; \omatch{\bit{1}, \inc} \seq J}{
% %     \infer[\llab{{\rimp}}]{\uctx ; \ofill{\bit{1}, \inc, (\bit{1} \fuse \inc \rimp \inc \fuse \bit{0})} \seq J}{
% %       \infer[\rlab{{\fuse}}]{\uctx ; \bit{1}, \inc \seq \bit{1} \fuse \inc}{
% %         \infer[\lab{id}]{\uctx ; \bit{1} \seq \bit{1}}{
% %           } &
% %         \infer[\lab{id}]{\uctx ; \inc \seq \inc}{
% %           }} &
% %       \infer[\llab{{\fuse}}]{\uctx ; \ofill{\inc \fuse \bit{0}} \seq J}{
% %         \uctx ; \ofill{\inc, \bit{0}} \seq J}}}
% % \end{equation*}




% \subsection{Example: Binary counter}\label{sec:exampl-binary-count-2}

% As an example of an ordered logic program, we can implement a binary counter that supports increments.
% Similarly to the process implementation from \cref{sec:exampl-binary-count}, the counter is represented as a list of $\bit{0}$ and $\bit{1}$s terminated at the most significant end by an $\eps$.
% Here, however, the $\bit{}$s and $\eps$ are not processes, but rather atomic propositions (or, in string rewriting terminology, letters).
% For instance, the ordered conjunction (or string) $\eps \fuse \bit{1} \fuse \bit{0}$ represents a counter with value $2$.

% % An increment instruction is represented by an $\inc$ atom at the counter's least significant end.
% % There are three rewrite rules that describe the increment operation:
% % \begin{align*}
% %   &\eps \fuse \inc \lrimp \monad{\eps \fuse \bit{1}} \\
% %   &\bit{0} \fuse \inc \lrimp \monad{\bit{1}} \\
% %   &\bit{1} \fuse \inc \lrimp \monad{\inc \fuse \bit{0}}
% % \end{align*}
% % By rewriting $\eps \fuse \inc$ as $\eps \fuse \bit{1}$, the first rule introduces $\bit{1}$ as a new most significant bit, and thereby serves to increment an $\eps$.
% % % By rewriting $\eps \fuse \inc$ as $\eps \fuse \bit{1}$ and thereby introducing $\bit{1}$ as a new most significant bit, the first rule serves to increment $\eps$s.
% % Likewise, the second rule serves to increment a counter whose least significant bit is $\bit{0}$, by rewriting $\bit{0} \fuse \inc$ as $\bit{1}$ and thereby flipping $\bit{0}$.
% % % Likewise, by rewriting $\bit{0} \fuse \inc$ as $\bit{1}$ and thereby flipping $\bit{0}$, the second rule serves to increment a counter whose least significant bit is $\bit{0}$.
% % Finally, by rewriting $\bit{1} \fuse \inc$ as $\inc \fuse \bit{0}$, the third rule flips $\bit{1}$ and propogates a carry to the more significant bits, thereby serving to increment a counter whose least significant bit is $\bit{1}$.

% An increment instruction is represented by an $\inc$ atom at the counter's least significant end.
% There are three rewrite rules that describe increments, the first of which is
% \begin{equation*}
%   \bit{1} \fuse \inc \lrimp \monad{\inc \fuse \bit{0}} \,.
% \end{equation*}
% By rewriting $\bit{1} \fuse \inc$ as $\inc \fuse \bit{0}$, this rule carries the $\inc$ up past any $\bit{1}$s at the counter's least significant end.
% Whenever the carried $\inc$ reaches the $\eps$ or right-most $\bit{0}$, the carry is resolved:
% \begin{align*}
%   &\eps \fuse \inc \lrimp \monad{\eps \fuse \bit{1}} \\
%   &\bit{0} \fuse \inc \lrimp \monad{\bit{1}} \,.
% \end{align*}
% By rewriting $\eps \fuse \inc$ as $\eps \fuse \bit{1}$, the second rule ensures that the carry becomes a new most significant $\bit{1}$ in the $\eps$ case.
% By rewriting $\bit{0} \fuse \inc$ as $\bit{1}$, the third rule ensures that the carry flips the $\bit{0}$ to $\bit{1}$ in that case.


% \subsection{Adequacy and generative invariants}\label{sec:gener-invar}

% Thus far, we have used ordered logic programs to specify concurrent systems, whereas the non-modal fragment of ordered logic was originally developed by \textcite{Lambek:AMM58} to describe sentence structure.
% However, these two modes of use of ordered logic are not as different as they might first appear.

% Recall that, in our running example of an incrementable binary counter, the counter is represented as a string of $\bit{0}$, $\bit{1}$, and $\inc$ atoms terminated at the most significant end by an $\eps$.
% More precisely, a string is a well-formed binary counter if it can be generated from the $\Cntr$ nonterminal by the context-free grammar
% \begin{gather*}
%   \Cntr ::= \eps \mid \Cntr \fuse \bit{0} \mid \Cntr \fuse \bit{1} \mid \Cntr \fuse \inc
%   \,,
% %
% \intertext{which is just an abbreviated notation for four distinct productions:}
% %
%   \begin{aligned}
%     &\Cntr \to \eps \\
%     &\Cntr \to \Cntr \fuse \bit{0} \\
%     &\Cntr \to \Cntr \fuse \bit{1} \\
%     &\Cntr \to \Cntr \fuse \inc
%     \,.
%   \end{aligned}
% \end{gather*}

% Building on \citeauthor{Lambek:AMM58}'s work, the same context-free grammar can be described in ordered logic using \vocab{generative signatures}~\autocite{Simmons:CMU12}.
% Each production in the grammar becomes a clause, with the atomic proposition $\cntr$ acting as the nonterminal:
% \begin{equation*}
%   \begin{aligned}
%     &\cntr \lrimp \monad{\eps} \\
%     &\cntr \lrimp \monad{\cntr \fuse \bit{0}} \\
%     &\cntr \lrimp \monad{\cntr \fuse \bit{1}} \\
%     &\cntr \lrimp \monad{\cntr \fuse \inc}
%     \,.
%   \end{aligned}
% \end{equation*}
% A string $S$ is then a \vocab{well-formed binary counter} whenever $\cntr \trans+[\sig_{\cntr}] S \ntrans[][\sig_{\cntr}]$, that is, whenever $S$ is a maximal rewriting of $\cntr$ under these clauses.%
% \footnote{%
% The relation $\trans+[\sig]$ is the transitive, but not reflexive, closure of $\trans[\sig]$;
% and $\trans*[\sig]$ is the reflexive, transitive closure of $\trans[\sig]$.
% }
% For example, the maximal trace
% \begin{equation*}
%   \mathul{\cntr}
%     \trans[\sig_{\cntr}] \mathul{\cntr} \fuse \inc
%     \trans[\sig_{\cntr}] \mathul{\cntr} \fuse \bit{1} \fuse \inc
%     \trans[\sig_{\cntr}] \eps \fuse \bit{1} \fuse \inc
%     \ntrans[\sig_{\cntr}]
%   \,.
% \end{equation*}
% witnesses that $\eps \fuse \bit{1} \fuse \inc$ is a well-formed binary counter.
% Note how important the choice of signature is: if we also allowed increment clauses from $\sig_{\inc}$ here, this trace would no longer be maximal.

% % Then, just as all well-formed binary counters are generated from the $\Cntr$ nonterminal according to the above productions, so are all binary counters generated as maximal rewritings of the $\cntr$ atom according to these clauses.
% % % Just as a string is well-formed binary counter if it can be generated from the $\Cntr$ nonterminal by the above context-free grammar, so too is a string well-formed if it can be generated by maximally rewriting the $\cntr$ atom.
% % For example, $\eps \fuse \bit{1} \fuse \inc$ is a well-formed binary counter because it is a maximal rewriting of $\cntr$ under the above clauses:
% % \begin{equation*}
% %   \mathul{\cntr}
% %     \trans[\sig_{\cntr}] \mathul{\cntr} \fuse \inc
% %     \trans[\sig_{\cntr}] \mathul{\cntr} \fuse \bit{1} \fuse \inc
% %     \trans[\sig_{\cntr}] \eps \fuse \bit{1} \fuse \inc
% %     \ntrans[\sig_{\cntr}]
% %   \,.
% % \end{equation*}
% % % Note that restricting to the $\sig_{\cntr}$ signature---and not $\sig_{\inc}$---
% % Note that this trace is maximal because only clauses from $\sig_{\cntr}$ are permitted; if .

% \begingroup
%   \RenewPredicate{\cntr}[Cntr]{1}%
% Generative signatures in fact generalize context-free grammars.
% An example generalization is to predicate $\cntr{}$ on a natural number (and renaming it $\cntr[_R]{}$), effectively giving a countably infinite family of nonterminals.
% We say that $S$ \vocab{represents} $N$ if it is a maximal rewriting of $\cntr[_R]{N}$ under the signature $\sig_{\cntr[_R]{}}$ consisting of the first-order clauses
% % Thus, a binary counter is well-formed \emph{and} represents $N$ if it is a maximal rewriting of $\cntr{N}$ according to the first-order ordered logic program
% \begin{equation*}
%   \begin{aligned}
%     &\cntr[_R]{0} \lrimp \monad{\eps} \\
%     &\cntr[_R]{(2N)} \lrimp \monad{\cntr[_R]{N} \fuse \bit{0}} \\
%     &\cntr[_R]{(2N{+}1)} \lrimp \monad{\cntr[_R]{N} \fuse \bit{1}} \\
%     &\cntr[_R]{(N{+}1)} \lrimp \monad{\cntr[_R]{N} \fuse \inc}
%     \,.
%   \end{aligned}
% \end{equation*}

% This generative signature allows us to formally state (and prove) adequacy of the incrementable binary counter program.
% % \begin{definition}
% %   A string is \vocab{quiescent} if $S \ntrans$.
% %   A string $S$ is a \vocab{well-formed counter} if $\cntr{} \trans+ S \ntrans$.
% %   A string $S$ \vocab{represents} natural number $N$ if $\cntr{N} \trans+ S \ntrans$.
% % \end{definition}

% \begin{theorem}[Preservation]\mbox{}
%   \begin{enumerate}[nosep]
%   \item For every well-formed counter $S$, if $S \trans[\sig_{\inc}] S'$, then $S'$ is a well-formed counter.
%   \item For all well-formed counters $S$ and $S'$, if $S$ represents $N$ and $S \trans[\sig_{\inc}] S'$, then $S'$ also represents $N$.
%   \end{enumerate}
% \end{theorem}
% \begin{proof}
%   By induction on the structure of the maximal trace that generates $S$.
% \end{proof}

% \begin{theorem}[Progress]
%   \mbox{}
%   \begin{enumerate}[nosep]
%   \item For every well-formed counter $S$, if $S$ is $\inc$-free, then $S \ntrans[\sig_{\inc}]$.
%   \item For every well-formed counter $S$, either $S$ is $\inc$-free or $S \trans[\sig_{\inc}] S'$ for some $S'$.
%   \end{enumerate}
% \end{theorem}
% \begin{proof}
%   By induction on the structure of the maximal trace that generates $S$.
% \end{proof}

% \begin{theorem}[Termination]
%   For every well-formed counter $S$, there is a unique string $S'$ such that $S \trans*[\sig_{\inc}] S' \ntrans[\sig_{\inc}]$.
% \end{theorem}
% \begin{proof}
%   By induction on the structure of the maximal trace that generates $S$.
% \end{proof}

% \begin{theorem}[Adequacy of counters]
%   \mbox{}
%   \begin{enumerate}[nosep]
%   \item For each natural number $N$, there is a unique well-formed counter $S$ that represents $N$, is $\inc$-free, and has no leading $\bit{0}$s.
%   \item\label{item:cntr-adeq-value}
%     For each well-formed counter $S$, there is a unique natural number $N$ such that $S$ represents $N$.
%   \end{enumerate}
% \end{theorem}
% \begin{proof}
%   Part~1 is by induction on $N$; part~\ref{item:cntr-adeq-value} is by induction on the structure of the maximal trace that generates $S$.
% \end{proof}


% \begin{corollary}[Adequacy of $\inc$]
%   For all natural numbers $N$ and $N'$ and every well-formed counter $S$, if $S$ represents $N$, then $N + 1 = N'$ if and only if $S \fuse \inc \trans+[\sig_{\inc}] S' \ntrans[\sig_{\inc}]$ for some $S'$ that represents $N'$.
%   % \begin{enumerate}
%   % \item For all natural numbers $N$ and $N'$, if $S$ represents $N$, $S'$ represents $N'$, and $N + 1 = N'$, then $S \fuse \inc \trans+ S' \ntrans$.
%   % %
%   % \item For all well-formed counters $S$ and $S'$, if $S$ represents $N$, $S'$ represents $N'$, and $S \fuse \inc \trans+ S' \ntrans$, then $N + 1 = N'$.
%   % \end{enumerate}
% \end{corollary}

% \begin{falseclaim}
%   For all natural numbers $N$ and $N'$ and well-formed counters $S$ and $S'$, if $S$ represents $N$ and $S'$ represents $N'$ and is $\inc$-free, then $N + 1 = N'$ if and only if $S \fuse \inc \trans+[\sig_{\inc}] S' \ntrans[\sig_{\inc}]$.
%   % \begin{enumerate}
%   % \item For all natural numbers $N$ and $N'$, if $S$ represents $N$, $S'$ represents $N'$, and $N + 1 = N'$, then $S \fuse \inc \trans+ S' \ntrans$.
%   % %
%   % \item For all well-formed counters $S$ and $S'$, if $S$ represents $N$, $S'$ represents $N'$, and $S \fuse \inc \trans+ S' \ntrans$, then $N + 1 = N'$.
%   % \end{enumerate}
% \end{falseclaim}
% \endgroup

% % stype C = &{ inc: C, dec: C', halt: X }
% %   and C' = +{ zero: C, succ: C }
% % eps : {C <- X}, bit0 : {C <- C}, bit1 : {C <- C}, inc : {C <- C},
% % dec : {C' <- C}, zero : {C' <- C}, succ : {C' <- C}, bit0' : {C' <- C'},
% % halt : {X <- C}
% % 
% %  C ::= X * eps | C * bit0 | C * bit1 | C * inc
% % C' ::= C * dec | C * zero | C * succ | C' * bit0'
% %  X ::= C * halt

% % stype C = &{ inc: C, dec: C' }
% %   and C' = +{ zero: C-, succ: C- }
% % bit0 : {C- <- C-} /\ {C <- C}, ...
% % 
% % C- ::= eps | C- * bit0 | C- * bit1
% %  C ::= eps | C * bit0 | C * bit1 | C * inc
% % C' ::= C * dec | C- * zero | C- * succ | C' * bit0'




% % \subsubsection{String rewriting rules as ordered implications.}

% % Using a focused proof search strategy~\autocite{Andreoli:JLC92}, ordered implications correspond to string rewriting rules.


% % \subsection{Example: Binary counter}\label{sec:exampl-binary-count-4}

% % As a running example, we can implement a binary counter that supports increments.
% % The counter is represented as a string of $\bit{0}$ and $\bit{1}$ letters terminated at the most significant end by an $\eps$.
% % So, for instance, the ordered conjunction, or string, $\eps \fuse \bit{1} \fuse \bit{0}$ represents a counter with value $2$.
% % Also interspersed are $\inc$ atoms, each of which serves as an increment instruction sent to the counter given by the more significant bits.
% % Thus, $\eps \fuse \bit{1} \fuse \inc$ represents a counter with value $1$ that has been sent an increment instruction.

% % Operationally, increments are described by three ordered implications that correspond to string rewriting rules; the first of these is
% % \begin{equation*}
% %   \bit{1} \fuse \inc \rimp \monad{\inc \fuse \bit{0}} \,.
% % \end{equation*}
% % By rewriting the string $\bit{1} \fuse \inc$ as $\inc \fuse \bit{0}$, this rule carries the $\inc$ up past any $\bit{1}$s at the counter's least significant end.
% % Whenever the carried $\inc$ reaches the $\eps$ or right-most $\bit{0}$, the carry is resolved:
% % \begin{align*}
% %   &\eps \fuse \inc \lrimp \monad{\eps \fuse \bit{1}} \\
% %   &\bit{0} \fuse \inc \lrimp \monad{\bit{1}} \,.
% % \end{align*}
% % By rewriting $\eps \fuse \inc$ as $\eps \fuse \bit{1}$, the second rule ensures that the carry becomes a new most significant $\bit{1}$ in the $\eps$ case.
% % By rewriting $\bit{0} \fuse \inc$ as $\bit{1}$, the third rule ensures that the carry flips the $\bit{0}$ to $\bit{1}$ in that case.



% % When this proposition is part of the persistent context $\uctx$, the following rule is derivable:
% % \begin{equation*}
% %   \infer{\uctx ; \omatch{\bit{1}, \inc} \seq J}{
% %     \uctx ; \ofill{\inc, \bit{0}} \seq J}
% %   \,.
% % \end{equation*}
% % Read bottom-up, this derived rule rewrites part of the ordered context so that $\bit{1}, \inc$ becomes $\inc, \bit{0}$.


% % % \begin{equation*}
% % %   \infer[\lab{copy}]{\uctx ; \omatch{\bit{1}, \inc} \seq J}{
% % %     \infer[\llab{{\rimp}}]{\uctx ; \ofill{\bit{1}, \inc, (\bit{1} \fuse \inc \rimp \inc \fuse \bit{0})} \seq J}{
% % %       \infer[\rlab{{\fuse}}]{\uctx ; \bit{1}, \inc \seq \bit{1} \fuse \inc}{
% % %         \infer[\lab{id}]{\uctx ; \bit{1} \seq \bit{1}}{
% % %           } &
% % %         \infer[\lab{id}]{\uctx ; \inc \seq \inc}{
% % %           }} &
% % %       \infer[\llab{{\fuse}}]{\uctx ; \ofill{\inc \fuse \bit{0}} \seq J}{
% % %         \uctx ; \ofill{\inc, \bit{0}} \seq J}}}
% % % \end{equation*}




% % \subsection{Example: Binary counter}\label{sec:exampl-binary-count-2}

% % As an example of an ordered logic program, we can implement a binary counter that supports increments.
% % Similarly to the process implementation from \cref{sec:exampl-binary-count}, the counter is represented as a list of $\bit{0}$ and $\bit{1}$s terminated at the most significant end by an $\eps$.
% % Here, however, the $\bit{}$s and $\eps$ are not processes, but rather atomic propositions (or, in string rewriting terminology, letters).
% % For instance, the ordered conjunction (or string) $\eps \fuse \bit{1} \fuse \bit{0}$ represents a counter with value $2$.

% % % An increment instruction is represented by an $\inc$ atom at the counter's least significant end.
% % % There are three rewrite rules that describe the increment operation:
% % % \begin{align*}
% % %   &\eps \fuse \inc \lrimp \monad{\eps \fuse \bit{1}} \\
% % %   &\bit{0} \fuse \inc \lrimp \monad{\bit{1}} \\
% % %   &\bit{1} \fuse \inc \lrimp \monad{\inc \fuse \bit{0}}
% % % \end{align*}
% % % By rewriting $\eps \fuse \inc$ as $\eps \fuse \bit{1}$, the first rule introduces $\bit{1}$ as a new most significant bit, and thereby serves to increment an $\eps$.
% % % % By rewriting $\eps \fuse \inc$ as $\eps \fuse \bit{1}$ and thereby introducing $\bit{1}$ as a new most significant bit, the first rule serves to increment $\eps$s.
% % % Likewise, the second rule serves to increment a counter whose least significant bit is $\bit{0}$, by rewriting $\bit{0} \fuse \inc$ as $\bit{1}$ and thereby flipping $\bit{0}$.
% % % % Likewise, by rewriting $\bit{0} \fuse \inc$ as $\bit{1}$ and thereby flipping $\bit{0}$, the second rule serves to increment a counter whose least significant bit is $\bit{0}$.
% % % Finally, by rewriting $\bit{1} \fuse \inc$ as $\inc \fuse \bit{0}$, the third rule flips $\bit{1}$ and propogates a carry to the more significant bits, thereby serving to increment a counter whose least significant bit is $\bit{1}$.

% % An increment instruction is represented by an $\inc$ atom at the counter's least significant end.
% % There are three rewrite rules that describe increments, the first of which is
% % \begin{equation*}
% %   \bit{1} \fuse \inc \lrimp \monad{\inc \fuse \bit{0}} \,.
% % \end{equation*}
% % By rewriting $\bit{1} \fuse \inc$ as $\inc \fuse \bit{0}$, this rule carries the $\inc$ up past any $\bit{1}$s at the counter's least significant end.
% % Whenever the carried $\inc$ reaches the $\eps$ or right-most $\bit{0}$, the carry is resolved:
% % \begin{align*}
% %   &\eps \fuse \inc \lrimp \monad{\eps \fuse \bit{1}} \\
% %   &\bit{0} \fuse \inc \lrimp \monad{\bit{1}} \,.
% % \end{align*}
% % By rewriting $\eps \fuse \inc$ as $\eps \fuse \bit{1}$, the second rule ensures that the carry becomes a new most significant $\bit{1}$ in the $\eps$ case.
% % By rewriting $\bit{0} \fuse \inc$ as $\bit{1}$, the third rule ensures that the carry flips the $\bit{0}$ to $\bit{1}$ in that case.

\end{document}

%%% Local Variables:
%%% TeX-master: "ordered-lp"
%%% End:
