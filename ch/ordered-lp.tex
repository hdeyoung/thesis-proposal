% arara: pdflatex
% arara: pdflatex
% arara: biber
% arara: pdflatex
% arara: pdflatex
\documentclass[
  class=../hdeyoung-proposal,
  crop=false
]{standalone}

\usepackage[subpreambles]{standalone}

\usepackage{ordered-logic}
\usepackage{binary-counter}

\addbibresource{../proposal.bib}

\begin{document}

\section{Background: Concurrent ordered logic programming}\label{sec:ordered-lp}

% As we saw in \cref{sec:linear-lp}, proof search in linear logic is the foundation for concurrent linear logic programming, which can be seen as a logically motivated form of multiset rewriting.
% Likewise, proof search in ordered logic forms the basis of concurrent ordered logic programming~\autocites{Pfenning+Simmons:LICS09}{Simmons:CMU12}, which is a form of string rewriting~\autocite[see, \eg,][]{Book+Otto:SRS93}.


Viewed through a computational lens, proof search in a fragment of ordered logic becomes a forward-chaining logic programming language~\autocite{Pfenning+Simmons:LICS09}.
It %[\st{This form of ordered logic programming}]
can be seen as a logically motivated generalization of string rewriting~\autocite[see, \eg,][]{Book+Otto:SRS93}, an analogy which we will exploit to provide some intuition for this form of ordered logic programming.

% We can exploit this analogy to provide some intuition for forward-chaining ordered logic programming.
From the perspective of string rewriting, an ordered logic program's atomic propositions are letters; ordered conjunctions of these atoms are strings; and, under a focused proof search strategy~\autocite{Andreoli:JLC92}, the ordered implications that serve as program clauses are string rewriting rules.
An example will help to clarify.


% arara: pdflatex
% arara: biber
% arara: pdflatex
% arara: pdflatex
\documentclass[
  class=../hdeyoung-proposal,
  crop=false
]{standalone}

\usepackage{ordered-logic}
\usepackage{binary-counter}
\usepackage{proof}
\usepackage{tikz}

\NewDocumentCommand{\trans}{t* t+ o}{%
  \longrightarrow
  \IfBooleanT{#1}{^*}\IfBooleanT{#2}{^+}%
  \IfValueT{#3}{_{#3}}%
}
\NewDocumentCommand{\ntrans}{}{
  \longarrownot\trans
}

\DeclareAcronym{CLF}{
  short = CLF,
  long = the concurrent logical framework
}

\NewPredicate{\Cntr}[Cntr][font = \mathit]{0}
\NewPredicate{\cntr}[Cntr]{0}

\NewPredicate{\coin}{0}
\NewPredicate{\heads}{0}
\NewPredicate{\tails}{0}
\NewPredicate{\win}{0}
\NewPredicate{\loss}{0}

\begin{document}

\subsection{Example: Binary counter}\label{sec:exampl-binary-count-5}

Using an ordered logic program, we can specify an incrementable binary counter.
% Here we give an ordered logic program for an incrementable binary counter.
% We can implement an incrementable binary counter as an ordered logic program.
% From the perspective of string rewriting, atomic propositions are letters, and ordered conjunctions of these atoms are strings.
% 
% % In the string rewriting terminology, atomic propositions correspond to letters; ordered conjunctions of these atoms correspond to strings; and ordered implications correspond to string rewriting rules.
% In the string rewriting terminology, atomic propositions correspond to letters, and ordered conjunctions of these atoms correspond to strings.
% % Using a focused proof search strategy~\autocite{Andreoli:JLC92}, ordered implications correspond to string rewriting rules.
The counter is represented as a string of $\bit{0}$ and $\bit{1}$ atoms terminated at the most significant end by an $\eps$.
For instance, the ordered conjunction $\eps \fuse \bit{1} \fuse \bit{0}$ is a string that represents a counter with value $2$.
%
Increment instructions are represented by $\inc$ atoms at the counter's least significant end. 
% Also interspersed are $\inc$ atoms, each of which serves as an increment instruction sent to the counter given by the more significant bits.
Thus, $\eps \fuse \bit{1} \fuse \inc$ represents a counter with value $1$ that has been instructed to increment once.

Operationally, increments are described by the program's three clauses, the first of which is
\begin{equation*}
  \bit{1} \fuse \inc \lrimp \monad{\inc \fuse \bit{0}}
  \,.
\end{equation*}
From a string rewriting perspective, this implication is a rule for rewriting the (sub)string $\bit{1} \fuse \inc$ as $\inc \fuse \bit{0}$.%
% From a string rewriting perspective, this implication is interpreted as a rule for rewriting the (sub)string $\bit{1} \fuse \inc$ as $\inc \fuse \bit{0}$.%
\footnote{This interpretation is justified because, logically, implications transform one resource into another.  This will be explained in more detail in \cref{??}.}
%  in the sense that the following bottom-up rule
% % This clause allows the string $\bit{1} \fuse \inc$ to be rewritten as $\inc \fuse \bit{0}$, in the sense that the rule
% is admissible whenever this clause is part of the persistent context, $\uctx$.
% \begin{equation*}
%   \infer{\uctx ; \omatch{\bit{1} \fuse \inc} \seq C \lax}{
%     \uctx ; \ofill{\inc \fuse \bit{0}} \seq C \lax}
% \end{equation*}}
%
% Just as this implication transforms 
% When this proposition is part of the persistent context $\uctx$, the rule
% \begin{equation*}
%   \infer{\uctx ; \omatch{\bit{1} \fuse \inc} \seq C \lax}{
%     \uctx ; \ofill{\inc \fuse \bit{0}} \seq C \lax}
%   \,.
% \end{equation*}
% is admissible.
% Read bottom-up, this rule serves to rewrite the (sub)string $\bit{1} \fuse \inc$ as the string $\inc \fuse \bit{0}$.
%
By rewriting $\bit{1} \fuse \inc$ as $\inc \fuse \bit{0}$, this clause serves to carry the $\inc$ up past any $\bit{1}$s that may exist at the counter's least significant end.

Whenever the carried $\inc$ reaches the $\eps$ or right-most $\bit{0}$, the carry is resolved by one of the other two program clauses:
\begin{align*}
  &\eps \fuse \inc \lrimp \monad{\eps \fuse \bit{1}} \\
  &\bit{0} \fuse \inc \lrimp \monad{\bit{1}} \,.
\end{align*}
By rewriting $\eps \fuse \inc$ as $\eps \fuse \bit{1}$, this second clause ensures that the carry becomes a new most significant $\bit{1}$ in the $\eps$ case.
By rewriting $\bit{0} \fuse \inc$ as $\bit{1}$, the third clause ensures that the carry flips the $\bit{0}$ to $\bit{1}$ in that case.

For example, the counter $\eps \fuse \bit{1} \fuse \inc$ can be maximally rewritten as
\ExplSyntaxOn
\NewDocumentCommand{\mathul}{m}{
  \mathpalette\mathul:nn{#1}
}
\cs_new:Npn \mathul:nn #1#2 {
  \tikz [baseline] {
    \node (pr) [anchor = base, inner~sep = 0em] {$#1#2$};
    \draw [overlay, ultra~thick, gray]
      ([yshift=-0.3em]pr.base~west) -- ([yshift=-0.3em]pr.base~east);
  }
}
\ExplSyntaxOff
\begin{equation*}
  \eps \fuse \mathul{\bit{1} \fuse \inc}
    \trans \mathul{\eps \fuse \inc} \fuse \bit{0}
    \trans \eps \fuse \bit{1} \fuse \bit{0}
    \ntrans
  \text{\,,}
\end{equation*}
where at each step the sites available for rewriting are underlined.
This trace computes $1 + 1 = 2$ in binary representation.
More generally, the above clauses adequately specify an increment operation:
string $S$ represents a counter with value $N$ if and only if $S \fuse \inc \trans+ S' \ntrans$ for some string $S'$ that represents a counter of value $N + 1$, where $\trans+$ denotes the transitive closure of the rewriting relation $\trans$.

\subsection{Concurrency}\label{sec:concurrency-1}

Some strings contain more than one rewrite site.
% several disjoint substrings that are amenable to rewriting.
% If these sites are disjoint, the rewrites can be thought of as happening concurrently.
For instance, the following binary counter has two $\inc$s in flight, which give rise to two disjoint rewrite sites.
\begin{equation*}
  \mathul{\eps \fuse \inc} \fuse \mathul{\bit{0} \fuse \inc}
\end{equation*}
The rewrites in this example can be interleaved in two ways: either
% there are two interleavings of these rewrites: either
\begin{align*}
  &\mathul{\eps \fuse \inc} \fuse \mathul{\bit{0} \fuse \inc}
     \trans \eps \fuse \bit{1} \fuse \mathul{\bit{0} \fuse \inc}
     \trans \eps \fuse \bit{1} \fuse \bit{1}
     \ntrans \\
  %
  \shortintertext{or}
  %
  &\mathul{\eps \fuse \inc} \fuse \mathul{\bit{0} \fuse \inc}
     \trans \mathul{\eps \fuse \inc} \fuse \bit{1}
     \trans \eps \fuse \bit{1} \fuse \bit{1}
     \ntrans
   \,.
\end{align*}

However, because these two rewrites are independent, they should be considered morally concurrent.
Rather than giving a truly concurrent semantics for string rewriting, we can treat different interleavings of independent steps as indistinguishable.
Then, because we can't observe which rewrite occurred first, the two rewrites appear to happen concurrently.
This is the idea of \vocab{concurrent equality} from {CLF}~\autocite{Watkins+:CMU02} that gives a pretense to true concurrency, borrowed for ordered logic programming.


\subsubsection{Infinite traces and fairness}

\NewPredicate{\incs}{0}%
Thus far, all traces have been finite, but this is not necessarily so.
Consider adding an atom, $\incs$, that generates a stream of $\inc$ atoms:
\begin{equation*}
  \incs \lrimp \monad{\inc \fuse \incs}
  \,.
\end{equation*}
Among the infinite traces now possible is
% Beginning from $\eps \fuse \incs$, there are now infinite trace
\begin{equation*}
  \eps \fuse \mathul{\incs}
    \trans \mathul{\eps \fuse \inc} \fuse \mathul{\incs}
    \trans \dotsb
    \trans \mathul{\eps \fuse \inc} \fuse \inc \fuse \dotsb \fuse \inc \fuse \mathul{\incs}
    \trans \dotsb
  \,.
\end{equation*}

Notice that this trace never rewrites the infinitely available $\eps \fuse \inc$, instead always choosing to rewrite $\incs$.
Because of its\fxnote{\ \st{scheduling}} bias against rewriting $\eps \fuse \inc$, we say that the trace is \vocab{(weakly) process-unfair}.

Process unfairness is fundamentally at odds with true concurrency.
Because true concurrency allows multiple independent events to occur simultaneously, one event cannot preclude another independent event;
in the above example, for instance, rewriting the $\incs$ atom would not preclude rewriting the independent $\eps \fuse \inc$ substring.
Therefore, to maintain the pretense of true concurrency, we must require all traces to be (weakly) process-fair.


% Transition unfairness is fundamentally at odds with true concurrency
% because true concurrency allows multiple independent events to occur simultaneously;
% the occurrence of one event cannot preclude another independent event from occurring at the same time.
% To maintain the pretense of true concurrency, 

% In the above example, $\eps \fuse \inc$ and $\incs$ are independent rewrite sites

% In the above example, $\eps \fuse \inc$ and $\incs$ are independent rewrite sites.
% % ; if they are truly concurrent, the rewrite of $\eps \fuse \inc$ should eventually occur.
% To maintain any pretense of true concurrency, the rewrite of $\eps \fuse \inc$ should eventually occur since true concurrency would allow multiple independent events to occur simultaneously.



% However, by treating different interleavings of independent steps as indistinguishable, these two rewrites happen concurrently.




% For instance, notice that this binary counter specification allows multiple $\inc$s to be in flight at once, each of which is amenable to rewriting.



% Sometimes several disjoint substrings are amenable to rewriting.


% Different interleavings of independent steps are indistinguishable.

% String rewriting (and therefore ordered logic programming) gives rise to 


% The counter $\eps \fuse \inc \bit{0} \fuse \inc$ has two $\inc$s in flight, which can be rewritten independently.


% Rewrites of disjoint substrings can be thought of as happening concurrently.

% Notice that we can also allow multiple $\inc$s to be in flight at once, and that independent rewrites can thought of as happening concurrently.
% For instance, the counter $\eps \fuse \inc \fuse \bit{0} \fuse \inc$ has two $\inc$s in flight, and they give rise to independent rewrites.
% \begin{center}
%   \begin{tikzpicture}
%     \matrix [matrix of math nodes, column sep = 1.5em]
%     {
%       % First row
%       & |(inc-1-2)| \eps \fuse \bit{1} \fuse \mathul{\bit{0} \fuse \inc} & \\
%       % Second row
%       |(inc-2-1)| \mathul{\eps \fuse \inc} \fuse \mathul{\bit{0} \fuse \inc}
%         && |(inc-2-3)| \eps \fuse \bit{1} \fuse \bit{1} \\
%       % Third row
%       & |(inc-3-2)| \mathul{\eps \fuse \inc} \fuse \bit{1} & \\
%     };

%     \begin{scope}
%     [ start chain, every join/.style={->} ]
%       \chainin (inc-2-1);
%       \begin{scope}[start branch=inc-1-2]
%         \chainin (inc-1-2) [join];
%       \end{scope}
%       \begin{scope}[start branch=inc-3-2]
%         \chainin (inc-3-2) [join];
%       \end{scope}
%       \chainin (inc-2-3) [join = with inc-1-2, join = with inc-3-2];
%     \end{scope}
%   \end{tikzpicture}
% \end{center}


\subsection{Committed choice}\label{sec:committed-choice}

% In our forward-chaining ordered logic programming language, we assume a \vocab{committed-choice} semantics, meaning that when multiple rewritings are possible at a given site the choice is never reconsidered.
In forward-chaining ordered logic programming, we assume a \vocab{committed-choice} semantics, meaning that when several rewritings are possible at a given site, one is chosen and that choice is never reconsidered.

Committed choice does not clearly arise in the binary counter example;
instead, consider a single-player game in which the player wins if a
% each of two coin tosses land heads.
coin toss lands heads.
As an forward-chaining ordered logic program, this game is specified
\begin{align*}
  &\coin \lrimp \monad{\heads} \\
  &\coin \lrimp \monad{\tails} \\
  &\heads \lrimp \monad{\win} \\
  &\tails \lrimp \monad{\loss}
  \,.
\end{align*}
The first two clauses specify that a $\coin$ may land either $\heads$ or $\tails$, and the remaining clauses specify the winning condition.

% One way in which the starting string $\coin \fuse \coin$ of two coins can be maximally rewritten is
One way in which the starting string can be maximally rewritten is
\begin{equation*}
  \mathul{\coin}
    \trans \mathul{\tails}
    \trans \loss
    \ntrans
  \,.
\end{equation*}
Here the $\coin$ lands $\tails$, resulting in a $\loss$.
To get a $\win$, we'd like to somehow take back the toss and have it instead land $\heads$.
% Here the first $\coin$ lands $\tails$ and the second $\coin$ lands $\heads$, resulting in a $\loss$.
% To get a $\win$, we'd like to somehow take back the first toss and have it instead land $\heads$.
This violation of the game's rules is just what the committed-choice semantics proscribes: once the coin is tossed, we must commit to that toss's outcome.



\subsection{Example: Binary counter with decrements}\label{sec:exampl-binary-count-3}

Returning to the binary counter, its also possible \dots

To reiterate some of the points made above, \dots

It's also possible to add support for decrements to the above ordered logic program.
Like increments, a decrement instruction is represented by a $\dec$ atom at the counter's least significant end.
To perform the decrement, a $\dec$ begins propagating up the counter.
As it passes over any $\bit{0}$s at the least significant end, they are marked as $\bit[']{0}$s to indicate that they are waiting to borrow from their more significant neighbors:
\begin{equation*}
  \bit{0} \fuse \dec \lrimp \monad{\dec \fuse \bit[']{0}} \,.
\end{equation*}
Whenever it reaches the $\eps$ or right-most $\bit{1}$, the $\dec$ is replaced with either $\zero$ or $\suc$, respectively, to show whether the borrow was possible; in the case of $\bit{1}$, the borrow is also effected:
\begin{align*}
  &\eps \fuse \dec \lrimp \monad{\eps \fuse \zero} \\
  &\bit{1} \fuse \dec \lrimp \monad{\bit{0} \fuse \suc} \,.
\end{align*}
Then the $\zero$ or $\suc$ travels back over all of the $\bit[']{0}$s that were waiting to borrow.
In the case of $\zero$, the bits are returned to their original $\bit{0}$ state because no borrow was possible;
in the case of $\suc$, a borrow was performed and so the bits are set to $\bit{1}$:
\begin{align*}
  &\zero \fuse \bit[']{0} \lrimp \monad{\bit{0} \fuse \zero} \\
  &\suc \fuse \bit[']{0} \lrimp \monad{\bit{1} \fuse \suc} \,.
\end{align*}


% According to the following rewrite rules, a $\dec$ propogates up the counter past any $\bit{0}$s until it reaches an $\eps$ or the right-most $\bit{1}$.
% At this point, the $\dec$ is replaced with either $\zero$ or $\suc$, respectively.
% \begin{align*}
%   &\bit{0} \fuse \dec \lrimp \monad{\dec \fuse \bit[']{0}} \\
%   &\eps \fuse \dec \lrimp \monad{\eps \fuse \zero} \\
%   &\bit{1} \fuse \dec \lrimp \monad{\bit{0} \fuse \suc}
% \end{align*}
% Then the $\zero$ or $\suc$ traveles back down the counter
% \begin{align*}
%   &\zero \fuse \bit[']{0} \lrimp \monad{\bit{0} \fuse \zero} \\
%   &\suc \fuse \bit[']{0} \lrimp \monad{\bit{1} \fuse \suc}
% \end{align*}


For example, the counter $\eps \fuse \bit{1} \fuse \bit{0} \fuse \dec$ can be maximally rewritten as
\begin{align*}
  \MoveEqLeft[0.5]
  \eps \fuse \bit{1} \fuse \mathul{\bit{0} \fuse \dec} \\
    &\trans \eps \fuse \mathul{\bit{1} \fuse \dec} \fuse \bit[']{0} \\
    &\trans \eps \fuse \mathul{\bit{0} \fuse \suc} \fuse \bit[']{0} \\
    &\trans \eps \fuse \bit{0} \fuse \bit{1} \fuse \suc \\
    &\ntrans
\end{align*}
Once again, there are possibilities for concurrency.
For example, the following two traces are indistinguishable because they differ only in the order of independent rewrites:
\begin{align*}
  &\mathul{\eps \fuse \inc} \fuse \mathul{\bit{0} \fuse \dec} \trans \mathul{\eps \fuse \inc} \fuse \dec \fuse \bit[']{0} \trans \eps \fuse \mathul{\bit{1} \fuse \dec} \fuse \bit[']{0} \\
  %
  \shortintertext{and}
  %
  &\mathul{\eps \fuse \inc} \fuse \mathul{\bit{0} \fuse \dec} \trans \eps \fuse \bit{1} \fuse \mathul{\bit{0} \fuse \dec} \trans \eps \fuse \mathul{\bit{1} \fuse \dec} \fuse \bit[']{0}
\end{align*}
This justifies treating those two rewrites as concurrent.


% C ::= eps | C * bit0 | C * bit1 | C * inc


% c <- dec <- d =
% { case d of
%     eps => wait d;
%            d' <- eps;
%            c <- zero <- d'
%   | bit0 => d' <- dec <- d
%             c <- bit0' <- d'
%   | bit1 => d' <- bit0 <- d
%             c <- succ <- d' }

% c <- zero <- d =
% { case c of
%     bit0' => d' <- bit0 <- d
%              c <- zero <- d' }

% Cntr = +{ eps: 1 , bit0: Cntr , bit1: Cntr }
% Cntr' = &{ bit0': Cntr' }

% bit0 : {Cntr |- Cntr}
% dec : {Cntr |- Cntr'}
% zero : {Cntr |- Cntr'}
% bit0' : {Cntr' |- Cntr'}


\begin{align*}
  &\eps \fuse \dec \lrimp \monad{\eps \fuse \zero} \\
  &\bit{0} \fuse \dec \lrimp \monad[auto]{
                               \dec \fuse \parens[auto, align=c@{\,}l]{
                                                & (\zero \limp \monad{\bit{0} \fuse \zero}) \\
                                          \with & (\suc \limp \monad{\bit{1} \fuse \suc})}} \\
  &\bit{1} \fuse \dec \lrimp \monad{\bit{0} \fuse \suc}
\end{align*}



\subsection{Adequacy and generative invariants}\label{sec:gener-invar}

Thus far, we have used ordered logic programs to specify concurrent systems, whereas the non-modal fragment of ordered logic was originally developed by \textcite{Lambek:AMM58} to describe sentence structure.
However, these two modes of use of ordered logic are not as different as they might first appear.

In our running example of an incrementable binary counter, the counter is represented as a string of $\bit{0}$, $\bit{1}$, and $\inc$ atoms terminated at the most significant end by an $\eps$.
More precisely, a string is a well-formed binary counter if it can be generated from the $\Cntr$ nonterminal by the context-free grammar
\begin{gather*}
  \Cntr ::= \eps \mid \Cntr \fuse \bit{0} \mid \Cntr \fuse \bit{1} \mid \Cntr \fuse \inc
  \,,
%
\intertext{which is an abbreviated notation for four distinct productions:}
%
  \begin{aligned}
    &\Cntr \to \eps \\
    &\Cntr \to \Cntr \fuse \bit{0} \\
    &\Cntr \to \Cntr \fuse \bit{1} \\
    &\Cntr \to \Cntr \fuse \inc
    \,.
  \end{aligned}
\end{gather*}

Building on \citeauthor{Lambek:AMM58}'s work, the same context-free grammar can be described in ordered logic using \vocab{generative signatures}~\autocite{Simmons:CMU12}.
Each production in the grammar corresponds to a clause, with the $\Cntr$ nonterminal represented as the atomic proposition $\cntr$:
\begin{equation*}
  \begin{aligned}
    &\cntr \lrimp \monad{\eps} \\
    &\cntr \lrimp \monad{\cntr \fuse \bit{0}} \\
    &\cntr \lrimp \monad{\cntr \fuse \bit{1}} \\
    &\cntr \lrimp \monad{\cntr \fuse \inc}
    \,.
  \end{aligned}
\end{equation*}
Then, just as all well-formed binary counters are generated from the $\Cntr$ nonterminal according to the above productions, so are all binary counters generated as maximal rewritings of the $\cntr$ atom according to these clauses.
% Just as a string is well-formed binary counter if it can be generated from the $\Cntr$ nonterminal by the above context-free grammar, so too is a string well-formed if it can be generated by maximally rewriting the $\cntr$ atom.
For example, $\eps \fuse \bit{1} \fuse \inc$ is a well-formed binary counter because it is a maximal rewriting of $\cntr$:
\begin{equation*}
  \mathul{\cntr}
    \trans \mathul{\cntr} \fuse \inc
    \trans \mathul{\cntr} \fuse \bit{1} \fuse \inc
    \trans \eps \fuse \bit{1} \fuse \inc
    \ntrans
  \,.
\end{equation*}
Note that 

\begingroup
  \RenewPredicate{\cntr}[Cntr]{1}%
Generative signatures in fact generalize context-free grammars.
One example is to augment $\cntr{}$ with a natural number, effectively \wc{giving}[\st{creating}] a countably infinite family of nonterminals.
Thus, a binary counter is well-formed \emph{and} represents value $N$ if it is a maximal rewriting of $\cntr{N}$ according to the first-order ordered logic program
\begin{equation*}
  \begin{aligned}
    &\cntr{0} \lrimp \monad{\eps} \\
    &\cntr{(2N)} \lrimp \monad{\cntr{N} \fuse \bit{0}} \\
    &\cntr{(2N{+}1)} \lrimp \monad{\cntr{N} \fuse \bit{1}} \\
    &\cntr{(N{+}1)} \lrimp \monad{\cntr{N} \fuse \inc}
    \,.
  \end{aligned}
\end{equation*}

This generative signature allows us to formally state (and prove) adequacy of the incrementable binary counter program:
\begin{definition}
  A string is \vocab{quiescent} if $S \ntrans$.
  A string $S$ is a \vocab{well-formed counter} if $\cntr{} \trans+ S \ntrans$.
  A string $S$ \vocab{represents} natural number $N$ if $\cntr{N} \trans+ S \ntrans$.
\end{definition}

\begin{theorem}[Preservation]
  For every well-formed counter $S$ such that $S \trans S'$, the string $S'$ is also a well-formed counter.
\end{theorem}
\begin{theorem}[Adequacy of counters]
  \mbox{}
  \begin{enumerate}
  \item For every natural number $N$, there is a unique well-formed counter $S$ such that
%     \item $\cntr{} \trans+ S \ntrans$;
 $S$ represents $N$ and % $\cntr{N} \trans+ S \ntrans$
 $S$ is quiescent.% $S \ntrans$.
%  \item For every string $S$ such that $\cntr{} \trans+ S \ntrans$, there is a unique natural number $N$ such that $\cntr{N} \trans+ S \ntrans$.
  \item For every well-formed counter $S$, there is a unique natural number $N$ such that $S$ represents $N$.
  \end{enumerate}
\end{theorem}
\begin{theorem}[Adequacy of $\inc$]\mbox{}
  \begin{enumerate}
  \item If $N + 1 = N'$, then there exist well-formed counters $S$ and unique $S'$ such that string $S$ represents $N$, string $S'$ represents $N'$, and $S \fuse \inc \trans+ S' \ntrans$.
  \item For all well-formed counters $S$ and $S'$ such that $S \fuse \inc \trans+ S' \ntrans$, there exist unique natural numbers $N$ and $N'$ such that string $S$ represents $N$, string $S'$ represents $N'$, and $N + 1 = N'$.
  \end{enumerate}
\end{theorem}

\begin{theorem}[Adequacy of $\inc$]
  $\cntr{N} \trans+ S \ntrans$ if and only if $S \fuse \inc \trans+ S' \ntrans$ and $\cntr{(N{+}1)} \trans+ S' \ntrans$.
  \begin{itemize}
  \item If $\cntr{N} \trans+ S \ntrans$ and $S \fuse \inc \trans+ S' \ntrans$, then $\cntr{(N{+}1)} \trans+ S' \ntrans$.
  \item If $\cntr{N} \trans+ S \ntrans$, then $S \fuse \inc \trans+ S' \ntrans$.
  \end{itemize}
\end{theorem}
\endgroup

% stype C = &{ inc: C, dec: C', halt: X }
%   and C' = +{ zero: C, succ: C }
% eps : {C <- X}, bit0 : {C <- C}, bit1 : {C <- C}, inc : {C <- C},
% dec : {C' <- C}, zero : {C' <- C}, succ : {C' <- C}, bit0' : {C' <- C'},
% halt : {X <- C}
% 
%  C ::= X * eps | C * bit0 | C * bit1 | C * inc
% C' ::= C * dec | C * zero | C * succ | C' * bit0'
%  X ::= C * halt

% stype C = &{ inc: C, dec: C' }
%   and C' = +{ zero: C-, succ: C- }
% bit0 : {C- <- C-} /\ {C <- C}, ...
% 
% C- ::= eps | C- * bit0 | C- * bit1
%  C ::= eps | C * bit0 | C * bit1 | C * inc
% C' ::= C * dec | C- * zero | C- * succ | C' * bit0'




% \subsubsection{String rewriting rules as ordered implications.}

% Using a focused proof search strategy~\autocite{Andreoli:JLC92}, ordered implications correspond to string rewriting rules.


% \subsection{Example: Binary counter}\label{sec:exampl-binary-count-4}

% As a running example, we can implement a binary counter that supports increments.
% The counter is represented as a string of $\bit{0}$ and $\bit{1}$ letters terminated at the most significant end by an $\eps$.
% So, for instance, the ordered conjunction, or string, $\eps \fuse \bit{1} \fuse \bit{0}$ represents a counter with value $2$.
% Also interspersed are $\inc$ atoms, each of which serves as an increment instruction sent to the counter given by the more significant bits.
% Thus, $\eps \fuse \bit{1} \fuse \inc$ represents a counter with value $1$ that has been sent an increment instruction.

% Operationally, increments are described by three ordered implications that correspond to string rewriting rules; the first of these is
% \begin{equation*}
%   \bit{1} \fuse \inc \rimp \monad{\inc \fuse \bit{0}} \,.
% \end{equation*}
% By rewriting the string $\bit{1} \fuse \inc$ as $\inc \fuse \bit{0}$, this rule carries the $\inc$ up past any $\bit{1}$s at the counter's least significant end.
% Whenever the carried $\inc$ reaches the $\eps$ or right-most $\bit{0}$, the carry is resolved:
% \begin{align*}
%   &\eps \fuse \inc \lrimp \monad{\eps \fuse \bit{1}} \\
%   &\bit{0} \fuse \inc \lrimp \monad{\bit{1}} \,.
% \end{align*}
% By rewriting $\eps \fuse \inc$ as $\eps \fuse \bit{1}$, the second rule ensures that the carry becomes a new most significant $\bit{1}$ in the $\eps$ case.
% By rewriting $\bit{0} \fuse \inc$ as $\bit{1}$, the third rule ensures that the carry flips the $\bit{0}$ to $\bit{1}$ in that case.



% When this proposition is part of the persistent context $\uctx$, the following rule is derivable:
% \begin{equation*}
%   \infer{\uctx ; \omatch{\bit{1}, \inc} \seq J}{
%     \uctx ; \ofill{\inc, \bit{0}} \seq J}
%   \,.
% \end{equation*}
% Read bottom-up, this derived rule rewrites part of the ordered context so that $\bit{1}, \inc$ becomes $\inc, \bit{0}$.


% % \begin{equation*}
% %   \infer[\lab{copy}]{\uctx ; \omatch{\bit{1}, \inc} \seq J}{
% %     \infer[\llab{{\rimp}}]{\uctx ; \ofill{\bit{1}, \inc, (\bit{1} \fuse \inc \rimp \inc \fuse \bit{0})} \seq J}{
% %       \infer[\rlab{{\fuse}}]{\uctx ; \bit{1}, \inc \seq \bit{1} \fuse \inc}{
% %         \infer[\lab{id}]{\uctx ; \bit{1} \seq \bit{1}}{
% %           } &
% %         \infer[\lab{id}]{\uctx ; \inc \seq \inc}{
% %           }} &
% %       \infer[\llab{{\fuse}}]{\uctx ; \ofill{\inc \fuse \bit{0}} \seq J}{
% %         \uctx ; \ofill{\inc, \bit{0}} \seq J}}}
% % \end{equation*}




% \subsection{Example: Binary counter}\label{sec:exampl-binary-count-2}

% As an example of an ordered logic program, we can implement a binary counter that supports increments.
% Similarly to the process implementation from \cref{sec:exampl-binary-count}, the counter is represented as a list of $\bit{0}$ and $\bit{1}$s terminated at the most significant end by an $\eps$.
% Here, however, the $\bit{}$s and $\eps$ are not processes, but rather atomic propositions (or, in string rewriting terminology, letters).
% For instance, the ordered conjunction (or string) $\eps \fuse \bit{1} \fuse \bit{0}$ represents a counter with value $2$.

% % An increment instruction is represented by an $\inc$ atom at the counter's least significant end.
% % There are three rewrite rules that describe the increment operation:
% % \begin{align*}
% %   &\eps \fuse \inc \lrimp \monad{\eps \fuse \bit{1}} \\
% %   &\bit{0} \fuse \inc \lrimp \monad{\bit{1}} \\
% %   &\bit{1} \fuse \inc \lrimp \monad{\inc \fuse \bit{0}}
% % \end{align*}
% % By rewriting $\eps \fuse \inc$ as $\eps \fuse \bit{1}$, the first rule introduces $\bit{1}$ as a new most significant bit, and thereby serves to increment an $\eps$.
% % % By rewriting $\eps \fuse \inc$ as $\eps \fuse \bit{1}$ and thereby introducing $\bit{1}$ as a new most significant bit, the first rule serves to increment $\eps$s.
% % Likewise, the second rule serves to increment a counter whose least significant bit is $\bit{0}$, by rewriting $\bit{0} \fuse \inc$ as $\bit{1}$ and thereby flipping $\bit{0}$.
% % % Likewise, by rewriting $\bit{0} \fuse \inc$ as $\bit{1}$ and thereby flipping $\bit{0}$, the second rule serves to increment a counter whose least significant bit is $\bit{0}$.
% % Finally, by rewriting $\bit{1} \fuse \inc$ as $\inc \fuse \bit{0}$, the third rule flips $\bit{1}$ and propogates a carry to the more significant bits, thereby serving to increment a counter whose least significant bit is $\bit{1}$.

% An increment instruction is represented by an $\inc$ atom at the counter's least significant end.
% There are three rewrite rules that describe increments, the first of which is
% \begin{equation*}
%   \bit{1} \fuse \inc \lrimp \monad{\inc \fuse \bit{0}} \,.
% \end{equation*}
% By rewriting $\bit{1} \fuse \inc$ as $\inc \fuse \bit{0}$, this rule carries the $\inc$ up past any $\bit{1}$s at the counter's least significant end.
% Whenever the carried $\inc$ reaches the $\eps$ or right-most $\bit{0}$, the carry is resolved:
% \begin{align*}
%   &\eps \fuse \inc \lrimp \monad{\eps \fuse \bit{1}} \\
%   &\bit{0} \fuse \inc \lrimp \monad{\bit{1}} \,.
% \end{align*}
% By rewriting $\eps \fuse \inc$ as $\eps \fuse \bit{1}$, the second rule ensures that the carry becomes a new most significant $\bit{1}$ in the $\eps$ case.
% By rewriting $\bit{0} \fuse \inc$ as $\bit{1}$, the third rule ensures that the carry flips the $\bit{0}$ to $\bit{1}$ in that case.

\end{document}

%%% Local Variables:
%%% TeX-master: "ordered-lp"
%%% End:


% arara: pdflatex
% arara: biber
% arara: pdflatex
% arara: pdflatex
\documentclass[
  class=../hdeyoung-proposal,
  crop=false
]{standalone}

\usepackage{ordered-logic}
\usepackage{basic-atoms}
\usepackage{ordered-lp-terms}
\usepackage{proof}
\usepackage{mathpartir}

\NewDocumentEnvironment{infers}{o O{}}%
  {%
    \noindent\IfValueT{#1}{\fbox{#1}#2}%
    \begin{mathpar}\ignorespaces
  }%
  {\end{mathpar}\ignorespacesafterend}

\NewDocumentCommand{\irlabel}{m m o}%
  {{#2}\text{\textsc{#1}}\IfValueT{#3}{_{#3}}}
\NewDocumentCommand{\rlab}{m o}%
  {\IfValueTF{#2}{\irlabel{r}{#1}[#2]}{\irlabel{r}{#1}}}
\NewDocumentCommand{\llab}{m o}%
  {\IfValueTF{#2}{\irlabel{l}{#1}[#2]}{\irlabel{l}{#1}}}

\NewDocumentCommand{\tctx}{}{\Psi}
\NewDocumentCommand{\tctxe}{}{\cdot}

% \NewDocumentCommand{\trans}{t* t+ o}{%
%   \longrightarrow
%   \IfBooleanT{#1}{^*}\IfBooleanT{#2}{^+}%
%   \IfValueT{#3}{_{#3}}%
% }
% \NewDocumentCommand{\ntrans}{}{
%   \longarrownot\trans
% }

\begin{document}

\subsection{Technical details}\label{sec:ordered-lp:technical}

The previous \lcnamecrefs{sec:??} have hopefully served to provide intuition for forward-chaining ordered logic programming.
In this \lcnamecref{sec:technical-details}, we review the technical details of this framework; 
we generally follow the lead of \citeauthor{Simmons:CMU12}'s SLS framework~\autocite*{Simmons:CMU12}.

\subsubsection{Propositions, terms, and traces}\label{sec:props-terms-traces}

\paragraph{Propositions.}\label{sec:propositions}

Propositions are polarized into \vocab{negative} and \vocab{positive} classes:
\begin{alignat*}{2}
  &\text{Negative propositions}\quad & A^- &::= \forall a{:}\tau. A^- \mid A^+ \rimp B^- \mid A^+ \limp B^- \mid A^- \with B^- \mid \monad{A^+} \\
  &\text{Positive propositions}      & A^+ &::= \exists a{:}\tau. A^+ \mid A^+ \fuse B^+ \mid \one \mid \p^+ \mid A^-
\end{alignat*}
The negative propositions, $A^-$, are those whose right rules are invertible, whereas the positive propositions, $A^+$, are those whose left rules are invertible.
As an example, the ordered implications $A^+ \rimp B^-$ and $A^+ \limp B^-$ are classified as negative because the $\rlab{\rimp}$ and $\rlab{\limp}$ rules are invertible.
% For the fragment in which we are interested, the grammar of polarized propositions is

A few additional comments on the polarized propositions are worthwhile.
First, a lax modality, $\monad{A^+}$, allows positive propositions to be treated as negative ones, while still maintaining an explicit separation of the classes.
% The lax modality gives logical force to 
The lax modality will be responsible for typing traces (\cref{??}).
Second, negative propositions can similiarly be treated as positive ones via the implicit inclusion $A^-$.
Positive atomic propositions $\p^+$ stand in for arbitrary positive propositions. 

\paragraph{Contexts.}\label{sec:contexts}

\begin{alignat*}{2}
  &\text{Ordered contexts}\quad & \octx &::= \octxe \mid x{:}\susp+{\p^+} \mid x{:}A^- \mid \octx_1, \octx_2
\end{alignat*}


\paragraph{Terms.}\label{sec:terms}

As previously alluded, a focused proof-search strategy~\autocite{Andreoli:JLC92} forms the basis of ordered logic programming.
In this focused calculus, there are several forms of sequent%
%---inversion sequents, right-focused sequents, and left-focused sequents---
, each corresponding to a syntactic class of proof terms.

\begin{figure}[!tbp]
  \begin{infers}[Inversion: $\tctx ; \octx \seq \nof{N : A^-}$][ with $\octx$ stable]
    \infer[\rlab{\forall}]{\tctx ; \octx \seq \nof{\tlam{a.N} : \forall a{:}\tau.A^-}}{
      \tctx, a{:}\tau ; \octx \seq \nof{N : A^-}}
    \and
    \infer[\rlab{\rimp}]{\tctx ; \octx \seq \nof{\rlam{p.N} : A^+ \rimp B^-}}{
      \tctx_{A^+} ; \octx_{A^+} \pseq \pof{p : A^+} &
      \tctx, \tctx_{A^+} ; \octx, \octx_{A^+} \seq \nof{N : B^-}}
    \and
    \infer[\rlab{\limp}]{\tctx ; \octx \seq \nof{\llam{p.N} : A^+ \limp B^-}}{
      \tctx_{A^+} ; \octx_{A^+} \pseq \pof{p : A^+} &
      \tctx, \tctx_{A^+} ; \octx_{A^+}, \octx \seq \nof{N : B^-}}
    \and
    \infer[\rlab{\with}]{\tctx ; \octx \seq \nof{\pair{N_1 , N_2} : A^-_1 \with A^-_2}}{
      \tctx ; \octx \seq \nof{N_1 : A^-_1} &
      \tctx ; \octx \seq \nof{N_2 : A^-_2}}
    \and
    \infer[\rlab{\monad{}}]{\tctx ; \octx \seq \nof{\lett{T in V} : \monad{A^+}}}{
      \tof{T :: (\tctx ; \octx) \trans* (\tctx' ; \octx')} &
      \tctx' ; \octx' \seq \vof{V : [A^+]}}
  \end{infers}
  \caption{Normal terms\label{fig:normal-terms}}
\end{figure}

Normal terms $N$ are typed by inversion sequents, $\tctx ; \octx \seq \nof{N : A^-}$ (\cref{fig:normal-terms}).
Inversion proceeds by eagerly applying right rules, which are invertible because the succedent $A^-$ is of negative polarity.
Rather than \enquote{inlining} invertible left rules along the derivation's trunk, the focused calculus maintains the invariant that the ordered context $\octx$ in an inversion sequent $\tctx ; \octx \seq \nof{N : A^-}$ is stable.
% the rules for inversion sequents maintain the invariant that the ordered context $\octx$ is stable.
It's for this reason that the lambda abstractions $\rlam{p.N}$ and $\llam{p.N}$ bind patterns $p$ instead of single variables---the patterns ensure that the argument is fully left-inverted.
% Among the normal terms are the lambda abstractions $\rlam{p.N}$ and $\llam{p.N}$, which are typed by the right rules for the ordered implications.
% These abstractions bind patterns, $p$, instead of single variables because, rather than \enquote{inlining} left inversion along the derivation's trunk, this judgment's rules maintain the invariant that the ordered context $\octx$ is stable.

Eventually, inversion terminates with a lax modality, $\monad{A^+}$, as the succedent.
Forward-chaining to construct a trace $T$ begins here.

\begin{figure}
  \begin{infers}[$\tof{T :: (\tctx ; \octx) \trans* (\tctx' ; \octx')}$]
    \infer{\tof{\tnil :: (\tctx ; \octx) \trans* (\tctx ; \octx)}}{
      }
    \and
    \infer{\tof{\tstep{p <- R; T} :: (\tctx ; \omatch{\octx}) \trans* (\tctx' ; \octx')}}{
      \tctx ; \octx \seq \aof{R : \susp-{\monad{A^+}}} &
      \tctx_{A^+} ; \octx_{A^+} \pseq \pof{p : A^+} &
      \tof{T :: (\tctx, \tctx_{A^+} ; \ofill{\octx_{A^+}}) \trans* (\tctx' ; \octx')}}
  \end{infers}
  \caption{Traces\label{fig:traces}}
\end{figure}

Traces $T$ are possibly empty sequences of steps $\tstep{p <- R}$.



\begin{figure}
  \begin{infers}[Stable: $\tctx ; \octx \seq \aof{R : \susp-{C^-}}$]
    \infer{\tctx ; \omatch{x{:}A^-} \seq \aof{\atm{x . S} : \susp-{C^-}}}{
      \tctx ; \ofill{\lfoc{A^-}} \seq \sof{S : \susp-{C^-}}}
    \and
    \infer{\tctx ; \omatch{\octxe} \seq \aof{\atm{c . S} : \susp-{C^-}}}{
      c{:}A^- \in \sig &
      \tctx ; \ofill{\lfoc{A^-}} \seq \sof{S : \susp-{C^-}}}    
  \end{infers}
  \caption{Atomic terms\label{fig:atomic-terms}}
\end{figure}

Rather than introducing terms for individual left-invertible rules

\begin{figure}[!t]
  \begin{infers}[Left-focus: $\tctx ; \omatch{\lfoc{A^-}} \seq \sof{S : \susp-{C^-}}$]
    \infer{\tctx ; \omatch{\lfoc{A^-}} \seq \sof{\snil : \susp-{A^-}}}{
    }
    \and
    \infer{\tctx ; \omatch{\lfoc{\forall a{:}\tau.A^-}} \seq \sof{\tapp{t ; S} : \susp-{C^-}}}{
      \tctx \seq t : \tau &
      \tctx ; \ofill{\lfoc{\subst{t/a}{A^-}}} \seq \sof{S : \susp-{C^-}}}
    \and
    \infer{\tctx ; \omatch{\lfoc{A^+ \rimp B^-}, \octx} \seq \sof{\rapp{V ; S} : \susp-{C^-}}}{
      \tctx ; \octx \seq \vof{V : \rfoc{A^+}} &
      \tctx ; \ofill{\lfoc{B^-}} \seq \sof{S : \susp-{C^-}}}
    \and
    \infer{\tctx ; \omatch{\octx, \lfoc{A^+ \limp B^-}} \seq \sof{\lapp{V ; S} : \susp-{C^-}}}{
      \tctx ; \octx \seq \vof{V : \rfoc{A^+}} &
      \tctx ; \ofill{\lfoc{B^-}} \seq \sof{S : \susp-{C^-}}}
    \and
    \infer{\tctx ; \omatch{\lfoc{A^-_1 \with A^-_2}} \seq \sof{\fst{S} : \susp-{C^-}}}{
      \tctx ; \ofill{\lfoc{A^-_1}} \seq \sof{S : \susp-{C^-}}}
    \and
    \infer{\tctx ; \omatch{\lfoc{A^-_1 \with A^-_2}} \seq \sof{\snd{S} : \susp-{C^-}}}{
      \tctx ; \ofill{\lfoc{A^-_2}} \seq \sof{S : \susp-{C^-}}}    
  \end{infers}
  \caption{Spines}
\end{figure}

\begin{figure}
  \begin{infers}[Right-focus: $\tctx ; \octx \seq \vof{V : \rfoc{A^+}}$]
    \infer{\tctx ; x{:}\susp+{\p^+} \seq \vof{x : \rfoc{\p^+}}}{
    }
    \and
    \infer{\tctx ; \octx_1, \octx_2 \seq \vof{\vfuse{V_1}{V_2} : \rfoc{A^+_1 \fuse A^+_2}}}{
      \tctx ; \octx_1 \seq \vof{V_1 : \rfoc{A^+_1}} &
      \tctx ; \octx_2 \seq \vof{V_2 : \rfoc{A^+_2}}}
    \and
    \infer{\tctx ; \octxe \seq \vof{\vone : \rfoc{\one}}}{
    }
    \and
    \infer{\tctx ; \octx \seq \vof{\vexists{t.V} : \rfoc{\exists a{:}\tau.A^+}}}{
      \tctx \seq t : \tau &
      \tctx ; \octx \seq \vof{V : \rfoc{\subst{t/a}{A^+}}}}
    \and
    \infer{\tctx ; \octx \seq \vof{N : \rfoc{A^-}}}{
      \tctx ; \octx \seq \nof{N : A^-}}
  \end{infers}
  \caption{Values}
\end{figure}

\begin{figure}
  \begin{infers}[Pattern: $\tctx ; \octx \pseq \pof{p : A^+}$]
    \infer{\tctxe ; x{:}\susp+{\p^+} \pseq \pof{x : \p^+}}{
    }
    \and
    \infer{\tctx_1, \tctx_2 ; \octx_1, \octx_2 \pseq \pof{\pfuse{p_1}{p_2} : A^+_1 \fuse A^+_2}}{
      \tctx_1 ; \octx_1 \pseq \pof{p_1 : A^+_1} &
      \tctx_2 ; \octx_2 \pseq \pof{p_2 : A^+_2}}
    \and
    \infer{\tctxe ; \octxe \pseq \pof{\pone : \one}}{
    }
    \and
    \infer{\tctx, a{:}\tau ; \octx \pseq \pof{\pexists{a.p} : \exists a{:}\tau.A^+}}{
      \tctx ; \octx \pseq \pof{p : A^+}}
    \and
    \infer{\tctxe ; x{:}A^- \pseq \pof{x : A^-}}{
    }
  \end{infers}
  \caption{Patterns}
\end{figure}

Traces $T$ are sequences of steps $\tstep{p <- R}$, where the atomic term $R$ specifies the transition's inputs and $p$ matchs the transition's outputs.   
\begin{alignat*}{2}
  &\text{Normal terms}\quad & N &::= \tlam{a.N} \mid \rlam{p.N} \mid \llam{p.N} \mid \pair{N_1, N_2} \mid \lett{T in V} \\
  &\text{Atomic terms}\quad & R &::= \atm{c . S} \mid \atm{x . S} \\
  &\text{Spines} & S &::= \tapp{t ; S} \mid \rapp{V ; S} \mid \lapp{V ; S} \mid \fst{S} \mid \snd{S} \mid \snil \\
  &\text{Values} & V &::= x \mid \vfuse{V_1}{V_2} \mid \vone \mid \vexists{t.V} \mid N \\
  &\text{Patterns} & p &::= x \mid \pfuse{p_1}{p_2} \mid \pone \mid \pexists{a.p} \\
  &\text{Traces} & T &::= \tnil \mid \tstep{p <- R; T}
\end{alignat*}



\subsubsection{Concurrent equality}\label{sec:concurrent-equality}






\subsubsection{Fairness}\label{sec:fairness}

\NewDocumentCommand{\headpath}{m}{(#1)^p}
A trace $\tof{T :: \octx_0 \trans[\omega][]}$ is unfair if there exist $i$ and $R'_i, R'_{i+1}, \dotsc$ such that:
\begin{enumerate}[label=\alph*., ref=\alph*]
\item for all $j \geq i$, there exist $p'_j$ and $\octx'_j$ such that $\tof{\tstep{p'_j <- R'_j} :: \octx_j \trans \octx'_j}$;
\item $\headpath{R'_j} = \headpath{R'_{j+1}}$ for all $j \geq i$;
\item $\headpath{R_j} \neq \headpath{R'_j}$ for all $j \geq i$.
\end{enumerate}


\begin{equation*}
  \octx_j = \omatch[_j]{\octx_{R_j}}
\end{equation*}

\begin{equation*}
  \octx_{R_j} \seq \aof{R_j : \susp-{\monad{A^+_j}}}
\end{equation*}
\end{document}

%%% Local Variables:
%%% TeX-master: "ordered-lp"
%%% End:


\end{document}
