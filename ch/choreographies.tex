% arara: pdflatex
% arara: pdflatex
% arara: biber
% arara: pdflatex
% arara: pdflatex
% \documentclass{../hdeyoung-proposal}
\documentclass[
  class=../hdeyoung-proposal,
  crop=false
]{standalone}

\usepackage[subpreambles]{standalone}

\usepackage{ordered-logic}
\usepackage{binary-counter}

\addbibresource{../proposal.bib}

\begin{document}

\section{Choreographies}\label{sec:choreographies}

Traditionally, concurrency is phrased as the composition of interacting, locally executing processes.
As the binary counter example from \cref{sec:olp-intuition:concurrency} demonstrates, a notion of concurrency based on indistinguishable interleavings of independent rewrites arises naturally in ordered logic programming.
% a notion of concurrency based on indestinguishable interleavings arises naturally in ordered logic programming.
% However, it is not as clear how to identify a notion of process
But where are the locally executing processes?

Taking a formula-as-process view \autocites{Miller:ELP92}{Cervesato+Scedrov:IC09}, the processes are the ordered logic program's atomic propositions.
% This thesis proposes that the atomic propositions in an ordered logic program are the processes.
% \fxnote{[How much of this is already implied by a formula-as-process interpretation?]}%
The program's clauses, accordingly, serve to specify the valid interactions among processes. 
In the binary counter (\cref{sec:olp-intuition:binary-counter}),
% \wc{specification}[\st{program}], 
for example, $\eps$, $\bit{0}$, $\bit{1}$, and $\inc$, among others,
% $\dec$, $\bit[']{0}$, $\zero$, and $\succ$ 
are
% all 
atoms-as-processes,
% whose interactions are governed by the program's rules.
% In particular, the rule
and the clause
\begin{equation*}
  \bit{1} \fuse \inc \lrimp \monad{\inc \fuse \bit{0}}
\end{equation*}
% says that neighboring $\bit{1}$ and $\inc$ processes \fxnote{should be able to} interact to form neighboring $\inc$ and $\bit{0}$ processes.
says that one valid interaction is for neighboring $\bit{1}$ and $\inc$ processes to \wc{coordinate}[\st{cooperate}] to become neighboring $\inc$ and $\bit{0}$ processes.
% (with similar readings for the other clauses).

The program's clauses don't tell the full story, however:
the clauses globally specify \emph{what} are valid interactions but not \emph{how} to realize \wc{them}[\st{those interactions}] locally.
% The program is thus only a \vocab{specification}, with the how instead being supplied by the logic programming language's operational semantics.
In ordered logic programming, the how is traditionally supplied by an operational semantics in which an omniscient central conductor, having the benefit of a global view of all atoms, directs the atoms' interactions according to the program's clauses.
% The how is instead supplied by the language's operational semantics.
% % The  language's operational semantics instead supplies the how.
% % The program is thus only a \vocab{specification}, with
% In the usual operational semantics for ordered logic programming, there is a central \enquote{conductor} who, having the benefit of a global view of all atoms, directs the atoms' interactions according to the program's clauses.
But because they rely so heavily on the central conductor, processes using this semantics are no more than superficially \wc{local}[\st{distributed}].
% Under this semantics, however, processes are only nominally distributed because they rely so heavily on the central conductor.


To be truly \wc{local}[\st{distributed}], the processes should instead communicate directly with their neighbors to identify which, if any, of the valid interactions are possible for them at that moment.
% For instance, by communicating directly with its left-hand neighbor, an $\inc$ process might learn that that neighbor is a $\bit{1}$ process and that the above clause therefore applies; further direct communication between the two processes would effect
For instance, by communicating directly with its left-hand neighbor, an $\inc$ process might learn that that neighbor is a $\bit{1}$ process; with further direct communication, the $\bit{1}$ and $\inc$ processes could coordinate to effect the above\fxnote{\ globally specified} interaction.
% How does $\bit{1}$, for example, learn that its right-hand neighbor is $\inc$ and that the above clause therefore applies?


So, the distinction being drawn here is one between a \vocab{specification} and its \vocab{choreography}---the what and the how.
% \fxnote{\st{The original program is only a specification of the valid process interactions, whereas a choreography is a pattern of communication that implements that specification.}}
A specification is the original program, which serves as a \emph{global} description of the valid process interactions; a choreography%
\footnote{We borrow the term \enquote*{choreography} from the literature on session-based concurrency.
The analogy is intended only as a loose one, however, and should not be taken to imply a precise, technical correspondence.}
is a \emph{local}\fxnote{\ message-passing} implementation of that specification.%
% A specification is the original program used to describe the valid process interactions, whereas a choreography is a pattern of communication that implements that specification%
% \footnote{Notice that we always speak of a choreography relative to a specification, just as an implementation is always relative to an abstraction.}%
% , and which must be given by the programmer, at least implicitly.%
%, which must also be given by the programmer, at least implicitly.%
\footnote{Notice that a choreography is always relative to a given specification.}

% Designing a one-size-fits-all distributed operational semantics appears to be difficult, however.
% We could try to design a one-size-fits-all local operational semantics, but this appears to be difficult.
Ideally, an operational semantics would automatically generate a choreography from the specification supplied by the programmer, but designing such a semantics\fxnote{\ \st{unfortunately}} appears to be difficult.
% % Designing a distributed operational semantics that is uniformly suitable appears to be difficult, however.
% % Designing a uniformly suitable distributed operational semantics appears to be difficult, however.
Different specifications will often require different patterns of interprocess communication;
sometimes a specification will even admit several choreographies, and, more often than not, the programmer will want to exercise control in those cases.
% % Therefore, rather than relying on a one-size-fits-all operational semantics, the programmer must indicate the intended pattern of communication for each program.
% Not having a one-size-fits-all local operational semantics, the programmer himself must indicate the intended pattern of communication for each program.
Therefore, not having a one-size-fits-all local operational semantics, the programmer himself must supply the choregraphy, at least implicitly.

In the previous \lcnamecref{sec:ordered-lp}, we saw several examples of specifications (characterized as forward-chaining ordered logic programs), including the aforementioned binary counter supporting increment and decrement operations (\cref{sec:olp-intuition:binary-counter,sec:olp-intuition:decrements}).
We'll now describe what counts as a choreography, first with informal examples and then with formal definitions.



% % Designing a one-size-fits-all distributed operational semantics appears to be difficult, however.
% % We could try to design a one-size-fits-all local operational semantics, but this appears to be difficult.
% Ideally, the operational semantics would localize programs in this way, but, unfortunately, designing such a semantics\fxnote{\ \st{that is also uniformly suitable}} appears to be difficult.
% % % Designing a distributed operational semantics that is uniformly suitable appears to be difficult, however.
% % % Designing a uniformly suitable distributed operational semantics appears to be difficult, however.
% Different programs will often require different patterns of interprocess communication;
% sometimes a program will even admit several communication patterns, and, more often than not, the programmer will want to exercise control in those cases.
% % Therefore, rather than relying on a one-size-fits-all operational semantics, the programmer must indicate the intended pattern of communication for each program.
% Not having a one-size-fits-all local operational semantics, the programmer himself must indicate the intended pattern of communication for each program.

% % The distinction being drawn here is one between a \vocab{specification} and its \vocab{choreography}---the what and the how.
% % The original program serves only as a specification of what are valid process interactions, whereas the choregraphy is the programmer's intended 

% So, the distinction being drawn here is one between a \vocab{specification} and its \vocab{choreography}---the what and the how.
% % \fxnote{\st{The original program is only a specification of the valid process interactions, whereas a choreography is a pattern of communication that implements that specification.}}
% A specification is the original program, which serves as a global description of the valid process interactions; a choreography is a local\fxnote{, message-passing} implementation of that specification.%
% % A specification is the original program used to describe the valid process interactions, whereas a choreography is a pattern of communication that implements that specification%
% % \footnote{Notice that we always speak of a choreography relative to a specification, just as an implementation is always relative to an abstraction.}%
% % , and which must be given by the programmer, at least implicitly.%
% \footnote{We borrow the term \enquote*{choreography} from the literature on session-based concurrency.
% The analogy is intended only as a loose one, however, and should not be taken to imply a precise, technical correspondence.}%
% %, which must also be given by the programmer, at least implicitly.%
% \footnote{Notice that a choreography is always relative to a given specification.}

% In the previous \lcnamecref{sec:??}, we saw several examples of specifications (characterized as forward-chaining ordered logic programs), including a binary counter supporting increment and decrement operations.
% We'll now describe what counts as a choreography, first with informal examples and then with formal definitions.

% To build intuition, we'll now describe choregraphies by example


% , adapting terminology from the concurrency literature,


% So, unfortunately, 


% Unfortunately, because different programs will require different patterns of communication among processes, we won't be able to leave the how up to the operational semantics.
% The programmer will want control over the communication patterns.



% Borrowing terminology from the literature on sessions

% In session terminology, the logic program with a centralized operational semantics is known as an orchestration of processes, whereas the desired distributed semantics is known as a choreography.




% \mbox{}\\

% The program's clauses don't tell the full story, however:
% The clauses specify what are valid interactions but not \emph{how} to realize those interactions; the \enquote*{how} is instead supplied by the logic programming language's operational semantics.
% In the usual operational semantics, there is a central \enquote{conductor} who, having the benefit of a global view of all atoms, directs the atoms' interactions according to the program's clauses.

% However, because they rely so heavily on the central conductor, processes using this semantics are no more than superficially distributed.
% % Under this semantics, however, processes are only nominally distributed because they rely so heavily on the central conductor.
% To be truly distributed, the processes should instead communicate directly with their neighbors to identify which, if any, of the valid interactions are possible for them at that moment.

% It's difficult to argue that this centralized \enquote{how} is suitable for \emph{distributed} processes, however.
% The distributed processes should instead communicate directly with their neighbors to identify which, if any, of the valid interactions are possible for them at that moment.
% How does $\bit{1}$, for example, learn that its right-hand neighbor is $\inc$ and that the above clause therefore applies?

% In session terminology, the logic program with a centralized operational semantics is known as an orchestration of processes, whereas the desired distributed semantics is known as a choreography.




% The program's clauses do not tell the full story, however: the clauses specify what are valid interactions but not \emph{how} to realize those interactions.
% In the usual operational semantics, the \enquote{how} is supplied by providing a central \enquote{conductor} that, having the benefit of a global view\fxnote{\ \st{of all atoms}}, manipulates the atoms according to the program's clauses.
% But it's difficult to argue that this centralized \enquote{how} is suitable for \emph{distributed} processes.
% Distributed processes should instead communicate directly with their neighbors to identify which, if any, of the valid interactions are possible for them at that moment.
% How does $\bit{1}$, for example, learn that its right-hand neighbor is $\inc$ and that the above clause therefore applies?






% This isn't the full story, however:
% % The program's clauses specify \emph{what} are valid interactions but not \emph{how} to achieve those interactions.
% the program's clauses specify what are valid interactions but not \emph{how} to achieve them.
% The \enquote{how} is provided by the logic programming language's operational semantics.
% The usual operational semantics


% How do $\bit{1}$ and $\inc$, for example, learn that they are neighbors and that the above clause therefore applies?


% the \enquote{how}it is provided by the logic programming language's operational semantics.
% The usual operational semantics for logic programming 

% This isn't the full story, however.
% The usual operational semantics for ordered logic programming assumes a central \enquote{puppeteer} that has a global view of all atoms and manipulates them according to the program's clauses.
% It's difficult to argue that this centralization is appropriate for distributed processes, however.
% Instead, the processes should communicate directly to identify their neighbors and thereby deduce which, if any, of the valid interactions are possible for them at that moment.
% But this communication is left unspecified in the original logic program.
% How does $\bit{1}$, for example, learn that its right-hand neighbor is $\inc$ and that the above clause therefore applies?

% % What's left unspecified in the ordered logic program is how the distributed processes communicate to identify their neighbors and thereby deduce which, if any, of the valid interactions are possible for them at that moment.
% % % Using a communication protocol that is left unspecified in the program, the atoms deduce 
% % How does $\bit{1}$, for example, learn that its right-hand neighbor is $\inc$ and that the above clause therefore applies?

% Orchestration vs. choreography


% arara: pdflatex
% arara: pdflatex
% arara: biber
% arara: pdflatex
% arara: pdflatex
% \documentclass{../hdeyoung-proposal}
\documentclass[
  class=../hdeyoung-proposal,
  crop=false
]{standalone}

\usepackage{ordered-logic}
\usepackage{basic-atoms}
\usepackage{binary-counter}

\crefname{choreography}{chor.}{chors.}
\Crefname{choreography}{Chor.}{Chors.}

\DeclareAcronym{BHK}{
  short = BHK,
  long  = Brouwer-Heyting-Kolmogorov
}

\DeclareAcronym{JILL}{
  short = JILL,
  long  = judgmental intuitionistic linear logic
}

\DeclareAcronym{ILL}{
  short = ILL,
  long  = intuitionistic linear logic
}

\DeclareAcronym{SML}{
  short = SML,
  long  = Standard ML
}

\DeclareAcronym{SOS}{
  short = SOS,
  short-indefinite = an,
  long = structural operational semantics
}

\DeclareAcronym{SSOS}{
  short = SSOS,
  short-indefinite = an,
  long = substructural operational semantics
}

\DeclareAcronym{SILL}{
  short = SILL,
  long = session-typed intuitionistic linear logic
}

\DeclareAcronym{CLF}{
  short = CLF,
  long = the concurrent logical framework
}


\begin{document}

\subsection{Choreographies by example}\label{sec:chor-by-example}

\subsubsection{The binary counter}\label{sec:chor-example-counter}

In giving the intuition behind the binary counter specification (\cref{sec:olp-intuition:binary-counter}), we described the $\inc$ atoms % as moving --- moving past any $\bit{1}$s and eventually stopping at the $\eps$ or right-most $\bit{0}$.
as moving up the counter.
% , a subliminal hint that $\inc$s are like messages.
% This suggests a choreography in which $\inc$ processes take the active lead:
This hints that $\inc$s are a bit like messages, and suggests a choreography in which $\inc$ processes initiate the interaction:
First, each $\inc$ process sends a message, $\inc[<-]$, to its left-hand neighbor, thereby notifying that neighbor of its existence, and then the $\inc$ process terminates.
If the neighbor is $\eps$, $\bit{0}$, or $\bit{1}$, then, upon receiving the $\inc$'s message, that neighbor takes full responsibility for completing the corresponding interaction.

Expressed as an ordered logic program in its own right, this choreography is:
\begin{equation}\label[choreography]{chor:oop-counter}
  \!\begin{aligned}
    &\inc \lrimp \monad{\inc[<-]} \\
    &\eps \fuse \inc[<-] \lrimp \monad{\eps \fuse \bit{1}} \\
    &\bit{0} \fuse \inc[<-] \lrimp \monad{\bit{1}} \\
    &\bit{1} \fuse \inc[<-] \lrimp \monad{\inc \fuse \bit{0}}
    \text{\,,}
  \end{aligned}
\end{equation}
where the $\inc$, $\eps$, $\bit{0}$, and $\bit{1}$ atoms are still viewed as processes, but the $\inc[<-]$ atom, which is formally distinct from $\inc$, is viewed as a message.
%
Two properties are noteworthy:
\begin{description}[font=\normalfont\itshape, leftmargin=\parindent, labelindent=\leftmargin]
\item[Locality.]
% First, notice that
In this choreography, each clause's premise depends on exactly one process atom and at most one message atom.
Consequently, each process's decisions are entirely local: the $\inc$ process always sends $\inc[<-]$ regardless of its neighbors, and the $\eps$ and $\bit{}$ processes act (independently) only after receiving an $\inc[<-]$ message.%
\footnote{In {SSOS} terminology, processes that act regardless of their neighbors, like $\inc$ here, would be termed \vocab{active} propositions; processes that wait to receive a message, like $\eps$, $\bit{0}$, and $\bit{1}$ here, would be termed \vocab{latent} propositions; and messages, like $\inc[<-]$ here, would be termed \vocab{passive} propositions.}

Locality serves to ensure that the choreography may be read as a sensible message-passing program.
A clause such as $\inc[<-] \lrimp \monad{{\dots}}$, whose premise does not contain a process atom, is not message-passing because no process receives the $\inc[<-]$ message.
%
\item[Specification-preserving.]
% % Second, notice that
% The choreography exposes the same $\eps$, $\bit{}$, and $\inc$ processes as the original binary counter specification; the last three clauses of the choreography differ from the specification's clauses only in the substitution of $\inc[<-]$ for $\inc$ in their premises.
The choreography exposes the same $\eps$, $\bit{}$, and $\inc$ processes as the original binary counter specification.
Its clauses differ from those of the specification only in the substitution of $\inc[<-]$ for $\inc$ in their premises.
(The choreography also includes an $\inc \lrimp \monad{\inc[<-]}$ clause to justify that substitution.)
In this sense, there is a very strong equivalence between the two programs.
The choreography does not fundamentally alter the specification---it only refines that specification by making the communication patterns explicit.
\end{description}
%
% In this sense, there is a strong equivalence between the 
% The choreography does not fundamentally alter the implementation given in the original program---it only refines that implementation by making the communication patterns explicit.
% In this sense, there is a strong equivalence between, which will be made precise in \cref{??}
%
% Notice that this choreography \wc{refactors} the original program so that each new clause depends on exactly one process atom and at most one message atom.
% In this way, each process's decisions are completely local: the $\inc$ process always sends $\inc[<-]$ regardless of its neighbors, and the $\eps$ and $\bit{}$ processes act only after receiving an $\inc[<-]$ message.%
% \footnote{In \ac{SSOS} terminology, processes that act regardless of their neighbors, like $\inc$, would be termed \vocab{active} propositions; processes that wait to receive a message, like $\eps$, $\bit{0}$, and $\bit{1}$, would be termed \vocab{latent} propositions; and messages, like $\inc[<-]$, would be termed \vocab{passive} propositions.}
%
It's convenient to think of the programmer as supplying this choreography in full, but in practice the programmer might only give the substitution, \eg\ $\inc[<-]$ for $\inc$.

\subsubsection{Messages can flow in both directions}\label{sec:chor-binary-count}

In our binary counter specification with decrements (\cref{sec:olp-intuition:decrements}), $\dec$s propagate up the counter similarly to $\inc$s, with the difference that each $\dec$ atom eventually gives rise to either a $\zero$ or $\suc$ that travels back down the counter.
Once again, this hints that $\dec$s, $\zero$s, and $\suc$s are like messages.
These can be incorporated into the counter's choreography:
% We can also incorporate decrements into the counter's choreography.
\begin{itemize}
\item Each $\dec$ process sends a message, $\dec[<-]$, to its left-hand neighbor and terminates.
      If the neighbor is $\eps$, $\bit{0}$, or $\bit{1}$, then, upon receiving the message, that neighbor completes the corresponding interaction given in the specification.
\item Each $\zero$ or $\suc$ process sends a message, $\zero[->]$ or $\suc[->]$, respectively, to its \emph{right-hand} neighbor and terminates.
      If the neighbor is $\bit[']{0}$, then, upon receiving the message from $\zero$ or $\suc$, that neighbor completes the corresponding interaction.
\end{itemize}
To account for decrements, the binary counter's choreography is therefore extended with the following clauses:
\begin{equation}
  \!\begin{aligned}
    &\dec \lrimp \monad{\dec[<-]} \\
    &\eps \fuse \dec[<-] \lrimp \monad{\eps \fuse \zero} \\
    &\bit{0} \fuse \dec[<-] \lrimp \monad{\dec \fuse \bit[']{0}} \\
    &\bit{1} \fuse \dec[<-] \lrimp \monad{\bit{0} \fuse \suc} \\[1.5\jot]
    % 
    &\zero \lrimp \monad{\zero[->]} \\
    &\suc \lrimp \monad{\suc[->]} \\
    &\zero[->] \fuse \bit[']{0} \lrimp \monad{\bit{0} \fuse \zero} \\
    &\suc[->] \fuse \bit[']{0} \lrimp \monad{\bit{1} \fuse \suc}
    \,.
  \end{aligned}
\end{equation}
% Once again, the atoms that are decorated with arrows are formally distinct from their undecorated counterparts.
(As before, the atoms that are decorated with arrows are formally distinct from their undecorated counterparts.)

This extended choreography illustrates that message atoms may be either left-directed, like $\inc[<-]$ and $\dec[<-]$, or right-directed, like $\zero[->]$ and $\suc[->]$.
% Moreover, a message's direction determines the structure of premises in which it is received:
% a left-directed (right-directed) message must arrive at the receiving process's right (resp., left) side, otherwise the message would not be traveling from left to right (resp., right to left).
% 
% Because it is traveling left-to-right, a left-directed message must always arrive at the right-hand side of its recipient; dually, a right-directed message must always arrive at the left-hand side of its recipient.
Because a left-directed message travels from right to left, it must always arrive at the right-hand side of its recipient; dually, a right-directed message must always arrive at the left-hand side of its recipient.
This directionality is another aspect of locality, and it further constrains the structure of a choreography's premises.  For instance, $\bit{1} \fuse \inc[<-]$ and $\zero[->] \fuse \bit[']{0}$ are well-formed premises because each message flows toward its recipient, whereas premises of the forms $\q[<-] \fuse \p$ or $\p \fuse \q[->]$ are not well-formed because no process is present to receive the $\q[<-]$ or $\q[->]$ message where it arrives.

As well as retaining locality, notice that this extended choreography continues to be specification-preserving:
the choreography's clauses differ from those of the specification only in the substitution of $\dec[<-]$, $\zero[->]$, and $\suc[->]$ for their undecorated counterparts (with three additional clauses in the choreography to justify those substitutions).


\subsubsection{Choreographies are not always unique}\label{sec:mult-chor-are}

As alluded to previously, for some specifications multiple choreographies are possible.
% for the same specification.

This is true of our binary counter specification.
% To illustrate, let's return to the binary counter specification without decrements.
% Instead of having the $\inc$ 
%
In the first choreography (\cref{chor:oop-counter}),
% In the above choreography, 
the $\inc$ processes initiate the interaction but leave all remaining work to the $\eps$, $\bit{0}$, and $\bit{1}$ processes alone.
(To simplify the example, we'll ignore decrements for now.)
Alternatively, the $\inc$ processes could wait for $\eps$, $\bit{0}$, or $\bit{1}$ to initiate the interaction, but thereafter take full responsibility for its completion.
%  Another choreography of the binary counter has $\inc$ taking responsibility:

% In this choreography, 
Specifically, each $\eps$, $\bit{0}$, and $\bit{1}$ process sends an identifying message, $\eps[->]$, $\bit{0}[->]$, or $\bit{1}[->]$, to its right-hand neighbor and then terminates.
If the neighbor is $\inc$, then, upon receiving the message, that $\inc$ completes the corresponding interaction.
% takes responsibility for carrying out the corresponding clause of the specification.
\begin{equation}
  \!\begin{aligned}
    &\eps \lrimp \monad{\eps[->]} \\
    &\bit{0} \lrimp \monad{\bit{0}[->]} \\
    &\bit{1} \lrimp \monad{\bit{1}[->]} \\[1.5\jot]
    % 
    &\eps[->] \fuse \inc \lrimp \monad{\eps \fuse \bit{1}} \\
    &\bit{0}[->] \fuse \inc \lrimp \monad{\bit{1}} \\
    &\bit{1}[->] \fuse \inc \lrimp \monad{\inc \fuse \bit{0}}
  \end{aligned}
\end{equation}
Once again, this choreography possesses the locality and specification-preserving properties.

% Owing to the difference in roles held by, these two choreographies have distinct flavors.
% These two choreographies
These two choreographies
% presented thus far
have distinct flavors, owing to the different sending and receiving roles that they assign to $\inc$ and $\eps$ and $\bit{}$ processes.
Our first choreography has an object-oriented flavor: by sending an $\inc[<-]$ message, the $\inc$ method dispatches on the receiving object's class---either $\eps$, $\bit{0}$, or $\bit{1}$.
In contrast, this new choreography has a more functional flavor, with $\inc$ acting like a function that receives its argument as a message---either $\eps[->]$, $\bit{0}[->]$, or $\bit{1}[->]$.

% The increment method dispatches on 
% $\inc$ invokes the increment method on the neighboring object by sending an $\inc[<-]$ message
% There, the $\inc$ method sends an $\inc[<-]$ message like a method that dispatches on the class of the recipient object---either $\eps$, $\bit{0}$, or $\bit{1}$.
% Our first choreography has an object-oriented flavor, with $\inc$ like a method that dispatches on the class of the recipient object---either $\eps$, $\bit{0}$, or $\bit{1}$.
% In contrast, this second choreography has a more functional flavor, with 

% This alternate choreography has a funcitonal flavor: $\inc$ can be viewed as a function on the $\eps$-and-$\bit{}$ representation of data.
% In contrast, the previous choreography has a more object-oriented flavor

% The difference in sender and recipient between this alternate choreography and the previous one gives the two choreographies different flavors.
% In this alternate choreography
% In constrast, the previous choreography has a more object-oriented flavor, with $\inc$ being a method that dispatches on the class of the recipient---either $\eps$, $\bit{0}$, or $\bit{1}$.

% If we view the $\eps$ and $\bit{}$ processes as data, then this alternate choreography has a functional flavor.
% In contrast, the previous choreography has an object-oriented flavor, with a dynamic dispatch of $\inc[<-]$ on the recipient.

\subsubsection{Two non-choreographies}\label{sec:non-choreographies}

Another, more complex implementation of the binary counter specification divides the work of completing the interaction among the $\eps$ and $\bit{}$ and (the continuation of) the $\inc$ processes:
\begin{equation}
  \!\begin{aligned}
    &\inc \lrimp \monad{\inc[^l, <-] \fuse \inc[^r]} \\[1.5\jot]
    % 
    &\eps \fuse \inc[^l, <-] \lrimp \monad{\eps \fuse \eps[^r, ->]} \\
    &\bit{0} \fuse \inc[^l, <-] \lrimp \monad{\bit{0}[^r, ->]} \\
    &\bit{1} \fuse \inc[^l, <-] \lrimp \monad{\inc \fuse \bit{1}[^r, ->]} \\[1.5\jot]
    % 
    &\eps[^r, ->] \fuse \inc[^r] \lrimp \monad{\bit{1}} \\
    &\bit{0}[^r, ->] \fuse \inc[^r] \lrimp \monad{\bit{1}} \\
    &\bit{1}[^r, ->] \fuse \inc[^r] \lrimp \monad{\bit{0}} \,.
  \end{aligned}
\end{equation}
In this implementation, $\inc$ first sends an $\inc[^l, <-]$ message to its left-hand neighbor and then waits for a response as process $\inc[^r]$.
Upon receiving an $\inc[^l, <-]$ message, the recipient process, either $\eps$, $\bit{0}$, or $\bit{1}$, partially completes the interaction and sends an identifying message to its right-hand neighbor, which is necessarily an $\inc[^r]$ process.
The $\inc[^r]$ process finishes the interaction once it receives the identifying message.

This implementation is equivalent to the binary counter specification in that it ultimately exposes the same $\eps$, $\bit{}$, and $\inc$ processes.
However, it is not specification-preserving under the informal definition that we have used thus far.
% In contrast with the choreographies, this implementation's main clauses are not in one-to-one correspondence with those of the specification.
% In contrast with the choreographies, this implementation's main clauses are not a simple decoration of the specification's clauses: each of the specification's clauses is spread across several clauses here.
In contrast with the choreographies, this implementation does more than simply refine the specification by making the communication explicit: using a temporary $\inc[^r]$ process, it spreads each of the specification's clauses across several clauses.

Although this implementation is not specification-preserving for the binary counter specification, and therefore not a choreography, the programmer can nevertheless achieve the same behavior by changing the \emph{specification}.
If the specification is changed to be
\begin{equation}
  % \!\begin{aligned}[t]
  %   &\inc[^l] \lrimp \monad{\inc[^l, <-]} \\
  %   &\eps[^r] \lrimp \monad{\eps[^r, ->]} \\
  %   &\bit{0}[^r] \lrimp \monad{\bit{0}[^r, ->]} \\
  %   &\bit{1}[^r] \lrimp \monad{\bit{1}[^r, ->]}
  % \end{aligned}
  % \qquad
  \!\begin{aligned}
    &\inc \lrimp \monad{\inc[^l] \fuse \inc[^r]} \\
    &\eps \fuse \inc[^l] \lrimp \monad{\eps \fuse \eps[^r]} \\
    &\bit{0} \fuse \inc[^l] \lrimp \monad{\bit{0}[^r]} \\
    &\bit{1} \fuse \inc[^l] \lrimp \monad{\inc \fuse \bit{1}[^r]} \\
    &\eps[^r] \fuse \inc[^r] \lrimp \monad{\bit{1}} \\
    &\bit{0}[^r] \fuse \inc[^r] \lrimp \monad{\bit{1}} \\
    &\bit{1}[^r] \fuse \inc[^r] \lrimp \monad{\bit{0}} \,,
  \end{aligned}
\end{equation}
then the above implementation is a choreography for \emph{this} specification.

% Although this implimentation is not specification-preserving, and therefore not a choreography, for the binary counter specification (\cref{??}), the programmer can nevertheless acheive the same behavior by changing the \emph{specification}.
% Instead of using the binary counter specification from \cref{??}, the following specification could be used because one of its choreographies, which uses $\inc[^l, <-]$, $\eps[^r, ->]$, $\bit{0}[^r, ->]$, and $\bit{1}[^r, ->]$ messages, is nearly identical the disallowed implimentation.
% \begin{equation*}
%   % \!\begin{aligned}[t]
%   %   &\inc[^l] \lrimp \monad{\inc[^l, <-]} \\
%   %   &\eps[^r] \lrimp \monad{\eps[^r, ->]} \\
%   %   &\bit{0}[^r] \lrimp \monad{\bit{0}[^r, ->]} \\
%   %   &\bit{1}[^r] \lrimp \monad{\bit{1}[^r, ->]}
%   % \end{aligned}
%   % \qquad
%   \!\begin{aligned}[t]
%     &\inc \lrimp \monad{\inc[^l] \fuse \inc[^r]} \\
%     &\eps \fuse \inc[^l] \lrimp \monad{\eps \fuse \eps[^r]} \\
%     &\bit{0} \fuse \inc[^l] \lrimp \monad{\bit{0}[^r]} \\
%     &\bit{1} \fuse \inc[^l] \lrimp \monad{\inc \fuse \bit{1}[^r]} \\
%     &\eps[^r] \fuse \inc[^r] \lrimp \monad{\bit{1}} \\
%     &\bit{0}[^r] \fuse \inc[^r] \lrimp \monad{\bit{1}} \\
%     &\bit{1}[^r] \fuse \inc[^r] \lrimp \monad{\bit{0}} \,.
%   \end{aligned}
% \end{equation*}


% Although this implementation ultimately exposes the same $\eps$, $\bit{}$, and $\inc$ processes

% Even so\fxnote{\ \st{Even though this choreography introduces and auxiliary process atom}}, we still consider it to be a valid choreography: $\inc[']$ is only temporary, leaving the underlying specification fundamentally unchanged.

Another program that is equivalent to the binary counter specification, in the sense that the two track the same value, is
\begin{equation}
  \!\begin{aligned}
    &\inc \lrimp \monad{\inc[<-]} \\
    &\num{N} \fuse \inc[<-] \lrimp \monad{\num{(N{+}1)}} \,.
  \end{aligned}
\end{equation}
Nevertheless, we wouldn't consider it to be a choreography of the binary counter specification because, by using a single number held by $\num{}$ instead of a string of $\bit{}$s, it fundamentally alters the specification.
We would, however, consider it to be a choreography of a different, simple counter specification: $\num{N} \fuse \inc \lrimp \monad{\num{(N{+}1)}}$.

% % Although it is equivalent to the binary counter in the sense that it tracks the same value, we wouldn't consider the following program to be a choreography of the binary counter specification because it fundamentally alters the implementation by using a single $\num{}$ instead of a string of $\bit{}$s.
% The following program is also equivalent to the binary counter specification, in the sense that the two track the same value.
% Nevertheless, we wouldn't consider it to be a choreography of the binary counter because it fundamentally alters the specification by using a single number held by $\num{}$ instead of a string of $\bit{}$s.
% \begin{align*}
%   &\inc \lrimp \monad{\inc[<-]} \\
%   &\num{N} \fuse \inc[<-] \lrimp \monad{\num{(N{+}1)}}
% \end{align*}
% We would, however, consider it to be a choreography of a different, simple counter specification: $\num{N} \fuse \inc \lrimp \monad{\num{(N{+}1)}}$.



\end{document}

%%% Local Variables:
%%% TeX-master: "choreographies"
%%% End:


% arara: pdflatex
% arara: biber
% arara: pdflatex
% arara: pdflatex
\documentclass[
  class=../hdeyoung-proposal,
  crop=false
]{standalone}

\usepackage{ordered-logic}
\usepackage{basic-atoms}
\usepackage{tikz-cd}

\NewDocumentCommand{\chor}{}{X}
\NewDocumentCommand{\spec}{}{\Sigma}

\NewDocumentCommand{\erasemsg}{m}{(#1)^{e}}

\NewDocumentCommand{\trans}{t* t+ o}{%
  \longrightarrow
  \IfBooleanT{#1}{^*}\IfBooleanT{#2}{^+}%
  \IfValueT{#3}{_{#3}}%
}

\begin{document}

\subsection{Choreographies, formally}\label{sec:chor-formal}

Hopefully the preceding examples have given some intuition for what counts as a choreography.
To make the definition precise, we need only formalize the locality and specification-preserving properties.

\subsubsection{Locality}\label{sec:locality}

As discussed in \cref{?}, clauses such as $\bit{1} \fuse \inc[<-] \lrimp \monad{\inc[<-] \fuse \bit{0}}$ satisfy locality because the premise consists of a process that receives a message.
Although it is possible to admit clauses like this one, they come at the expense of complicating the statement of locality slightly.
Instead, it is more convenient to require that clauses be given in curried form, such as $\bit{1} \lrimp (\inc[<-] \rimp \monad{\inc[<-] \fuse \bit{0}})$.
% Because currying is easy enough to do in a preprocessing phase, this requirement is not onerous.

For curried clauses, locality holds if there is exactly one process-like atom, $\p^+$, in the premises and if any message-like atoms, $\matom[<-]$ and $\matom[->]$, are used only with right- and left-ordered implications, respectively.


\begin{definition}[Locality]
  A clause $\p^+ \lrimp A^-$ satisfies \vocab{locality} if it adheres to the following refined grammar.
  \begin{alignat*}{3}
    &\text{Negative propositions} &\quad&& A^-,B^- &::= A^- \with B^- \mid \monad{A^+} \mid \matom[<-]^+ \rimp \monad{A^+} \mid \matom[->]^+ \limp \monad{A^+} \\
    &\text{Positive propositions}      &&& A^+,B^+ &::= A^+ \fuse B^+ \mid \one \mid \matom[<-]^+ \fuse A^+ \mid A^+ \fuse \matom[->]^+ \mid \p^+ \mid A^-
  \end{alignat*}
\end{definition}
Notice that, for technical reasons related to the translation to session-typed processes (\cref{?}), this definition also requires that message-like atoms $\matom[<-]$ and $\matom[->]$ never appear in clause heads except as part of a conjunction---either $\matom[<-] \fuse A^+$ or $A^+ \fuse \matom[->]$.
Because $\one$ is available, this requirement does not restrict expressiveness.

For example, the $\inc[<-]$-choreography indeed satisfies locality because it can be rewritten as 
\begin{equation*}
  \!\begin{aligned}
    &\eps \lrimp (\inc[<-] \rimp \monad{\eps \fuse \bit{1}}) \\
    &\bit{0} \lrimp (\inc[<-] \rimp \monad{\bit{1}}) \\
    &\bit{1} \lrimp (\inc[<-] \rimp \monad{\inc[<-] \fuse \bit{0}})
    \text{\,.}
  \end{aligned}
\end{equation*}
Likewise, the $\bit{}[->]$-choreography satisfies locality:
\begin{equation*}
  \!\begin{aligned}
    &\inc \lrimp (\eps[->] \limp \monad{(\one \fuse \eps[->]) \fuse \bit{1}[->]}) \\
    &\inc \lrimp (\bit{0}[->] \limp \monad{\one \fuse \bit{1}[->]}) \\
    &\inc \lrimp (\bit{1}[->] \limp \monad{\inc \fuse \bit{0}[->]})
    \text{\,.}
  \end{aligned}
\end{equation*}

\subsubsection{Specification-preserving}\label{sec:spec-pres}

To judge that an ordered logical specification is specification-preserving, we rely on a notion of erasure that removes the assigned roles, translating message- and process-like atoms to ordinary atoms.
\begin{definition}[Role erasure]
  For atomic propositions, the \vocab{role erasure} $\erasemsg{-}$ is given by
  \begin{align*}
    \erasemsg{\matom[<-]^+} &= \matom^+ = \erasemsg{\matom[->]^+} \\
    % \erasemsg{\matom[<-]^+ \fuse A^+} &= \matom^+ \fuse \erasemsg{A^+} \\
    % \erasemsg{A^+ \fuse \matom[->]^+} &= \erasemsg{A^+} \fuse \matom^+ \\
    \erasemsg{\p^+} &= \p^+
    \,.
  \end{align*}
  Role erasures for propositions and contexts, $\erasemsg{A^+}$, $\erasemsg{A^-}$, and $\erasemsg{\octx}$, are defined compositionally, lifting role erasure for atoms.
\end{definition}
For instance, the role erasure of $\inc[<-]$ is $\erasemsg{\inc[<-]} = \inc$, matching the intuition that the message-like atom $\inc[<-]$ in the $\inc[<-]$-choreography (\cref{sec:chor-example-counter}) serves to implement the specification's $\inc$ atom.

Using role erasure, we can define the specification-preserving property as follows:
\begin{definition}[Specification-preserving]
  An ordered logic program $\chor$ is \vocab{specification-preserving} for specification $\spec$ if:
  $\octx \trans[\chor] \octx'$ in the program $\chor$ if and only if $\erasemsg{\octx} \trans[\spec] \erasemsg{\octx'}$ in the specification $\spec$.
  In other words, $\chor$ is specification-preserving for $\spec$ if $\erasemsg{-}$ is a bisimulation.
\end{definition}

Thus, to be specification-preserving, a choreography must be lock-step equivalent with its specification.
For example, the $\inc[<-]$-choreography for the binary counter specification is indeed specification-preserving because it is in lock-step equivalence with the specification.
\Cref{fig:spec-pres-inc} shows how the steps correspond, using bisimulation diagrams.

\begin{figure}[!t]
\begin{gather*}
  \begin{tikzcd}[arrow style=math font, ampersand replacement=\&]
    \omatch[^e]{\eps , \inc} \rar \dar[dash, "\erasemsg{-}"'] \& \omatch[^e]{\eps , \bit{1}} \dar[dash, "\erasemsg{-}"] \\
    \omatch{\eps , \inc[<-]} \rar \& \omatch{\eps , \bit{1}}
  \end{tikzcd}
  \qquad\qquad
  \begin{tikzcd}[arrow style=math font, ampersand replacement=\&]
    \omatch[^e]{\bit{0} , \inc} \rar \dar[dash, "\erasemsg{-}"'] \& \omatch[^e]{\bit{1}} \dar[dash, "\erasemsg{-}"] \\
    \omatch{\bit{0} , \inc[<-]} \rar \& \omatch{\bit{1}}
  \end{tikzcd}
  \\[3\jot]
  \begin{tikzcd}[arrow style=math font, ampersand replacement=\&]
    \omatch[^e]{\bit{1} , \inc} \rar \dar[dash, "\erasemsg{-}"'] \& \omatch[^e]{\inc , \bit{0}} \dar[dash, "\erasemsg{-}"] \\
    \omatch{\bit{1} , \inc[<-]} \rar \& \omatch{\inc[<-] , \bit{0}}
  \end{tikzcd}
\end{gather*}
\caption{The $\inc[<-]$-choreography of \cref{sec:chor-example-counter} is specification-preserving.\label{fig:spec-pres-inc}}
\end{figure}


% \subsection{}

% Hopefully the preceding examples have given some intuition for what counts as a choreography.
% To make the definition precise, we need only formalize the locality and specification-preserving properties.

% We use the forward-chaining ordered logic programming language described in \cref{sec:ordered-lp}, with a few restrictions.
% First, we confine ourselves to the purely propositional fragment of ordered logic programming.
% Second, to simplify the statement of locality, we require that each implication have exactly one premise: each implication has the form of $A^+ \rimp \monad{B^+}$, $A^+ \limp \monad{B^+}$, or $A^+ \lrimp \monad{B^+}$.
% These uncurried implications would not seem to place a large burden on the programmer, for it is easy enough to write $\p[_{\mathrm{2}}]^+ \fuse \p[_{\mathrm{1}}]^+ \fuse \p[_{\mathrm{3}}]^+ \lrimp \monad{B^+}$ in place of $\p[_{\mathrm{1}}]^+ \lrimp \p[_{\mathrm{2}}]^+ \limp \p[_{\mathrm{3}}]^+ \rimp \monad{B^+}$, for example.

% These restrictions can be lifted, but at the expense of complicating the presentation; because the restrictions are orthogonal to the main points of discussion, we prefer the simplifying restrictions for the present.

% \begin{alignat*}{2}
%   &\text{Negative propositions}\quad & A^- &::= A^+ \rimp \monad{B^+} \mid A^+ \limp \monad{B^+} \mid A^- \with B^- \\
%   &\text{Positive propositions}      & A^+ &::= A^+ \fuse B^+ \mid \one \mid \p^+ \mid \p[->]^+ \mid \p[<-]^+ \mid A^-
% \end{alignat*}
 


% \subsubsection{Locality}\label{sec:locality}

% \begin{definition}[Locality]
%   A clause $A^+ \lrimp \monad{B^+}$ is \emph{local} if its premise adheres to the grammar refinement $O^+$:
%   \begin{alignat*}{2}
%     O^+ &::= L^+ \fuse O^+ \mid \p^+ \mid O^+ \fuse R^+ \\
%     L^+ &::= \p[->]^+ \mid L^+_1 \fuse L^+_2 \mid \one \\
%     R^+ &::= \p[<-]^+ \mid R^+_1 \fuse R^+_2 \mid \one
%   \end{alignat*}
%   % Similarly, a right implication, $A^+ \rimp \monad{B^+}$, is local if its premise adheres to the grammar of $R^+$.
%   % Dually, a left implication, $A^+ \limp \monad{B^+}$, is local if its premise adheres to the grammar of $L^+$.
% \end{definition}


% A clause $A^+ \lrimp \monad{B^+}$ is \emph{local} if its premise adheres to the grammar refinement $O^+$:
% \begin{alignat*}{2}
%   O^+ &::= L^+ \fuse O^+ \mid \p^+ \mid O^+ \fuse R^+ \\
%   L^+ &::= \p[->]^+ \mid L^+_1 \fuse L^+_2 \mid \one \\
%   R^+ &::= \p[<-]^+ \mid R^+_1 \fuse R^+_2 \mid \one
% \end{alignat*}
% The grammar may appear to be a bit complicated, but the idea behind it is simple enough and matches the intuition behind locality:
% a local premise $O^+$ contains exactly one process atom $\p^+$ that receives right-directed messages, $\p[->]^+$, from its left and left-directed messages, $\p[<-]^+$, from its right.

% Similarly, a right implication, $A^+ \rimp \monad{B^+}$, is local if its premise adheres to the grammar of $R^+$.
% That is, right implications must receive only left-directed messages that are arriving at the implication's right-hand side.
% No process atom may appear in the premise because, as an ephemeral resource, the implication itself acts as the recipient process.
% Dually, a left implication, $A^+ \limp \monad{B^+}$, is local if its premise adhers to the grammar of $L^+$.


% % At the expense of a more complicated grammar, locality could be extended to implications with multiple premises.
% % , we could allow curried clauses, such as $\p[_1, ->]^+ \lrimp \p[_2, ->]^+ \limp \p^+ \rimp \p[_3, <-]^+ \rimp \monad{A^+}$.
% % Uncurrying clauses would not seem to place a large burden on the programmer, for it is easy enough to write $\q[->]^+ \fuse \p[->]^+ \fuse \rr^+ \fuse \s[<-]^+ \lrimp \monad{A^+}$, and this detail is anyway orthogonal to what follows.

% \subsubsection{Specification-preserving}\label{sec:spec-pres}

% To judge that an ordered logic program is specification-preserving, we rely on a notion of erasure that relates message atoms to process atoms.
% \begin{definition}[Message erasure]
%   For atomic propositions, the \vocab{message erasure} $\erasemsg{-}$ is given by
%   \begin{equation*}
%     \erasemsg{\p[->]^+} = \erasemsg{\p[<-]^+} = \erasemsg{\p^+} = \p^+
%     \,.
%   \end{equation*}
%   Message erasure of propositions and contexts, $\erasemsg{A^+}$, $\erasemsg{A^-}$, and $\erasemsg{\octx}$, is defined compositionally as the lifting of message erasure for atoms.
% \end{definition}
% For instance, the message erasure of $\inc[<-]$ is $\erasemsg{\inc[<-]} = \inc$, matching the intuition that the message $\inc[<-]$ serves to implement the specification's atom $\inc$ in the choreography from \cref{sec:chor-example-counter}.

% Using this definition, we can define the property of specification-preserving as follows:
% \begin{definition}[Specification-preserving]
%   An ordered logic program $\chor$ is \vocab{specification-preserving} for specification $\spec$ if:
%   \begin{enumerate}
%   \item\label{defn:specification-preserving:completeness} for each step $\octx \trans[\spec] \octx'$ in the specification $\spec$, there is a non-empty trace $\octx \trans+[\chor] \octx'$ in the program $\chor$; and
%   \item\label{defn:specification-preserving:soundness} for each step $\octx \trans[\chor] \octx'$ in the program $\chor$, either $\erasemsg{\octx} = \erasemsg{\octx'}$ or there is a step $\erasemsg{\octx} \trans[\spec] \erasemsg{\octx'}$ in the specification $\spec$.
%   \end{enumerate}
% \end{definition}
% Thus, to be specification-preserving for $\spec$, a choreography $\chor$ must be a sound and complete implementation of specification $\spec$:
% every step in $\spec$ must be reproducible by a non-empty trace in $\chor$ (part \labelcref{defn:specification-preserving:completeness}, completeness); and every step in $\chor$ must be either silent or its erasure reproducible by a single step in $\spec$ (part \labelcref{defn:specification-preserving:soundness}, soundness).

% % The first part of this definition requires the choreography $\chor$ to be a complete implementation of the specification $\spec$: every step in $\spec$ must be reproducible by a trace in $\chor$.
% % The second part requires $\chor$ to be a sound implementation of $\spec$: every step in $\chor$ must be either silent or reproducible by a single step in $\spec$.

% \tikzcdset{arrow style=math font}
% \tikzset{subscript/.style={shorten >=0.5em, "\ensuremath{#1}" {inner sep=0pt, sloped, at end, below right}}}

% As an example, the object-oriented choreography for the binary counter, given in \cref{sec:chor-example-counter}, is indeed specification-preserving.
% \begin{equation*}
%   \begin{tikzcd}
%     \omatch{\bit{1}, \inc}
%       \arrow[rr, subscript=\spec]
%       \arrow[dr, start anchor=south east, end anchor=north west, subscript=\chor]
%       &
%       & \ofill{\inc, \bit{0}} \\
%     & \ofill{\bit{1}, \inc[<-]} \arrow[ur, start anchor=north east, end anchor=south west, subscript=\chor] &
%   \end{tikzcd}
% \end{equation*}

% \begin{equation*}
%   \begin{tikzcd}
%     \omatch{\bit{1}, \inc[<-]}  \rar[subscript=\chor]  \dar[dash, "\erasemsg{-}" {below, sloped}]
%       & \ofill{\inc, \bit{0}}   \dar[dash, "\erasemsg{-}"' {above, sloped}] \\
%     \ofill[^e]{\bit{1}, \inc}   \rar[subscript=\spec]
%       & \ofill[^e]{\inc, \bit{0}}
%   \end{tikzcd}
% \end{equation*}
% \begin{gather*}
%   \begin{lgathered}
%     \omatch{\bit{1}, \inc} \trans[\spec] \ofill{\inc, \bit{0}} \\
%     \ofill{\bit{1}, \inc} \trans[\chor] \ofill{\bit{1}, \inc[<-]} \trans[\chor] \ofill{\inc, \bit{0}}
%   \end{lgathered}
%   \\
%   \begin{lgathered}
%     \omatch{\eps, \inc} \trans[\spec] \ofill{\eps, \bit{1}} \\
%     \ofill{\eps, \inc} \trans[\chor] \ofill{\eps, \inc[<-]} \trans[\chor] \ofill{\eps, \bit{1}}
%   \end{lgathered}
%   \\
%   \begin{lgathered}
%     \omatch{\bit{0}, \inc} \trans[\spec] \ofill{\bit{1}} \\
%     \ofill{\bit{0}, \inc} \trans[\chor] \ofill{\bit{0}, \inc[<-]} \trans[\chor] \ofill{\bit{1}}
%   \end{lgathered}
% \end{gather*}

% \begin{gather*}
%   \begin{lgathered}
%     \omatch{\inc} \trans[\chor] \ofill{\inc[<-]} \\
%     \erasemsg{\ofill{\inc}} = \ofill[^e]{\inc} = \erasemsg{\ofill{\inc[<-]}}
%   \end{lgathered}
%   \\
%   \begin{lgathered}
%     \omatch{\bit{1}, \inc[<-]} \trans[\chor] \ofill{\inc, \bit{0}} \\
%     \erasemsg{\ofill{\bit{1}, \inc[<-]}} = \ofill[^e]{\bit{1}, \inc} \trans[\spec] \ofill[^e]{\inc, \bit{0}} = \erasemsg{\ofill{\inc, \bit{0}}}
%   \end{lgathered}
%   \\
%   \begin{lgathered}
%     \omatch{\eps, \inc[<-]} \trans[\chor] \ofill{\eps, \bit{1}} \\
%     \ofill{\eps, \inc} \trans[\spec] \ofill{\eps, \bit{1}}
%   \end{lgathered}
%   \\
%   \begin{lgathered}
%     \omatch{\bit{0}, \inc[<-]} \trans[\chor] \ofill{\bit{1}} \\
%     \ofill{\bit{0}, \inc} \trans[\spec] \ofill{\bit{1}}
%   \end{lgathered}
% \end{gather*}
\end{document}

%%% Local Variables:
%%% TeX-master: "choreographies"
%%% End:


\end{document}