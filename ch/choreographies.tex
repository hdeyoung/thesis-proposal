% arara: pdflatex
% arara: pdflatex
% arara: biber
% arara: pdflatex
% arara: pdflatex
% \documentclass{../hdeyoung-proposal}
\documentclass[
  class=../hdeyoung-proposal,
  crop=false
]{standalone}

\usepackage[subpreambles]{standalone}

\usepackage{ordered-logic}
\usepackage{binary-counter}

\addbibresource{../proposal.bib}

\begin{document}

As the binary counter from \cref{sec:olp-intuition:binary-counter,sec:olp-intuition:concurrency} exemplifies, there is a notion of concurrency, based on indistinguishable interleavings of independent rewritings, that arises naturally in ordered logical specifications.
And, under a forward-chaining logic programming interpretation, these specifications can be executed by an omniscient \enquote{puppeteer} that 

And these specifications can be executed by a forward-chaining logic programming interpreter that omnisciently rewrites 

In contrast, concurrency is traditionally phrased as the composition of \vocab{communicating processes}.
Processes are not omniscient but instead execute locally; (asyncronous) processes do not communicate directly, but instead communicate by exchanging messages.
processes do not communicate on an arbitrary scale but instead communicate 

Are there communicating processes hidden within ordered logical specifications that would allow these two seemingly distinct notions of concurrency to be reconciled?

Inspired by the process-as-formula view of linear logic~\autocites{Miller:ELP92}{Cervesato+Scedrov:IC09}, we propose that each atomic proposition in an ordered logical specification serves in one of two roles:
an atom is either process-like or message-like.
These role assignments, or \vocab{choreographies}, refine the original specification to a local, message-passing specification.
% We call this role assignment a \vocab{choreography}, and it\fxnote{\ \st{serves to refine}} refines the original specification to a local, message-passing specification.
For example, in the increment fragment of the binary counter specification, $\eps$, $\bit{0}$, and $\bit{1}$ might all be process-like atoms and $\inc$ might be a message-like atom.
The clause
% Continuing with this atoms-as-processes view, the program's clauses should serve as\fxnote{\ \st{(a global description of)}} an implementation of\fxnote{\ \st{interactions among}} these processes.
% For example, the clause
\begin{equation*}
  \bit{1} \fuse \inc \lrimp \monad{\inc \fuse \bit{0}}
\end{equation*}
would specify a $\bit{1}$ process that, upon receiving an $\inc$ message at its right side, sends an $\inc$ message to the left and continues as a $\bit{0}$ process.




So, the distinction being drawn here is one between a \vocab{specification} and its \vocab{choreography}---the what and the how.
% \fxnote{\st{The original program is only a specification of the valid process interactions, whereas a choreography is a pattern of communication that implements that specification.}}
A specification is the original program, which serves as a global description of the valid interactions among atoms; a choreography%
\footnote{We borrow the term \enquote*{choreography} from the literature on session-based concurrency.
The analogy is intended only as a loose one, however, and should not be taken to imply a precise, technical correspondence.}
is a local message-passing implementation of that specification that is obtained by assigning each atom a role---either message or process.
is an assignment of message or process roles to the atoms that transforms the specification into a local message-passing implementation.
is a local message-passing implementation of that specification.%
% A specification is the original program used to describe the valid process interactions, whereas a choreography is a pattern of communication that implements that specification%
% \footnote{Notice that we always speak of a choreography relative to a specification, just as an implementation is always relative to an abstraction.}%
% , and which must be given by the programmer, at least implicitly.%
%, which must also be given by the programmer, at least implicitly.%
\footnote{Notice that a choreography is always relative to a given specification.}

Ideally, choreographies would be mechanically generated from ordered logical specifications.
But this appears to be difficult:
Different specifications often require different patterns of communication.
And some specifications even admit several choreographies, amoung which the programmer should exercise his choice.
Therefore, in the absence of a choreography generator, the programmer himself must supply the choreography.

We'll now describe what counts as a choreography, first with informal examples and then with formal definitions.


Ideally, an operational semantics would automatically generate a choreography from the specification\fxnote{\ \st{supplied by the programmer}}, but designing such a semantics appears to be difficult.
Different specifications will often require different patterns of communication;
some specifications even admit several choreographies!
sometimes one specification will even admit several choreographies, and\fxnote{\st{, more often than not,}} usually the programmer will want to exercise control in\fxnote{\ \st{those cases}} choosing one.
Therefore, being unable to rely on a one-size-fits-all\fxnote{\ \st{local}} operational semantics, the programmer himself must supply the choreography\fxnote{\st{, at least implicitly}}.

In the previous \lcnamecref{sec:ordered-lp}, we saw the aforementioned binary counter supporting increment and decrement operations (\cref{sec:olp-intuition:binary-counter,sec:olp-intuition:decrements}).
% several examples of specifications (characterized as forward-chaining ordered logic programs), including the aforementioned binary counter supporting increment and decrement operations (\cref{sec:olp-intuition:binary-counter,sec:olp-intuition:decrements}).
% In the previous \lcnamecref{sec:ordered-lp}, we saw several examples of specifications (characterized as forward-chaining ordered logic programs), including the aforementioned binary counter supporting increment and decrement operations (\cref{sec:olp-intuition:binary-counter,sec:olp-intuition:decrements}).
We'll now describe what counts as a choreography, first with informal examples and then with formal definitions.


one might suggest that an ordered logic program's atomic propositions---or, from a string rewriting perspective, letters---should be processes and that the program's clauses should serve as\fxnote{\ \st{(a global description of)}} an implementation of these processes.





Traditionally, concurrency is phrased as the composition of communicating, but locally executing, processes.
As the binary counter example from \cref{sec:olp-intuition:binary-counter,sec:olp-intuition:concurrency} demonstrates, a notion of concurrency based on indistinguishable interleavings of independent rewritings arises naturally in ordered logical specifications.

As ..., ordered logical specifications naturally give rise to a notion of concurrency based not on communicating processes but on ... .
Are there communicating processes hidden within the specifications that would allow these two seemingly distinct notions of concurrency to be reconciled?

But where are the communicating\fxnote{\ \st{locally executing}} processes?

Inspired by the process-as-formula view of linear logic~\autocites{Miller:ELP92}{Cervesato+Scedrov:IC09}, one might suggest that an ordered logic program's atomic propositions---or, from a string rewriting perspective, letters---should be processes and that the program's clauses should serve as\fxnote{\ \st{(a global description of)}} an implementation of these processes.
In the increment fragment of the binary counter program,
% \wc{specification}[\st{program}], 
for example, $\eps$, $\bit{0}$, $\bit{1}$, and $\inc$ all would be atoms-as-processes, and the clause
% Continuing with this atoms-as-processes view, the program's clauses should serve as\fxnote{\ \st{(a global description of)}} an implementation of\fxnote{\ \st{interactions among}} these processes.
% For example, the clause
\begin{equation*}
  \bit{1} \fuse \inc \lrimp \monad{\inc \fuse \bit{0}}
\end{equation*}
would implement $\bit{1}$ and $\inc$ processes that, when neighboring, react to become neighboring $\inc$ and $\bit{0}$ processes.

But this atoms-as-processes and clauses-as-implementations view is somewhat flawed.
If a clause such as the above is itself to be an implementation, then we must assume that the atoms-as-processes can \enquote{see} and identify their neighbors without any communication; otherwise how would $\bit{1}$ and $\inc$ know when to become $\inc$ and $\bit{0}$, for example?
% possess some amount of omniscience
But, in contradiction, true processes, in the sense of standard process calculi, are quite nearsighted:
% do not have the advantage of being farsighted:
\fxnote{\st{true}\ }processes can see only messages that arrive at their doorstep, not neighboring processes.

Nevertheless, a clauses-as-implementations view is still valid if some atoms serve as messages and other atoms serve as processes.
For example, if both of the $\inc$ atoms in the above clause are treated as message atoms, then that clause is an valid implementation.

% A serious conceptual objection can be raised against this atoms-as-processes and clauses-as-implentations view, however:
% If clauses are (global descriptions of) process implementations, then atoms cannot be locally executing.

So, the distinction being drawn here is one between a \vocab{specification} and its \vocab{choreography}---the what and the how.
% \fxnote{\st{The original program is only a specification of the valid process interactions, whereas a choreography is a pattern of communication that implements that specification.}}
A specification is the original program, which serves as a global description of the valid interactions among atoms; a choreography%
\footnote{We borrow the term \enquote*{choreography} from the literature on session-based concurrency.
The analogy is intended only as a loose one, however, and should not be taken to imply a precise, technical correspondence.}
is a local message-passing implementation of that specification that is obtained by assigning each atom a role---either message or process.
is an assignment of message or process roles to the atoms that transforms the specification into a local message-passing implementation.
is a local message-passing implementation of that specification.%
% A specification is the original program used to describe the valid process interactions, whereas a choreography is a pattern of communication that implements that specification%
% \footnote{Notice that we always speak of a choreography relative to a specification, just as an implementation is always relative to an abstraction.}%
% , and which must be given by the programmer, at least implicitly.%
%, which must also be given by the programmer, at least implicitly.%
\footnote{Notice that a choreography is always relative to a given specification.}

Ideally, an operational semantics would automatically generate a choreography from the specification\fxnote{\ \st{supplied by the programmer}}, but designing such a semantics appears to be difficult.
Different specifications will often require different patterns of interprocess communication;
sometimes a specification will even admit several choreographies, and\fxnote{\st{, more often than not,}} usually the programmer will want to exercise control in\fxnote{\ \st{those cases}} choosing one.
Therefore, unable to rely on a one-size-fits-all\fxnote{\ \st{local}} operational semantics, the programmer himself must supply the choreography\fxnote{\st{, at least implicitly}}.

In the previous \lcnamecref{sec:ordered-lp}, we saw the aforementioned binary counter supporting increment and decrement operations (\cref{sec:olp-intuition:binary-counter,sec:olp-intuition:decrements}).
% several examples of specifications (characterized as forward-chaining ordered logic programs), including the aforementioned binary counter supporting increment and decrement operations (\cref{sec:olp-intuition:binary-counter,sec:olp-intuition:decrements}).
% In the previous \lcnamecref{sec:ordered-lp}, we saw several examples of specifications (characterized as forward-chaining ordered logic programs), including the aforementioned binary counter supporting increment and decrement operations (\cref{sec:olp-intuition:binary-counter,sec:olp-intuition:decrements}).
We'll now describe what counts as a choreography, first with informal examples and then with formal definitions.






\section{Choreographies}\label{sec:choreographies}

Traditionally, concurrency is phrased as the composition of interacting, locally executing processes.
As the binary counter example from \cref{sec:olp-intuition:binary-counter,sec:olp-intuition:concurrency} demonstrates, a notion of concurrency based on indistinguishable interleavings of independent rewritings arises naturally in forward-chaining ordered logic programming.
But where are the locally executing processes?

Borrowing from the process-as-formula view of linear logic~\autocites{Miller:ELP92}{Cervesato+Scedrov:IC09}, one might suggest that an ordered logic program's atomic propositions---or, from a string rewriting perspective, letters---should be processes and that the program's clauses should serve as\fxnote{\ \st{(a global description of)}} an implementation of these processes.
In the increment fragment of the binary counter program,
% \wc{specification}[\st{program}], 
for example, $\eps$, $\bit{0}$, $\bit{1}$, and $\inc$ all would be atoms-as-processes, and the clause
% Continuing with this atoms-as-processes view, the program's clauses should serve as\fxnote{\ \st{(a global description of)}} an implementation of\fxnote{\ \st{interactions among}} these processes.
% For example, the clause
\begin{equation*}
  \bit{1} \fuse \inc \lrimp \monad{\inc \fuse \bit{0}}
\end{equation*}
would implement $\bit{1}$ and $\inc$ processes that, when neighboring, react to become neighboring $\inc$ and $\bit{0}$ processes.

But this atoms-as-processes and clauses-as-implementations view is somewhat flawed.
If a clause such as the above is itself to be an implementation, then we must assume that the atoms-as-processes can \enquote{see} and identify their neighbors without any communication; otherwise how would $\bit{1}$ and $\inc$ know when to become $\inc$ and $\bit{0}$, for example?
% possess some amount of omniscience
But, in contradiction, true processes, in the sense of standard process calculi, are quite nearsighted:
% do not have the advantage of being farsighted:
\fxnote{\st{true}\ }processes can see only messages that arrive at their doorstep, not neighboring processes.

Nevertheless, a clauses-as-implementations view is still valid if some atoms serve as messages and other atoms serve as processes.
For example, if both of the $\inc$ atoms in the above clause are treated as message atoms, then that clause is an valid implementation.

% A serious conceptual objection can be raised against this atoms-as-processes and clauses-as-implentations view, however:
% If clauses are (global descriptions of) process implementations, then atoms cannot be locally executing.

So, the distinction being drawn here is one between a \vocab{specification} and its \vocab{choreography}---the what and the how.
% \fxnote{\st{The original program is only a specification of the valid process interactions, whereas a choreography is a pattern of communication that implements that specification.}}
A specification is the original program, which serves as a global description of the valid interactions among atoms; a choreography%
\footnote{We borrow the term \enquote*{choreography} from the literature on session-based concurrency.
The analogy is intended only as a loose one, however, and should not be taken to imply a precise, technical correspondence.}
is a local message-passing implementation of that specification that is obtained by assigning each atom a role---either message or process.
is an assignment of message or process roles to the atoms that transforms the specification into a local message-passing implementation.
is a local message-passing implementation of that specification.%
% A specification is the original program used to describe the valid process interactions, whereas a choreography is a pattern of communication that implements that specification%
% \footnote{Notice that we always speak of a choreography relative to a specification, just as an implementation is always relative to an abstraction.}%
% , and which must be given by the programmer, at least implicitly.%
%, which must also be given by the programmer, at least implicitly.%
\footnote{Notice that a choreography is always relative to a given specification.}

Ideally, an operational semantics would automatically generate a choreography from the specification\fxnote{\ \st{supplied by the programmer}}, but designing such a semantics appears to be difficult.
Different specifications will often require different patterns of interprocess communication;
sometimes a specification will even admit several choreographies, and\fxnote{\st{, more often than not,}} usually the programmer will want to exercise control in\fxnote{\ \st{those cases}} choosing one.
Therefore, unable to rely on a one-size-fits-all\fxnote{\ \st{local}} operational semantics, the programmer himself must supply the choreography\fxnote{\st{, at least implicitly}}.

In the previous \lcnamecref{sec:ordered-lp}, we saw the aforementioned binary counter supporting increment and decrement operations (\cref{sec:olp-intuition:binary-counter,sec:olp-intuition:decrements}).
% several examples of specifications (characterized as forward-chaining ordered logic programs), including the aforementioned binary counter supporting increment and decrement operations (\cref{sec:olp-intuition:binary-counter,sec:olp-intuition:decrements}).
% In the previous \lcnamecref{sec:ordered-lp}, we saw several examples of specifications (characterized as forward-chaining ordered logic programs), including the aforementioned binary counter supporting increment and decrement operations (\cref{sec:olp-intuition:binary-counter,sec:olp-intuition:decrements}).
We'll now describe what counts as a choreography, first with informal examples and then with formal definitions.






% Under a process-as-formula view \autocites{Miller:ELP92}{Cervesato+Scedrov:IC09}, the ordered logic program's atomic propositions---or, from a string rewriting perspective, letters---would be processes.
% In the binary counter program,
% % \wc{specification}[\st{program}], 
% for example, $\eps$, $\bit{0}$, and $\bit{1}$ would be atoms-as-processes.
% % If one tries to take\fxnote{\ \st{Taking}} a formula-as-process view \autocites{Miller:ELP92}{Cervesato+Scedrov:IC09}, the processes would be\fxnote{\ \st{are}} the ordered logic program's atomic propositions (or, from a string rewriting perspective, the program's letters).
% Continuing with this atoms-as-processes view, the ordered logic program's clauses should serve as (a global description of) an implementation of\fxnote{\ \st{interactions among}} these processes.
% For example, the clause
% \begin{equation*}
%   \bit{1} \fuse \inc \lrimp \monad{\inc \fuse \bit{0}}
% \end{equation*}
% should implement $\bit{1}$ and $\inc$ processes that, when neighbors, should react to become neighboring $\inc$ and $\bit{0}$ processes.



% The program's clauses don't tell the full story, however:
% the clauses globally specify \emph{what} are valid interactions but not \emph{how} to realize them locally.
% In ordered logic programming, the how is traditionally supplied by an operational semantics in which an omniscient central conductor, having the benefit of a global view of all atoms, directs the atoms' interactions according to the program's clauses.
% % The how is instead supplied by the language's operational semantics.
% % % The  language's operational semantics instead supplies the how.
% % % The program is thus only a \vocab{specification}, with
% % In the usual operational semantics for ordered logic programming, there is a central \enquote{conductor} who, having the benefit of a global view of all atoms, directs the atoms' interactions according to the program's clauses.
% But because they rely so heavily on the central conductor, processes using this semantics are no more than superficially \wc{local}[\st{distributed}].
% % Under this semantics, however, processes are only nominally distributed because they rely so heavily on the central conductor.


% To be truly \wc{local}[\st{distributed}], the processes should instead communicate directly with their neighbors to identify which, if any, of the valid interactions are possible for them at that moment.
% % For instance, by communicating directly with its left-hand neighbor, an $\inc$ process might learn that that neighbor is a $\bit{1}$ process and that the above clause therefore applies; further direct communication between the two processes would effect
% For instance, by communicating directly with its left-hand neighbor, an $\inc$ process might learn that that neighbor is a $\bit{1}$ process; with further direct communication, the $\bit{1}$ and $\inc$ processes could coordinate to effect the above\fxnote{\ globally specified} interaction.
% % How does $\bit{1}$, for example, learn that its right-hand neighbor is $\inc$ and that the above clause therefore applies?



% \section{Choreographies}\label{sec:choreographies}

% Traditionally, concurrency is phrased as the composition of interacting, locally executing processes.
% As the binary counter example from \cref{sec:olp-intuition:binary-counter,sec:olp-intuition:concurrency} demonstrates, a notion of concurrency based on indistinguishable interleavings of independent rewritings arises naturally in forward-chaining ordered logic programming.
% But where are the locally executing processes?

% Under a formula-as-process view \autocites{Miller:ELP92}{Cervesato+Scedrov:IC09}, the ordered logic program's atomic propositions---or, from a string rewriting perspective, letters---would be processes.
% % If one tries to take\fxnote{\ \st{Taking}} a formula-as-process view \autocites{Miller:ELP92}{Cervesato+Scedrov:IC09}, the processes would be\fxnote{\ \st{are}} the ordered logic program's atomic propositions (or, from a string rewriting perspective, the program's letters).
% The program's clauses, accordingly, would serve to specify the valid interactions among processes. 
% In the binary counter (\cref{sec:olp-intuition:binary-counter,sec:olp-intuition:decrements}),
% % \wc{specification}[\st{program}], 
% for example, $\eps$, $\bit{0}$, and $\bit{1}$ are atoms-as-processes, and the clause
% \begin{equation*}
%   \bit{1} \fuse \inc \lrimp \monad{\inc \fuse \bit{0}}
% \end{equation*}
% says that one valid interaction is for neighboring $\bit{1}$ and $\inc$ processes to coordinate to become neighboring $\inc$ and $\bit{0}$ processes.
% % (with similar readings for the other clauses).

% The program's clauses don't tell the full story, however:
% the clauses globally specify \emph{what} are valid interactions but not \emph{how} to realize them locally.
% In ordered logic programming, the how is traditionally supplied by an operational semantics in which an omniscient central conductor, having the benefit of a global view of all atoms, directs the atoms' interactions according to the program's clauses.
% % The how is instead supplied by the language's operational semantics.
% % % The  language's operational semantics instead supplies the how.
% % % The program is thus only a \vocab{specification}, with
% % In the usual operational semantics for ordered logic programming, there is a central \enquote{conductor} who, having the benefit of a global view of all atoms, directs the atoms' interactions according to the program's clauses.
% But because they rely so heavily on the central conductor, processes using this semantics are no more than superficially \wc{local}[\st{distributed}].
% % Under this semantics, however, processes are only nominally distributed because they rely so heavily on the central conductor.


% To be truly \wc{local}[\st{distributed}], the processes should instead communicate directly with their neighbors to identify which, if any, of the valid interactions are possible for them at that moment.
% % For instance, by communicating directly with its left-hand neighbor, an $\inc$ process might learn that that neighbor is a $\bit{1}$ process and that the above clause therefore applies; further direct communication between the two processes would effect
% For instance, by communicating directly with its left-hand neighbor, an $\inc$ process might learn that that neighbor is a $\bit{1}$ process; with further direct communication, the $\bit{1}$ and $\inc$ processes could coordinate to effect the above\fxnote{\ globally specified} interaction.
% % How does $\bit{1}$, for example, learn that its right-hand neighbor is $\inc$ and that the above clause therefore applies?


% So, the distinction being drawn here is one between a \vocab{specification} and its \vocab{choreography}---the what and the how.
% % \fxnote{\st{The original program is only a specification of the valid process interactions, whereas a choreography is a pattern of communication that implements that specification.}}
% A specification is the original program, which serves as a \emph{global} description of the valid process interactions; a choreography%
% \footnote{We borrow the term \enquote*{choreography} from the literature on session-based concurrency.
% The analogy is intended only as a loose one, however, and should not be taken to imply a precise, technical correspondence.}
% is a \emph{local}\fxnote{\ message-passing} implementation of that specification.%
% % A specification is the original program used to describe the valid process interactions, whereas a choreography is a pattern of communication that implements that specification%
% % \footnote{Notice that we always speak of a choreography relative to a specification, just as an implementation is always relative to an abstraction.}%
% % , and which must be given by the programmer, at least implicitly.%
% %, which must also be given by the programmer, at least implicitly.%
% \footnote{Notice that a choreography is always relative to a given specification.}

% % Designing a one-size-fits-all distributed operational semantics appears to be difficult, however.
% % We could try to design a one-size-fits-all local operational semantics, but this appears to be difficult.
% Ideally, an operational semantics would automatically generate a choreography from the specification supplied by the programmer, but designing such a semantics\fxnote{\ \st{unfortunately}} appears to be difficult.
% % % Designing a distributed operational semantics that is uniformly suitable appears to be difficult, however.
% % % Designing a uniformly suitable distributed operational semantics appears to be difficult, however.
% Different specifications will often require different patterns of interprocess communication;
% sometimes a specification will even admit several choreographies, and, more often than not, the programmer will want to exercise control in those cases.
% % % Therefore, rather than relying on a one-size-fits-all operational semantics, the programmer must indicate the intended pattern of communication for each program.
% % Not having a one-size-fits-all local operational semantics, the programmer himself must indicate the intended pattern of communication for each program.
% Therefore, not having a one-size-fits-all local operational semantics, the programmer himself must supply the choregraphy, at least implicitly.

% In the previous \lcnamecref{sec:ordered-lp}, we saw the aforementioned binary counter supporting increment and  several examples of specifications (characterized as forward-chaining ordered logic programs), including the aforementioned binary counter supporting increment and decrement operations (\cref{sec:olp-intuition:binary-counter,sec:olp-intuition:decrements}).
% % In the previous \lcnamecref{sec:ordered-lp}, we saw several examples of specifications (characterized as forward-chaining ordered logic programs), including the aforementioned binary counter supporting increment and decrement operations (\cref{sec:olp-intuition:binary-counter,sec:olp-intuition:decrements}).
% We'll now describe what counts as a choreography, first with informal examples and then with formal definitions.



% % % Designing a one-size-fits-all distributed operational semantics appears to be difficult, however.
% % % We could try to design a one-size-fits-all local operational semantics, but this appears to be difficult.
% % Ideally, the operational semantics would localize programs in this way, but, unfortunately, designing such a semantics\fxnote{\ \st{that is also uniformly suitable}} appears to be difficult.
% % % % Designing a distributed operational semantics that is uniformly suitable appears to be difficult, however.
% % % % Designing a uniformly suitable distributed operational semantics appears to be difficult, however.
% % Different programs will often require different patterns of interprocess communication;
% % sometimes a program will even admit several communication patterns, and, more often than not, the programmer will want to exercise control in those cases.
% % % Therefore, rather than relying on a one-size-fits-all operational semantics, the programmer must indicate the intended pattern of communication for each program.
% % Not having a one-size-fits-all local operational semantics, the programmer himself must indicate the intended pattern of communication for each program.

% % % The distinction being drawn here is one between a \vocab{specification} and its \vocab{choreography}---the what and the how.
% % % The original program serves only as a specification of what are valid process interactions, whereas the choregraphy is the programmer's intended 

% % So, the distinction being drawn here is one between a \vocab{specification} and its \vocab{choreography}---the what and the how.
% % % \fxnote{\st{The original program is only a specification of the valid process interactions, whereas a choreography is a pattern of communication that implements that specification.}}
% % A specification is the original program, which serves as a global description of the valid process interactions; a choreography is a local\fxnote{, message-passing} implementation of that specification.%
% % % A specification is the original program used to describe the valid process interactions, whereas a choreography is a pattern of communication that implements that specification%
% % % \footnote{Notice that we always speak of a choreography relative to a specification, just as an implementation is always relative to an abstraction.}%
% % % , and which must be given by the programmer, at least implicitly.%
% % \footnote{We borrow the term \enquote*{choreography} from the literature on session-based concurrency.
% % The analogy is intended only as a loose one, however, and should not be taken to imply a precise, technical correspondence.}%
% % %, which must also be given by the programmer, at least implicitly.%
% % \footnote{Notice that a choreography is always relative to a given specification.}

% % In the previous \lcnamecref{sec:??}, we saw several examples of specifications (characterized as forward-chaining ordered logic programs), including a binary counter supporting increment and decrement operations.
% % We'll now describe what counts as a choreography, first with informal examples and then with formal definitions.

% % To build intuition, we'll now describe choregraphies by example


% % , adapting terminology from the concurrency literature,


% % So, unfortunately, 


% % Unfortunately, because different programs will require different patterns of communication among processes, we won't be able to leave the how up to the operational semantics.
% % The programmer will want control over the communication patterns.



% % Borrowing terminology from the literature on sessions

% % In session terminology, the logic program with a centralized operational semantics is known as an orchestration of processes, whereas the desired distributed semantics is known as a choreography.




% % \mbox{}\\

% % The program's clauses don't tell the full story, however:
% % The clauses specify what are valid interactions but not \emph{how} to realize those interactions; the \enquote*{how} is instead supplied by the logic programming language's operational semantics.
% % In the usual operational semantics, there is a central \enquote{conductor} who, having the benefit of a global view of all atoms, directs the atoms' interactions according to the program's clauses.

% % However, because they rely so heavily on the central conductor, processes using this semantics are no more than superficially distributed.
% % % Under this semantics, however, processes are only nominally distributed because they rely so heavily on the central conductor.
% % To be truly distributed, the processes should instead communicate directly with their neighbors to identify which, if any, of the valid interactions are possible for them at that moment.

% % It's difficult to argue that this centralized \enquote{how} is suitable for \emph{distributed} processes, however.
% % The distributed processes should instead communicate directly with their neighbors to identify which, if any, of the valid interactions are possible for them at that moment.
% % How does $\bit{1}$, for example, learn that its right-hand neighbor is $\inc$ and that the above clause therefore applies?

% % In session terminology, the logic program with a centralized operational semantics is known as an orchestration of processes, whereas the desired distributed semantics is known as a choreography.




% % The program's clauses do not tell the full story, however: the clauses specify what are valid interactions but not \emph{how} to realize those interactions.
% % In the usual operational semantics, the \enquote{how} is supplied by providing a central \enquote{conductor} that, having the benefit of a global view\fxnote{\ \st{of all atoms}}, manipulates the atoms according to the program's clauses.
% % But it's difficult to argue that this centralized \enquote{how} is suitable for \emph{distributed} processes.
% % Distributed processes should instead communicate directly with their neighbors to identify which, if any, of the valid interactions are possible for them at that moment.
% % How does $\bit{1}$, for example, learn that its right-hand neighbor is $\inc$ and that the above clause therefore applies?






% % This isn't the full story, however:
% % % The program's clauses specify \emph{what} are valid interactions but not \emph{how} to achieve those interactions.
% % the program's clauses specify what are valid interactions but not \emph{how} to achieve them.
% % The \enquote{how} is provided by the logic programming language's operational semantics.
% % The usual operational semantics


% % How do $\bit{1}$ and $\inc$, for example, learn that they are neighbors and that the above clause therefore applies?


% % the \enquote{how}it is provided by the logic programming language's operational semantics.
% % The usual operational semantics for logic programming 

% % This isn't the full story, however.
% % The usual operational semantics for ordered logic programming assumes a central \enquote{puppeteer} that has a global view of all atoms and manipulates them according to the program's clauses.
% % It's difficult to argue that this centralization is appropriate for distributed processes, however.
% % Instead, the processes should communicate directly to identify their neighbors and thereby deduce which, if any, of the valid interactions are possible for them at that moment.
% % But this communication is left unspecified in the original logic program.
% % How does $\bit{1}$, for example, learn that its right-hand neighbor is $\inc$ and that the above clause therefore applies?

% % % What's left unspecified in the ordered logic program is how the distributed processes communicate to identify their neighbors and thereby deduce which, if any, of the valid interactions are possible for them at that moment.
% % % % Using a communication protocol that is left unspecified in the program, the atoms deduce 
% % % How does $\bit{1}$, for example, learn that its right-hand neighbor is $\inc$ and that the above clause therefore applies?

% % Orchestration vs. choreography


% % arara: pdflatex
% % arara: pdflatex
% % arara: biber
% % arara: pdflatex
% % arara: pdflatex
% % \documentclass{../hdeyoung-proposal}
% \documentclass[
%   class=../hdeyoung-proposal,
%   crop=false
% ]{standalone}

% \usepackage{ordered-logic}
% \usepackage{basic-atoms}
% \usepackage{binary-counter}

% \crefname{choreography}{chor.}{chors.}
% \Crefname{choreography}{Chor.}{Chors.}

% \DeclareAcronym{BHK}{
  short = BHK,
  long  = Brouwer-Heyting-Kolmogorov
}

\DeclareAcronym{JILL}{
  short = JILL,
  long  = judgmental intuitionistic linear logic
}

\DeclareAcronym{ILL}{
  short = ILL,
  long  = intuitionistic linear logic
}

\DeclareAcronym{SML}{
  short = SML,
  long  = Standard ML
}

\DeclareAcronym{SOS}{
  short = SOS,
  short-indefinite = an,
  long = structural operational semantics
}

\DeclareAcronym{SSOS}{
  short = SSOS,
  short-indefinite = an,
  long = substructural operational semantics
}

\DeclareAcronym{SILL}{
  short = SILL,
  long = session-typed intuitionistic linear logic
}

\DeclareAcronym{SISLL}{
  short = singleton \acs{SILL},
  long = session-typed intuitionistic singleton linear logic
}

\DeclareAcronym{CLF}{
  short = CLF,
  long = the Concurrent Logical Framework
}


% \begin{document}

\subsection{Choreographies by example}\label{sec:chor-by-example}

\subsubsection{The binary counter}\label{sec:chor-example-counter}

In giving the intuition behind the binary counter specification (\cref{sec:olp-intuition:binary-counter}), we described the $\inc$ atoms % as moving --- moving past any $\bit{1}$s and eventually stopping at the $\eps$ or right-most $\bit{0}$.
as moving up the counter.
% % , a subliminal hint that $\inc$s are like messages.
% % This suggests a choreography in which $\inc$ processes take the active lead:
% This hints that $\inc$s are a bit like messages, and suggests a choreography in which $\inc$ processes initiate the interaction:
% First, each $\inc$ process sends a message, $\inc[<-]$, to its left-hand neighbor, thereby notifying that neighbor of its existence, and then the $\inc$ process terminates.
% If the neighbor is $\eps$, $\bit{0}$, or $\bit{1}$, then, upon receiving the $\inc$'s message, that neighbor takes full responsibility for completing the corresponding interaction.
This hints at a choreography in which $\inc$ atoms act as messages that trigger the increment action:
Whenever an $\inc$ message arrives at an $\eps$, $\bit{0}$, or $\bit{1}$ process, that process takes responsiblity for completing the increment action.
% First, each $\inc$ process sends a message, $\inc[<-]$, to its left-hand neighbor, thereby notifying that neighbor of its existence, and then the $\inc$ process terminates.
% If the neighbor is $\eps$, $\bit{0}$, or $\bit{1}$, then, upon receiving the $\inc$'s message, that neighbor takes full responsibility for completing the corresponding interaction.


% In giving the intuition behind the binary counter specification (\cref{sec:olp-intuition:binary-counter}), we described the $\inc$ atoms as moving up the counter.
% % This hints that $\inc$s are a bit like messages, and suggests a choreography in which $\inc$ processes initiate the interaction:
% % First, each $\inc$ process sends a message, $\inc[<-]$, to its left-hand neighbor, thereby notifying that neighbor of its existence, and then the $\inc$ process terminates.
% % If the neighbor is $\eps$, $\bit{0}$, or $\bit{1}$, then, upon receiving the $\inc$'s message, that neighbor takes full responsibility for completing the corresponding interaction.
% This hints at a choreography in which $\inc$ atoms are messages that trigger the increment action by $\eps$, $\bit{0}$, and $\bit{1}$ atoms that act as processes.
% When the $\eps$, $\bit{0}$, or $\bit{1}$, processes receive the $\inc[<-]$ message,  then, upon receiving the $\inc$'s message, that neighbor takes full responsibility for completing the corresponding interaction.

Expressed as an annotation of the original ordered logical specification, this choreography is:
\begin{equation}\label[choreography]{chor:oop-counter}
  \!\begin{aligned}
    &\eps \fuse \inc[<-] \lrimp \monad{\eps \fuse \bit{1}} \\
    &\bit{0} \fuse \inc[<-] \lrimp \monad{\bit{1}} \\
    &\bit{1} \fuse \inc[<-] \lrimp \monad{\inc[<-] \fuse \bit{0}}
    \text{\,,}
  \end{aligned}
\end{equation}
where the $\eps$, $\bit{0}$, and $\bit{1}$ atoms are viewed as processes, but the $\inc[<-]$ atoms are viewed as messages.
%
Two properties are crucial:
\begin{description}[font=\normalfont\itshape, leftmargin=\parindent, labelindent=\leftmargin, listparindent=\parindent, parsep=0pt]
\item[Locality.]
  Each clause's premise depends on exactly one process-like atom and (at most) one message-like atom.
  Consequently, each process's decisions are entirely local: the $\eps$, $\bit{0}$, and $\bit{1}$ processes act (independently) only after receiving an $\inc[<-]$ message.%
  \footnote{In {SSOS} terminology, processes that wait to receive a message, like $\eps$, $\bit{0}$, and $\bit{1}$ here, would be termed \vocab{latent} propositions; and messages, like $\inc[<-]$ here, would be termed \vocab{passive} propositions.}

  Locality serves to ensure that the choreography describes sensible message-passing behaviors.
  A clause such as $\inc[<-] \lrimp \monad{{\dots}}$, whose premise does not contain a process-like atom, is not message-passing because no process receives the $\inc[<-]$ message.
%
\item[Specification-preserving.]
% % % Second, notice that
% % The choreography exposes the same $\eps$, $\bit{}$, and $\inc$ processes as the original binary counter specification; the last three clauses of the choreography differ from the specification's clauses only in the substitution of $\inc[<-]$ for $\inc$ in their premises.
% The choreography exposes the same $\eps$, $\bit{}$, and $\inc$ processes as the original binary counter specification.
% Its clauses differ from those of the specification only in the substitution of $\inc[<-]$ for $\inc$ in their premises.
% (The choreography also includes an $\inc \lrimp \monad{\inc[<-]}$ clause to justify that substitution.)
% In this sense, there is a very strong equivalence between the two programs.
% The choreography does not fundamentally alter the specification---it only refines that specification by making the communication patterns explicit.
%
The choreography exposes the same behaviors for $\eps$, $\bit{}$, and $\inc$ as in the original specification.
Its clauses are exactly those of the specification, except that each $\inc$ atom in the specification has been annotated as an $\inc[<-]$ message-like atom in the choreography.

In this sense, there is a very strong, lock-step equivalence between the choreography and its specification.
The choreography does not fundamentally alter the specification---it only refines that specification by making the communication patterns explicit.
\end{description}
%
% In this sense, there is a strong equivalence between the 
% The choreography does not fundamentally alter the implementation given in the original program---it only refines that implementation by making the communication patterns explicit.
% In this sense, there is a strong equivalence between, which will be made precise in \cref{??}
%
% Notice that this choreography \wc{refactors} the original program so that each new clause depends on exactly one process atom and at most one message atom.
% In this way, each process's decisions are completely local: the $\inc$ process always sends $\inc[<-]$ regardless of its neighbors, and the $\eps$ and $\bit{}$ processes act only after receiving an $\inc[<-]$ message.%
% \footnote{In \ac{SSOS} terminology, processes that act regardless of their neighbors, like $\inc$, would be termed \vocab{active} propositions; processes that wait to receive a message, like $\eps$, $\bit{0}$, and $\bit{1}$, would be termed \vocab{latent} propositions; and messages, like $\inc[<-]$, would be termed \vocab{passive} propositions.}
%
It's convenient to think of the programmer as supplying this choreography in full, but in practice the programmer might only give the assignment of roles to atoms, \eg\ $\inc[<-]$ for $\inc$.

\subsubsection{Messages can flow in both directions}\label{sec:chor-binary-count}

In our binary counter specification with decrements (\cref{sec:olp-intuition:decrements}), $\dec$ atoms propagate up the counter similarly to $\inc$s, with the difference that each $\dec$ atom eventually gives rise to either a $\fail$ or $\suc$ atom that travels back down the counter.
Once again, this hints at a choreography in which $\dec$, $\fail$, and $\suc$ atoms are message-like:
\begin{itemize}
\item Whenever a $\dec[<-]$ message arrives at an $\eps$, $\bit{0}$, or $\bit{1}$ process's right-hand side, that process completes the local decrement action:
      the $\eps$ and $\bit{1}$ processes send a $\fail[->]$ or $\suc[->]$ message, respectively, to their right;
      the $\bit{0}$ process forwards the $\dec[<-]$ message to its left and continues as a $\bit[']{0}$ process.
\item Whenever a $\fail[->]$ or $\suc[->]$ message arrives at a $\bit[']{0}$ process's left-hand side, that process forwards the message to its right-hand neighbor and continues as a $\bit{0}$ or $\bit{1}$ process, respectively.
% \item Each $\dec$ process sends a message, $\dec[<-]$, to its left-hand neighbor and terminates.
%       If the neighbor is $\eps$, $\bit{0}$, or $\bit{1}$, then, upon receiving the message, that neighbor completes the corresponding interaction given in the specification.
% \item Each $\fail$ or $\suc$ process sends a message, $\fail[->]$ or $\suc[->]$, respectively, to its \emph{right-hand} neighbor and terminates.
%       If the neighbor is $\bit[']{0}$, then, upon receiving the message from $\fail$ or $\suc$, that neighbor completes the corresponding interaction.
\end{itemize}
To account for decrements, the binary counter's choreography is therefore extended with the following clauses:
\begin{equation}
  \!\begin{aligned}
    &\eps \fuse \dec[<-] \lrimp \monad{\eps \fuse \fail[->]} \\
    &\bit{0} \fuse \dec[<-] \lrimp \monad{\dec[<-] \fuse \bit[']{0}} \\
    &\bit{1} \fuse \dec[<-] \lrimp \monad{\bit{0} \fuse \suc[->]} \\[1.5\jot]
    % 
    &\fail[->] \fuse \bit[']{0} \lrimp \monad{\bit{0} \fuse \fail[->]} \\
    &\suc[->] \fuse \bit[']{0} \lrimp \monad{\bit{1} \fuse \suc[->]}
    \,.
  \end{aligned}
\end{equation}
Once again, these clauses are just an annotation of the original specification's clauses, with $\dec$, $\suc$, and $\fail$ annotated as $\dec[<-]$, $\suc[->]$, and $\fail[->]$.
% Once again, the atoms that are decorated with arrows are formally distinct from their undecorated counterparts.
% (As before, the atoms that are decorated with arrows are formally distinct from their undecorated counterparts.)
The extended choreography thus continues to be specification-preserving.

This extended choreography illustrates that message atoms may be either left-directed, like $\inc[<-]$ and $\dec[<-]$, or right-directed, like $\fail[->]$ and $\suc[->]$.
% Moreover, a message's direction determines the structure of premises in which it is received:
% a left-directed (right-directed) message must arrive at the receiving process's right (resp., left) side, otherwise the message would not be traveling from left to right (resp., right to left).
% 
% Because it is traveling left-to-right, a left-directed message must always arrive at the right-hand side of its recipient; dually, a right-directed message must always arrive at the left-hand side of its recipient.
Because a left-directed message travels from right to left, it must always arrive at the right-hand side of its recipient; dually, a right-directed message must always arrive at the left-hand side of its recipient.
This directionality is another aspect of locality, and it further constrains the structure of a choreography's premises.
For example, this choreography's premises are well-formed because each message flows toward its recipient, whereas premises of the forms $\matom[<-] \fuse \patom$ or $\patom \fuse \matom[->]$ are not well-formed because process $\patom$ doesn't receive the $\matom[<-]$ or $\matom[->]$ message.
% For instance, $\bit{1} \fuse \inc[<-]$ and $\fail[->] \fuse \bit[']{0}$ are well-formed premises because each message flows toward its recipient, whereas premises of the forms $\matom[<-] \fuse \p$ or $\p \fuse \matom[->]$ are not well-formed because process $\p$ doesn't receive the $\matom[<-]$ or $\matom[->]$ message.

% As well as retaining locality, notice that this extended choreography continues to be specification-preserving:
% the choreography's clauses differ from those of the specification only in using the decorated forms
% % the substitution of
% $\dec[<-]$, $\fail[->]$, and $\suc[->]$
% in place of
% % for
% their undecorated counterparts.


\subsubsection{Choreographies are not always unique}\label{sec:mult-chor-are}

As alluded to previously, multiple choreographies are possible for some specifications.

This is true of our binary counter specification, for instance.
(To simplify the example, we'll ignore decrements for now.)
In the $\inc[<-]$-choreography (\cref{sec:chor-example-counter}),
% the $\inc$ processes initiate the interaction but leave all remaining work to the $\eps$, $\bit{0}$, and $\bit{1}$ processes alone.
the counter's value is represented by a chain of $\eps$, $\bit{0}$, and $\bit{1}$ processes that are acted upon by $\inc[<-]$ messages.
%
% Alternatively, the $\inc$ processes could wait for $\eps$, $\bit{0}$, or $\bit{1}$ to initiate the interaction, but thereafter take full responsibility for its completion.
Alternatively, the counter's value could be represented by a sequence of $\eps[->]$, $\bit{0}[->]$, and $\bit{1}[->]$ messages; when fed such a message sequence, an $\inc$ process would emit another sequence that represents the result:
%
% Specifically, each $\eps$, $\bit{0}$, and $\bit{1}$ process sends an identifying message, $\eps[->]$, $\bit{0}[->]$, or $\bit{1}[->]$, to its right-hand neighbor and then terminates.
% If the neighbor is $\inc$, then, upon receiving the message, that $\inc$ completes the corresponding interaction.
% % takes responsibility for carrying out the corresponding clause of the specification.
\begin{equation}
  \!\begin{aligned}
    &\eps[->] \fuse \inc \lrimp \monad{\eps[->] \fuse \bit{1}[->]} \\
    &\bit{0}[->] \fuse \inc \lrimp \monad{\bit{1}[->]} \\
    &\bit{1}[->] \fuse \inc \lrimp \monad{\inc \fuse \bit{0}[->]}
      \,.
  \end{aligned}
\end{equation}
Once again, this choreography possesses the locality and specification-preserving properties.

% Owing to the difference in roles held by, these two choreographies have distinct flavors.
% These two choreographies
These two choreographies
% presented thus far
have distinct flavors, owing to the different process and message roles that they assign to the $\inc$ and $\eps$, $\bit{0}$, and $\bit{1}$ atoms.
The $\inc[<-]$-choreography has an object-oriented character: by sending an $\inc[<-]$ message, the increment method dispatches on the receiving object's class---either $\eps$, $\bit{0}$, or $\bit{1}$.
In contrast, this new $\bit{}[->]$-choreography has a functional character: $\inc$ is a function that receives its argument as a sequence of messages---either $\eps[->]$, $\bit{0}[->]$, or $\bit{1}[->]$.

% The increment method dispatches on 
% $\inc$ invokes the increment method on the neighboring object by sending an $\inc[<-]$ message
% There, the $\inc$ method sends an $\inc[<-]$ message like a method that dispatches on the class of the recipient object---either $\eps$, $\bit{0}$, or $\bit{1}$.
% Our first choreography has an object-oriented flavor, with $\inc$ like a method that dispatches on the class of the recipient object---either $\eps$, $\bit{0}$, or $\bit{1}$.
% In contrast, this second choreography has a more functional flavor, with 

% This alternate choreography has a funcitonal flavor: $\inc$ can be viewed as a function on the $\eps$-and-$\bit{}$ representation of data.
% In contrast, the previous choreography has a more object-oriented flavor

% The difference in sender and recipient between this alternate choreography and the previous one gives the two choreographies different flavors.
% In this alternate choreography
% In constrast, the previous choreography has a more object-oriented flavor, with $\inc$ being a method that dispatches on the class of the recipient---either $\eps$, $\bit{0}$, or $\bit{1}$.

% If we view the $\eps$ and $\bit{}$ processes as data, then this alternate choreography has a functional flavor.
% In contrast, the previous choreography has an object-oriented flavor, with a dynamic dispatch of $\inc[<-]$ on the recipient.

\subsubsection{Two non-choreographies}\label{sec:non-choreographies}

Another, slightly more complex reformulation of the binary counter specification chooses to treat the $\inc$ atom as a simple process, not a message:
\begin{equation}
  \!\begin{aligned}
    &\inc \lrimp \monad{\incm[<-]} \\[1.5\jot]
    % 
    &\eps \fuse \incm[<-] \lrimp \monad{\eps \fuse \bit{1}} \\
    &\bit{0} \fuse \incm[<-] \lrimp \monad{\bit{1}} \\
    &\bit{1} \fuse \incm[<-] \lrimp \monad{\inc \fuse \bit{0}}
      \,.
  \end{aligned}
\end{equation}
In fact, the $\inc$ process does nothing but send an $\incm[<-]$ message.

This signature is equivalent to the binary counter specification in that it ultimately exposes the same $\eps$, $\bit{}$, and $\inc$ behaviors.
However, it is \emph{not} specification-preserving under the informal definition that we have used thus far.
% In contrast with the choreographies, this implementation's main clauses are not a simple decoration of the specification's clauses: each of the specification's clauses is spread across several clauses here.
In contrast with the choreographies, this formulation does more than simply refine the specification by making the communication explicit: it introduces a new message-like atom, $\incm[<-]$, a new clause, $\inc \lrimp \monad{\incm[<-]}$, and modifies the existing clauses.






% Another, more complex implementation of the binary counter specification divides the work of completing the interaction among the $\eps$ and $\bit{}$ and (the continuation of) the $\inc$ processes:
% \begin{equation}
%   \!\begin{aligned}
%     &\inc \lrimp \monad{\inc[^l, <-] \fuse \inc[^r]} \\[1.5\jot]
%     % 
%     &\eps \fuse \inc[^l, <-] \lrimp \monad{\eps \fuse \eps[^r, ->]} \\
%     &\bit{0} \fuse \inc[^l, <-] \lrimp \monad{\bit{0}[^r, ->]} \\
%     &\bit{1} \fuse \inc[^l, <-] \lrimp \monad{\inc \fuse \bit{1}[^r, ->]} \\[1.5\jot]
%     % 
%     &\eps[^r, ->] \fuse \inc[^r] \lrimp \monad{\bit{1}} \\
%     &\bit{0}[^r, ->] \fuse \inc[^r] \lrimp \monad{\bit{1}} \\
%     &\bit{1}[^r, ->] \fuse \inc[^r] \lrimp \monad{\bit{0}} \,.
%   \end{aligned}
% \end{equation}
% In this implementation, $\inc$ first sends an $\inc[^l, <-]$ message to its left-hand neighbor and then waits for a response as process $\inc[^r]$.
% Upon receiving an $\inc[^l, <-]$ message, the recipient process, either $\eps$, $\bit{0}$, or $\bit{1}$, partially completes the interaction and sends an identifying message to its right-hand neighbor, which is necessarily an $\inc[^r]$ process.
% The $\inc[^r]$ process finishes the interaction once it receives the identifying message.

% This implementation is equivalent to the binary counter specification in that it ultimately exposes the same $\eps$, $\bit{}$, and $\inc$ processes.
% However, it is not specification-preserving under the informal definition that we have used thus far.
% % In contrast with the choreographies, this implementation's main clauses are not in one-to-one correspondence with those of the specification.
% % In contrast with the choreographies, this implementation's main clauses are not a simple decoration of the specification's clauses: each of the specification's clauses is spread across several clauses here.
% In contrast with the choreographies, this implementation does more than simply refine the specification by making the communication explicit: using a temporary $\inc[^r]$ process, it spreads each of the specification's clauses across several clauses.

So, this signature is not specification-preserving, and therefore not a choreography, for the binary counter specification.
But that doesn't mean that the programmer cannot achieve the same behavior anyway: the programmer is free to rewrite the \emph{specification} to incorporate the behavior at the specification level.
% Although this implementation is not specification-preserving for the binary counter specification, and therefore not a choreography, the programmer can nevertheless achieve the same behavior by changing the \emph{specification}.
If the specification is changed to be
\begin{equation}
  \!\begin{aligned}
    &\inc \lrimp \monad{\incm} \\[1.5\jot]
    % 
    &\eps \fuse \incm \lrimp \monad{\eps \fuse \bit{1}} \\
    &\bit{0} \fuse \incm \lrimp \monad{\bit{1}} \\
    &\bit{1} \fuse \incm \lrimp \monad{\inc \fuse \bit{0}}
      \,,
  \end{aligned}
\end{equation}
then the above signature is indeed a choreography for \emph{this} specification.

% Although this implimentation is not specification-preserving, and therefore not a choreography, for the binary counter specification (\cref{??}), the programmer can nevertheless acheive the same behavior by changing the \emph{specification}.
% Instead of using the binary counter specification from \cref{??}, the following specification could be used because one of its choreographies, which uses $\inc[^l, <-]$, $\eps[^r, ->]$, $\bit{0}[^r, ->]$, and $\bit{1}[^r, ->]$ messages, is nearly identical the disallowed implimentation.
% \begin{equation*}
%   % \!\begin{aligned}[t]
%   %   &\inc[^l] \lrimp \monad{\inc[^l, <-]} \\
%   %   &\eps[^r] \lrimp \monad{\eps[^r, ->]} \\
%   %   &\bit{0}[^r] \lrimp \monad{\bit{0}[^r, ->]} \\
%   %   &\bit{1}[^r] \lrimp \monad{\bit{1}[^r, ->]}
%   % \end{aligned}
%   % \qquad
%   \!\begin{aligned}[t]
%     &\inc \lrimp \monad{\inc[^l] \fuse \inc[^r]} \\
%     &\eps \fuse \inc[^l] \lrimp \monad{\eps \fuse \eps[^r]} \\
%     &\bit{0} \fuse \inc[^l] \lrimp \monad{\bit{0}[^r]} \\
%     &\bit{1} \fuse \inc[^l] \lrimp \monad{\inc \fuse \bit{1}[^r]} \\
%     &\eps[^r] \fuse \inc[^r] \lrimp \monad{\bit{1}} \\
%     &\bit{0}[^r] \fuse \inc[^r] \lrimp \monad{\bit{1}} \\
%     &\bit{1}[^r] \fuse \inc[^r] \lrimp \monad{\bit{0}} \,.
%   \end{aligned}
% \end{equation*}


% Although this implementation ultimately exposes the same $\eps$, $\bit{}$, and $\inc$ processes

% Even so\fxnote{\ \st{Even though this choreography introduces and auxiliary process atom}}, we still consider it to be a valid choreography: $\inc[']$ is only temporary, leaving the underlying specification fundamentally unchanged.

Another signature that is equivalent to the binary counter specification, in the sense that the two track the same value, is
\begin{equation}
  \!\begin{aligned}
    &\num{N} \fuse \inc[<-] \lrimp \monad{\num{(N{+}1)}} \,.
  \end{aligned}
\end{equation}
Nevertheless, we wouldn't consider this to be a choreography of the binary counter specification because, by using a single number held by $\num{}$ instead of a string of $\bit{}$s, it fundamentally alters the specification.
We would, however, consider this signature to be a choreography of a different, simple counter specification, namely $\num{N} \fuse \inc \lrimp \monad{\num{(N{+}1)}}$.

% % Although it is equivalent to the binary counter in the sense that it tracks the same value, we wouldn't consider the following program to be a choreography of the binary counter specification because it fundamentally alters the implementation by using a single $\num{}$ instead of a string of $\bit{}$s.
% The following program is also equivalent to the binary counter specification, in the sense that the two track the same value.
% Nevertheless, we wouldn't consider it to be a choreography of the binary counter because it fundamentally alters the specification by using a single number held by $\num{}$ instead of a string of $\bit{}$s.
% \begin{align*}
%   &\inc \lrimp \monad{\inc[<-]} \\
%   &\num{N} \fuse \inc[<-] \lrimp \monad{\num{(N{+}1)}}
% \end{align*}
% We would, however, consider it to be a choreography of a different, simple counter specification: $\num{N} \fuse \inc \lrimp \monad{\num{(N{+}1)}}$.



% \end{document}

%%% Local Variables:
%%% TeX-master: "choreographies"
%%% End:


% arara: pdflatex
% arara: biber
% arara: pdflatex
% arara: pdflatex
\documentclass[
  class=../hdeyoung-proposal,
  crop=false
]{standalone}

\usepackage{ordered-logic}
\usepackage{basic-atoms}

\NewDocumentCommand{\chor}{}{X}
\NewDocumentCommand{\spec}{}{\Sigma}

\NewDocumentCommand{\erasemsg}{m}{#1^{e}}

\NewDocumentCommand{\trans}{t* t+ o}{%
  \longrightarrow
  \IfBooleanT{#1}{^*}\IfBooleanT{#2}{^+}%
  \IfValueT{#3}{_{#3}}%
}

% \cs_new_protected:Nn \trans: {
%   \peek_meaning:nTF {*}
%     { \@@_trans_star: }
%     { \peek_meaning:nTF {+}
%         { \@@_trans_plus: }
%    \longrightarrow
%   \IfBooleanT{#1}{^*}\IfBooleanT{#2}{^+}%
%   \IfValueT{#3}{_{#3}}%
% }

\begin{document}

\subsection{Choreographies, formally}\label{sec:chor-formal}

Hopefully the preceding examples have given some intuition for what counts as a choreography.
To make the definition precise, we need only formalize the locality and specification-preserving properties.

We use the forward-chaining ordered logic programming language described in \cref{??}, with a few restrictions.


\subsubsection{Locality}\label{sec:locality}

Each clause must have the form $U^+ \lrimp \monad{A^+}$, where $U^+$ is an \vocab{uncurried local premise} that adheres to the following grammar:
\begin{alignat*}{2}
  A^- &::= L^+ \limp \monad{A^+} \mid R^+ \rimp \monad{A^+} \mid A^-_1 \with A^-_2 \\
  A^+ &::= \p^+ \mid \p[->]^+ \mid \p[<-]^+ \mid A^+_1 \fuse A^+_2 \mid \one \mid A^- \\
  U^+ &::= L^+ \fuse U^+ \mid \p^+ \mid U^+ \fuse R^+ \\
  L^+ &::= \p[->]^+ \mid L^+_1 \fuse L^+_2 \mid \one \\
  R^+ &::= \p[<-]^+ \mid R^+_1 \fuse R^+_2 \mid \one
\end{alignat*}
The grammar is a bit complicated, but the idea behind it is simple and matches the intuition behind locality:
an uncurried local premise $U^+$ contains exactly one process atom $\p^+$ that receives right-directed messages, $\p[->]^+$, from its left and left-directed messages, $\p[<-]^+$, from its right.

At the expense of a more complicated grammar, we could allow curried clauses, such as $\p[_1, ->]^+ \lrimp \p[_2, ->]^+ \limp \p^+ \rimp \p[_3, <-]^+ \rimp \monad{A^+}$.
Uncurrying clauses would not seem to place a large burden on the programmer, for it is easy enough to write $\q[->]^+ \fuse \p[->]^+ \fuse \rr^+ \fuse \s[<-]^+ \lrimp \monad{A^+}$, and this detail is anyway orthogonal to what follows.

\subsubsection{Specification-preserving}\label{sec:spec-pres}

\begin{definition}[Specification-preserving]
  An ordered logic program $\chor$ is \vocab{specification-preserving} for specification $\spec$ if:
  \begin{enumerate}
  \item for each step $\octx \trans[\spec] \octx'$ in the specification $\spec$, there is a non-empty trace $\octx \trans+[\chor] \octx'$ in the program $\chor$; and
  \item for each step $\octx \trans[\chor] \octx'$ in the program $\chor$, either $\erasemsg{(\octx)} = \erasemsg{(\octx')}$ or there is a step $\erasemsg{(\octx)} \trans[\spec] \erasemsg{(\octx')}$ in the specification $\spec$.
  \end{enumerate}
\end{definition}

\end{document}

%%% Local Variables:
%%% TeX-master: "choreographies"
%%% End:


\end{document}