\section{Introduction}\label{sec:introduction}

Concurrency has become a provasive method for structuring computations, but, like mutable state, is notoriously tricky to use correctly.
...

Decades of research on mathematical logics and programming languages have established the power of deductive computation to ensure programs' clarity and correctness.
Deductive computation itself can be divided into two classes: proof-reduction as computation and proof-construction as computation.
Proof-reduction as computationis the foundation for functional programming languages, such as ML and Haskell, and originates from the BHK interpretation of intuitionistic logic.
Proof-construction as computaion is the foundation for logic programming languages, such as Prolog and Datalog.

Both proof-reduction and proof-construction techniques have been applied to the problem of clearly specifying and correctly implementing concurrent systems.
Using proof-construction, the \ac{CLF} \autocite{?} has been used to specify concurrent systems from ... to ... , and  those same concurrent systems can be simulated using the Lollimon and Celf logic programming interpreters.

Using proof-reduction, SILL has been used to implement 

%%% Local Variables:
%%% TeX-master: "proposal"
%%% End:
