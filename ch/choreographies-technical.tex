% arara: pdflatex
% arara: biber
% arara: pdflatex
% arara: pdflatex
\documentclass[
  class=../hdeyoung-proposal,
  crop=false
]{standalone}

\usepackage{ordered-logic}
\usepackage{basic-atoms}

\NewDocumentCommand{\chor}{}{X}
\NewDocumentCommand{\spec}{}{\Sigma}

\NewDocumentCommand{\erasemsg}{m}{#1^{e}}

\NewDocumentCommand{\trans}{t* t+ o}{%
  \longrightarrow
  \IfBooleanT{#1}{^*}\IfBooleanT{#2}{^+}%
  \IfValueT{#3}{_{#3}}%
}

% \cs_new_protected:Nn \trans: {
%   \peek_meaning:nTF {*}
%     { \@@_trans_star: }
%     { \peek_meaning:nTF {+}
%         { \@@_trans_plus: }
%    \longrightarrow
%   \IfBooleanT{#1}{^*}\IfBooleanT{#2}{^+}%
%   \IfValueT{#3}{_{#3}}%
% }

\begin{document}

\subsection{Choreographies, formally}\label{sec:chor-formal}

Hopefully the preceding examples have given some intuition for what counts as a choreography.
To make the definition precise, we need only formalize the locality and specification-preserving properties.

We use the forward-chaining ordered logic programming language described in \cref{??}, with a few restrictions.


\subsubsection{Locality}\label{sec:locality}

Each clause must have the form $U^+ \lrimp \monad{A^+}$, where $U^+$ is an \vocab{uncurried local premise} that adheres to the following grammar:
\begin{alignat*}{2}
  A^- &::= L^+ \limp \monad{A^+} \mid R^+ \rimp \monad{A^+} \mid A^-_1 \with A^-_2 \\
  A^+ &::= \p^+ \mid \p[->]^+ \mid \p[<-]^+ \mid A^+_1 \fuse A^+_2 \mid \one \mid A^- \\
  U^+ &::= L^+ \fuse U^+ \mid \p^+ \mid U^+ \fuse R^+ \\
  L^+ &::= \p[->]^+ \mid L^+_1 \fuse L^+_2 \mid \one \\
  R^+ &::= \p[<-]^+ \mid R^+_1 \fuse R^+_2 \mid \one
\end{alignat*}
The grammar is a bit complicated, but the idea behind it is simple and matches the intuition behind locality:
an uncurried local premise $U^+$ contains exactly one process atom $\p^+$ that receives right-directed messages, $\p[->]^+$, from its left and left-directed messages, $\p[<-]^+$, from its right.

At the expense of a more complicated grammar, we could allow curried clauses, such as $\p[_1, ->]^+ \lrimp \p[_2, ->]^+ \limp \p^+ \rimp \p[_3, <-]^+ \rimp \monad{A^+}$.
Uncurrying clauses would not seem to place a large burden on the programmer, for it is easy enough to write $\q[->]^+ \fuse \p[->]^+ \fuse \rr^+ \fuse \s[<-]^+ \lrimp \monad{A^+}$, and this detail is anyway orthogonal to what follows.

\subsubsection{Specification-preserving}\label{sec:spec-pres}

\begin{definition}[Specification-preserving]
  An ordered logic program $\chor$ is \vocab{specification-preserving} for specification $\spec$ if:
  \begin{enumerate}
  \item for each step $\octx \trans[\spec] \octx'$ in the specification $\spec$, there is a non-empty trace $\octx \trans+[\chor] \octx'$ in the program $\chor$; and
  \item for each step $\octx \trans[\chor] \octx'$ in the program $\chor$, either $\erasemsg{(\octx)} = \erasemsg{(\octx')}$ or there is a step $\erasemsg{(\octx)} \trans[\spec] \erasemsg{(\octx')}$ in the specification $\spec$.
  \end{enumerate}
\end{definition}

\end{document}

%%% Local Variables:
%%% TeX-master: "choreographies"
%%% End:
