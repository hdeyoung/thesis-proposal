% arara: pdflatex
% arara: biber
% arara: pdflatex
% arara: pdflatex
\documentclass[
  class=../hdeyoung-proposal,
  crop=false
]{standalone}

\usepackage{ordered-logic}
\usepackage{basic-atoms}
\usepackage{tikz-cd}

\NewDocumentCommand{\chor}{}{X}
\NewDocumentCommand{\spec}{}{\Sigma}

\NewDocumentCommand{\erasemsg}{m}{(#1)^{e}}

\NewDocumentCommand{\trans}{t* t+ o}{%
  \longrightarrow
  \IfBooleanT{#1}{^*}\IfBooleanT{#2}{^+}%
  \IfValueT{#3}{_{#3}}%
}

\begin{document}

\subsection{Choreographies, formally}\label{sec:chor-formal}

Hopefully the preceding examples have given some intuition for what counts as a choreography.
To make the definition precise, we need only formalize the locality and specification-preserving properties.

We use the forward-chaining ordered logic programming language described in \cref{sec:ordered-lp}, with a few restrictions.
First, we confine ourselves to the purely propositional fragment of ordered logic programming.
Second, to simplify the statement of locality, we require that each implication have exactly one premise: each implication has the form of $A^+ \rimp \monad{B^+}$, $A^+ \limp \monad{B^+}$, or $A^+ \lrimp \monad{B^+}$.
These uncurried implications would not seem to place a large burden on the programmer, for it is easy enough to write $\p[_{\mathrm{2}}]^+ \fuse \p[_{\mathrm{1}}]^+ \fuse \p[_{\mathrm{3}}]^+ \lrimp \monad{B^+}$ in place of $\p[_{\mathrm{1}}]^+ \lrimp \p[_{\mathrm{2}}]^+ \limp \p[_{\mathrm{3}}]^+ \rimp \monad{B^+}$, for example.

These restrictions can be lifted, but at the expense of complicating the presentation; because the restrictions are orthogonal to the main points of discussion, we therefore prefer the simplifying restrictions for the present.

\begin{alignat*}{2}
  &\text{Negative propositions}\quad & A^- &::= A^+ \rimp \monad{B^+} \mid A^+ \limp \monad{B^+} \mid A^- \with B^- \\
  &\text{Positive propositions}      & A^+ &::= A^+ \fuse B^+ \mid \one \mid \p^+ \mid \p[->]^+ \mid \p[<-]^+ \mid A^-
\end{alignat*}
 


\subsubsection{Locality}\label{sec:locality}

\begin{definition}[Locality]
  A clause $A^+ \lrimp \monad{B^+}$ is \emph{local} if its premise adheres to the grammar refinement $O^+$:
  \begin{alignat*}{2}
    O^+ &::= L^+ \fuse O^+ \mid \p^+ \mid O^+ \fuse R^+ \\
    L^+ &::= \p[->]^+ \mid L^+_1 \fuse L^+_2 \mid \one \\
    R^+ &::= \p[<-]^+ \mid R^+_1 \fuse R^+_2 \mid \one
  \end{alignat*}
  % Similarly, a right implication, $A^+ \rimp \monad{B^+}$, is local if its premise adheres to the grammar of $R^+$.
  % Dually, a left implication, $A^+ \limp \monad{B^+}$, is local if its premise adheres to the grammar of $L^+$.
\end{definition}


A clause $A^+ \lrimp \monad{B^+}$ is \emph{local} if its premise adheres to the grammar refinement $O^+$:
\begin{alignat*}{2}
  O^+ &::= L^+ \fuse O^+ \mid \p^+ \mid O^+ \fuse R^+ \\
  L^+ &::= \p[->]^+ \mid L^+_1 \fuse L^+_2 \mid \one \\
  R^+ &::= \p[<-]^+ \mid R^+_1 \fuse R^+_2 \mid \one
\end{alignat*}
The grammar may appear to be a bit complicated, but the idea behind it is simple enough and matches the intuition behind locality:
a local premise $O^+$ contains exactly one process atom $\p^+$ that receives right-directed messages, $\p[->]^+$, from its left and left-directed messages, $\p[<-]^+$, from its right.

Similarly, a right implication, $A^+ \rimp \monad{B^+}$, is local if its premise adheres to the grammar of $R^+$.
That is, right implications must receive only left-directed messages that are arriving at the implication's right-hand side.
No process atom may appear in the premise because, as an ephemeral resource, the implication itself acts as the recipient process.
Dually, a left implication, $A^+ \limp \monad{B^+}$, is local if its premise adhers to the grammar of $L^+$.


% At the expense of a more complicated grammar, locality could be extended to implications with multiple premises.
% , we could allow curried clauses, such as $\p[_1, ->]^+ \lrimp \p[_2, ->]^+ \limp \p^+ \rimp \p[_3, <-]^+ \rimp \monad{A^+}$.
% Uncurrying clauses would not seem to place a large burden on the programmer, for it is easy enough to write $\q[->]^+ \fuse \p[->]^+ \fuse \rr^+ \fuse \s[<-]^+ \lrimp \monad{A^+}$, and this detail is anyway orthogonal to what follows.

\subsubsection{Specification-preserving}\label{sec:spec-pres}

To judge that an ordered logic program is specification-preserving, we rely on a notion of erasure that relates message atoms to process atoms.
\begin{definition}[Message erasure]
  For atomic propositions, the \vocab{message erasure} $\erasemsg{-}$ is given by
  \begin{equation*}
    \erasemsg{\p[->]^+} = \erasemsg{\p[<-]^+} = \erasemsg{\p^+} = \p^+
    \,.
  \end{equation*}
  Message erasure of propositions and contexts, $\erasemsg{A^+}$, $\erasemsg{A^-}$, and $\erasemsg{\octx}$, is defined compositionally as the lifting of message erasure for atoms.
\end{definition}
For instance, the message erasure of $\inc[<-]$ is $\erasemsg{\inc[<-]} = \inc$, matching the intuition that the message $\inc[<-]$ serves to implement the specification's atom $\inc$ in the choreography from \cref{sec:chor-example-counter}.

Using this definition, we can define the property of specification-preserving as follows:
\begin{definition}[Specification-preserving]
  An ordered logic program $\chor$ is \vocab{specification-preserving} for specification $\spec$ if:
  \begin{enumerate}
  \item\label{defn:specification-preserving:completeness} for each step $\octx \trans[\spec] \octx'$ in the specification $\spec$, there is a non-empty trace $\octx \trans+[\chor] \octx'$ in the program $\chor$; and
  \item\label{defn:specification-preserving:soundness} for each step $\octx \trans[\chor] \octx'$ in the program $\chor$, either $\erasemsg{\octx} = \erasemsg{\octx'}$ or there is a step $\erasemsg{\octx} \trans[\spec] \erasemsg{\octx'}$ in the specification $\spec$.
  \end{enumerate}
\end{definition}
Thus, to be specification-preserving for $\spec$, a choreography $\chor$ must be a sound and complete implementation of specification $\spec$:
every step in $\spec$ must be reproducible by a non-empty trace in $\chor$ (part \labelcref{defn:specification-preserving:completeness}, completeness); and every step in $\chor$ must be either silent or its erasure reproducible by a single step in $\spec$ (part \labelcref{defn:specification-preserving:soundness}, soundness).

% The first part of this definition requires the choreography $\chor$ to be a complete implementation of the specification $\spec$: every step in $\spec$ must be reproducible by a trace in $\chor$.
% The second part requires $\chor$ to be a sound implementation of $\spec$: every step in $\chor$ must be either silent or reproducible by a single step in $\spec$.

\tikzcdset{arrow style=math font}
\tikzset{subscript/.style={shorten >=0.5em, "\ensuremath{#1}" {inner sep=0pt, sloped, at end, below right}}}

As an example, the object-oriented choreography for the binary counter, given in \cref{sec:chor-example-counter}, is indeed specification-preserving.
\begin{equation*}
  \begin{tikzcd}
    \omatch{\bit{1}, \inc}
      \arrow[rr, subscript=\spec]
      \arrow[dr, start anchor=south east, end anchor=north west, subscript=\chor]
      &
      & \ofill{\inc, \bit{0}} \\
    & \ofill{\bit{1}, \inc[<-]} \arrow[ur, start anchor=north east, end anchor=south west, subscript=\chor] &
  \end{tikzcd}
\end{equation*}

\begin{equation*}
  \begin{tikzcd}
    \omatch{\bit{1}, \inc[<-]}  \rar[subscript=\chor]  \dar[dash, "\erasemsg{-}" {below, sloped}]
      & \ofill{\inc, \bit{0}}   \dar[dash, "\erasemsg{-}"' {above, sloped}] \\
    \ofill[^e]{\bit{1}, \inc}   \rar[subscript=\spec]
      & \ofill[^e]{\inc, \bit{0}}
  \end{tikzcd}
\end{equation*}
\begin{gather*}
  \begin{lgathered}
    \omatch{\bit{1}, \inc} \trans[\spec] \ofill{\inc, \bit{0}} \\
    \ofill{\bit{1}, \inc} \trans[\chor] \ofill{\bit{1}, \inc[<-]} \trans[\chor] \ofill{\inc, \bit{0}}
  \end{lgathered}
  \\
  \begin{lgathered}
    \omatch{\eps, \inc} \trans[\spec] \ofill{\eps, \bit{1}} \\
    \ofill{\eps, \inc} \trans[\chor] \ofill{\eps, \inc[<-]} \trans[\chor] \ofill{\eps, \bit{1}}
  \end{lgathered}
  \\
  \begin{lgathered}
    \omatch{\bit{0}, \inc} \trans[\spec] \ofill{\bit{1}} \\
    \ofill{\bit{0}, \inc} \trans[\chor] \ofill{\bit{0}, \inc[<-]} \trans[\chor] \ofill{\bit{1}}
  \end{lgathered}
\end{gather*}

\begin{gather*}
  \begin{lgathered}
    \omatch{\inc} \trans[\chor] \ofill{\inc[<-]} \\
    \erasemsg{\ofill{\inc}} = \ofill[^e]{\inc} = \erasemsg{\ofill{\inc[<-]}}
  \end{lgathered}
  \\
  \begin{lgathered}
    \omatch{\bit{1}, \inc[<-]} \trans[\chor] \ofill{\inc, \bit{0}} \\
    \erasemsg{\ofill{\bit{1}, \inc[<-]}} = \ofill[^e]{\bit{1}, \inc} \trans[\spec] \ofill[^e]{\inc, \bit{0}} = \erasemsg{\ofill{\inc, \bit{0}}}
  \end{lgathered}
  \\
  \begin{lgathered}
    \omatch{\eps, \inc[<-]} \trans[\chor] \ofill{\eps, \bit{1}} \\
    \ofill{\eps, \inc} \trans[\spec] \ofill{\eps, \bit{1}}
  \end{lgathered}
  \\
  \begin{lgathered}
    \omatch{\bit{0}, \inc[<-]} \trans[\chor] \ofill{\bit{1}} \\
    \ofill{\bit{0}, \inc} \trans[\spec] \ofill{\bit{1}}
  \end{lgathered}
\end{gather*}
\end{document}

%%% Local Variables:
%%% TeX-master: "choreographies"
%%% End:
