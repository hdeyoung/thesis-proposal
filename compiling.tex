\section{Compiling choreographies}\label{sec:compile-choreo}

\NewDocumentCommand{\compilebasic}{m m m m}{%
  \llbracket #2\rrbracket^{#1}_{#3} = #4%
}
\NewDocumentCommand{\compilepos}{m m m m}{%
  \compilebasic{#1}{#2}{#3}{#4}%
}
\NewDocumentCommand{\compileneg}{m m m m}{%
  \compilebasic{#1}{#2}{#3}{#4}%
}

\begin{mathpar}
  \infer{\compilepos{d}{A^+_1 \fuse A^+_2}{c}{\mbind{\bv{c'} <- \mspawn{P_1} <- d; P_2}}}{
    \compilepos{d}{A^+_1}{c'}{P_1} &
    \compilepos{c'}{A^+_2}{c}{P_2}}
  \and
  \infer{\compilepos{d}{\one}{c}{\mfwd{c <- d}}}{
    }
  \\
  \infer{\compilepos{d}{\p^+}{c}{\mbind{c <- \p^+ <- d}}}{
    }
  \and
  \infer{\compilepos{d}{\p[->]^+}{c}{\moutputl{c.\p[->]^+; \mfwd{c <- d}}}}{
    }
  \and
  \infer{\compilepos{d}{\p[<-]^+}{c}{\moutputl{d.\p[<-]^+; \mfwd{c <- d}}}}{
    }
  \\
  \infer{\compilepos{d}{A^-}{c}{P}}{
    \compileneg{d}{A^-}{c}{P}}
\end{mathpar}

\begin{mathpar}
  \infer{\compileneg{d}{\with_i (\p[->, _i]^+ \limp \monad{A^+_i})}{c}{\mcase{d}{\p[->, _i]^+ => P_i}}}{
    \compileneg{d}{A^+_i}{c}{P_i}}
  \and
  \infer{\compileneg{d}{\with_i (\p[<-, _i]^+ \rimp \monad{A^+_i})}{c}{\mcase{c}{\p[<-, _i]^+ => P_i}}}{
    \compileneg{d}{A^+_i}{c}{P_i}}
  \and
  \infer{\compileneg{d}{\monad{A^+}}{c}{P}}{
    \compilepos{d}{A^+}{c}{P}}
\end{mathpar}

\NewDocumentCommand{\compilectx}{m m m m}{%
  \compilebasic{#1}{#2}{#3}{#4}%
}

\begin{mathpar}
  \infer{\compilectx{d}{\octx_1, \octx_2}{c}{\existq c'. A^+_1 \tensor A^+_2}}{
    \compilectx{d}{\octx_1}{c'}{A^+_1} &
    \compilectx{c'}{\octx_2}{c}{A^+_2}}
  \and
  \infer{\compilectx{d}{\octxe}{c}{c \eq d}}{
    }
  \\
  \infer{\compilectx{d}{A^+}{c}{\exec{P}}}{
    \compilepos{d}{A^+}{c}{P}}
  \and
  \infer{\compilectx{d}{\suspp{\p[->]^+}}{c}{\msgl{c,\p[->]^+,d}}}{
    }
  \and
  \infer{\compilectx{d}{\suspp{\p[<-]^+}}{c}{\msgl{d,\p[<-]^+,c}}}{
    }
\end{mathpar}




\subsection{}\label{sec:xyz}

\begin{theorem}[Completeness]
  If $\octx \trans_{\chor} \octx'$ and $\compilectx{d}{\octx}{c}{(\tctx ; \lctx)}$,
  then $(\tctx ; \lctx) \trans+ (\tctx' ; \lctx')$ for some $(\tctx' ; \lctx')$ such that $\compilectx{d}{\octx'}{c}{(\tctx' ; \lctx')}$.
\end{theorem}
The claim does not hold for stutters involving $\monad{A^+}$.
There is a transition
\begin{equation*}
  \monad{\p[<-]^+ \rimp \monad{\one}}
    \trans \p[<-]^+ \rimp \monad{\one}
\end{equation*}
but there is no nonempty trace
\begin{equation*}
  \exec{(\mcase{c}{\p[<-]^+ => \mfwd{c <- d}})}
    \trans+ \exec{(\mcase{c}{\p[<-]^+ => \mfwd{c <- d}})} \,.
\end{equation*}
Changing 



\begin{theorem}[Soundness]
  If $\mathcal{T} :: \lctx \trans\trans[\infty]$ is a maximal trace and $\compilectx{d}{\octx}{c}{\lctx}$, then
  \begin{itemize}
  \item there is a step $\octx \trans_{\chor} \octx'$ for some $\octx'$; and
  \item there are traces $\mathcal{T}_1 :: \lctx \trans+ \lctx'$ and $\mathcal{T}_2 :: \lctx' \trans[\infty]$ for some $\lctx'$ such that $\compilectx{d}{\octx'}{c}{\lctx'}$ and $\mathcal{T} \equiv \mathcal{T}_1; \mathcal{T}_2$.
  \end{itemize}
\end{theorem}

Given this soundness theorem, we cannot allow $\with_i (\p[_i, ->]^+ \limp A^-_i)$ and $\with_i (\p[_i, <-]^+ \rimp A^-_i)$ in general.
There is a transition
\begin{align*}\MoveEqLeft[0.5]
  \msgl{d', \lft[->], d} \;{,}\; \exec{(\mcase{d'}{\lft[->] => \mcase{c}{\rgt[<-] => \mbind{c <- \mdl <- d'}}})} \\
    &\trans \exec{(\mcase{c}{\rgt[<-] => \mbind{c <- \mdl <- d}})}
\end{align*}
but
\begin{equation*}
  \lft[->] \;{,}\; \lft[->] \limp (\rgt[<-] \rimp \monad{\mdl}) \ntrans
\end{equation*}

Even with this restriction, the soundness claim does not hold for unstable contexts.
There is a maximal trace
\begin{equation*}
  \exec{(\moutputl{d.\p[<-]^+; \mfwd{c <- d}})}
    \trans \msgl{d, \p[<-]^+, d'} \;{,}\; \exec{(\mfwd{c <- d'})}
    \trans \msgl{d, \p[<-]^+, c}
    \ntrans
\end{equation*}
but no transition $\p[<-]^+ \ntrans$.

\begin{sillcode*}
stype L = +{ left: L }
stype R = &{ right: R }

mid : {R <- L}
c <- mid <- d =
{ case d of
    left => (case c of
               right => c <- mid <- d) }
\end{sillcode*}



\begin{mathpar}
  \infer{\cdot ; x{:}\suspp{p^+} \Vdash_{\cdot} x :: p^+}{
    }
  \and
  \infer{\tctx_1, \tctx_2 ; \lctx_1, \lctx_2 \Vdash_{\sigma_2 \circ \sigma_1} p_1 \tensor p_2 :: A^+_1 \tensor A^+_2}{
    \tctx_1 ; \lctx_1 \Vdash_{\sigma_1} p_1 :: A^+_1 &
    \tctx_2 ; \lctx_2 \Vdash_{\sigma_2} p_2 :: A^+_2}
  \and
  \infer{a{:}\tau, \tctx ; \lctx \Vdash_\sigma a.p :: \exists a{:}\tau.A^+}{
    \tctx ; \lctx \Vdash_\sigma p :: A^+}
  \and
  \infer{\cdot ; \cdot \Vdash_{[t/a]} t/a :: a \eq t}{
    }
\end{mathpar}
If $\tctx_0 ; \lctx_0, A^+ \seq J \;\rightsquigarrow\; \sigma (\tctx_0, \tctx) ; \sigma(\lctx_0, \lctx) \seq \sigma\,J$, then $\tctx ; \lctx \Vdash_\sigma A^+$.

\subsection{Correctness}\label{sec:correctness}

\begin{itemize}
\item Well-typed choreographies compile to well-typed processes.
\item Executions of a choreography and its compiled form are bisimilar.
\end{itemize}

%%% Local Variables:
%%% TeX-master: "proposal"
%%% End:
