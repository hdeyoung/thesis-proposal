Proof-relevant computation is divided into two classes: proof search as computation, which is the basis of logic programming languages such as Celf~\autocite{Schack-Nielsen+Schuermann:IJCAR08}, Elf, and, of course, Prolog; and proof reduction as computation, which is the basis of functional programming languages such as ML and Haskell.

This thesis proposes to better understand the relationship between proof-search-as-computation and proof-reduction-as-computation in the context of concurrent, distributed programming.

\begin{itemize}
\item Two paradigms for logic-based computation: proof search (logic programming) and proof reduction (functional programming).
      Understand the relationship between them.
\item Proof search allows very concise, expressive programs (e.g., [parallel] bubblesort in one line), but does not provide many (any?) guarantees.
      Proof reduction provides strong guarantees (e.g., preservation, progress, and termination), but requires the programmer to supply many details (e.g., bubblesort).
\end{itemize}

%%% Local Variables:
%%% TeX-master: "proposal"
%%% End:
